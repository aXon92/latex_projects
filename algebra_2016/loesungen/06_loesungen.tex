\section{Lösungen zum Abschnitt 6}

\begin{loes}\hypertarget{loes:6.1}\
	Es gilt
	\begin{align*}
	|G| = 80 = 2^4 \cdot 5^1
	\end{align*}
	und nach den Sylowsätzen \ref{skript:6.4} müssen
	\begin{align*}
	n_2(G) &\equiv 1 \mod 2 \quad  \wedge \quad n_2(G) \mid 5\\
	n_5(G) &\equiv 1 \mod 5 \quad  \wedge \quad  n_5(G) \mid 16
	\end{align*}
	erfüllt.
	Angenommen es gibt $ 16  $ verschiedene $ 5 $-Sylowuntergruppen.
	Da Untergruppen der Ordnung $ 5 $ zyklisch sind, besitzen die Gruppen
	in $ \Syl_5(G) $ nur den trivialen Schnitt.
	Damit erhalten wir $ 16 \cdot 4  +1 = 65  $ Elemente.
	Also kann es nur eine $ 2 $-Sylowuntergruppe mit Ordnung $ 16 $ geben, womit
	\begin{align*}
	\Syl_2(G) = \lbrace P \rbrace 
	\Leftrightarrow
	P \nt G
	\end{align*}
	gilt.
	Also ist $ G $ nicht einfach.
 \end{loes}
 
 \begin{loes}\hypertarget{loes:6.2}\
 	\begin{enumerate}
 		\item[a)]
 		Wir nehmen ohne Beschränkung der Allgemeinheit an, dass $ p < q $ gilt.
 		Wegen
 		\begin{align*}
 		n_q(G) \equiv 1 \mod q \quad \wedge \quad n_q(G)  \mid p
 		\end{align*}
 		kann nur $ n_q(G) = 1 $ erfüllt sein.
 		Damit gilt nach \ref{skript:6.5}
 		\begin{align*}
 		\Syl_q(G) = \lbrace Q \rbrace 
 		\Leftrightarrow
 		Q \nt G,
 		\end{align*}
 		womit $ G $ nicht einfach ist.
 		
 		\item[b)]
 		Mit den Voraussetzungen erhalten wir
 		\begin{align*}
 		p \notin 1 + q \Z \quad \wedge \quad q \notin 1 + p \Z,
 		\end{align*}
 		womit $ p \neq n_q(G) $ und $ q \neq n_p(G) $ gilt.
 		Da $ p $ und $ q $ Primzahlen sind, 
 		kann nur
 		\begin{align*}
 		n_p(G) = 1  \quad \wedge \quad n_q(G) = 1 
 		\end{align*}
 		erfüllt sein.
 		Damit folgt dann
 		\begin{align*}
 		\Syl_p(G) = \lbrace P \rbrace &\Leftrightarrow P \nt G\\
 		\Syl_q(G) = \lbrace Q \rbrace &\Leftrightarrow Q \nt G
 		\end{align*}
 		und $ P,Q $ sind aufgrund ihrer Ordnungen zyklisch.
 		Nun gilt noch $ P \cap Q = 1  $ und $ P \cdot Q = G $.
 		Damit folgt dann
 		\begin{align*}
 		G \cong P \times Q \cong C_p \times C_q \cong C_{pq},
 		\end{align*}
 		womit $ G $ zyklisch ist.
 		Eine andere Variante wäre es über den Kommutator zu gehen.
 		Da $ P \cap Q  = 1$ und $ P,Q \nt G $ muss 
 		\begin{align*}
 		[p,q] = 1
 		\end{align*}
 		für beliebige $ p \in P $ und $ q \in Q $ gelten.
 		Wegen $ G = P \cdot Q $ ist $ G $ somit abelsch.
 		Damit folgt dann sofort
 		\begin{align*}
 		(pq)^{\ord(p) \cdot \ord(g)} = 1
 		\end{align*}
 		und wir sind fertig.
 	\end{enumerate}
 \end{loes}
 
 \begin{loes}\hypertarget{loes:6.3}\
 	Wir wissen zunächst, dass
 	\begin{align*}
 	n_p(G) \equiv 1 &\mod p \quad \wedge \quad n_p(G) \mid q \\
 	n_q(G) \equiv 1 &\mod q \quad \wedge \quad n_q(G) \mid p^2
 	\end{align*}
 	erfüllt sein müssen.
 	Nun müssen wir drei Fälle betrachten:
 	\begin{itemize}
 		\item $ q < p $:
 		 In diesem Fall folgt sofort, dass $ n_p(G) = 1 $ sein muss.
 		 Damit gilt 
 		 \begin{align*}
 		 \Syl_p(G) = \lbrace P \rbrace \Leftrightarrow P \nt G,
 		 \end{align*}
 		 womit $ G $ nicht einfach ist und eine normale $ p $-Sylowuntergruppe existiert.
 		 Nach \ref{skript:5.5} sind $ P $ und $ G/P $ auflösbar 
 		 und mit $ \ref{skript:4.11}  $ auch $ G $.
 		 
 		 \item $ q > p^2 $:
 		 Hier folgt wieder sofort, dass $ n_q(G) = 1 $ sein muss.
 		 Mit der selben Argumentation wie im ersten Fall
 		 existiert eine $ q $-Sylowuntergruppe und $ G $ ist auflösbar.
 		 
 		 \item $ p < q < p^2 $:
 		 Aufgrund der Voraussetzungen wissen wir, dass 
 		 $ n_q(G)  \in \lbrace 1 , p , p^2 \rbrace$
 		 gilt und erhalten schon wieder drei Fälle:
 		 \begin{itemize}
 		 	\item $ n_q(G) = 1 $:
 		 	Hierfür einfach die zwei ersten Punkte anschauen.
 		 	
 		 	\item $ n_q(G) = p $:
 		 	Dieser Fall kann wegen
 		 	\begin{align*}
 		 	n_q(G) \equiv 1 \mod q \quad \wedge \quad p<q
 		 	\end{align*}
 		 	nicht erfüllt sein.
 		 	
 		 	\item $ n_q(G) = p^2 $:
 		 	In diesem Fall gibt es $ p^2 $ Untergruppen der Ordnung $ q $.
 		 	Also erhalten wir $ p^2 \cdot (q-1) $ Elemente der Ordnung $ q $
 		 	und es bleiben noch $ p^2 $ Elemente die fehlen.
 		 	Somit ist $ n_p(G) = 1 $.
 		 	Den Rest kennen wir ja schon.
 		 \end{itemize} 
 	\end{itemize}
 	Insgesamt finden wir immer entweder eine normale $ p $-Sylowgruppe oder eine normale $ q $-Sylowgruppe und $ G $ ist auflösbar.
 \end{loes}
 
\begin{loes}\hypertarget{loes:6.4}\
	Wenn wir uns die Eigenschaft \textbf{(3)} aus \ref{skript:6.4} aufschreiben,
	erhalten wir sofort
	\begin{align*}
	n_5(G) = 1 \quad \wedge \quad n_7(G) = 1. 
	\end{align*}
	Damit folgt mit \ref{skript:4.11} und \ref{skript:5.5} auch direkt die Auflösbarkeit.
	Also sind Gruppen der Ordnung $ 6125 $ immer auflösbar.
\end{loes}

\begin{loes}\hypertarget{loes:6.4}\
	\begin{enumerate}
		\item[a)]
		Sei $ |G| = p^a \cdot m $ mit $ \ggT(p, m ) = 1 $.
		Da $p$ kein Teiler von $ [G:N] $ ist, muss
		\begin{align*}
		|N| = p^a \cdot \tilde{m}
		\end{align*}
		mit $ \ggT(p,\tilde{m}) = 1 $ und $ m = c \cdot \tilde{m} $ gelten.
		Also erhalten wir mit
		\begin{align*}
		P \in Syl_p(N) \Rightarrow |P| = p^a
		\quad \wedge \quad
		Q \in Syl_p(G) \Rightarrow |G| = p^a
		\end{align*}
		die gesuchte Aussage.
		
		\item[b)]
		Sei $ P \in \Syl_p(N)$.
		Dann gilt insbesondere auch $ P \leq G $ mit $ |P| = p^a $.
		Nach den Sylowsätzen existiert dann ein $ S \in \Syl_p(G)$
		mit $ P \leq S $.
		Aufgrund von a) folgt somit $ P = S $.
		Nun wählen wir ein $ Q \in \Syl_p(G) $ und wollen zeigen,
		dass $ Q $ in $ \Syl_p(N) $ liegt.
		Da die Menge der Sylowuntergruppen nicht leer ist,
		existiert ein $ P \in \Syl_p(N) $.
		Mit der selben Argumentation wie in der Hinrichtung liegt $ P $ in $ \Syl_p(G) $.
		Also existiert ein $ g \in G $ mit $ g^{-1} P g = Q $.
		Wegen $ N \nt G $ gilt
		\begin{align*}
		g^{-1} P g \in \Syl_p(N)
		\end{align*}
		für alle $ g \in G $, womit auch $ Q $ in $ \Syl_p(N) $ liegt.
		Insgesamt erhalten wir $ \Syl_p(N) = \Syl_p(G)$.
	\end{enumerate}
\end{loes}