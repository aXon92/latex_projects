\section{Lösungen zum Abschnitt 5}

\begin{loes}\hypertarget{loes:5.1}\
	Es gilt 
	\begin{align*}
	|G| = 1 + 8 + 18 = 27,
	\end{align*}
	womit $ G $ eine abelsche $ p $-Gruppe ist.
	Damit können wir \ref{skript:5.8} und \ref{skript:5.10}
	anwenden.
	Somit wir erhalten als Möglichkeiten
	\begin{align*}
	G &\cong C_{27} \\
	G &\cong C_3 \times C_3 \times C_3\\
	G &\cong C_9 \times C_3.	
	\end{align*}
	Die ersten beiden können nicht erfüllt sein.
	Die erste enthält Elemente der Ordnung $ 27 $ und
	die zweite nur Elemente der Ordnung $ 3 $.
\end{loes}

\begin{loes}\hypertarget{loes:5.2}\
	Sei $ \overline{a} \in G $ beliebig.
	Dann ist
	\begin{align*}
	\langle \overline{a} \rangle
	= \langle a + p^m \Z \rangle 
	\end{align*}	
	die kleinste Untergruppe, die $ \overline{a} $ enthält.
	Für $ 0 \leq i \leq m $ gilt
	\begin{align*}
	\ggT(a,p^m) = p^i,
	\end{align*}
	falls $ a = k \cdot p^i $, $ p \nmid k $ und $ k \in \Z $.
	Mit dem Lemma von Bezout existieren $ z_1 , z_2 \in \Z  $, sodass 
	\begin{align*}
	z_1 \cdot a + z_2 \cdot p^m  = \ggT(a,p^m) = p^i
	\end{align*} 
	gilt.
	Damit folgt $ p^i \in \langle a + p^m \Z \rangle$ und wegen $ a = k \cdot p^i $
	auch $ \overline{a} \in \langle p^i + p^m \Z  \rangle$.
	Insgesamt erhalten wir also
	\begin{align*}
	\langle a + p^m \Z \rangle = \langle p^i + p^m \Z \rangle.
	\end{align*}
	Wir wissen, dass
	\begin{align*}
	G = U_0 \unrhd U_1 \unrhd \dots \unrhd U_m = 1
	\end{align*}
	mit $ U_{i-1} / U_i \cong C_p $ einfach und $ U_i \nt U_{i-1} $ maximal
	eine Kompositionsreihe ist.
	Nun werden wir zeigen, dass es keine andere gibt.
	Sei $ j > i $, dann folgt mit $ U_i \cong \Z / p^{m-i} \Z $
	\begin{align*}
	U_i / U_j 
	\cong
	(\Z / p^{m-i} \Z) / (\Z / p^{m - j } \Z )
	\cong \Z / p^{j-i} \Z
	\cong U_{m-j+i} \ \text{einfach}. 
	\end{align*}
	Damit hat $ U_{m-j+i} $ nur sich selbst und die $ 1 $ als Normalteiler.
	Es folgt
	\begin{align*}
	m - j + i = m - 1 
	\Leftrightarrow
	j = i +1 ,
	\end{align*}
	womit es keine weitere Kompositionsreihe gibt.
\end{loes}

\begin{loes}\hypertarget{loes:5.2}\
	Es gilt
	\begin{align*}
	[G:N] = \frac{72}{8}  = 9,
	\end{align*}
	womit $ G/N $ eine $ p^2 $ Gruppe ist.
	Nach \ref{aufgabe:2.8} ist diese abelsch.
	Es existiert eine Untergruppe $ K \leq G/N $ mit Ordnung $ 3 $.
	Da $ G/N $ abelsch ist, ist diese auch ein Normalteiler.
	Nun betrachten wir den surjektiven Gruppenhomomorphismus
	\begin{align*}
	\varphi : G \to G/N , \ g \mapsto gN
	\end{align*}
	und wenden den Korresponedzsatz \ref{skript:5.4} an.
	Wir setzen
	\begin{align*}
	U := \varphi^{-1}(K) \leq G
	\end{align*}
	und durch eine kurze Überlegung erhält man auch $ \Ker \varphi \leq U $.
	Da nun $ \varphi(U) = K $ gilt, folgt $ U \nt G $ und
	\begin{align*}
	G/U \cong (G/N) / K 
	\Rightarrow 
	[G : U ] = 3
	\Rightarrow 
	|  U | = 24.
	\end{align*}
	Damit existiert eine solche Gruppe und diese ist normal.
\end{loes}

\begin{loes}\hypertarget{loes:5.3}\
	Sei $ G $ eine abelsche Gruppe der Ordnung $ 200 = 2^3 \cdot 5^2$.
	Da $ G $ abelsch ist, gibt es für $ 2 $ und $ 3 $ jeweils nur ein Sylowuntergruppe.
	Mit \ref{skript:6.9} erhalten wir dann
	\begin{align*}
	G \cong P \times Q
	\end{align*}
	für $ P \in \Syl_5(G)  $ und $ Q \in \Syl_2(G) $.
	Nun gibt es nach \ref{skript:5.8}
	\begin{align*}
	P \cong
	\begin{cases}
	C_5 \times C_5 \\
	C_{5^2} 
	\end{cases}
	\end{align*}
	zwei Isomorphietypen für $ P $.
	Analog erhalten wir mit
	\begin{align*}
	Q \cong
	\begin{cases}
	C_2 \times C_2 \times C_2 \\
	C_{2^2} \times C_2\\
	C_{2^3} 
	\end{cases}
	\end{align*}
	drei Isomorphietypen für $ Q $.
	Daraus erhalten wir mit 
	\begin{align*}
	C_5 \ &\times C_5 \times C_2 \times C_2 \times C_2 \\
	C_5 \  &\times C_5 \times C_{2^2} \times C_2 \\
	C_5 \ &\times C_5 \times C_{2^3}\\
	C_{5^2} &\times C_2 \times C_2 \times C_2 \\
	C_{5^2} &\times C_{2^2} \times C_2 \\
	C_{5^2} &\times C_{2^3}
	\end{align*}
	sechs mögliche Isomorphietypen für $ G $.
	Wenn wir lustig sind, können wir daraus auch die Elementarteilerzerlegungen konstruieren. 
\end{loes}

