\section{Lösungen zum Abschnitt 4}

\begin{loes}\hypertarget{loes:4.1}\
	\begin{enumerate}\
		\item[a)]
		 Wir wissen bereits, dass 
		 \begin{align*}
		 D_8 = \lbrace \id , r , r^2, r^3, s , sr , sr^2, sr^3 \rbrace
		 \end{align*}
		 und $ r^i s = s r^{-i} $ für $ i \in \lbrace 0 , 1 , 2 ,3 \rbrace $ gilt.
		 Durch nerviges Ausrechnen von
		 \begin{align*}
		 &[\id, \id]\\
		 &[r^i,r^j]\\
		 &[r^i,sr^j]\\
		 &[sr^i, sr^j]
		 \end{align*}
		 erhalten wir $ D_8^\prime = \lbrace \id , r^2 \rbrace = \langle r^2 \rangle $.
		 Diese Kommutatorgruppe ist abelsch, damit ist $ D^{(2)}_8 = 1 $.
		 Also ist $ D_8 $ auflösbar.
		 
		 \item[b)] 
		 Wir müssen analog zu a) alle Möglichkeiten für 
		 \begin{align*}
		 Q_8 = \lbrace \pm 1 , \pm i , \pm j , \pm k  \rbrace
		 \end{align*}
		 mit $ i^2 = j^2 = k^2 = i \cdot j \cdot k = -1 $
		 ausprobieren.
		 Damit erhalten wir $ D^\prime_8 = \lbrace \pm 1 \rbrace $.
		 Diese Gruppe ist abelsch, damit ist  $ D_8 $ auflösbar.
		  
		 
	\end{enumerate}
\end{loes}

