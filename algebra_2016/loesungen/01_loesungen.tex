\section{Lösungen zum Abschnitt 1}

\begin{loes}\hypertarget{loes:1.1}
Seien $U_1$ und $U_2$ zwei echte Untergruppen von $G$.
Nun nehmen wir an, dass 
\begin{align*}
G = U_1 \cup U_2
\end{align*}
gilt.
Zwischen $U_1$ und $U_2 $ kann es keine Teilmengenbeziehung geben, ansonsten wäre $U_1$ oder $U_2$ keine echte Untergruppe.
Nun wählen wir $g \in U_1 \setminus U_2$ und $h \in U_2 \setminus U_1$.
Es gilt $g \ast h \in G$ und mit unserer Annahme muss $g \ast h \in U_1 \cup U_2$ erfüllt sein.
Mit 
\begin{align*}
g \ast h \in U_1 \Rightarrow h \in U_1 \quad \text{und} \quad g \ast h \in U_2 \Rightarrow g \in U_2 
\end{align*}
erhalten wir einen Widerspruch zu unserer Wahl der Elemente. Damit war unsere Annahme falsch und das Gegenteil ist richtig.
\end{loes}

\begin{loes}\hypertarget{loes:1.2}
Die Aussage ist äquivalent zu
\begin{align*}
|G| = \infty \quad \Rightarrow \quad \text{es gibt unendlich viele Untergruppen von } G.
\end{align*}
Für den Beweis müssen wir zwei Fälle unterscheiden.
Der \textit{erste Fall} ist, dass alle Gruppenelemente endliche Ordnung besitzen.
Das heißt für alle $g \in G$ gilt $\ord(g) < \infty$.
Aus der Identität 
\begin{align*}
G = \bigcup\limits_{g \in G} \langle g \rangle
\end{align*}
folgt das es unendlich viele Untergruppen geben muss, ansonsten wäre $G$ endlich.
Der \textit{zweite Fall} ist, dass mindestens ein Gruppenelement mit unendlicher Ordnung existiert.
Sei $g \in G$ dieses Gruppenelement. Nun gilt
\begin{align*}
\langle g \rangle &\cong \Z \\
\langle g^2 \rangle &\cong 2\Z\\
&\vdots
\end{align*}
und somit haben wir auch in diesem Fall unendlich viele Untergruppen.
\end{loes}

\begin{loes}\hypertarget{loes:1.3}
Zunächst zeigen wir, dass $U$ eine Untergruppe von $G$ ist. 
Es gilt $E \in U$, wobei $E$ die Einheitsmatrix bezeichnet.
Mit
\begin{align*}
\begin{pmatrix}
1 & a & b \\
0 & 1 & c \\
0 & 0 & 1
\end{pmatrix}
\cdot
\begin{pmatrix}
1 & d & e \\
0 & 1 & f \\
0 & 0 & 1
\end{pmatrix}
= 
\begin{pmatrix}
1 & a+d & b +af + e \\
0 & 1 & c+f \\
0 & 0 & 1
\end{pmatrix}
\end{align*}
für $a,b,c,d,e,f \in \Z / 3\Z$ liegt auch das Produkt in $U$.
Wenden wir nun auf den ersten Faktor im obigen Produkt den Gauß-Algorithmus an, so erhalten wir 
\begin{align*}
\begin{pmatrix}
1 & -a & ac-b \\
0 & 1 & -c \\
0 & 0 & 1
\end{pmatrix}
\end{align*}
als Inverse. Damit ist auch die letzte Untergruppenbedingung erfüllt.
Für die Anzahl betrachten wir 
\begin{align*}
\begin{pmatrix}
1 & \ast & \ast \\
0 & 1 & \ast \\
0 & 0 & 1
\end{pmatrix}
\end{align*} 
und mit $| \lbrace 0,1,2 \rbrace^3 | = 27$ erhalten wir $|U| = 27$. 
Da die Ordnung eines Elements die Gruppenordnung teilt, haben wir die möglichen Ordnungen $1,3 ,9$ und $27$.
Es gilt offensichtlich $\ord(E) = 1$. Sei nun $A \in U \setminus \lbrace E \rbrace$.
Dann gilt
\begin{align*}
A &=
\begin{pmatrix}
1 & a & b \\
0 & 1 & c \\
0 & 0 & 1
\end{pmatrix}\\
A^2 &=
\begin{pmatrix}
1 & 2a & 2b+ac \\
0 & 1 & 2c \\
0 & 0 & 1
\end{pmatrix}\\
A^2 &=
\begin{pmatrix}
1 & 3a & 3b+3ac \\
0 & 1 & 3c \\
0 & 0 & 1
\end{pmatrix} 
= E
\end{align*}
für passende $a,b$ und $c$.
Damit gilt $\ord(A) = 3$ für alle $A \in U \setminus \lbrace E \rbrace$.
\end{loes}

\begin{loes}\hypertarget{loes:1.4} \
\begin{enumerate}
\item[a)]
Wir bemerken das $ g^2 = 1_G$ für alle $g \in G$ gilt.
Sei nun $g,h \in G$ beliebig. Dann folgt mit
\begin{align*}
(gh)^2 = 1_G
\Leftrightarrow
ghgh = 1 
\Leftrightarrow
gh = (gh)^{-1} = h^{-1} g^{-1} = h g
\end{align*}
die Kommutativität der Gruppe.
\item[b)]
Wegen $[G:U] = 2 $ wählen wir $g \in G \setminus U $ beliebig. Da $G$ eine disjunkte Vereinigung von Nebenklassen ist gilt
\begin{align*}
G = U   \overset{.}{\cup} g U = U  \overset{.}{\cup} U g
\end{align*}
und somit auch 
\begin{align*}
g U = U g \Leftrightarrow g^{-1} U g = U
\end{align*}
für alle $g \in G \setminus U$. Für alle anderen Gruppenelemente ist diese Identität so oder so erfüllt.
Insgesamt folgt $U \nt G$.
\item[c)]
Hierfür können wir ein Gegenbeispiel angeben.
Wir betrachten $S_3 $ mit $U = \langle (12) \rangle$. 
Mit Lagrange gilt
\begin{align*}
[S_3 : U ] = \frac{|S_3|}{|U|} = 3.
\end{align*}
Nun wählen wir $(23) \in S_3$ und erhalten mit
\begin{align*}
(23) \id &= \id (23)  \\
(231) = (23)(12) &= (12)(23) = (13)
\end{align*}
das Gegenbeispiel.
\item[d)]
Wir verwenden die Aufgabe 1.3 als Gegenbeispiel.
Es gilt
\begin{align*}
\begin{pmatrix}
1 & a & b \\
0 & 1 & c \\
0 & 0 & 1
\end{pmatrix}
\cdot
\begin{pmatrix}
1 & d & e \\
0 & 1 & f \\
0 & 0 & 1
\end{pmatrix}
&= 
\begin{pmatrix}
1 & a+d & b +af + e \\
0 & 1 & c+f \\
0 & 0 & 1
\end{pmatrix}\\
\begin{pmatrix}
1 & d & e \\
0 & 1 & f \\
0 & 0 & 1
\end{pmatrix}
\cdot
\begin{pmatrix}
1 & a & b \\
0 & 1 & c \\
0 & 0 & 1
\end{pmatrix}
&=
\begin{pmatrix}
1 & a+d & b +d c + e \\
0 & 1 & c+f \\
0 & 0 & 1
\end{pmatrix}
\end{align*}
und mit der Wahl $a = f = 1$ und $0$ sonst erhalten wir das Gegenbeispiel.
\end{enumerate}


\end{loes}

\begin{loes}\hypertarget{loes:1.5} \
\begin{enumerate}
\item[a)]
Sei $g \in G$ mit $\ord(g) = n \in \N$. Mit der Homomorphismuseigenschaft erhalten wir
\begin{align*}
\varphi(g)^n = \varphi(g^n) = \varphi(1_G) = 1_H.
\end{align*}
Damit muss $\ord(\varphi(g)) \leq n$ sein. Sei $m $ diese Ordnung.
Mit $n =  k \cdot m + r$ für $k \in \N$ mit $0 \leq r < m$ erhalten wir
\begin{align*}
1_H = \varphi(g)^n = \varphi(g)^{k \cdot m + r} = 1_H \cdot \varphi(g)^r =  \varphi(g)^r,
\end{align*}
womit $r = 0 $ sein muss. Dies folgt sofort, da $r < m $ ist. 
Also ist $m$ Teiler von $n$.
\item[b)]
Zunächst müssen wir die Untergruppeneigenschaft für $ \varphi^{-1}(M) $ zeigen.
Wegen $ \varphi(1_G)  = 1_H$ ist $ 1_G \in \varphi^{-1}(M) $.
Sei nun $ a,b \in \varphi^{-1}(M)   $. Dann folgt aus $ M \leq H $ durch
\begin{align*}
\varphi(a \ast b) = \varphi(a) \cdot \varphi(b) \in M \quad
\Rightarrow \quad
a \ast b \in \varphi^{-1}(M) 
\end{align*}
die Abgeschlossenheit unter der Verknüpfung aus $ G $. Durch eine ähnliche Argumentation
folgt auch
\begin{align*}
a \in \varphi^{-1}(M) \quad \Rightarrow \quad a^{-1} \in \varphi^{-1}(M) .
\end{align*}
Für die Normalteilereigenschaft wählen wir $ g \in \varphi^{-1}(M)  $ und $ x \in G $ beliebig.
Aufgrund $ M \nt H $ folgt mit
\begin{align*}
\varphi(x^{-1} g x) = \varphi(x)^{-1} \cdot \varphi(g) \cdot \varphi(x) \in M,
\end{align*}
dass $ x^{-1} g x \in \varphi^{-1}(M)  $ gilt.
Durch die beliebige Wahl gilt $ \varphi^{-1}(M)  \nt G $.
\item[c)] 
Hierfür haben wir ein Gegenbeispiel.
Sei $ G = S_2 $, $ H = S_3 $ und $ \varphi : G \to H $ die Einbettung von $ S_2$ in $ S_3 $.
Es gilt also $ \varphi(S_2) \cong S_2 $. 
Nun gilt $ S_2 \nt S_2 $, jedoch folgt mit Aufgabe 4 c), dass $ S_2 \ntrianglelefteq S_3 $ gilt.
\item[d)] 
Es gilt $ \varphi(U) \leq H $. Nun ist $ \varphi(U)  $ invariant unter allen Automorphismen von $ H $.
Damit gilt insbesondere auch
\begin{align*}
\gamma_x(\varphi(U)) = x^{-1} \varphi(U) x = \varphi(U)
\end{align*}
für alle $ x \in H $. Daraus folgt $ \varphi(U) \nt H $.
\end{enumerate}


\end{loes}