\section{Lösungen zum Abschnitt 10}

\begin{loes}\hypertarget{loes:10.1}\
	\begin{enumerate}
		\item[a)]
		Wir wenden Eisenstein für $ p = 2 $ an.
		Der Leitkoeffizient wird nicht von $ 2 $ geteilt,
		alle anderen Koeffizienten jedoch schon und es gilt $ 4 \nmid 2 $.
		Damit ist das Polynom irreduzibel.
		
		\item[b)]
		Dasselbe wie in a), nur für $ p=3 $.
		
		\item[c)]
		Dieses Polynom besitzt keine ganzzahligen Nullstellen.
		Da der Grad des Polynoms gleich drei ist, ist es irreduzibel.  
		
		\item[d)]
		Wir wenden das Reduktionskriterium für $ p=2 $ an.
		Das Polynom
		\begin{align*}
		x^4+x+1
		\end{align*}
		besitzt über $ \Z / 2 \Z $ keine Nullstellen.
		Es könnte also nur noch in Polynome vom Grad zwei zerfallen. 
		Da $ \Z / 2 \Z $ faktoriell ist, gilt dies auch für $ \Z/2 \Z[x] $.
		Also muss das Polynom ein Produkt aus irreduziblen Elementen sein,
		jedoch gilt
		\begin{align*}
		x^4+x+1 \neq (x^2 + x +1)^2 = x^4 +x^2 + 1
		\end{align*}
		und $ x^2 + x +1 $ ist das einzige irreduzible Polynom vom Grad zwei.
		Damit ist unser Polynom irreduzibel.
	\end{enumerate}
\end{loes}

\begin{loes}\hypertarget{loes:10.2}\
	\begin{enumerate}
		\item[a)]
		Es gilt
		\begin{align*}
		x^4+ x^2 +1
		=(x^2 + x +1 )^2 
		\end{align*}
		woraus die Reduziblität folgt.
		
		\item[b)]
		Mit Eisenstein für $ p = 2 $ erhalten wir die Irreduziblität in
		$ \Z[x] $ und nach dem Satz von Gauß folgt diese auch für $ \Q[x] $.
		
		\item[c)]
		Wir wenden das Reduktionskriterium für $ p = 2 $ an.
		Durch schnelles Nachrechnen sehen wir, dass
		\begin{align*}
		 x^6 + x^3 + 1		
		\end{align*} 
		keine Nullstellen besitzt.
		Das Polynom kann also noch zwei irreduzible Polynome vom Grad drei,
		vom Grad zwei und vier oder drei irreduzible Polynome vom Grad zwei
		zerfallen.
		Nach dem Ausprobieren aller Möglichkeiten, erfahren wir, dass dies nicht möglich ist.
		Also ist das Polynom irreduzibel.  
	\end{enumerate}
\end{loes}

\begin{loes}\hypertarget{loes:10.3}\
	\begin{enumerate}
		\item[a)]
		Wir nehmen an, dass
		\begin{align*}
		\prod \limits_{i=1}^n (x-a_i) -1
		\end{align*}
		reduzibel ist.
		Dann folgt 
		\begin{align*}
		f = g \cdot h
		\end{align*}
		mit $ \Grad(f) = n $ und $ \Grad(g),\Grad(h) < n $.
		Weiter erhalten wir 
		\begin{align*}
		f(a_i) = -1
		\Rightarrow
		h(a_i) \cdot g(a_i) = -1
		\Rightarrow
		h(a_i) = -g(a_i) \wedge h(a_i), g(a_i) \in \Z[x]^\ast
		\end{align*}
		für alle $ i \in \lbrace 1,\dots, n \rbrace $.
		Wegen $ n \geq \Grad(g)  + 1$ und $ n \geq \Grad(h)  +  1$
		folgt somit
		\begin{align*}
		h(x) = - g(x)
		\Rightarrow
		f(x) = -h^2(x)
		\Rightarrow
		\Grad(h) = \frac{n}{2}
		\end{align*}
		für alle $ x \in \Z $.
		Nun ist der Leitkoeffizient unseres Polynoms eins.
		Sei $ b $ der Leitkoeffizient von $ h $.
		Durch
		\begin{align*}
		-b^2 \cdot x^{\frac{n}{2} + \frac{n}{2}}
		= 1 \cdot x^n
		\end{align*}
		erhalten wir einen Widerspruch.
		Also war unsere Annahme falsch und unser Polynom ist irreduzibel.
		
		\item[b)]
		Zuerst betrachten wir zerlegbare normierte Polynome zweiten Grades.
		Diese können wir durch
		\begin{align*}
		(x-a)(x-b)
		\end{align*}
		für $ a,b \in \F_p $ zerlegen.
		Wir haben also
		\begin{align*}
		\binom{p+2-1}{2} =\frac{(p+1)p}{2}
		\end{align*}
		mögliche Zerlegungen, denn die Reihenfolge ist egal und wir können zurücklegen.
		Nun gibt es insgesamt $ p^2 $ Polynome der Form
		\begin{align*}
		x^2 + c\cdot x + d
		\end{align*}
		für $ c,d \in \F_p $.
		Damit erhalten wir 
		\begin{align*}
		p^2 - \frac{(p+1)p}{2}
		\end{align*}
		irreduzible Polynome von Grad zwei.
	\end{enumerate}
\end{loes}

\begin{loes}\hypertarget{loes:10.4}\
	\begin{enumerate}
		\item[a)]
		Seien $ a, b \in R $ koprim und $ I := (a,b) \unlhd R $.
		Da $ R $ ein Hauptidealbereich ist, existiert ein $ x \in R $
		mit 
		\begin{align*}
		(x) = (a,b)
		\Rightarrow
		(a) \subseteq (x) \wedge (b) \subseteq (x)
		\Rightarrow
		x \mid a \wedge x \mid b
		\Rightarrow
		x \in R^\ast
		\end{align*}
		und es existieren $ r ,s \in R $, sodass
		\begin{align*}
		x = a \cdot r + b \cdot s
		\end{align*}
		gilt. Da $ R $ kommutativ ist erhalten wir
		\begin{align*}
		1 = a \cdot \underbrace{r \cdot x^{-1}}_{c:=}  + b \cdot 
		\underbrace{s \cdot x^{-1} }_{d:=}
		\end{align*}
		als Resultat.
		
		\item[b)]
		Sei $ I \unlhd F $ mit $ \nicefrac{a}{b}  \in I$ .
		Da $ R $ ein Hauptidealring ist, existiert ein $ x \in R $ mit
		$ (x)_R  =(a,b)_R$ und es folgt
		\begin{align*}
		a = a^\prime \cdot x, \ b = b^\prime \cdot x
		\Rightarrow
		R (a^\prime \cdot x) +R (b^\prime \cdot x) = R x
		\Rightarrow
		R a^\prime + R b^\prime = R
		\Rightarrow
		(a^\prime, b^\prime) = R.
		\end{align*}
		Gilt nun $ x^\prime \mid a^\prime $ und $ x^\prime \mid b^\prime $,
		so folgt
		\begin{align*}
		(a^\prime, b^\prime) \subseteq (x^\prime)
		\Rightarrow 
		(x^\prime) = R
		\Rightarrow
		x^\prime \in R^\ast,
		\end{align*}
		womit $ a^\prime $ und $ b^\prime $ koprim sind.
		Nun gilt 
		\begin{align*}
		\frac{a}{b} = \frac{a^\prime}{b^\prime}
		\end{align*}
		und mit a) existieren $ c,d \in R $, sodass
		\begin{align*}
		1 = a^\prime \cdot c + b^\prime \cdot d
		\Rightarrow
		\frac{1}{b^\prime} = \frac{a^\prime}{b^\prime} \cdot c + d
		\Rightarrow
		\frac{1}{b^\prime} \in F
		\end{align*}
		folgt.
		Damit erhalten wir direkt
		\begin{align*}
		\left(\frac{a^\prime}{b^\prime}\right)_F = (a^\prime)_F.
		\end{align*}
		Diese Aussage wenden wir nun auf
		\begin{align*}
		\left(\frac{a_1}{b_1}\right) 
		\subseteq
		\left(\frac{a_1}{b_1}, \frac{a_2}{b_2}\right) 
		\subseteq
		\dots
		\subseteq
		I \unlhd F
		\end{align*}
		an und erhalten
		\begin{align*}
		(a_1^\prime) 
		\subseteq
		(a_1^\prime, a_2^\prime)
		\subseteq
		\dots
		\subseteq
		I \unlhd F.
		\end{align*}
		Wir betrachten nun $ J \unlhd R $ mit
		\begin{align*}
		J = (a_1\prime)_R \cup (a_1^\prime,a_2^\prime)\cup \dots
		=(z_1)_R \cup (z_2)_R \cup \dots = (z)_R,
		\end{align*}
		womit ein $ i \in \N $ mit $ z \in (z_i) $ existiert.
		Damit folgt dann
		\begin{align*}
		(z_i) = (z_{i+1}) = \dots
		\end{align*}
		und es gilt
		\begin{align*}
		a_i^\prime \in (z)_R \subseteq (z)_F
		\Rightarrow 
		I = (z)_F.
		\end{align*}
	\end{enumerate}
\end{loes}