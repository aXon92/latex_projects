\section{Lösungen zum Abschnitt 11}

\begin{loes}\hypertarget{loes:11.1}\
	\begin{enumerate}
		\item[a)]
		Zunächst bemerken wir, falls $ d $ teilt $ n $ gilt, folgt
		\begin{align*}
		E_d \subseteq E_n \Rightarrow E_d^\ast \subseteq E_d \subseteq E_n.
		\end{align*}
		Nun wählen wir $ \alpha \in E_n $ beliebig
		und setzen $ d = \ord(\alpha) $.
		Damit folgt dann $ \alpha \in E_d^\ast $ und mit Lagrange gilt
		\begin{align*}
		d \mid n 
		\Rightarrow 
		E_n = \biguplus \limits_{d \mid n} E_d^\ast.
		\end{align*}
		Die Vereinigung muss disjunkt sein, sonst gäbe es Elemente mit verschiedenen Ordnungen.
		Nun gilt
		\begin{align*}
		x^n -1 
		= \prod \limits_{\alpha \in E_n} (x - \alpha)
		= \prod \limits_{d \mid n} \cdot \left( \prod \limits_{\alpha \in E_d^\ast} (x- \alpha) \right)
		=\prod \limits_{d \mid n} \Phi_d
		\end{align*}
		und damit folgt
		\begin{align*}
		n = \Grad(x^n - 1)
		=\Grad\left(\prod \limits_{d \mid n} \Phi_d\right)
		= \sum \limits_{d \mid n} \Grad(\Phi_d)
		= \sum \limits_{d \mid n} \Phi_d.
		\end{align*}
		
		\item[b)]
		Die zweite Aussage wurde bereits in a) gezeigt.
		Die Erste werden wir mit Induktion zeigen.
		Für $ n=1 $ folgt $ \Phi_1 = x - 1 \in \Z[x] $.
		Sei nun $ n > 1 $, dann ist nach Induktionsvoraussetzung $ \Phi_d \in \Z[x] $
		für $ d < n $.
		Nun gilt
		\begin{align*}
		x^n - 1 
		=\prod \limits_{d \mid n} \Phi_d
		= \Phi_n \cdot 
		\underbrace{\prod \limits_{d \mid n, d < n} \Phi_d}_{=: g \in \Z[x]}
		= \Phi_n \cdot g 
		\end{align*}
		und da $ x^n -1  $ normiert ist folgt mit Divison mit Rest, dass 
		$ \Phi_n \in \Z[x] $ ist. 
	\end{enumerate}
\end{loes}