\section{Lösungen zum Abschnitt 8}

\begin{loes}\hypertarget{loes:8.1}\
	\begin{enumerate}
		\item[a)]
		Wir haben bereits gezeigt, dass alle Untergruppen von $ \Z $
		die Form $ m\Z $ besitzen.
		Sei $ m \in \Z $, $ a \in \Z $ und $ b \in m \Z $ beliebig.
		Dann gilt
		\begin{align*}
		a \cdot b = a \cdot k \cdot m \in m \Z
		\end{align*}
		und wir sind aufgrund der Kommutativität von $ \Z $ fertig.
		
		\item[b)]
		Sei $ I \unlhd K[x] $ ein Ideal.
		Wir wählen $ f \in I $ mit $ \Grad(f) $ minimal
		und $ x \in I $ beliebig.
		Nun wenden wir Polynomdivision an und erhalten
		\begin{align*}
		x = \underbrace{q \cdot f}_{\in I} + r , \ \Grad(r) < \Grad(g). 
		\end{align*}
		Also ist auch $ r \in I $ und aus der Minimalität von $ f $ folgt $ r = 0 $.
		Insgesamt gilt $ x = q \cdot f $, womit $ K[x] $ ein Hauptidealring ist.
		
		\item[c)]
		Fehlt noch.
		
		\item[d)]
		Fehlt noch.  
	\end{enumerate}
\end{loes}

\begin{loes}\hypertarget{loes:8.2}\
	Sei $ I $ ein Primideal von $ R $
	und $ a + I,b+I \in R/I $ mit
	\begin{align*}
	0 + I = (a  + I ) \cdot (b + I ) = ab + I.
	\end{align*}
	Daraus erhalten wir mit der Primidealeigenschaft
	\begin{align*}
	a \cdot b \in I
	\Rightarrow
	a \in I \vee b \in I
	\Rightarrow 
	(a + I) = 0 \vee (b + I) = 0
	\end{align*}
	die Nullteilerfreiheit.
	Da $ I \neq R $ ist, ist $ R/I $ ein Integritätsring.
	Umgekehrt nehmen wir nun an, dass $ R/I $ ein Integritätsring ist.
	Sei $ a \cdot b \in I $, dann folgt
	\begin{align*}
	0 + I = ab + I = (a+I)\cdot (b+I)
	\end{align*}
	und aus der Nullteilerfreiheit erhalten wir
	\begin{align*}
	a + I = 0 \vee b + I = 0
	\Rightarrow 
	a \in I \vee b \in I.
	\end{align*} 
	Damit $ I $ ein Primideal.
\end{loes}

\begin{loes}\hypertarget{loes:8.3}\
	Sei $ r \in R $ beliebig
	und $ f,h \in R[x] $ mit $ f \cdot g \in P := \Ker \varphi_r $.
	Dann gilt
	\begin{align*}
	0 = (f \cdot g)(r) = f(r) \cdot g(r)
	\Rightarrow
	f(r) = 0 \vee g(r) = 0
	\Rightarrow
	f \in P \vee g \in P,
	\end{align*}
	womit $ P $ ein Primideal ist.
\end{loes}

\begin{loes}\hypertarget{loes:8.4}\
	Wir verwenden den chinesischen Restsatz.
	Durch 
	\begin{align*}
	17^{68} \mod 10
	\rightarrow
	(17^{68} \mod 2, 17^{68} \mod 5 )
	\rightarrow
	 \dots
	\rightarrow
	(1 \mod 2, 1 \mod 5 )
	\rightarrow
	1 \mod 10
	\end{align*}
	erhalten wir als letzte Ziffer die $ 1 $.
	Nun zum zweiten Teil.
	Hier erhalten wir nach ein wenig herumrechnen
	\begin{align*}
	14^{200} \mod 100
	\rightarrow
	\dots
	\rightarrow
	(0 \mod 4, 1 \mod 25),
	\end{align*}
	wodurch
	\begin{align*}
	1 + 25 \cdot k \equiv 0 \mod 4
	\end{align*}
	erfüllt sein muss.
	Dies tritt zuerst bei $ k= 3 $ auf.
	Damit sind die letzten beiden Ziffern $ 76 $.
\end{loes}