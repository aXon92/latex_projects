\subsection{Aufgaben zum Abschnitt 3}

\begin{exe}\label{aufgabe:3.1} 
	\begin{enumerate}\
		\item[a)]
		Sei $ G $ eine Gruppe und sei $ h : G \to H $ ein Gruppenhomomorphismus.
		Zeigen Sie, dass $ h $ ein Isomorphismus ist genau dann, wenn
		es einen Gruppenhomomorphismus $ h^\prime: H \to G $ gibt,
		sodass $ h \circ h^\prime  = \id_H $ und $ h^\prime \circ h = \id_G $ ist.
		
		\item[b)] 
		Sei $ G $ eine Gruppe,
		$ H \nt G $ und $ K \nt H $.
		Gilt dann auch $ K \nt G $?.
	\end{enumerate}
	\hyperlink{loes:3.1}{Lösung}
\end{exe}

\begin{exe}\label{aufgabe:3.2}
	Seien $ G ,H$ Gruppen und $ S $ eine Teilmenge von $ G $, die $ G $ erzeugt.
	\begin{enumerate}
		\item[a)]
		Seien $ \phi,\psi $ Gruppenhomomorphismen von $ G $ nach $ H $.
		Zeigen Sie:
		\begin{align*}
		\forall s \in S: \ \phi(s) = \psi(s)
		\Rightarrow
		\phi = \psi
		\end{align*}  
		
		\item[b)]
		Wie viele Gruppenhomomorphismen gibt es von $ S_3 $ nach $ C_2 $? 
	\end{enumerate}
	\hyperlink{loes:3.2}{Lösung}
\end{exe}

\begin{exe}\label{aufgabe:3.3}
	Sei $ N $ eine normale Untergruppe von $ G $
	und $ L $ eine normale Untergruppe von $ G/N $.
	\begin{enumerate}
		\item[a)]
		Zeigen Sie,
		dass es eine normale Untergruppe $ K $ von $ G $ gibt
		mit $ N \subseteq K $, sodass
		$ L = K/N = \lbrace kN \ | \ k \in K \rbrace  $ gilt.
		
		\item[b)] 
		Zeigen Sie,
		dass die Abbildung
		\begin{align*}
		\Phi : G/N \to G/K, \ gN \mapsto gK
		\end{align*}
		für $ g \in G $ ein wohldefinierter Gruppenhomomorphismus ist
		und bestimmen sie den Kern von $ \Phi $.
	\end{enumerate}
	\hyperlink{loes:3.3}{Lösung}
\end{exe}