\subsection{Aufgaben zu Abschnitt 1}

\begin{exe}\label{aufgabe:1.1} 
Zeigen Sie: Eine endliche Gruppe ist nicht Vereinigung zweier echter Untergruppen.
\hyperlink{loes:1.1}{Lösung}
\end{exe}

\begin{exe}\label{aufgabe:1.2} 
Zeigen Sie: Eine Gruppe mit endlich vielen Untergruppen ist endlich.
\hyperlink{loes:1.2}{Lösung}
\end{exe}

\begin{exe}\label{aufgabe:1.3} 
Sei $G$ die Gruppe der regulären oberen $3 \times 3$ Dreiecksmatritzen über $\Z / 3 \Z$.
Nun sei 
\begin{align*}
U :=  \lbrace A = (a_{ij}) \in G \ | \ a_{11} = a_{22} = a_{33} = 1 \rbrace.
\end{align*}
Zeigen sie, dass $U $ eine Untergruppe von $G$ ist.
Berechnen sie außerdem die Anzahl der Elemente in $U$ und deren mögliche Ordnungen.
\hyperlink{loes:1.3}{Lösung}
\end{exe}

\begin{exe}\label{aufgabe:1.4} 
Sei $G$ eine Gruppe. Welche der folgenden Aussagen sind korrekt?
\begin{enumerate}
\item[a)]
Falls für alle $g \in G \setminus \lbrace 1_G \rbrace$ gilt, dass $\ord(g) = 2$ ist, dann ist $G$ abelsch.
\item[b)]
Sei $U$ eine Untergruppe vom Index $2$ in $G$. Dann ist $U$ ein Normalteiler von $G$.
\item[c)]
Sei $U$ eine Untergruppe vom Index $3$ in $G$. Dann ist $U$ ein Normalteiler von $G$.
\item[d)]
Falls für alle $g \in G \setminus \lbrace 1_G \rbrace$ gilt, dass $\ord(g) = 3$ ist, dann ist $G$ abelsch.
\end{enumerate}
\hyperlink{loes:1.4}{Lösung}
\end{exe}

\begin{exe}\label{aufgabe:1.5} 
Sei $\varphi : G \to H$ ein Gruppenhomomorphismus.
Welche der folgenden Aussagen sind korrekt?

\begin{enumerate}
\item[a)]
$\ord(\varphi(g)) | \ord(g)$.
\item[b)]
$M \nt H \quad \Rightarrow \quad \varphi^{-1}(M) \nt G$.
\item[c)]
$N \nt G \quad \Rightarrow \quad \varphi(N) \nt H$.
\item[d)]
Sei $U \leq G$ und $\varphi(U)$ invariant unter allen Automorphismen von $H$.
Dann gilt $\varphi(U) \nt H$.
\end{enumerate}
\hyperlink{loes:1.5}{Lösung}
\end{exe}

\begin{exe}\label{aufgabe:1.6} 
Zeigen sie das Untergruppenkriterium.
\end{exe}