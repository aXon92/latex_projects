\subsection{Aufgaben zu Abschnitt 6}

\begin{exe}\label{aufgabe:6.1}
	Zeigen Sie, dass es keine einfache Gruppe der Ordnung $ 80  $ gibt.
	\hyperlink{loes:6.1}{Lösung}
\end{exe}

\begin{exe}\label{aufgabe:6.2}
	Sei $ G $ eine Gruppe der Ordnung $ p \cdot q $, wobei $ p $ und $ q $
	zwei verschiedene Primzahlen sind.
	\begin{enumerate}
		\item[a)]
		Zeigen Sie, dass $ G $ nicht einfach ist.
		
		\item[b)]
		 Zeigen Sie, wenn
		 \begin{align*}
		 p \not\equiv 1 \mod q \quad 
		 \wedge
		 \quad q \not\equiv 1 \mod p
		 \end{align*}
		 gilt, ist $ G $ zyklisch.
	\end{enumerate}
	\hyperlink{loes:6.2}{Lösung}
\end{exe}

\begin{exe}\label{aufgabe:6.3}
	Sei $ G $ eine Gruppe der Ordnung $ p^2q $, wobei $ p $ und $ q $ verschiedene Primzahlen sind.
	Zeigen Sie, dass $ G $ entweder eine normale $ p $-Sylowuntergruppe oder eine normale $ q $
	Sylowuntergruppe besitzt.
	Ist $ G $ auflösbar?
	\hyperlink{loes:6.3}{Lösung}
\end{exe}

\begin{exe}\label{aufgabe:6.4}
	Sei $ G $ eine Gruppe der Ordnung $ 6125 = 5^3 \cdot 7^2  $.
	\begin{enumerate}
		\item[a)]
		Bestimmen Sie die Anzahl der $ 5 $-Sylowuntergruppen und der $ 7 $-Sylowuntergruppen.
		
		\item[b)] 
		Ist eine Gruppe dieser Ordnung immer auflösbar?		
	\end{enumerate}
	\hyperlink{loes:6.4}{Lösung}
\end{exe}

\begin{exe}
	Sei $ p $ eine Primzahl, $ G $ eine endliche Gruppe und $ N \nt G $ ein Normalteiler von $ G $, sodass $ p $ kein Teiler von $ [G:N] $ ist.
	\begin{enumerate}
		\item[a)]
		Zeigen Sie:
		\begin{align*}
		P \in \Syl_p(N) \wedge Q \in \Syl_p(G) 
		\Rightarrow
		 |P | = |Q| 
		\end{align*} 
		
		\item[b)]
		Zeigen Sie:
		$ \Syl_p(N) = \Syl_p(G) $
	\end{enumerate}
	\hyperlink{loes:6.5}{Lösung}
\end{exe}