\subsection{Aufgaben zum Abschnitt 2}

\begin{exe}\label{aufgabe:2.1} 
	Bestimmen Sie in der Gruppe $ \Gl_2(\R) $ den Zentralisator
	\begin{align*}
	C_{\Gl_2(\R)}(S) := \lbrace A \in \Gl_2(\R) \ | \ AS = SA \rbrace
	\quad
	\text{für}
	\quad
	S = 
	\begin{pmatrix}
	1 & 0 \\
	0 & -1
	\end{pmatrix}.
	\end{align*}
	Ist dieser Zentralisator eine Untergruppe bzw. ein Normalteiler von $ \Gl_2(\R) $?
	\hyperlink{loes:2.1}{Lösung}
\end{exe}

\begin{exe}\label{aufgabe:2.2} 
	Sei $ G $ die Symmetriegruppe eines Quadrats.
	Die Gruppe besteht also aus allen Bewegungen der euklidischen Ebene, welche ein Quadrat in sich selbst überführen.
	\begin{itemize}
		\item[a)]
		Beschreiben Sie $ G $ als Permutationsgruppe auf der Eckmenge eines Quadrats.
		\item[b)] 
		Berechnen Sie das Zentrum $ Z(G) $.
		\item[c)]
		Bestimmen Sie alle Untergruppen von Untergruppen von $ G $ und ordnen Sie diese anhand ihrer Teilmengenbeziehung. 
	\end{itemize}
	\hyperlink{loes:2.2}{Lösung}
\end{exe}

\begin{exe}\label{aufgabe:2.3} 
	Zeigen Sie, dass jede Untergruppe einer zyklischen Untergruppe zyklisch ist.
	\hyperlink{loes:2.3}{Lösung}
\end{exe}

\begin{exe}\ \label{aufgabe:2.4} 
	\begin{enumerate}
		\item[a)]
		Zeigen Sie, ist $ G $ eine Gruppe und $ h \in G $ fest, so ist $ \gamma_h(g) = h^{-1}g h $
		ein Automorphismus von $ G $.
		\item[b)] 
		Zeigen Sie: 
		Die inneren Automorphismen einer Gruppe bilden einen Normalteiler der Gruppe aller Automorphismen von $ G $.
	\end{enumerate}
	\hyperlink{loes:2.4}{Lösung}
\end{exe}

\begin{exe}\label{aufgabe:2.5} 
	Sei $ G $ eine Gruppe und seien $ H,K $ Untergruppen von $ G $.
	\begin{enumerate}
		\item[a)]
		Zeigen Sie:
		$ H \nt K \Rightarrow K \subset N_G(H) $.
		
		\item[b)]
		Zeigen Sie:
		$ K \leq N_G(H) \Rightarrow K \cdot H \leq H  \wedge H \nt K \cdot H$. 
	\end{enumerate}
	\hyperlink{loes:2.5}{Lösung}
\end{exe}

\begin{exe}\label{aufgabe:2.6} 
	$ G \leq \Gl_n(K) $ operiere durch Matrixmultiplikation auf $ \R^n $.
	Bestimmen sie alle Bahnen der Operation,
	\begin{enumerate}
		\item[a)]
		wenn $ G = \Sl_n(\R) = \lbrace M \in \Gl_n(\R) \ | \ \det(M) = 1 \rbrace $
		die spezielle lineare Gruppe ist.
		
		\item[b)]
		wenn $ G = O(n) = \lbrace M \in \Gl_n(\R) \ | \ M^\top \cdot M = M \cdot M^\top = I $ die orthogonale Gruppe ist.
		
		\item[c)]
		wenn $ G $ die Gruppe der invertierbaren Diagonalmatrizen ist.
		
		\item[d)] 
		wenn $ G = B_n(\R) $ die Gruppe der regulären oberen Dreiecksmatrizen ist.
	\end{enumerate}
	\hyperlink{loes:2.6}{Lösung}
\end{exe}

\begin{exe}\ \label{aufgabe:2.7} 
	\begin{enumerate}
		\item[a)]
		Bestimmen Sie alle endlichen Gruppen mit zwei Konjugationsklassen.
		
		\item[b)]
		Bestimmen Sie alle endlichen $ p $ Gruppen mit drei Konjugationsklassen.
	\end{enumerate}
	\hyperlink{loes:2.7}{Lösung}
\end{exe}

\begin{exe}\label{aufgabe:2.8} 
	Sei $ G $ eine $ p $- Gruppe.
	\begin{enumerate}
		\item[a)]
		Zeigen Sie, dass der Index von $ Z(G) $ in $ G $ ungleich $ p $ ist.
		
		\item[b)]
		Zeigen Sie:
		Wenn $ |G| = p^2 $ ist, dann ist $ G $ abelsch.  
	\end{enumerate}
	\hyperlink{loes:2.8}{Lösung}
\end{exe}

\begin{exe}\label{aufgabe:2.9}
	Sei $ Y = \lbrace 1,2,3,4,5\rbrace $, $ X = Y \times Y $,
	$ G = S_5 $ und
	\begin{align*}
	G \times X \to X, \ (\sigma,(a_1,a_2)) \mapsto (\sigma(a_1), \sigma(a_2) = \sigma.(a_1,a_2).
	\end{align*}
	\begin{enumerate}
		\item[a)]
		Zeigen Sie, 
		dass dies eine Gruppenoperation von $ G $ auf $ X $ definiert.
		
		\item[b)]
		Zeigen Sie,
		dass es unter dieser Operation genau zwei Bahnen gibt.
		
		\item[c)]
	    Zeigen Sie,
	    dass $ x_1,x_2 \in X $ mit 
	    $ \Stab_G(x_1) \cong S_4 $ und $ \Stab_G(x_2) \cong S_3 $
	    existieren.
	    Geben Sie außerdem den Zusammenhang zwischen der Ordnung des Stabilisators eines Elements und der Mächtigkeit der zugehörigen Bahn an. 
	\end{enumerate}
	\hyperlink{loes:2.9}{Lösung}
\end{exe}