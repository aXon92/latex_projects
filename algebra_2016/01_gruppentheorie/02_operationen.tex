\section{Operationen von Gruppen auf Mengen}

\begin{df}\label{skript:2.1}\index{Operation} \index{G-Menge}
Sei $X $ eine nicht-leere Menge und $G$ eine Gruppe.
Wir sagen \textbf{\textit{$G$ operiert auf $X$}},
falls eine Abbildung  $G \times X \to X, \ (g,x) \mapsto g.x$
mit
\begin{enumerate}
\item[\textbf{(i)}] $\forall x \in X: \ 1_G.x=x$
\item[\textbf{(ii)}] $\forall g,h \in G \ \forall x \in X: \ g.(h.x)=(gh).x $
\end{enumerate}
existiert. In diesem Fall sagen wir, dass $G$ \textbf{\textit{von links}} auf $X$ operiert.
Ersetzen wir jedoch \textbf{(ii)} durch
\begin{enumerate}
\item[$\textbf{(ii)}^\prime$] $\forall g,h \in G \ \forall x \in X: \ g.(h.x) = (hg).x$,
\end{enumerate}
sprechen wir von einer \textbf{\textit{Rechtsoperation}}.
Falls wir $x.g$ anstatt $g.x$ schreiben, wird $\textbf{(ii)}^\prime  $ durch $(x.h).g = x.(hg)$ ersetzt.
In der Regel werden wir aber Linksoperationen betrachten.
$X$ bezeichnen wir auch als \textbf{\textit{$G$-(Links)Menge}}.
\end{df}

\begin{genericdf}{Beispiele}\label{2.2} \
\begin{enumerate}
\item[\textbf{(1)}]
Die symmetrische Gruppe $S_n$ operiert durch
\begin{align*}
\sigma.i := \sigma(i)
\end{align*}
für $i \in \lbrace 1,...,n \rbrace$ auf $\lbrace 1,...,n \rbrace$.
\item[\textbf{(2)}]
Die lineare Gruppe $\Gl_n(K)$ operiert durch Matrixmultiplikation auf $V= K^n$.
Es gilt also 
\begin{align*}
A.v := A \cdot v
\end{align*}
für $A \in \Gl_n(K)$ und $v \in V$.
\end{enumerate}

\end{genericdf}

\begin{genericthm}{Lemma}\label{2.3} \index{Stabilisator}
Sei $X$ eine $G$-Menge und $x \in X$ beliebig, aber fest.
Dann gilt
\begin{align*}
G_x := \lbrace g \in G \ | \ g.x=x \rbrace \leq G.
\end{align*}
Wir bezeichnen $G_x$ als \textbf{\textit{(Punkt)-Stabilisator von $x$}} und schreiben hierfür $\Stab_G(x)$.
\end{genericthm}

\begin{proof}
Wir arbeiten nun die Punkte der Untergruppendefinition ab.
Wegen $1_G.x= x$ folgt direkt, dass $1_G$ in $G_x$ liegt.
Nun wählen wir $g,h \in G_x$ beliebig.
Mit 
\begin{align*}
(gh).x &=g.(h.x) =g.x  = x\\
x = 1_G.x = (g^{-1}g).x &=g^{-1}.(g.x)= g^{-1}.x
\end{align*} 
folgen die geforderten Eigenschaften sofort.
\end{proof}

\begin{sz}\label{2.4}\index{Bahn} \index{G-Bahn}
Sei $X$ eine $G$-Menge und $ x \in X$ beliebig, aber fest.
Dann nennen wir 
\begin{align*}
O_x := \lbrace g.x \ | \ g \in G \rbrace
\end{align*}
die \textbf{\textit{Bahn von $x$ unter der Operation}} von $G$.
Wir sprechen auch von \textbf{\textit{$G$-Bahnen}}.
Außerdem ist die Abbildung
\begin{align*}
\mu_x: O_x \to G / G_x , \ g.x \mapsto g G_x
\end{align*}
bijektiv.
\end{sz}

\begin{proof}
Wir werden nun zeigen, dass $\mu_x$ eine wohldefinierte, bijektive Abbildung ist.
\begin{itemize}
\item $\mu_x$ ist wohldefiniert: 
Wir wählen $g,h \in G$ mit $g.x=h.x$.
Damit folgt  $x = (g^{-1}h).x$ und somit auch $g^{-1}h \in G_x$.
Also existiert ein $u \in G_x$ mit $u = g^{-1}h $, was äquivalent zu $h = gu$ ist.
Mit 
\begin{align*}
h G_x = gu G_x = g G_x
\end{align*}
erhalten wir die Wohldefiniertheit.
\item $\mu_x$ ist offensichtlich surjektiv. Warum eigentlich?
Die einzelnen Klassen von $G / G_x$ enthalten die Elemente $g,h \in G$ mit $g.x = h.x$.
Durch 
\begin{align*}
g \sim h &\Leftrightarrow
\exists u \in G_x : \ gu = h \Leftrightarrow \exists u \in G_x: \ u = g^{-1}h 
\Leftrightarrow g^{-1}h \in G_x 
\Leftrightarrow (g^{-1}h).x = x \\
 &\Leftrightarrow g.x = h.x
\end{align*}
erhalten wir die Aussage.
\item
$\mu_x$ ist injektiv:
Durch 
\begin{align*}
g G_x = h G_x 
&\Rightarrow g^{-1}h G_x = G_x 
\Rightarrow g^{-1}h \in G_x
\Rightarrow (g^{-1}h).x = x \\
&\Rightarrow g.x = g.(g^{-1}h).x = h.x
\end{align*}
erhalten wir die Injektivität.
\end{itemize}
\end{proof}

\begin{genericthm}{Bahnensatz}\label{2.5} \index{Bahnensatz}
Sei $X$ eine $G$-Menge.
\begin{enumerate}
\item[\textbf{(1)}]
Dann ist $X$ eine disjunkte Vereinigung von Bahnen.
\item[\textbf{(2)}]
Sei $O$ eine Bahn und $x,y \in O$.
Dann existiert ein $h \in G$ mit
\begin{align*}
h^{-1} G_y h = G_x.
\end{align*}
Damit sind die Punktstabilisatoren verschiedener Elemente einer Bahn zueinander konjugiert.
\item[\textbf{(3)}]
Sei $|G| < \infty$. Dann gilt
\begin{align*}
|O_x| = \frac{|G|}{|G_x|}
\end{align*}
für alle $x \in X$.
Die Länge einer Bahn ist also der Index des Punktstabilisators eines Elementes der Bahn.
\end{enumerate}
\end{genericthm}

\begin{proof}\
\begin{enumerate}
\item[\textbf{(1)}]
Wir definieren die Relation
\begin{align*}
x \sim y \Leftrightarrow \exists g \in G : \ g.x=y 
\end{align*}
auf $X$. Mit 
\begin{align*}
1_G.x &= x  \Rightarrow x \sim x\\
x \sim y  &\Rightarrow \exists g \in G : \ g.x= y
\Rightarrow g^{-1}.(g.x) = g^{-1} y
\Rightarrow ... \Rightarrow x = g^{-1}.y 
\Leftrightarrow y \sim x\\
x \sim y &\wedge y \sim z 
\Rightarrow h.(g.x) = h.y = z  
\Rightarrow x \sim z
\end{align*}
für $x,y,z \in X$ erhalten wir die Eigenschaften einer Äquivalenzrelation.
Die Bahnen sind die Äquivalenzklassen dieser Relation und bilden somit eine disjunkte Vereinigung von $X$.
\item[\textbf{(2)}]
Wir wählen $x,y \in O$. Damit existiert ein $k \in G$ mit $k.x = y$.
Sei nun $g \in G_x$, dann gilt
\begin{align*}
(kgk^{-1}).y= (kgk^{-1})k.x =  (kgk^{-1}k).x = (kg).x = k.(g.x) = k.x = y
\end{align*}
und es folgt $kgk^{-1} \in G_y$. Nun haben wir $g$ beliebig gewählt, womit dann auch 
\begin{align*}
k G_x k^{-1} \subseteq G_y 
\end{align*}
gilt.
Nun müssen wir noch die Teilmengenbeziehung in die andere Richtung zeigen.
Sei nun $\tilde{g} \in G_y$. Mit $k^{-1}.y =x$ sehen wir durch analoges Vorgehen, dass
$k^{-1} \tilde{g} k \in G_x$ gilt. Insbesondere folgt
\begin{align*}
k (k^{-1} \tilde{g} k ) k^{-1}  \in G_y
\Rightarrow \tilde{g} \in k G_x k^{-1}
\Rightarrow G_y  \subseteq k G_x k^{-1},
\end{align*}
womit wir die andere Richtung erledigt haben.
Insgesamt haben wir 
\begin{align*}
G_y = k G_x k^{-1} \Leftrightarrow k^{-1} G_y k = G_x
\end{align*}
als Resultat.
\item[\textbf{(3)}]
Sei $G_x = \lbrace g \in G \ | \ g.x= x \rbrace$ ein Punktstabilisator.
Netterweise ist die Abbildung aus \ref{2.4} bijektiv, woraus wir direkt
\begin{align*}
|O_x| = | G / G_x| = \frac{|G|}{|G_x|} 
\end{align*}
erhalten.
\end{enumerate}
\end{proof}

\begin{genericdf}{Beispiele}\label{2.6} \
\begin{enumerate}
	\item[\textbf{(1)}]\index{Operation!transitiv} \index{G-Menge!transitiv}
	Sei $G = S_n$ und $X= \lbrace 1,...,n \rbrace$.
	Dann gibt es genau eine Bahn, denn wir finden zu jedem $i \in \lbrace 1,...,n \rbrace$ ein
	$\sigma_i \in S_n$ mit $\sigma_i(1) = i$.
	Eine Operation mit nur einer Bahn heißt \textbf{\textit{transitiv}} und X heißt dann auch 
	\textbf{\textit{transitive G-Menge}}.
	Sei nun $m \in \lbrace 1,...,n \rbrace$. Dann gilt
	\begin{align*}
	\Stab_{S_n}(m) = \lbrace \sigma \in S_n \ | \ \sigma(m) = m \rbrace
	\end{align*}
	und für $n = m$ sehen wir sofort, dass $\Stab_{S_n}(n) \simeq S_{n-1}$ gilt.
	Wir erhalten aus \ref{2.5} \textbf{(3)}
	\begin{align*}
	|O_n| = \frac{|S_n|}{|S_{n-1}|} = n,
	\end{align*}
	weswegen es nur eine Bahn geben kann.
\item[\textbf{(2)}]
	Sei $G = \Gl_n(K)$ und $X = K^n$ mit $A.v = A \cdot v$.
	Für den Nullvektor $0$ gilt $A.0 = 0$.
	Somit bildet $0$ eine einelementige Bahn und es gilt 
	$\Stab_{\Gl_n(K)}(0) = \Gl_n(K)$.
	Zu jeden $v \neq 0 $ aus $K^n$ gibt es eine reguläre Matrix $A$ mit $A.e_1 = v$,
	wobei $e_1$ der erste Standartbasisvektor und $v$ die erste Spalte der Matrix $A$ ist.
	Die andere Spalten erhalten wir, indem wir $v$ zu einer Basis von $K^n$ ergänzen.
	Ingesamt haben wir hier also die Bahnen $\lbrace 0 \rbrace$ und $K^n \setminus \lbrace 0 \rbrace$.
\end{enumerate}
\end{genericdf}

\begin{genericdf}{Beispiel}\label{2.7}
Die lineare Gruppe $\Gl_n(K)$ operiert auf $M_n(K)$ durch Konjugation, dass heißt es gilt
\begin{align*}
T.A = T^{-1}\cdot A \cdot T
\end{align*}
für $T \in \Gl_n(K)$ und $A \in M_n(K)$.
Die Bahnen bestehen dann aus zueinander ähnlichen Matrizen. 
Das Normalformproblem können wir nun durch Finden eines möglichst einfachen Repräsentanten einer Bahn lösen.
\end{genericdf}

\begin{genericdf}{Beispiel}\label{2.8} \index{Konjugation!Klassen} \index{Klassengleichung} \index{Stabilisator!Zentralisator}
Sei $G$ eine beliebige Gruppe. Durch 
\begin{align*}
h.g = h^{-1} g h 
\end{align*}
für $g,h \in G$ operiert $G$ durch Konjugation auf sich selbst.
Die Bahnen dieser Operation nennen wir \textbf{\textit{Konjugationsklassen}} der Gruppe $G$.
Für $g \in G $ bezeichnen wir mit $g^G$ oder $C_g$ die Konjugationsklasse von $g$.
Nach \ref{2.5} \textbf{(1)} ist
\begin{align*}
G =  \bigcup\limits_{g_i \in T} g_i^G
\end{align*}
eine disjunkte Vereinigung, wobei $T$ ein Repräsentantensystem der Konjugationsklassen von $G$ ist.
Falls $G$ endlich ist setzen wir $n := |T|$. Damit erhalten wir mit
\begin{align*}
|G| = |g_1^G| + ... + |g_n^G|
\end{align*}
die sogenannte \textbf{\textit{Klassengleichung}}. Weiter betrachten wir den Stabilisator
\begin{align*}
\Stab_G(x) = \lbrace h \in G \ | \ h^{-1} x h = x \rbrace =\lbrace h \in G \ | \ x h = h x \rbrace
\end{align*}
und bezeichnen diesen als \textbf{\textit{Zentralisator}} von $x$ in $G$.
Hierfür schreiben wir auch $C_G(x)$.
\end{genericdf}

\begin{df}\label{2.9} \index{Zentrum}
Das \textbf{\textit{Zentrum}} von $G$ ist
\begin{align*}
Z(G) = \lbrace g \in G \ | \ \forall x \in G : \ gx = xg \rbrace
\end{align*}
und es gilt $Z(G) \nt G$.
Das Zentrum $Z(G)$ ist also die Menge der Elemente aus $G$, deren Konjugationsklassen die Länge 1 haben.
Außerdem gelten $Z(G) \subseteq C_G(x)$ für alle $x \in X$ und 
\begin{align*}
Z(G) = \bigcap\limits_{x \in G } C_G(x).
\end{align*}
\end{df}

\begin{sz}\label{skript:2.10} \index{p-Gruppe}
Sei $p$ eine Primzahl und $G$ eine endliche Gruppe mit $|G| = p^n$ und $n \geq 1$.
Dann gilt
\begin{align*}
Z(G) \neq 1 = \lbrace 1_G \rbrace.
\end{align*}
Gruppen dieser Form bezeichnen wir auch als \textbf{\textit{$p$-Gruppen}}.
\end{sz}

\begin{proof}
Seien $C_1,...,C_k$ die Konjugationsklassen von $G$ mit $C_i = x_i^G$.
Dann erhalten wir 
\begin{align*}
|G| = \sum\limits_{i=1}^k |C_i|
\end{align*}
aus der Klassengleichung, wobei wir aus dem Bahnensatz \ref{2.5} und dem Satz von Lagrange \ref{1.14}
\begin{align*}
|C_i| = \frac{|G|}{|C_G(x_i)|} = p^{n_i}, \quad 1 \leq n_i \leq n
\end{align*}
erhalten. Setzen wir $x_1 = 1_G$, so gilt $|C_1| = 1$ und es folgt
\begin{align*}
|G| = p^n = \sum \limits_{i=1}^k p^{n_i} = 1 + \sum \limits_{i=2}^k p^{n_i}
\end{align*}
als Resultat. Nun nehmen wir an, dass $1_G$ das einzige zentrale Element ist.
Es gilt also $Z(G) = 1$, womit $n_i \geq 1$ für $i \geq 2$ folgt. Sonst hätte das Zentrum mehr als ein Element.
Nun gilt jedoch 
\begin{align*}
|G| \equiv 0 \mod p \wedge 1 + \sum \limits_{i=2}^k p^{n_i} \equiv 1 \mod p,
\end{align*}
wodurch $1$ durch $p$ teilbar ist. Dies ist ein Widerspruch.
Insgesamt gilt also $Z(G) \neq 1$.
\end{proof}

\begin{df}\label{skript:2.11} \index{Stabilisator!Normalisator} \index{Untergruppe!konjugierte}
Sei $G$ eine beliebige Gruppe und $X = \lbrace U \subseteq G \ | \ U \ \text{Untergruppe} \rbrace$.
Durch
\begin{align*}
g.U := g^{-1} U g
\end{align*}
für $g \in G$ und $U \in X$ ist eine Operation von $G$ auf $X$ definiert.
Die Konjugation $\gamma_g$ mit $g \in G$ ist ein Automorphismus, womit $g^{-1} U g \leq U $ gilt.
Wir nennen $g^{-1} U g $ die zu $U$ \textbf{\textit{konjugierte}} Untergruppe von $g$. 
Den Stabilisator 
\begin{align*}
\Stab_G(U) = \lbrace g \in G \ | \ g^{-1} U g = U \rbrace
\end{align*}
nennen wir \textbf{\textit{Normalisator}} von $U$ in $G$ und schreiben $N_G(U)$.
Mit \ref{2.4} und \ref{2.5} gilt außerdem $N_G(U) \leq G$.
\end{df}

\begin{genericdf}{Bemerkung} \label{2.12}
Es gilt $U \leq N_G(U)$ und $U \nt G \Leftrightarrow N_G(U) = G$.
\end{genericdf}

\begin{genericdf}{Bemerkung} \label{skript:2.13}
Sei $U \leq G$. Dann gilt für den Zentralisator
\begin{align*}
C_G(U) = \lbrace g \in G \ | \ \forall u \in U : g^{-1} u g = u \rbrace \leq G
\end{align*}
und $U \leq C_G(U) \Leftrightarrow U \ \text{abelsch}$.
Wir betrachten beispielsweise $G=U=S_3$.
Dort gilt $N_G(G) = N_{S_3}(S_3) = S_3$ und $C_G(G) = C_{S_3}(S_3) = Z(S_3) = 1$.
\end{genericdf}

\begin{genericdf}{Beispiel} \label{skript:2.14}
Sei $G = S_3$. Dann ist $\lbrace <(12)>, <(13)>, <(23)> \rbrace$ die Menge aller zu $<(12)>$ konjugierten Untergruppen. Dies ist die Bahn $O_{<(12)>}$ bezüglich der Operation in \ref{skript:2.11}.
Hierauf können wir nun den Bahnensatz anwenden
\begin{align*}
3 = |O_{<(12)>}| = |S_3 / \Stab_G(<(1,2)>)| = | S_3 / N_{S_3}(<(12)>) |  = \frac{|S_3|}{|N_{S_3}(<(12)>)|}
\end{align*}
und erhalten damit $|N_{S_3}(<(12)>)| = 2$.
\end{genericdf}

\begin{genericdf}{Bemerkung} \label{skript:2.15}
Sei $U \leq G$ und $|G / U | = 2$, dann gilt $U \nt G$. Hierfür vergleiche mit Blatt 1, Aufgabe 4b).
Die Aussage gilt nicht mehr für $|G / U | = 3$. Ein Gegenbeispiel haben wir im Prinzip in \ref{skript:2.14}
angegeben.
\end{genericdf}

\begin{genericdf}{Bemerkung} \label{skript:2.16}
Es gilt $Z(G) \nt G$. Außerdem sehen wir mit 
\begin{align*}
C_G(Z(G)) = G = N_G(Z(G)),
\end{align*}
dass $Z(G)$ ein abelscher Normalteiler ist.
\end{genericdf}

\begin{genericdf}{Bemerkung} \label{skript:2.17} \index{Core}
Sei $U \leq G$ und $G / U$ die Menge der Linksnebenklassen. Durch
\begin{align*}
G \times G / U \to G / U, \ (g,hU) \mapsto (gh) U
\end{align*}
wird $G / U$ zu einer transitiven $G$-Menge. Wir sehen durch schnelles Nachrechnen, dass diese Operation
transitiv ist.
Für $U \in G / U$ gilt
\begin{align*}
\Stab_G(gU) &= g U g^{-1}\\
\bigcap \limits_{g \in G } \Stab_G(gU) &\leq G
\end{align*}
und $G / U$ ist bezüglich Konjugation abgeschlossen. Nun setzen wir $U^g := g^{-1} U g$ und erhalten durch
\begin{align*}
\bigcap \limits_{g \in G } \Stab_G(gU) = \bigcap \limits_{g \in G } U^g \nt G.
\end{align*}
den kleinsten Normalteiler der in $U$ liegt. Dieser wird auch oft \textbf{\textit{Core}} von $G$ genannt.

\end{genericdf}