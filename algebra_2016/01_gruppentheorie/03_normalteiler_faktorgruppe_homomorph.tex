
\section{Normalteiler, Faktorgruppe und Homomorphismen}

\begin{df}\label{skript:3.1} \index{Faktorgruppe} \index{Quotientengruppe}
Sei $G$ eine Gruppe mit $N \nt G$.
Durch die Verknüpfung
\begin{align*}
G / N \times G / N \to G / N, \ (gN,hN) \mapsto (gh)N
\end{align*}
wird eine \textbf{\textit{Faktorgruppe von $G$ nach $N$}} auf der Nebenklassenmenge $G / N$ 
definiert.
Diese Gruppe nennen wir auch \textbf{\textit{Quotient}} oder \textbf{\textit{Quotientengruppe}}.
\end{df}

\begin{proof}
Leider wissen wir noch nicht, ob diese Verknüpfung wohldefiniert ist. Dies werden wir jetzt zeigen.
Sei $g_1N = g_2 N $ und $h_1N = h_2 N$.
Dann existieren $n_1, n_2 \in N $, sodass $g_1n_1 = g_2$ und $h_1n_2 = h_2$.
Mit der Normalteilereigenschaft erhalten wir durch
\begin{align*}
g_2 h_2 = g_1n_1h_1n_2 = g_1h_1 \underbrace{h_1^{-1}n_1h_1n_2}_{=: \tilde{n }\in N} = g_1h_1\tilde{n} 
\end{align*}
die Wohldefiniertheit. Die Assoziativität der Verknüpfung erhalten wir direkt aus der von $G$.
Das neutrale Element ist $N$ und das zu $gN$ inverse Element ist $g^{-1} N$.
\end{proof}

\begin{genericdf}{Beispiel}\label{skript:3.2} \index{Restklassenring}
Sei $G = (\Z,+)$ und $N = m\Z$. Da $G$ abelsch ist, gilt $N \nt G$. Die  Faktorgruppe
ist dann $(\Z / m \Z,+)$. Nun können wir auf $\Z / m \Z$ durch
\begin{align*}
(a + m\Z)(b + m\Z) := (a\cdot b + m\Z) 
\end{align*}
eine Multiplikation (eine zweite Verknüpfung) definieren. 
Dann ist $(\Z / m \Z,+, \cdot ) $ ein kommutativer Ring und wir bezeichnen diesen mit 
\textbf{\textit{Restklassenring modulo $m$}}.
\end{genericdf}

\begin{genericdf}{Beispiel}\label{skript:3.3}
Sei $G$ eine Gruppe mit $N \nt G$ und $\varphi : G \to Q $ eine Gruppenhomomorphismus mit $\varphi(N) = 1$.
Dann ist
\begin{align*}
\kappa : G / N \to Q , \ gN \mapsto \varphi(g)
\end{align*}
ein wohldefinierter Gruppenhomomorphismus. 
\begin{proof}\
\begin{itemize}
\item $\kappa$ ist wohldefiniert:
Sei $g_1N = g_2N$, dann existiert ein $n \in N $ mit $g_1 n = g_2$.
Damit erhalten wir durch
\begin{align*}
\varphi(g_2) = \varphi(g_1 n) = \varphi(g_1) \varphi(n) = \varphi(g_1) 1_Q = \varphi(g_1)
\end{align*}
die Wohldefiniertheit.
\item $\kappa$ ist ein Gruppenhomomorphismus:
Durch 
\begin{align*}
\kappa(g_1 N \cdot g_2 N ) = \kappa((g_1g_2) N ) = \varphi(g_1g_2) = \varphi(g_1) \cdot\varphi(g_2) = \kappa(g_1 N) \cdot \kappa(g_2 N)
\end{align*}
folgt dies sofort.
\end{itemize}
\end{proof}
\end{genericdf}

\begin{genericthm}{Grundeigenschaften von Gruppenhomomorphismen} \label{skript:3.4} 
\index{Homomorphismus!Gruppen!Eigenschaften}
Sei $\varphi : G \to H$ ein Gruppenhomomorphismus. Dann sind
\begin{enumerate}
\item[\textbf{(1)}]
$\varphi(1_G) = 1_H$
\item[\textbf{(2)}]
$\forall g \in G: \varphi(g)^{-1} = \varphi(g^{-1})$
\end{enumerate}
erfüllt.
\end{genericthm}

\begin{proof}
Den ersten Teil erhalten wir mit 
\begin{align*}
\varphi(1_G) = \varphi(1_G \cdot 1_G) = \varphi(1_G) \cdot \varphi(1_G)
\stackrel{\varphi(1_G)^{-1}}{\Rightarrow} 1_H = \varphi(1_G)
\end{align*}
sofort. Den zweiten erhalten wir durch
\begin{align*}
1_H = \varphi(1_G) = \varphi(g \cdot g^{-1}) = \varphi(g) \cdot \varphi(g^{-1}),
\end{align*}
denn $\varphi(g^{-1})$ ist invers zu $\varphi(g)$. Der Rest folgt aus der Eindeutigkeit des Inversen.
\end{proof}

\begin{sz}\label{skript:3.5}
Sei $\varphi : G \to H$ ein Gruppenhomomorphismus. Dann gelten:
\begin{enumerate}
\item[\textbf{(1)}] \index{Homomorphismus!Kern}
Wir nennen $\Ker \varphi := \lbrace g \in G \ | \ \varphi(g)= 1_H \rbrace$ den \textbf{\textit{Kern}} von $\varphi$. Für diesen gilt 
\begin{align*}
\Ker \varphi = 1 \Leftrightarrow \varphi \ \text{injektiv}
\end{align*}
und $\Ker \varphi \nt G$.
\item[\textbf{(2)}] \index{Homomorphismus!Bild}
Wir bezeichnen mit $\Bild \varphi := \lbrace h \in H \ | \ \exists g \in G : \varphi(g)=h \rbrace$
das \textbf{\textit{Bild}} von $\varphi$.
Sei $U \leq G$, dann gilt $\varphi(U) \leq H$.
Aus dieser Aussage folgt sofort $\Bild \varphi \leq H$.
\item[\textbf{(3)}] \index{volles Urbild}
Sei $V \leq H $, dann gilt $\varphi^{-1}(V) = \lbrace g \in G \ | \ \varphi(g) \in V \rbrace \leq G$.
Falls zusätzlich $V \nt H$ gilt, folgt $\varphi^{-1}(V) \nt G$.
Dies bezeichnen wir dann als \textbf{\textit{volles Urbild}}.
\end{enumerate}
\end{sz}

\begin{proof}\
\begin{itemize}
\item $\Ker \varphi$ ist Untergruppe von $G$: Wegen $\varphi(1_G) = 1_H$ gilt $1_G \in \Ker \varphi$.
Sei nun $g,h \in \Ker \varphi$.
Mit
\begin{align*}
\varphi(g^{-1}) =\varphi(g)^{-1} = 1_H^{-1} = 1_H &\Rightarrow g^{-1} \in \Ker \varphi\\
\varphi(g\cdot h) = \varphi(g) \cdot \varphi(h) = 1_H  &\Rightarrow gh \in \Ker \varphi 
\end{align*}
erhalten wir die anderen Untergruppeneigenschaften.
\item $\Ker \varphi$ ist Normalteiler von $G$: 
Sei $x \in G$ und $g \in \Ker \varphi$.
Dann gilt
\begin{align*}
\varphi(x^{-1} g x ) = \varphi(x^{-1}) \varphi(g) \varphi(x) = \varphi(x)^{-1} 1_H \varphi(x) = 1_H,
\end{align*}
womit $\Ker \varphi$ unter Konjugation invariant ist. Also folgt $\Ker \varphi \nt G$. 
\item $\Ker \varphi = 1 \Leftrightarrow \varphi \ \text{injektiv}$: 
Für die Hinrichtung nehmen wir an, dass $\varphi$ nicht injektiv ist.
Damit existieren $g,h \in G$ mit $g \neq h$ und $\varphi(g) = \varphi(h)$.
Wir erhalten durch 
\begin{align*}
\varphi(g^{-1}h) = \varphi(g)^{-1} \cdot \varphi(h) = 1_H
\end{align*}
eine Widerspruch zu $\Ker \varphi = 1$. Für die Rückrichtung nehmen wir an, dass $\Ker \varphi \neq \lbrace 1_G \rbrace$ gilt. Dann würde aber ein $g \neq 1_G$ existieren mit $\varphi(g) = 1_H$.
Dies ist ein Widerspruch zur Injektivität.
\end{itemize}
Die Aussagen aus \textbf{(2)} und \textbf{(3)} können wir durch ähnliches Vorgehen beweisen.
\end{proof}

\begin{genericdf}{Bemerkung}\label{3.6} \index{Homomorphismus!natürlich}
	Sei $ G $ eine Gruppe, $ N \nt G$ und $ G/N $ die zugehörige Faktorgruppe. Dann ist 
	\begin{align*}
	\kappa : G \to G/N, \ g \mapsto gN
	\end{align*}
	ein Gruppenhomomorphismus. Diesen nennen wir \textit{\textbf{natürlicher Homomorphismus}}.
	Für den Kern von $ \kappa $ gilt
	\begin{align*}
	\kappa(g) = N \Leftrightarrow g \in \Ker \kappa \Leftrightarrow g \in N,
	\end{align*}
	womit $ \Ker \kappa  = N $ und $ \Bild \kappa = G/ N $.
	
\end{genericdf}

\begin{genericdf}{Beispiel} \label{skript:3.7} \
	\begin{enumerate}
		\item[\textbf{(1)}]
		Sei $ K $ ein Körper. Wir betrachten die Determinante $ \det : \Gl_n(K) \to K^\ast $. Für deren Kern gilt
		\begin{align*}
		\Ker \det = \Sl_n(K))= \lbrace A \in \Gl_n(K) \ | \ \det(A) = 1 \rbrace,
		\end{align*}
		womit $ \Sl_n(K) \nt \Gl_n(K)$ folgt. Außerdem gilt $ \Bild \det = K^\ast $.
		\item[\textbf{(2)}] 
		Wir betrachten die Abbildung $ \exp : (\R,+) \to (\R^\ast, \cdot) , \ x \mapsto e^x$. Wegen
		\begin{align*}
		\exp(x+y) = e^{x+y} = e^x \cdot e^y = \exp(x) + \exp(y)
		\end{align*}
		ist $ \exp $ ein Homomorphismus. Dieser ist injektiv und es gilt $ \Bild \exp = \R^\ast_{\geq 0} $
	\end{enumerate}
\end{genericdf}

\begin{genericthm}{Lemma}\label{skript:3.8}
	Sei $ X $ eine nicht-leere $ G $ Menge. Dann ist 
	\begin{align*}
	\mu_g : X \to X, \ x \mapsto g.x
	\end{align*}
	für ein festes $ g \in G $ eine Bijektion. Die Abbildung 
	\begin{align*}
	\varphi: G \to \Sym(X), \ g \mapsto \mu_g
	\end{align*}
	ist ein Gruppenhomomorphismus mit 
	\begin{align*}
	\Ker \varphi = \lbrace g \in G \ | \ \forall x \in X: \ g.x = x \rbrace = \bigcap\limits_{x\in X} \Stab_G(x).
	\end{align*}
\end{genericthm}

\begin{proof}\
	\begin{itemize}
		\item $ \mu_g $ ist bijektiv: Sei $ g \in G $ fest. Dann können wir durch
		\begin{align*}
		\mu_{g^{-1}} : X \to X, \ x \to g^{-1}.x 
		\end{align*}
		die Umkehrabbildung explizit angeben. Damit gilt $ \mu_g \in \Sym(X) $ für alle $ g \in G $.
		\item $ \varphi $ ist ein Gruppenhomomorphismus: Aufgrund von
		\begin{align*}
		\varphi(gh) = \mu_{gh}: \ &x \mapsto (gh).x \\
		\varphi(g) \circ \varphi(h) = \mu_g \circ \mu_h : \  &x \mapsto g.(h.x)
		\end{align*}
		erhalten wir die Homomorphismuseigenschaft aus Axiom \textbf{(2)} aus \ref{skript:2.1}.
	\end{itemize}
\end{proof}

\begin{genericthm}{Satz von Cayley} \label{skript:3.9} \index{Satz!Cayley}
	Sei $ G $ eine endliche Gruppe mit $ |G| = n $. Dann gibt es einen injektiven
	Gruppenhomomorphismus $ \varphi : G \to S_n $. Damit ist $ G $ isomorph zu einer Untergruppe von $ S_n $.
\end{genericthm}

\begin{proof}
	Sei $ X = G $, dann ist $ X $ eine $ G $-Menge mit $ g.x = g \cdot x $. 
	Aus \ref{skript:3.8} erhalten wir dann
	\begin{align*}
	\varphi : G \to \Sym(X) \cong S_{|G|}
	\end{align*}
	als Gruppenhomomorphismus. Weiter ist $ g.x = x $ für alle $ x \in X $ nur für $ g = 1_G $ erfüllt.
	Damit gilt $ \Ker \varphi = 1 $ und somit ist $ \varphi $ injektiv.
\end{proof}

\begin{generic_no_num}{Bemerkung}
	Im Allgemeinen ist $ |G| = n $ aus \ref{skript:3.9} bestmöglich.
	Hat $ G $ eine Untergruppe $ U $ vom Index $ m $ mit 
	\begin{align*}
	\bigcap \limits_{g \in G} U^g = \bigcap \limits_{g \in G} g^{-1} U g ,
	\end{align*}
	so lässt sich \ref{skript:3.8} auf $ X = G/U $ mit Linksmultiplikation anwenden.
	Wir erhalten $ \varphi $ wie in \ref{skript:3.8}. 
	Es gilt $ \Ker \varphi \nt G $, $ \Ker \varphi \leq  U = \Stab_G(U) $ und in $ U $ 
	liegt kein nichttrivialer Normalteiler.
 mit \end{generic_no_num}

\begin{df} \label{skript:3.10} \index{Permutation!Matrix}
	Sei $ K $ ein Körper, $ n \geq 1 $ und $ \sigma \in S_n $.
	Die zu $ \sigma $-gehörige \textbf{\textit{Permutaionsmatrix}} $ A(\sigma) $ ist definiert durch
	\begin{align*}
	A(\sigma) := (a_{ij}) \in M_n(K)
	\end{align*}
	mit
	\begin{align*}
	a_{ij} = 
	\begin{cases}
	1 & \text{ falls } i=\sigma(j) \\
	0 & \text{ falls}
	\end{cases}.
	\end{align*}
	Außerdem gilt $ A( \sigma ) \cdot e_i = e_{\sigma(i)} $.
\end{df}

\begin{generic_no_num}{Beispiel}
	Für $ (123),(12) \in S_3 $ gilt
	\begin{align*}
	A((123)) =  
	\begin{pmatrix}
	0 & 0 & 1 \\
	1 & 0 & 0 \\
	0 & 1 & 0
	\end{pmatrix}
	\ \text{und} \ 
	A((12)) = 
	\begin{pmatrix}
	0 & 1 & 0 \\
	1 & 0 & 0 \\
	0 & 0 & 1
	\end{pmatrix}.
	\end{align*}
\end{generic_no_num}

\begin{genericthm}{Lemma} \label{skript:3.11}
	Die Abbildung $ \rho : S_n \to \Gl_n(K), \ \sigma \mapsto A(\sigma) $
	ist ein injektiver Gruppenhomomorphismus.
\end{genericthm}

\begin{proof}
	Sei $ \lbrace e_1,...,e_n \rbrace $ die kanonische Basis  von $ K^n $.
	Es gilt $ A ( \sigma )\cdot e_i = e_{\sigma(i)} $ und $ A(\sigma)  $ ist invertierbar.
	Wir wählen nun $ \sigma, \tau \in S_n$ beliebig, aber fest. Dann gilt 
	\begin{align*}
	(A(\tau) \cdot A(\sigma)) \cdot e_i = A(\tau) \cdot e_{\sigma(i)} = e_{\tau \circ \sigma (i)} = A(\tau \circ \sigma ) \cdot e_i
	\end{align*}
	für alle $ i \in \lbrace 1,...,n \rbrace $. Damit folgt $ \rho(\tau) \cdot \rho(\sigma) = \rho(\tau \circ \sigma) $, womit 
	$ \rho  $ ein Gruppenhomomorphismus ist.
	Die Injektivität folgt direkt, da $ \rho(\sigma) = \id $ nur für $ \sigma = \id  $ erfüllt ist.
\end{proof}

\begin{genericthm}{Folgerung} \label{skript:3.12}
	Sei $ G $ eine endliche Gruppe mit $ |G|= n $ und $ K $ ein Körper.
	Dann ist $ G $ isomorph zu einer Untergruppe von $ \Gl_n(K) $.
\end{genericthm}

\begin{proof}
	Die Aussage erhalten wir direkt durch Anwenden von \ref{skript:3.9} und \ref{skript:3.11}.
\end{proof}

\begin{generic_no_num}{Bemerkung}
	Wir können also das Rechnen auf endlichen Gruppen auf Matrizen übertragen.	
\end{generic_no_num}

\begin{sz} \label{skript:3.13} \index{Gruppe!alternierend} \index{Signumsfunktion}
	Sei $ K = \Q $. Die Abbildung
	\begin{align*}
	\varepsilon : S_n \to \Q^\ast , \ \sigma \mapsto \det(A(\sigma))
	\end{align*}
	ist ein Gruppenhomomorphismus. Falls $ \tau  $ eine Transposition ist gilt $ \varepsilon(\tau) = -1 $ und für das Bild
	gilt $ \Bild \varepsilon = \lbrace -1, 1 \rbrace $.
	Wir nennen $ \varepsilon $ \textbf{\textit{Signumsfunktion}} und kürzen diese mit $ \sgn $ ab.
	Der Kern $ \Ker \varepsilon $ heißt \textbf{\textit{alternierende Gruppe}} vom Grad $ n $. Für diese führen wir die Abkürzung
	$ A_n $ ein.
	Außerdem gilt $ A_n \nt S_n $ und $ \nicefrac{|S_n|}{|A_n|} = 2$.
\end{sz}

\begin{proof}
	Die Abbildung $ \varepsilon = \det \circ \rho $ ist eine Komposition von Gruppenhomomorphismen, also auch ein Gruppenhomomorphismus. Für $ \tau = (ij) $ entsteht $ A(\tau) $ durch Vertauschen der $ i $-ten und $ j $-ten Spalte.
	Damit folgt $ \det(A(\tau)) = -1 $. Insbesondere lässt sich jede Permutation als Produkt von Transpositionen schreiben, womit $ \varepsilon(\sigma) ) = \pm 1 $ für alle $ \sigma \in S_n  $ folgt. 
	Darüber hinaus erhalten wir die Äquivalenz
	\begin{align*}
	\sigma \in A_n \Leftrightarrow \sigma \ \text{ist gerades Produkt von Transpositionen}
	\end{align*}
	und mit \ref{skript:3.5} \textbf{(1)} gilt $ A_n = \Ker \varepsilon \nt S_n $.
	Weiter gilt
	\begin{align*}
	\sigma \in A_n &\Rightarrow (12)\sigma \notin A_n\\
	\sigma \notin A_n &\Rightarrow (12) \sigma \in A_n,
	\end{align*}
	woraus $ S_n = A_n \stackrel{.}{\cup} (12)A_n $ resultiert. Es gibt also zwei disjunkte Nebenklassen, damit ist der Index von $ A_n $ gleich $ 2 $.
\end{proof}

\begin{genericdf}{Bemerkung} \label{skript:3.14}
	Sei $ \sigma $ ein $ k $-Zykel. Dann gilt $ \sigma \in A_n \Leftrightarrow k \ \text{ungerade} $.
\end{genericdf}