\documentclass[11pt,a4paper]{scrreprt}
\usepackage[utf8]{inputenc}
\usepackage[T1]{fontenc}
\usepackage[ngerman]{babel}

\usepackage{makeidx}
\makeindex
\usepackage{paralist}
\usepackage{enumitem} 

%\usepackage {picins}
\usepackage{color}
\usepackage{float}

% -- Mathe
%\usepackage{gensymb}
%fancyhdr, lastpage, booktabs, xy
\usepackage{graphicx}
\usepackage{amsmath}
\usepackage{amsfonts}
\usepackage{amssymb}

\usepackage{amsthm}
\usepackage{rotating}
\usepackage{hyperref}
\usepackage{mathtools}
\usepackage{marginnote}
\usepackage{MnSymbol}
\usepackage[left=2cm,top=1.5cm,bottom=3cm,includeheadfoot]{geometry} 

% -- Malen
\usepackage{tikz}

\usetikzlibrary{patterns, decorations.pathreplacing, decorations.pathmorphing, arrows}

\usepackage{units}

\author{Philipp Beck, Paul Schwahn, Daniel Winkle}

\subtitle{Sommersemester 2016}
\title{Algebra}

\makeatletter
\renewcommand*{\thesection}{\arabic{section}}
\renewcommand*{\p@section}{\thechapter.}

%Befehle für bestimmte mathematische Symbole
\newcommand{\Gl}{\operatorname{GL}}
\newcommand{\Sl}{\operatorname{SL}}
\newcommand{\Sym}{\operatorname{Sym}}
\newcommand{\ord}{\operatorname{o}}
\newcommand{\id}{\operatorname{id}}
\newcommand{\Aut}{\operatorname{Aut}}
\newcommand{\nt}{\unlhd}
\newcommand{\td}[1]{\text{\rmfamily d} #1}
\newcommand{\Stab}{\operatorname{Stab}}
\newcommand{\Ker}{\operatorname{Ker}}
\newcommand{\Bild}{\operatorname{Im}}
\newcommand{\sgn}{\operatorname{sgn}}
\newcommand{\ggT}{\operatorname{ggT}}
\newcommand{\Syl}{\operatorname{Syl}}
\newcommand{\Grad}{\operatorname{Grad}}
\newcommand{\Char}{\operatorname{char}}
\newcommand{\Quot}{\operatorname{Quot}}
\newcommand{\Hom}{\operatorname{Hom}}
\newcommand{\Gal}{\operatorname{Gal}}

%Mengensymbole
\newcommand{\N}{\mathbb{N}}
\newcommand{\Z}{\mathbb{Z}}
\newcommand{\Zn}[1]{\mathbb{Z}/#1\mathbb{Z}}
\newcommand{\Q}{\mathbb{Q}}
\newcommand{\R}{\mathbb{R}}
\newcommand{\C}{\mathbb{C}}
\newcommand{\F}{\mathbb{F}}
\renewcommand{\i}{\textrm{i}}

\renewcommand\qedsymbol{$\blacksquare$}

%Big and italic
\newcommand{\bi}[1]{\textbf{\textit{#1}}}

\newtheoremstyle{exercise}% name
{15pt}% Space above
{5pt}% Space below
{}% Body font
{}% Indent amount: Indent amount: empty = no indent, \parindent = normal paragraph indent
{\bfseries}% Theorem head font
{:}% Punctuation after theorem head
{0.5em}% Space after theorem head: Space after theorem head: { } = normal interword space; \newline = linebreak
{}% Theorem head spec (can be left empty, meaning `normal')





\newcommand{\thistheoremname}{}

%Satz
\newtheorem{sz}{Satz}[section]

%Lemma
\newtheorem{lemma}[sz]{Lemma}
\newtheorem{gsz}[sz]{\thistheoremname}
\newtheorem*{gthm_no_num}{\thistheoremname}


\theoremstyle{definition}
\newtheorem{df}[sz]{Definition}
\newtheorem{gd}[sz]{\thistheoremname}

\newtheorem*{gdf_no_num}{\thistheoremname}


\theoremstyle{exercise}
\newtheorem{exe}{Aufgabe}[section]
\newtheorem{loes}{Lösung}[section]

%Für Beispiele , Bemerkungen
%Also alles was wie eine Definition aussehen soll, aber keine ist.
%\begin{genericdf}{(Hier den Text einfügen)}
%	Inhalt...
%\end{genericdf}
\newenvironment{genericdf}[1]{\renewcommand{\thistheoremname}{#1}\begin{gd}}{\end{gd}}
%Dasselbe ohne Nummerierung
\newenvironment{generic_no_num}[1]{\renewcommand{\thistheoremname}{#1}\begin{gdf_no_num}}{\end{gdf_no_num}}

%Hier für Sätze
\newenvironment{genericthm}[1]{\renewcommand{\thistheoremname}{#1}\begin{gsz}}{\end{gsz}}
%Analog zu Definitionen
\newenvironment{genericthm_no_num}[1]{\renewcommand{\thistheoremname}{#1}\begin{gthm_no_num}}{\end{gthm_no_num}}
\parindent0pt
%\parskip1ex


%----Dokument----

\begin{document}

\maketitle
\tableofcontents
\newpage



\chapter{Gruppentheorie}
\section{Grundbegriffe}

\begin{df}\label{1.1}
Eine nicht-leere Menge $G$ mit einer Verknüpfung $\ast$
heißt \textbf{\textit{Gruppe}}, falls
\begin{enumerate}
\item[\textbf{(i)}]
$\forall g,h,k \in G: \ (g \ast h) \ast k = g \ast (h \ast k) \ \quad  \quad \quad \quad \quad \quad \quad
  (\textit{assoziativ}) $   
\item[\textbf{(ii)}] 
$\exists ! e \in G \ \forall g \in G: \ g \ast e = e \ast g  = g  \  \quad  \quad \quad \quad \quad \quad \quad \quad \quad  (\textit{neutrales Element}) $   
\item[\textbf{(iii)}]
$\forall g \in G \ \exists ! g^{-1} \in G: \ g \ast g^{-1} = g^{-1} \ast g = e \ \  \quad \quad \quad \quad \quad \quad (\textit{inverses Element})$  
\end{enumerate}
erfüllt sind. Man schreibt dann auch $(G,\ast)$.
\index{Gruppe}
\end{df}



\begin{generic_no_num}{Bemerkungen}
\ 
\begin{enumerate}
\item[\textbf{(1)}]
Die Axiome $\textbf{(ii)}$ und $\textbf{(iii)}$ lassen sich mit
\begin{enumerate}
\item[$ \textbf{(ii)}^\prime $]
$\exists e \in G \ \forall g \in G : \ e \ast g = g$
\item[$ \textbf{(iii)}^\prime $]
$\forall g \in G  \ \exists h \in G : \ g \ast h = e$
\end{enumerate}
abschwächen und herleiten.
\item[\textbf{(2)}]
\textit{(Rechenregeln einer Gruppe)}
\begin{enumerate}
\item[ \textbf{(i)}]
$g  \ast x = g \ast y  \Rightarrow x = y  \qquad \qquad \ \textit{(Kürzungsregel)}$
\item[ \textbf{(ii)}]
$\forall g \in G : \ (g^{-1})^{-1} = g $
\item[\textbf{(iii)}]
$g \ast g = g \Rightarrow g = e  \qquad \qquad \qquad \textit{($e$ ist als einziges Element idempotent)}$
\item[\textbf{(iv)}]
$(g \ast h)^{-1} = h^{-1} \ast g^{-1}$
\item[\textbf{(v)}]
$\forall a,b \in G \ \exists ! x \in G : \ a \ast x = b$
\end{enumerate}
\item[\textbf{(3)}]
Eine Verknüpfung $\ast$ ist eine Abbildung der Form	$G \times G \to G , \ (g,h) \mapsto g \ast h$.
\end{enumerate}

\end{generic_no_num}

\begin{df}
Eine Gruppe $(G,\ast) $ mit $g \ast h = h \ast g$ für alle $g,h \in G$
heißt \textbf{\textit{kommutativ}} oder \textbf{\textit{abelsch}}.
\index{Gruppe!abelsch}
\end{df}

\begin{genericdf}{Beispiele}\label{1.3}\ 
\begin{enumerate}
\item[\textbf{(1)}]
Sei $X$ eine nicht-leere Menge und $\Sym(X) := \lbrace f : X \to X \ | \ f \ \text{ist bijektiv} \rbrace$.
$\Sym(X)$ bildet mit der Komposition von Abbildungen die \textbf{\textit{symmetrische Gruppe}} auf $X$.
Falls $X = \lbrace 1,...,n \rbrace$ ist, definieren wir die \textbf{\textit{symmetrische Gruppe vom Grad $n$}} durch
$S_n := \Sym(X)$ und es gilt $|S_n| = n!$.
\item[\textbf{(2)}]
$(\mathbb{Z},+)$, $(\mathbb{Q}^\ast,\cdot)$ und $O(n,\mathbb{R})$ sind Gruppen. Mit $\Q^\ast$ ist hier $\Q \setminus \lbrace 0 \rbrace$ gemeint.
\end{enumerate}
\index{Gruppe!symmetrisch}
\end{genericdf}

\begin{df}\label{1.4}
Sei $G$ eine Gruppe.
$U \subseteq G$ heißt \textbf{\textit{Untergruppe}}, falls
\begin{enumerate}
\item[\textbf{(i)}]
$1_G \in G$ 
\item[\textbf{(ii)}]
$\forall u,v \in U : \ u \cdot v \in U$
\item[\textbf{(iii)}]
$\forall u \in U: \ u^{-1} \in U $
\end{enumerate}
erfüllt sind.
Wir schreiben dann $U \leq G$.
\index{Untergruppe}
\end{df}

\begin{generic_no_num}{Bemerkungen} \

\begin{enumerate}
\item[\textbf{(1)}]
Sei $G$ eine Gruppe und $A,B \subseteq G$.
Dann definieren wir $A^{-1} := \lbrace a^{-1} \ | \ a \in A\rbrace$
und \\ 
$A \cdot B := \lbrace a \cdot b \ | \ a \in A \wedge b \in B \rbrace$.
Hierbei bezeichnen wir $A \cdot B$ auch als \textbf{\textit{Komplexprodukt}}. 

\item[\textbf{(2)}]
$U \subseteq G$ ist eine Untergruppe \textbf{genau dann}, \textbf{wenn}
$1_G \in U$, $U \cdot U \subseteq U$ und $ U^{-1} \subseteq U$. 

\item[\textbf{(3)}]
Sei $\lbrace U_i \rbrace_{i \in I}$ eine Familie von Untergruppen einer Gruppe $G$ und 
$I$ eine beliebige Indexmenge.
Dann gilt $\bigcap_{i \in I} U_i \leq G$.
\begin{proof}
\begin{align*}
x \in \bigcap_{i \in I} U_i \Rightarrow 
\forall i \in I : \ x \in U_i \stackrel{U_i \leq G}{\Rightarrow }
\forall i \in I : \ x^{-1} \in U_i \Rightarrow
x^{-1} \in \bigcap_{i \in I} U_i
\end{align*}
Die anderen Eigenschaften aus \ref{1.4} folgen durch analoges Vorgehen.
\end{proof}

\item[\textbf{(4)}]
\textbf{\textit{Untergruppenkriterium:}} \index{Untergruppe!Kriterium}
Sei $G$ eine Gruppe. Dann gilt
\begin{align*}
\emptyset \neq U \subseteq G \ \text{Untergruppe} 
\Leftrightarrow
\forall u,v \in U : \ u \cdot v^{-1} \in U.
\end{align*}
Falls $G$ endlich ist gilt sogar
\begin{align*}
\emptyset \neq U \subseteq G \ \text{Untergruppe} 
\Leftrightarrow
\forall u,v \in U : \ u \cdot v \in U.
\end{align*}
\end{enumerate}
\end{generic_no_num}

\begin{genericdf}{Beispiel}
$\Sl_n(K) := \lbrace A \in \Gl_n(K) \ | \det(A) = 1   \rbrace \leq \Gl_n(K)$
\end{genericdf}

\begin{df} \label{1.6}
Sei $G$ eine Gruppe und $S \subseteq G$.
Dann ist 
\begin{align*}
<S> := \bigcap \limits_{U_i \leq G, S \subseteq U_i} U_i
\end{align*}
das \textbf{\textit{Erzeugnis von $S$}}.\index{Erzeugnis}
\begin{enumerate}
\item[\textbf{(1)}]
$<S>$ ist die kleinste Untergruppe von $G$, die $S$ enthält.
Gilt $S = \lbrace g \rbrace$ für $g \in G$, dann schreiben wir $<g>$ anstatt $<S>$.
Wir nennen dies die von \textbf{\textit{$g$ erzeugte zyklische Untergruppe von $G$}}.\index{Untergruppe!zyklische}
Falls ein $g \in G$ existiert mit $G = < g>$, nennen wir $G$ \textbf{\textit{zyklisch}}.\index{Gruppe!zyklisch}  
\item[\textbf{(2)}]
Falls $S = \emptyset$ ist, gilt $<S> = \lbrace 1_G \rbrace =: 1$.
\end{enumerate}
\end{df}

\begin{genericthm}{Lemma}\label{1.7}
Sei $G$ eine Gruppe und $\emptyset \neq S \subseteq G$.
Dann gilt
\begin{align*}
<S> = \lbrace s_1 \cdot s_2 \cdots s_r \ | \ r \in \mathbb{N}, \ s_i \in S \cup S^{-1} \rbrace
\end{align*}
und für $S= \lbrace g \rbrace $ somit $<g> = \lbrace g^m \ | \ m \in \mathbb{Z} \rbrace$. \\
Hier fließt die Konvention $g^0 = 1_G$ und $g^{-k} = (g^{-1})^k$ für $k \in \mathbb{N}$ ein.
\end{genericthm}

\begin{proof}
Sei $H = \lbrace s_1 \cdot s_2 \cdots s_r \ | \ r \in \mathbb{N}, \ s_i \in S \cup S^{-1} \rbrace$.
Zunächst zeigen wir die Hinrichtung.
Wegen $1_G = s \cdot s^{-1}$ für beliebiges $s \in S$ gilt $1_G \in H$.
Durch schnelles Nachprüfen sind auch Produkte und Inverse von $H$ in $H$.
Also folgt $H \leq G$, $S \subseteq H$ und mit \ref{1.6} gilt $<S> \subseteq H$.
Für die Rückrichtung sei $U \leq G$ mit $S \subseteq U$.
Damit gilt $H \subseteq U$ und 
\begin{align*}
<S> = \bigcap \limits_{U \leq G, S \subseteq U} U
\Rightarrow 
H \subseteq <S>.
\end{align*}
Damit sind wir fertig.
\end{proof} 

\begin{genericthm}{Lemma}\label{1.8}
Sei $G = <g>$ mit $|G| = d$.
Dann gilt $G = \lbrace 1_G, g ,g^2,...,g^{d-1} \rbrace$
und \\ 
$d = \min\lbrace i \in \mathbb{N} \ | \ g^i = 1_G \rbrace$.
Für $k \in \mathbb{Z}$ mit $g^k = 1_G$ gilt $d | k$.
\end{genericthm}

\begin{proof}
Nach \ref{1.7} ist $G = \lbrace g^m \ | \ m \in \mathbb{Z} \rbrace$.
Aufgrund der Endlichkeit von $G$ existieren $i$ und $j$ 
mit $i > j > 0$ und $g^i = g^j$. Also folgt
\begin{align*}
g^{i-j} = g^j \cdot g^{-j} = 1_G, \quad i-j \in \mathbb{N},
\end{align*}
womit ein minimales $d^\prime \in \mathbb{N}$ mit $g^{d^\prime} = 1_G$ existieren muss. 
Nun wählen wir $k \in \mathbb{Z}$ beliebig, aber fest.
Dann existieren $q \in \mathbb{Z}$ und $0 \leq r \leq d^{\prime} $, sodass 
$k = q \cdot d^{\prime} + r$ gilt.
Damit folgt
\begin{align*}
g^k = g^{q \cdot d^{\prime} + r} = g^{q \cdot d^{\prime}} \cdot g^r
= \left( g^{d^{\prime}} \right)^q \cdot g^r = g^r
\end{align*}
und somit $<g> = \lbrace 1_G, g ,g^2,...,g^{d^\prime-1} \rbrace$.
Für $g^k = 1_G$ muss dann $g^r = 1_G$ gelten. 
In diesem Fall erhalten wir aufgrund der Minimalität von $d^\prime$, dass 
$r = 0 $ ist. Also gilt $ k = q \cdot d^\prime$ und somit $d^\prime | k$.
Wäre $g^i = g^j$ für $d^\prime \geq j > i >0$, würde
$g^{j-i} = 1_G$ für $1 \leq j-i < r$ folgen.
Dies ist ein Widerspruch zur Minimalität von $d^\prime$.
Insgesamt folgt also $d = d^\prime$.
\end{proof}

\begin{df}\label{1.9}\index{Ordnung!Gruppe}\index{Ordnung!Element}
Sei $G$ eine Gruppe. Die Kardinalität $|G|$ bezeichnet man als \textit{Ordnung von $G$}.
Für $g \in G$ nennen wir $|<g>|$ die \textbf{\textit{Ordnung des Elements $g$}} und schreiben hierfür $o(g)$.
\begin{enumerate}
\item[\textbf{(1)}]
Falls $G$ endlich ist, beschreibt $|G|$ die Anzahl der Elemente.
\item[\textbf{(2)}]
An $(\mathbb{R},+)$ sehen wir, dass $|G|$ überabzählbar sein kann.
\item[\textbf{(3)}]
Die Ordnung eines Elements $o(g)$ kann endlich oder abzählbar unendlich sein.
Im zweiten Fall schreiben wir $o(g) = \infty$.
\end{enumerate}
\end{df}

\begin{genericdf}{Beispiel}
\index{Permutation}
Sei $G = S_n$ die symmetrische Gruppe vom Grad $n$.
Ein Element $\pi \in S_n$ nennen wir \textit{Permutation}.
Durch
\begin{align*}
 \pi = \bigl(\begin{smallmatrix}
    1 & 2 & 3 & \cdots & n-1 & n \\
    \pi(1) & \pi(2) & \pi(3) & \cdots &  \pi(n-1)  & \pi(n)
  \end{smallmatrix}\bigr)
\end{align*}
können wir $\pi$ als Abbildung beschreiben. Jedoch ist dies ziemlich umständlich.
Um dies zu umgehen führen wir die \textbf{\textit{Zykelschreibweise}} ein.\index{Permutation!Zykelschreibweise}
\begin{figure}[H]
  \centering
    \begin{tikzpicture}
		\node (A) at (0,0)  {$a$};
		\node (B) at (3,0) {$\pi(a)$};
		\node (C) at (6,0) {$\pi^2(a)$};
		\node (D) at (6,-1) {$\pi^3(a)$};
		\node (E) at (3,-1) {$\pi^4(a)$};
		\node (F) at (0,-1) {$\pi^{k-1}(a)$};
		\draw[->] (A) to (B);
		\draw[->] (B) to (C);
		\draw[->] (C) to (D);
		\draw[->] (D) to (E);
		\draw[->,dashed] (E) to (F);
		\draw[->] (F) to (A);
\end{tikzpicture}
\caption{Zykel}
\end{figure}
Für ein $a \in \lbrace 1,...,n \rbrace$ ist $(a , \pi(a), \pi^2(a),...,\pi^{k-1}(a) )$ ein Zykel in $\pi$.
Hierbei setzen wir \\ 
$k := \min\lbrace i \in \mathbb{N}_0 \ | \ \pi^k(a) = a \rbrace$. Deswegen kommt eine Ziffer in einem Zykel auch nur ein einmal vor.
Mit den Äquivalenzklassen der Relation 
\begin{align*}
i \sim j \Leftrightarrow  \exists l \in \mathbb{N} : \pi(i)^l = j
\end{align*}
erhalten wir die verschiedenen Zykel bezüglich $\pi$.
Wir betrachten nun 
\begin{align*}
\pi = \bigl(\begin{smallmatrix}
    1 & 2 & 3 & 4 & 5  \\
    3 & 5 & 4 & 1 &  2  
  \end{smallmatrix}\bigr)
\end{align*}
aus $S_5$. In dieser Permutation sind die Zykel
\begin{figure}[H]
  \centering
    \begin{tikzpicture}
		\node (A) at (0,0)  {$1$};
		\node (B) at (4,0) {$2$};
		\node (C) at (1,0) {$3$};
		\node (D) at (2,0) {$4$};
		\node (E) at (5,0) {$5$};
		
		\draw[->] (A) to (C);
		\draw[->] (B) to (E);
		\draw[->] (C) to (D);
		\draw[->,out=90, in=90] (D) to (A);
		\draw[->,out=90, in=90] (E) to (B);
\end{tikzpicture}
\end{figure}
enthalten. Wir lassen die Zykel der Länge $1$ weg, wenn klar ist, in welchem $S_n$ gerechnet wird.
Nun betrachten wir noch ein konkretes Beispiel aus der $S_3$.
In $S_3$ sind $(1 2),(1 3), (2 3), (1 2 3),$ $(1 3 2)$ und $\id$ enthalten.
Es sind also fünf Untergruppen, die von einem Element erzeugt werden. Die zyklischen Untergruppen sind
$<(1 2)>, <(1 3)>, <(2 3)>, <(1 2 3)>$ und $< \id >$.
%\textcolor{red}{\textbf{Sind es wirklich 6? $(1 3 2)$ ist nämlich in $<(1 2 3)>$ enthalten. }}
\end{genericdf}

\begin{genericdf}{Beispiel}\label{1.11}
Sei $G = ( \mathbb{Z}, + )$. Nun fragen wir uns, wie die Untergruppen von $G$ aussehen.
Für $m \in \mathbb{N}_0$ sind $m  \mathbb{Z} = \lbrace z \in \mathbb{Z} \ | \ z = m \cdot x, \ x \in \mathbb{Z} \rbrace = < m >$  Untergruppen von $\mathbb{Z}$.
Aber haben alle Untergruppen von $\mathbb{Z}$ diese Form? Wir können diese Frage mit ja beantworten.
\begin{proof}
Sei $U \leq Z$ eine beliebige Untergruppe. Interessant wird es wenn $U \neq \lbrace 0 \rbrace $ ist.
Zunächst wissen wir, falls $ a \in U \setminus \lbrace 0 \rbrace$ ist auch $ -a \in U \setminus \lbrace 0 \rbrace$.
Da $\mathbb{Z}$ netterweise geordnet ist, finden wir auch $m := \min \lbrace a \in \mathbb{U} \ | \ a \in \mathbb{N} \rbrace$. Damit folgt $m \cdot n \in U$ für alle $n \in \mathbb{Z}$.
Damit wissen wir schon mal, dass $m  \mathbb{Z} \subseteq U$ gilt.
Sei $n \in U$ beliebig. Dann gilt $n = q \cdot m + r$ für $q,r \in \mathbb{Z}$ mit $0 \leq r < m$.
Aufgrund der Minimalität von $m$ folgt mit $r = n -q \cdot m \in U$, dass $r = 0$ sein muss. Also erhalten wir $n = q \cdot m \in m\mathbb{Z}$.
\end{proof}

\end{genericdf}

\begin{df}\label{1.12}\index{Linksnebenklassen}\index{Untergruppe!Nebenklassen} \index{Index}
Sei $G$ eine Gruppe und $U \leq G$. Mit 
\begin{align*}
g \sim h \Leftrightarrow \ \exists u \in U: \ g \cdot u = h
\end{align*}
für $g,h \in G$ erhalten wir eine Äquivalenzrelation auf $G$.
Deren Äquivalenzklassen bezeichnen wir als \textit{\textbf{Linksnebenklassen}} von $U$ in $G$.
Falls deren Anzahl endlich ist, nennen wir diese den \textit{\textbf{Index}} von $U$ in $G$ und schreiben hierfür auch $[G:U].$
Die Nebenklassen haben die Form
\begin{align*}
gU := \lbrace g\cdot u \ | \ u \in U \rbrace,
\end{align*}
wobei $g$ der Repräsentant von $gU$ ist.
\end{df}

\begin{genericdf}{Bemerkungen}\label{1.13} \ 
\begin{enumerate}
\item[\textbf{(1)}] \index{Rechtsnebenklassen}
Wir definieren analog durch
\begin{align*}
g \sim h \Leftrightarrow \ \exists u \in U: \ g = u \cdot h, \ Ug = \lbrace u \cdot g \ | \ u \in U \rbrace
\end{align*}
die \textbf{\textit{Rechtsnebenklassen}}.
Wichtig ist, dass im Allgemeinen $gU \neq Ug$ gilt.
Um dies zu verdeutlichen betrachten wir $G = S_3$ mit $U= <(12)> = \lbrace \id, (12)\rbrace $.
Mit 
\begin{align*}
(23)U &= \lbrace (23),(123) \rbrace \\
U(23) &= \lbrace(23),(132) \rbrace 
\end{align*}
folgt $(23)U  \neq U(23)$.
\item[\textbf{(2)}]\index{Normalteiler} \index{Untergruppe!Normalteiler} \index{normal}
Wir bezeichnen $U \leq G$ als \textbf{\textit{Normalteiler}} bzw. als \textbf{\textit{normal}} in $G$, falls
\begin{align*}
\forall g \in G: \ gU = Ug
\end{align*}
gilt und schreiben hierfür $ U \nt G$.
Sollte $G$ abelsch sein, sind alle Untergruppen von $G$ Normalteiler.
Mit \index{Zentrum}
\begin{align*}
Z(G) := \lbrace g \in G \ | \ \forall x \in G: \ g\cdot x = x \cdot g \rbrace \nt G
\end{align*}
bezeichnen wir das \textit{\textbf{Zentrum}} von $G$.
\end{enumerate}
\end{genericdf}

\begin{genericthm}{Satz von Lagrange}\label{1.14} \index{Satz!Lagrange}
Sei $G$ eine endliche Gruppe mit $U \leq G$. Dann wird $|G|$ von $|U|$ geteilt.
Dies bedeutet, dass
\begin{align*}
|G| =|U| \cdot |G  	/ U | = |U| \cdot |U  	\backslash  G |
\end{align*}
gilt. Mit $G  	/  U$ bezeichnen wir die Menge der Linksnebenklassen und mit
$U  	\backslash  G$ die der Rechtsnebenklassen.
Für eine beliebige Gruppe $G$ stehen alle Nebenklassen von $U$, unabhängig ob rechts- oder links, in Bijektion zu $U$. 
\end{genericthm}

\begin{proof}
Zunächst wählen wir $g \in G$ beliebig aber fest und zeigen, dass 
\begin{align*}
\varphi : U \to gU, \ u \mapsto gu
\end{align*}
bijektiv ist.
\begin{itemize}
\item $\varphi$ ist surjektiv: Sei $x \in gU$, dann existiert ein $u_1 \in U$ mit $x = g \cdot u_1$.
Damit gilt $\varphi(u_1) = x$ und $\varphi$ ist surjektiv.
\item $\varphi$ ist injektiv: Aus $\varphi(u_1) = \varphi(u_2)$ folgt $g \cdot u_1 = g \cdot u_2 $.
Weiter erhalten wir mit 
\begin{align*}
u_1 = g^{-1} \cdot (g \cdot u_1) = g^{-1} \cdot (g \cdot u_2) = u_2
\end{align*}
die Injektivität von $\varphi$.
\end{itemize}
Nun ist $G$ eine disjunkte Vereinigung seiner Linksnebenklassen.
Da $G$ endlich ist, gibt es auch nur endlich viele Repräsentanten, wir setzen hierfür $m := | G / U |$ und 
betrachten die Repräsentanten $\lbrace g_1,...,g_m \rbrace$.
Wir wissen also 
\begin{align*}
G = g_1U \cup ... \cup g_mU
\end{align*}
und mit der Bijektivität von $\varphi$ gilt 
\begin{align*}
|G| = |g_1U |+ ... + |g_mU| = m \cdot |U|.
\end{align*}
Damit haben wir die Aussage für die Linksnebenklassen gezeigt, dass Vorgehen für die Rechtsnebenklassen ist vollkommen analog.
\end{proof}

\begin{genericdf}{Beispiele und Anwendungen}\label{1.15} \
\begin{enumerate}
\item[\textbf{(1)}]
Sei $G$ eine Gruppe mit $|G| = p$ prim.
Mit \ref{1.14} haben alle Untergruppen die Ordnung $1$ oder $p$, wodurch
nur $G$ und $1$ Untergruppen sein können.
Außerdem ist $G$ zyklisch mit $G =<g>$ für $g \neq 1_G$.
Falls die zugehörige Verknüpfung additiv ist, schreiben wir $\Zn{p}$.
Falls diese multiplikativ ist, schreiben wir $C_p$.
\item[\textbf{(2)}]
Sei $G = S_3$. Wegen $|G| = 2 \cdot 3$ erhalten wir mit
\begin{align*}
1, <(12)>, <(13)>, <(23)>, <(123)>, S_3
\end{align*}
alle Untergruppen von $S_3$.
Also haben alle echten Untergruppen die Ordnung $1$, $2$ oder $3$.
In der \textbf{(1)} haben wir gesehen, dass diese dann zyklisch sind.
\item[\textbf{(3)}]
Existiert auch eine Umkehrung des Satzes von Lagrange? 
Diese können wir so formulieren:
Sei $G$ eine Gruppe mit $|G| = n$ und $d | n$.
Gibt es eine Untergruppe $U$ mit $|U| = d$?
Wir können diese Frage mit ja beantworten, wenn $G$ zyklisch ist.
\begin{proof}
Sei $G = <g>$ und $|G| = n$. Wegen $d | n $ existiert ein $k \in N$ mit $n = k \cdot d$.
Wir betrachten $g^k \in G$ mit $<g^k> \leq G$. Durch
\begin{align*}
(g^k)^d = g^{k\cdot g} = g^n = 1_G
\end{align*}
erhalten wir, dass $d$ von $|<g^k>|$ geteilt wird.
Sei nun $(g^k)^i = g^{k \cdot i}= 1_G$. Die Ordnung von $g$ ist $n$, wodurch $k \cdot i$ von $n$ geteilt wird.
Also ist $k\cdot i$ ein Vielfaches von $n$, womit $o(g^k) \geq d$ gilt.
Insgesamt folgt $d = | <g^k> | $.
\end{proof}
In den Übungen werden wir sehen, dass dies im Allgemeinen nicht gilt.
\end{enumerate}
\end{genericdf}

\begin{df}\label{1.16}\index{Homomorphismus!Gruppen}
Seien $(G,\cdot)$ und $(H,\ast)$ Gruppen.
Eine Abbildung $\varphi : G \to H$ mit
\begin{align*}
\varphi(g_1 \cdot g_2) = \varphi(g_1) \ast \varphi(g_2)
\end{align*}
für alle $g_1, g_2 \in G$ nennen wir \textbf{\textit{Gruppenhomomorphismus}}.
Sollte $\varphi$ bijektiv sein, sagen wir auch \textbf{\textit{Isomorphismus}} dazu.
Einen Isomorphismus der Form $\varphi: G \to G$ bezeichnen wir als \textbf{\textit{Automorphismus}}.
\end{df}

\begin{genericdf}{Beispiele} \
\begin{enumerate}
\item[\textbf{(1)}]
Sei $K$ ein Körper und $K^\ast = K \setminus \lbrace 0 \rbrace$ die zugehörige multiplikative Gruppe.
Dann ist 
\begin{align*}
\varphi: \Gl_n(K) \to K^\ast,\ A \mapsto \det(A)
\end{align*}
ein Gruppenhomomorphismus.
\item[\textbf{(2)}]Sei $G$ eine Gruppe.
Die Abbildung $\varphi : G \to 1,\ g \mapsto 1$ nennen wir trivialer Homomorphismus.
Die Identität $\varphi : G \to G,\ g \mapsto g$ ist ein Automorphismus.
\item[\textbf{(3)}]
Sei $G$ eine Gruppe und $h \in G$ fest.
Dann ist 
\begin{align*}
\gamma_h : G \to G, \ g \mapsto h^{-1}gh
\end{align*}
ein Automorphismus. Einen Automorphismus dieser Art nennen wir
\textbf{\textit{innerer Automorphismus}}. \index{innerer Automorphismus}
Die Äquivalenz
\begin{align*}
h \in Z(G) \Leftrightarrow \gamma_h = \id
\end{align*}
erhalten wir durch schnelles Nachrechnen.
Wir nennen $\gamma_h$ die \textbf{\textit{Konjugation mit $h$}}. \index{Konjugation}
\item[\textbf{(4)}]
Sei $G$ eine Gruppe. Dann ist 
\begin{align*}
\Aut(G)  := \lbrace \varphi : G \to G \ | \ \varphi \ \text{Automorphismus} \rbrace
\end{align*}
eine Gruppe bezüglich der Komposition.
Die Abgeschlossenheit bezüglich der Komposition ist erfüllt, denn es gilt 
\begin{align*}
(\sigma \circ \tau)(gh) 
= \sigma(\tau(gh)) = \sigma(\tau(g)\tau(h)) = ... = (\sigma \circ \tau)(g) (\sigma \circ \tau)(h)
\end{align*}
für $\sigma, \tau \in \Aut(G)$ und $g,h \in G$.
Das neutrale Element und die Assoziativität sind klar, da $\id_G$ und Automorphismus und die Komposition assoziativ ist.
Vermutlich ist zu \\
$\sigma \in \Aut(G)$ die Umkehrabbildung $\sigma^{-1}$ das inverse Element.
Über die Existenz und die Bijektivität müssen wir uns keine Gedanken machen.
Wir müssen also noch zeigen, dass $\sigma^{-1}$ ein Homomorphismus ist.
Dafür setzen wir $x =\sigma(g)$ und $y= \sigma(h)$ für beliebige $g,h \in G$.
Damit folgt dann mit
\begin{align*}
\sigma^{-1}(xy)= \sigma^{-1}(\sigma(g)\sigma(h)) = \sigma^{-1}(\sigma(gh)) = gh = \sigma^{-1}(y) \sigma^{-1}(y)
\end{align*}
die Homomorphismuseigenschaft.
\end{enumerate}

\end{genericdf}


\subsection{Aufgaben zu Abschnitt 1}

\begin{exe}\label{aufgabe:1.1} 
Zeigen Sie: Eine endliche Gruppe ist nicht Vereinigung zweier echter Untergruppen.
\hyperlink{loes:1.1}{Lösung}
\end{exe}

\begin{exe}\label{aufgabe:1.2} 
Zeigen Sie: Eine Gruppe mit endlich vielen Untergruppen ist endlich.
\hyperlink{loes:1.2}{Lösung}
\end{exe}

\begin{exe}\label{aufgabe:1.3} 
Sei $G$ die Gruppe der regulären oberen $3 \times 3$ Dreiecksmatritzen über $\Z / 3 \Z$.
Nun sei 
\begin{align*}
U :=  \lbrace A = (a_{ij}) \in G \ | \ a_{11} = a_{22} = a_{33} = 1 \rbrace.
\end{align*}
Zeigen sie, dass $U $ eine Untergruppe von $G$ ist.
Berechnen sie außerdem die Anzahl der Elemente in $U$ und deren mögliche Ordnungen.
\hyperlink{loes:1.3}{Lösung}
\end{exe}

\begin{exe}\label{aufgabe:1.4} 
Sei $G$ eine Gruppe. Welche der folgenden Aussagen sind korrekt?
\begin{enumerate}
\item[a)]
Falls für alle $g \in G \setminus \lbrace 1_G \rbrace$ gilt, dass $\ord(g) = 2$ ist, dann ist $G$ abelsch.
\item[b)]
Sei $U$ eine Untergruppe vom Index $2$ in $G$. Dann ist $U$ ein Normalteiler von $G$.
\item[c)]
Sei $U$ eine Untergruppe vom Index $3$ in $G$. Dann ist $U$ ein Normalteiler von $G$.
\item[d)]
Falls für alle $g \in G \setminus \lbrace 1_G \rbrace$ gilt, dass $\ord(g) = 3$ ist, dann ist $G$ abelsch.
\end{enumerate}
\hyperlink{loes:1.4}{Lösung}
\end{exe}

\begin{exe}\label{aufgabe:1.5} 
Sei $\varphi : G \to H$ ein Gruppenhomomorphismus.
Welche der folgenden Aussagen sind korrekt?

\begin{enumerate}
\item[a)]
$\ord(\varphi(g)) | \ord(g)$.
\item[b)]
$M \nt H \quad \Rightarrow \quad \varphi^{-1}(M) \nt G$.
\item[c)]
$N \nt G \quad \Rightarrow \quad \varphi(N) \nt H$.
\item[d)]
Sei $U \leq G$ und $\varphi(U)$ invariant unter allen Automorphismen von $H$.
Dann gilt $\varphi(U) \nt H$.
\end{enumerate}
\hyperlink{loes:1.5}{Lösung}
\end{exe}

\begin{exe}\label{aufgabe:1.6} 
Zeigen sie das Untergruppenkriterium.
\end{exe}

\newpage
\section*{TODO}
In \ref{skript:2.17} $\Stab_G(gU) = g U g^{-1}$ nachprüfen oder nachfragen.\\
In \ref{skript:2.17} Notation von $U^g$ nachfragen.
\section{Operationen von Gruppen auf Mengen}

\begin{df}\label{skript:2.1}\index{Operation} \index{G-Menge}
Sei $X $ eine nicht-leere Menge und $G$ eine Gruppe.
Wir sagen \textbf{\textit{$G$ operiert auf $X$}},
falls eine Abbildung  $G \times X \to X, \ (g,x) \mapsto g.x$
mit
\begin{enumerate}
\item[\textbf{(i)}] $\forall x \in X: \ 1_G.x=x$
\item[\textbf{(ii)}] $\forall g,h \in G \ \forall x \in X: \ g.(h.x)=(gh).x $
\end{enumerate}
existiert. In diesem Fall sagen wir, dass $G$ \textbf{\textit{von links}} auf $X$ operiert.
Ersetzen wir jedoch \textbf{(ii)} durch
\begin{enumerate}
\item[$\textbf{(ii)}^\prime$] $\forall g,h \in G \ \forall x \in X: \ g.(h.x) = (hg).x$,
\end{enumerate}
sprechen wir von einer \textbf{\textit{Rechtsoperation}}.
Falls wir $x.g$ anstatt $g.x$ schreiben, wird $\textbf{(ii)}^\prime  $ durch $(x.h).g = x.(hg)$ ersetzt.
In der Regel werden wir aber Linksoperationen betrachten.
$X$ bezeichnen wir auch als \textbf{\textit{$G$-(Links)Menge}}.
\end{df}

\begin{genericdf}{Beispiele}\label{2.2} \
\begin{enumerate}
\item[\textbf{(1)}]
Die symmetrische Gruppe $S_n$ operiert durch
\begin{align*}
\sigma.i := \sigma(i)
\end{align*}
für $i \in \lbrace 1,...,n \rbrace$ auf $\lbrace 1,...,n \rbrace$.
\item[\textbf{(2)}]
Die lineare Gruppe $\Gl_n(K)$ operiert durch Matrixmultiplikation auf $V= K^n$.
Es gilt also 
\begin{align*}
A.v := A \cdot v
\end{align*}
für $A \in \Gl_n(K)$ und $v \in V$.
\end{enumerate}

\end{genericdf}

\begin{genericthm}{Lemma}\label{2.3} \index{Stabilisator}
Sei $X$ eine $G$-Menge und $x \in X$ beliebig, aber fest.
Dann gilt
\begin{align*}
G_x := \lbrace g \in G \ | \ g.x=x \rbrace \leq G.
\end{align*}
Wir bezeichnen $G_x$ als \textbf{\textit{(Punkt)-Stabilisator von $x$}} und schreiben hierfür $\Stab_G(x)$.
\end{genericthm}

\begin{proof}
Wir arbeiten nun die Punkte der Untergruppendefinition ab.
Wegen $1_G.x= x$ folgt direkt, dass $1_G$ in $G_x$ liegt.
Nun wählen wir $g,h \in G_x$ beliebig.
Mit 
\begin{align*}
(gh).x &=g.(h.x) =g.x  = x\\
x = 1_G.x = (g^{-1}g).x &=g^{-1}.(g.x)= g^{-1}.x
\end{align*} 
folgen die geforderten Eigenschaften sofort.
\end{proof}

\begin{sz}\label{2.4}\index{Bahn} \index{G-Bahn}
Sei $X$ eine $G$-Menge und $ x \in X$ beliebig, aber fest.
Dann nennen wir 
\begin{align*}
O_x := \lbrace g.x \ | \ g \in G \rbrace
\end{align*}
die \textbf{\textit{Bahn von $x$ unter der Operation}} von $G$.
Wir sprechen auch von \textbf{\textit{$G$-Bahnen}}.
Außerdem ist die Abbildung
\begin{align*}
\mu_x: O_x \to G / G_x , \ g.x \mapsto g G_x
\end{align*}
bijektiv.
\end{sz}

\begin{proof}
Wir werden nun zeigen, dass $\mu_x$ eine wohldefinierte, bijektive Abbildung ist.
\begin{itemize}
\item $\mu_x$ ist wohldefiniert: 
Wir wählen $g,h \in G$ mit $g.x=h.x$.
Damit folgt  $x = (g^{-1}h).x$ und somit auch $g^{-1}h \in G_x$.
Also existiert ein $u \in G_x$ mit $u = g^{-1}h $, was äquivalent zu $h = gu$ ist.
Mit 
\begin{align*}
h G_x = gu G_x = g G_x
\end{align*}
erhalten wir die Wohldefiniertheit.
\item $\mu_x$ ist offensichtlich surjektiv. Warum eigentlich?
Die einzelnen Klassen von $G / G_x$ enthalten die Elemente $g,h \in G$ mit $g.x = h.x$.
Durch 
\begin{align*}
g \sim h &\Leftrightarrow
\exists u \in G_x : \ gu = h \Leftrightarrow \exists u \in G_x: \ u = g^{-1}h 
\Leftrightarrow g^{-1}h \in G_x 
\Leftrightarrow (g^{-1}h).x = x \\
 &\Leftrightarrow g.x = h.x
\end{align*}
erhalten wir die Aussage.
\item
$\mu_x$ ist injektiv:
Durch 
\begin{align*}
g G_x = h G_x 
&\Rightarrow g^{-1}h G_x = G_x 
\Rightarrow g^{-1}h \in G_x
\Rightarrow (g^{-1}h).x = x \\
&\Rightarrow g.x = g.(g^{-1}h).x = h.x
\end{align*}
erhalten wir die Injektivität.
\end{itemize}
\end{proof}

\begin{genericthm}{Bahnensatz}\label{2.5} \index{Bahnensatz}
Sei $X$ eine $G$-Menge.
\begin{enumerate}
\item[\textbf{(1)}]
Dann ist $X$ eine disjunkte Vereinigung von Bahnen.
\item[\textbf{(2)}]
Sei $O$ eine Bahn und $x,y \in O$.
Dann existiert ein $h \in G$ mit
\begin{align*}
h^{-1} G_y h = G_x.
\end{align*}
Damit sind die Punktstabilisatoren verschiedener Elemente einer Bahn zueinander konjugiert.
\item[\textbf{(3)}]
Sei $|G| < \infty$. Dann gilt
\begin{align*}
|O_x| = \frac{|G|}{|G_x|}
\end{align*}
für alle $x \in X$.
Die Länge einer Bahn ist also der Index des Punktstabilisators eines Elementes der Bahn.
\end{enumerate}
\end{genericthm}

\begin{proof}\
\begin{enumerate}
\item[\textbf{(1)}]
Wir definieren die Relation
\begin{align*}
x \sim y \Leftrightarrow \exists g \in G : \ g.x=y 
\end{align*}
auf $X$. Mit 
\begin{align*}
1_G.x &= x  \Rightarrow x \sim x\\
x \sim y  &\Rightarrow \exists g \in G : \ g.x= y
\Rightarrow g^{-1}.(g.x) = g^{-1} y
\Rightarrow ... \Rightarrow x = g^{-1}.y 
\Leftrightarrow y \sim x\\
x \sim y &\wedge y \sim z 
\Rightarrow h.(g.x) = h.y = z  
\Rightarrow x \sim z
\end{align*}
für $x,y,z \in X$ erhalten wir die Eigenschaften einer Äquivalenzrelation.
Die Bahnen sind die Äquivalenzklassen dieser Relation und bilden somit eine disjunkte Vereinigung von $X$.
\item[\textbf{(2)}]
Wir wählen $x,y \in O$. Damit existiert ein $k \in G$ mit $k.x = y$.
Sei nun $g \in G_x$, dann gilt
\begin{align*}
(kgk^{-1}).y= (kgk^{-1})k.x =  (kgk^{-1}k).x = (kg).x = k.(g.x) = k.x = y
\end{align*}
und es folgt $kgk^{-1} \in G_y$. Nun haben wir $g$ beliebig gewählt, womit dann auch 
\begin{align*}
k G_x k^{-1} \subseteq G_y 
\end{align*}
gilt.
Nun müssen wir noch die Teilmengenbeziehung in die andere Richtung zeigen.
Sei nun $\tilde{g} \in G_y$. Mit $k^{-1}.y =x$ sehen wir durch analoges Vorgehen, dass
$k^{-1} \tilde{g} k \in G_x$ gilt. Insbesondere folgt
\begin{align*}
k (k^{-1} \tilde{g} k ) k^{-1}  \in G_y
\Rightarrow \tilde{g} \in k G_x k^{-1}
\Rightarrow G_y  \subseteq k G_x k^{-1},
\end{align*}
womit wir die andere Richtung erledigt haben.
Insgesamt haben wir 
\begin{align*}
G_y = k G_x k^{-1} \Leftrightarrow k^{-1} G_y k = G_x
\end{align*}
als Resultat.
\item[\textbf{(3)}]
Sei $G_x = \lbrace g \in G \ | \ g.x= x \rbrace$ ein Punktstabilisator.
Netterweise ist die Abbildung aus \ref{2.4} bijektiv, woraus wir direkt
\begin{align*}
|O_x| = | G / G_x| = \frac{|G|}{|G_x|} 
\end{align*}
erhalten.
\end{enumerate}
\end{proof}

\begin{genericdf}{Beispiele}\label{2.6} \
\begin{enumerate}
	\item[\textbf{(1)}]\index{Operation!transitiv} \index{G-Menge!transitiv}
	Sei $G = S_n$ und $X= \lbrace 1,...,n \rbrace$.
	Dann gibt es genau eine Bahn, denn wir finden zu jedem $i \in \lbrace 1,...,n \rbrace$ ein
	$\sigma_i \in S_n$ mit $\sigma_i(1) = i$.
	Eine Operation mit nur einer Bahn heißt \textbf{\textit{transitiv}} und X heißt dann auch 
	\textbf{\textit{transitive G-Menge}}.
	Sei nun $m \in \lbrace 1,...,n \rbrace$. Dann gilt
	\begin{align*}
	\Stab_{S_n}(m) = \lbrace \sigma \in S_n \ | \ \sigma(m) = m \rbrace
	\end{align*}
	und für $n = m$ sehen wir sofort, dass $\Stab_{S_n}(n) \simeq S_{n-1}$ gilt.
	Wir erhalten aus \ref{2.5} \textbf{(3)}
	\begin{align*}
	|O_n| = \frac{|S_n|}{|S_{n-1}|} = n,
	\end{align*}
	weswegen es nur eine Bahn geben kann.
\item[\textbf{(2)}]
	Sei $G = \Gl_n(K)$ und $X = K^n$ mit $A.v = A \cdot v$.
	Für den Nullvektor $0$ gilt $A.0 = 0$.
	Somit bildet $0$ eine einelementige Bahn und es gilt 
	$\Stab_{\Gl_n(K)}(0) = \Gl_n(K)$.
	Zu jeden $v \neq 0 $ aus $K^n$ gibt es eine reguläre Matrix $A$ mit $A.e_1 = v$,
	wobei $e_1$ der erste Standartbasisvektor und $v$ die erste Spalte der Matrix $A$ ist.
	Die andere Spalten erhalten wir, indem wir $v$ zu einer Basis von $K^n$ ergänzen.
	Ingesamt haben wir hier also die Bahnen $\lbrace 0 \rbrace$ und $K^n \setminus \lbrace 0 \rbrace$.
\end{enumerate}
\end{genericdf}

\begin{genericdf}{Beispiel}\label{2.7}
Die lineare Gruppe $\Gl_n(K)$ operiert auf $M_n(K)$ durch Konjugation, dass heißt es gilt
\begin{align*}
T.A = T^{-1}\cdot A \cdot T
\end{align*}
für $T \in \Gl_n(K)$ und $A \in M_n(K)$.
Die Bahnen bestehen dann aus zueinander ähnlichen Matrizen. 
Das Normalformproblem können wir nun durch Finden eines möglichst einfachen Repräsentanten einer Bahn lösen.
\end{genericdf}

\begin{genericdf}{Beispiel}\label{2.8} \index{Konjugation!Klassen} \index{Klassengleichung} \index{Stabilisator!Zentralisator}
Sei $G$ eine beliebige Gruppe. Durch 
\begin{align*}
h.g = h^{-1} g h 
\end{align*}
für $g,h \in G$ operiert $G$ durch Konjugation auf sich selbst.
Die Bahnen dieser Operation nennen wir \textbf{\textit{Konjugationsklassen}} der Gruppe $G$.
Für $g \in G $ bezeichnen wir mit $g^G$ oder $C_g$ die Konjugationsklasse von $g$.
Nach \ref{2.5} \textbf{(1)} ist
\begin{align*}
G =  \bigcup\limits_{g_i \in T} g_i^G
\end{align*}
eine disjunkte Vereinigung, wobei $T$ ein Repräsentantensystem der Konjugationsklassen von $G$ ist.
Falls $G$ endlich ist setzen wir $n := |T|$. Damit erhalten wir mit
\begin{align*}
|G| = |g_1^G| + ... + |g_n^G|
\end{align*}
die sogenannte \textbf{\textit{Klassengleichung}}. Weiter betrachten wir den Stabilisator
\begin{align*}
\Stab_G(x) = \lbrace h \in G \ | \ h^{-1} x h = x \rbrace =\lbrace h \in G \ | \ x h = h x \rbrace
\end{align*}
und bezeichnen diesen als \textbf{\textit{Zentralisator}} von $x$ in $G$.
Hierfür schreiben wir auch $C_G(x)$.
\end{genericdf}

\begin{df}\label{2.9} \index{Zentrum}
Das \textbf{\textit{Zentrum}} von $G$ ist
\begin{align*}
Z(G) = \lbrace g \in G \ | \ \forall x \in G : \ gx = xg \rbrace
\end{align*}
und es gilt $Z(G) \nt G$.
Das Zentrum $Z(G)$ ist also die Menge der Elemente aus $G$, deren Konjugationsklassen die Länge 1 haben.
Außerdem gelten $Z(G) \subseteq C_G(x)$ für alle $x \in X$ und 
\begin{align*}
Z(G) = \bigcap\limits_{x \in G } C_G(x).
\end{align*}
\end{df}

\begin{sz}\label{skript:2.10} \index{p-Gruppe}
Sei $p$ eine Primzahl und $G$ eine endliche Gruppe mit $|G| = p^n$ und $n \geq 1$.
Dann gilt
\begin{align*}
Z(G) \neq 1 = \lbrace 1_G \rbrace.
\end{align*}
Gruppen dieser Form bezeichnen wir auch als \textbf{\textit{$p$-Gruppen}}.
\end{sz}

\begin{proof}
Seien $C_1,...,C_k$ die Konjugationsklassen von $G$ mit $C_i = x_i^G$.
Dann erhalten wir 
\begin{align*}
|G| = \sum\limits_{i=1}^k |C_i|
\end{align*}
aus der Klassengleichung, wobei wir aus dem Bahnensatz \ref{2.5} und dem Satz von Lagrange \ref{1.14}
\begin{align*}
|C_i| = \frac{|G|}{|C_G(x_i)|} = p^{n_i}, \quad 1 \leq n_i \leq n
\end{align*}
erhalten. Setzen wir $x_1 = 1_G$, so gilt $|C_1| = 1$ und es folgt
\begin{align*}
|G| = p^n = \sum \limits_{i=1}^k p^{n_i} = 1 + \sum \limits_{i=2}^k p^{n_i}
\end{align*}
als Resultat. Nun nehmen wir an, dass $1_G$ das einzige zentrale Element ist.
Es gilt also $Z(G) = 1$, womit $n_i \geq 1$ für $i \geq 2$ folgt. Sonst hätte das Zentrum mehr als ein Element.
Nun gilt jedoch 
\begin{align*}
|G| \equiv 0 \mod p \wedge 1 + \sum \limits_{i=2}^k p^{n_i} \equiv 1 \mod p,
\end{align*}
wodurch $1$ durch $p$ teilbar ist. Dies ist ein Widerspruch.
Insgesamt gilt also $Z(G) \neq 1$.
\end{proof}

\begin{df}\label{skript:2.11} \index{Stabilisator!Normalisator} \index{Untergruppe!konjugierte}
Sei $G$ eine beliebige Gruppe und $X = \lbrace U \subseteq G \ | \ U \ \text{Untergruppe} \rbrace$.
Durch
\begin{align*}
g.U := g^{-1} U g
\end{align*}
für $g \in G$ und $U \in X$ ist eine Operation von $G$ auf $X$ definiert.
Die Konjugation $\gamma_g$ mit $g \in G$ ist ein Automorphismus, womit $g^{-1} U g \leq U $ gilt.
Wir nennen $g^{-1} U g $ die zu $U$ \textbf{\textit{konjugierte}} Untergruppe von $g$. 
Den Stabilisator 
\begin{align*}
\Stab_G(U) = \lbrace g \in G \ | \ g^{-1} U g = U \rbrace
\end{align*}
nennen wir \textbf{\textit{Normalisator}} von $U$ in $G$ und schreiben $N_G(U)$.
Mit \ref{2.4} und \ref{2.5} gilt außerdem $N_G(U) \leq G$.
\end{df}

\begin{genericdf}{Bemerkung} \label{2.12}
Es gilt $U \leq N_G(U)$ und $U \nt G \Leftrightarrow N_G(U) = G$.
\end{genericdf}

\begin{genericdf}{Bemerkung} \label{skript:2.13}
Sei $U \leq G$. Dann gilt für den Zentralisator
\begin{align*}
C_G(U) = \lbrace g \in G \ | \ \forall u \in U : g^{-1} u g = u \rbrace \leq G
\end{align*}
und $U \leq C_G(U) \Leftrightarrow U \ \text{abelsch}$.
Wir betrachten beispielsweise $G=U=S_3$.
Dort gilt $N_G(G) = N_{S_3}(S_3) = S_3$ und $C_G(G) = C_{S_3}(S_3) = Z(S_3) = 1$.
\end{genericdf}

\begin{genericdf}{Beispiel} \label{skript:2.14}
Sei $G = S_3$. Dann ist $\lbrace <(12)>, <(13)>, <(23)> \rbrace$ die Menge aller zu $<(12)>$ konjugierten Untergruppen. Dies ist die Bahn $O_{<(12)>}$ bezüglich der Operation in \ref{skript:2.11}.
Hierauf können wir nun den Bahnensatz anwenden
\begin{align*}
3 = |O_{<(12)>}| = |S_3 / \Stab_G(<(1,2)>)| = | S_3 / N_{S_3}(<(12)>) |  = \frac{|S_3|}{|N_{S_3}(<(12)>)|}
\end{align*}
und erhalten damit $|N_{S_3}(<(12)>)| = 2$.
\end{genericdf}

\begin{genericdf}{Bemerkung} \label{skript:2.15}
Sei $U \leq G$ und $|G / U | = 2$, dann gilt $U \nt G$. Hierfür vergleiche mit Blatt 1, Aufgabe 4b).
Die Aussage gilt nicht mehr für $|G / U | = 3$. Ein Gegenbeispiel haben wir im Prinzip in \ref{skript:2.14}
angegeben.
\end{genericdf}

\begin{genericdf}{Bemerkung} \label{skript:2.16}
Es gilt $Z(G) \nt G$. Außerdem sehen wir mit 
\begin{align*}
C_G(Z(G)) = G = N_G(Z(G)),
\end{align*}
dass $Z(G)$ ein abelscher Normalteiler ist.
\end{genericdf}

\begin{genericdf}{Bemerkung} \label{skript:2.17} \index{Core}
Sei $U \leq G$ und $G / U$ die Menge der Linksnebenklassen. Durch
\begin{align*}
G \times G / U \to G / U, \ (g,hU) \mapsto (gh) U
\end{align*}
wird $G / U$ zu einer transitiven $G$-Menge. Wir sehen durch schnelles Nachrechnen, dass diese Operation
transitiv ist.
Für $U \in G / U$ gilt
\begin{align*}
\Stab_G(gU) &= g U g^{-1}\\
\bigcap \limits_{g \in G } \Stab_G(gU) &\leq G
\end{align*}
und $G / U$ ist bezüglich Konjugation abgeschlossen. Nun setzen wir $U^g := g^{-1} U g$ und erhalten durch
\begin{align*}
\bigcap \limits_{g \in G } \Stab_G(gU) = \bigcap \limits_{g \in G } U^g \nt G.
\end{align*}
den kleinsten Normalteiler der in $U$ liegt. Dieser wird auch oft \textbf{\textit{Core}} von $G$ genannt.

\end{genericdf}
\subsection{Aufgaben zum Abschnitt 2}

\begin{exe}\label{aufgabe:2.1} 
	Bestimmen Sie in der Gruppe $ \Gl_2(\R) $ den Zentralisator
	\begin{align*}
	C_{\Gl_2(\R)}(S) := \lbrace A \in \Gl_2(\R) \ | \ AS = SA \rbrace
	\quad
	\text{für}
	\quad
	S = 
	\begin{pmatrix}
	1 & 0 \\
	0 & -1
	\end{pmatrix}.
	\end{align*}
	Ist dieser Zentralisator eine Untergruppe bzw. ein Normalteiler von $ \Gl_2(\R) $?
	\hyperlink{loes:2.1}{Lösung}
\end{exe}

\begin{exe}\label{aufgabe:2.2} 
	Sei $ G $ die Symmetriegruppe eines Quadrats.
	Die Gruppe besteht also aus allen Bewegungen der euklidischen Ebene, welche ein Quadrat in sich selbst überführen.
	\begin{itemize}
		\item[a)]
		Beschreiben Sie $ G $ als Permutationsgruppe auf der Eckmenge eines Quadrats.
		\item[b)] 
		Berechnen Sie das Zentrum $ Z(G) $.
		\item[c)]
		Bestimmen Sie alle Untergruppen von Untergruppen von $ G $ und ordnen Sie diese anhand ihrer Teilmengenbeziehung. 
	\end{itemize}
	\hyperlink{loes:2.2}{Lösung}
\end{exe}

\begin{exe}\label{aufgabe:2.3} 
	Zeigen Sie, dass jede Untergruppe einer zyklischen Untergruppe zyklisch ist.
	\hyperlink{loes:2.3}{Lösung}
\end{exe}

\begin{exe}\ \label{aufgabe:2.4} 
	\begin{enumerate}
		\item[a)]
		Zeigen Sie, ist $ G $ eine Gruppe und $ h \in G $ fest, so ist $ \gamma_h(g) = h^{-1}g h $
		ein Automorphismus von $ G $.
		\item[b)] 
		Zeigen Sie: 
		Die inneren Automorphismen einer Gruppe bilden einen Normalteiler der Gruppe aller Automorphismen von $ G $.
	\end{enumerate}
	\hyperlink{loes:2.4}{Lösung}
\end{exe}

\begin{exe}\label{aufgabe:2.5} 
	Sei $ G $ eine Gruppe und seien $ H,K $ Untergruppen von $ G $.
	\begin{enumerate}
		\item[a)]
		Zeigen Sie:
		$ H \nt K \Rightarrow K \subset N_G(H) $.
		
		\item[b)]
		Zeigen Sie:
		$ K \leq N_G(H) \Rightarrow K \cdot H \leq H  \wedge H \nt K \cdot H$. 
	\end{enumerate}
	\hyperlink{loes:2.5}{Lösung}
\end{exe}

\begin{exe}\label{aufgabe:2.6} 
	$ G \leq \Gl_n(K) $ operiere durch Matrixmultiplikation auf $ \R^n $.
	Bestimmen sie alle Bahnen der Operation,
	\begin{enumerate}
		\item[a)]
		wenn $ G = \Sl_n(\R) = \lbrace M \in \Gl_n(\R) \ | \ \det(M) = 1 \rbrace $
		die spezielle lineare Gruppe ist.
		
		\item[b)]
		wenn $ G = O(n) = \lbrace M \in \Gl_n(\R) \ | \ M^\top \cdot M = M \cdot M^\top = I $ die orthogonale Gruppe ist.
		
		\item[c)]
		wenn $ G $ die Gruppe der invertierbaren Diagonalmatrizen ist.
		
		\item[d)] 
		wenn $ G = B_n(\R) $ die Gruppe der regulären oberen Dreiecksmatrizen ist.
	\end{enumerate}
	\hyperlink{loes:2.6}{Lösung}
\end{exe}

\begin{exe}\ \label{aufgabe:2.7} 
	\begin{enumerate}
		\item[a)]
		Bestimmen Sie alle endlichen Gruppen mit zwei Konjugationsklassen.
		
		\item[b)]
		Bestimmen Sie alle endlichen $ p $ Gruppen mit drei Konjugationsklassen.
	\end{enumerate}
	\hyperlink{loes:2.7}{Lösung}
\end{exe}

\begin{exe}\label{aufgabe:2.8} 
	Sei $ G $ eine $ p $- Gruppe.
	\begin{enumerate}
		\item[a)]
		Zeigen Sie, dass der Index von $ Z(G) $ in $ G $ ungleich $ p $ ist.
		
		\item[b)]
		Zeigen Sie:
		Wenn $ |G| = p^2 $ ist, dann ist $ G $ abelsch.  
	\end{enumerate}
	\hyperlink{loes:2.8}{Lösung}
\end{exe}

\begin{exe}\label{aufgabe:2.9}
	Sei $ Y = \lbrace 1,2,3,4,5\rbrace $, $ X = Y \times Y $,
	$ G = S_5 $ und
	\begin{align*}
	G \times X \to X, \ (\sigma,(a_1,a_2)) \mapsto (\sigma(a_1), \sigma(a_2) = \sigma.(a_1,a_2).
	\end{align*}
	\begin{enumerate}
		\item[a)]
		Zeigen Sie, 
		dass dies eine Gruppenoperation von $ G $ auf $ X $ definiert.
		
		\item[b)]
		Zeigen Sie,
		dass es unter dieser Operation genau zwei Bahnen gibt.
		
		\item[c)]
	    Zeigen Sie,
	    dass $ x_1,x_2 \in X $ mit 
	    $ \Stab_G(x_1) \cong S_4 $ und $ \Stab_G(x_2) \cong S_3 $
	    existieren.
	    Geben Sie außerdem den Zusammenhang zwischen der Ordnung des Stabilisators eines Elements und der Mächtigkeit der zugehörigen Bahn an. 
	\end{enumerate}
	\hyperlink{loes:2.9}{Lösung}
\end{exe}

\newpage
\section*{TODO}
Zusammenhang zwischen \ref{skript:2.17} und Bemerkung zwischen \ref{skript:3.9} und \ref{skript:3.10} klären.

\section{Normalteiler, Faktorgruppe und Homomorphismen}

\begin{df}\label{skript:3.1} \index{Faktorgruppe} \index{Quotientengruppe}
Sei $G$ eine Gruppe mit $N \nt G$.
Durch die Verknüpfung
\begin{align*}
G / N \times G / N \to G / N, \ (gN,hN) \mapsto (gh)N
\end{align*}
wird eine \textbf{\textit{Faktorgruppe von $G$ nach $N$}} auf der Nebenklassenmenge $G / N$ 
definiert.
Diese Gruppe nennen wir auch \textbf{\textit{Quotient}} oder \textbf{\textit{Quotientengruppe}}.
\end{df}

\begin{proof}
Leider wissen wir noch nicht, ob diese Verknüpfung wohldefiniert ist. Dies werden wir jetzt zeigen.
Sei $g_1N = g_2 N $ und $h_1N = h_2 N$.
Dann existieren $n_1, n_2 \in N $, sodass $g_1n_1 = g_2$ und $h_1n_2 = h_2$.
Mit der Normalteilereigenschaft erhalten wir durch
\begin{align*}
g_2 h_2 = g_1n_1h_1n_2 = g_1h_1 \underbrace{h_1^{-1}n_1h_1n_2}_{=: \tilde{n }\in N} = g_1h_1\tilde{n} 
\end{align*}
die Wohldefiniertheit. Die Assoziativität der Verknüpfung erhalten wir direkt aus der von $G$.
Das neutrale Element ist $N$ und das zu $gN$ inverse Element ist $g^{-1} N$.
\end{proof}

\begin{genericdf}{Beispiel}\label{skript:3.2} \index{Restklassenring}
Sei $G = (\Z,+)$ und $N = m\Z$. Da $G$ abelsch ist, gilt $N \nt G$. Die  Faktorgruppe
ist dann $(\Z / m \Z,+)$. Nun können wir auf $\Z / m \Z$ durch
\begin{align*}
(a + m\Z)(b + m\Z) := (a\cdot b + m\Z) 
\end{align*}
eine Multiplikation (eine zweite Verknüpfung) definieren. 
Dann ist $(\Z / m \Z,+, \cdot ) $ ein kommutativer Ring und wir bezeichnen diesen mit 
\textbf{\textit{Restklassenring modulo $m$}}.
\end{genericdf}

\begin{genericdf}{Beispiel}\label{skript:3.3}
Sei $G$ eine Gruppe mit $N \nt G$ und $\varphi : G \to Q $ eine Gruppenhomomorphismus mit $\varphi(N) = 1$.
Dann ist
\begin{align*}
\kappa : G / N \to Q , \ gN \mapsto \varphi(g)
\end{align*}
ein wohldefinierter Gruppenhomomorphismus. 
\begin{proof}\
\begin{itemize}
\item $\kappa$ ist wohldefiniert:
Sei $g_1N = g_2N$, dann existiert ein $n \in N $ mit $g_1 n = g_2$.
Damit erhalten wir durch
\begin{align*}
\varphi(g_2) = \varphi(g_1 n) = \varphi(g_1) \varphi(n) = \varphi(g_1) 1_Q = \varphi(g_1)
\end{align*}
die Wohldefiniertheit.
\item $\kappa$ ist ein Gruppenhomomorphismus:
Durch 
\begin{align*}
\kappa(g_1 N \cdot g_2 N ) = \kappa((g_1g_2) N ) = \varphi(g_1g_2) = \varphi(g_1) \cdot\varphi(g_2) = \kappa(g_1 N) \cdot \kappa(g_2 N)
\end{align*}
folgt dies sofort.
\end{itemize}
\end{proof}
\end{genericdf}

\begin{genericthm}{Grundeigenschaften von Gruppenhomomorphismen} \label{skript:3.4} 
\index{Homomorphismus!Gruppen!Eigenschaften}
Sei $\varphi : G \to H$ ein Gruppenhomomorphismus. Dann sind
\begin{enumerate}
\item[\textbf{(1)}]
$\varphi(1_G) = 1_H$
\item[\textbf{(2)}]
$\forall g \in G: \varphi(g)^{-1} = \varphi(g^{-1})$
\end{enumerate}
erfüllt.
\end{genericthm}

\begin{proof}
Den ersten Teil erhalten wir mit 
\begin{align*}
\varphi(1_G) = \varphi(1_G \cdot 1_G) = \varphi(1_G) \cdot \varphi(1_G)
\stackrel{\varphi(1_G)^{-1}}{\Rightarrow} 1_H = \varphi(1_G)
\end{align*}
sofort. Den zweiten erhalten wir durch
\begin{align*}
1_H = \varphi(1_G) = \varphi(g \cdot g^{-1}) = \varphi(g) \cdot \varphi(g^{-1}),
\end{align*}
denn $\varphi(g^{-1})$ ist invers zu $\varphi(g)$. Der Rest folgt aus der Eindeutigkeit des Inversen.
\end{proof}

\begin{sz}\label{skript:3.5}
Sei $\varphi : G \to H$ ein Gruppenhomomorphismus. Dann gelten:
\begin{enumerate}
\item[\textbf{(1)}] \index{Homomorphismus!Kern}
Wir nennen $\Ker \varphi := \lbrace g \in G \ | \ \varphi(g)= 1_H \rbrace$ den \textbf{\textit{Kern}} von $\varphi$. Für diesen gilt 
\begin{align*}
\Ker \varphi = 1 \Leftrightarrow \varphi \ \text{injektiv}
\end{align*}
und $\Ker \varphi \nt G$.
\item[\textbf{(2)}] \index{Homomorphismus!Bild}
Wir bezeichnen mit $\Bild \varphi := \lbrace h \in H \ | \ \exists g \in G : \varphi(g)=h \rbrace$
das \textbf{\textit{Bild}} von $\varphi$.
Sei $U \leq G$, dann gilt $\varphi(U) \leq H$.
Aus dieser Aussage folgt sofort $\Bild \varphi \leq H$.
\item[\textbf{(3)}] \index{volles Urbild}
Sei $V \leq H $, dann gilt $\varphi^{-1}(V) = \lbrace g \in G \ | \ \varphi(g) \in V \rbrace \leq G$.
Falls zusätzlich $V \nt H$ gilt, folgt $\varphi^{-1}(V) \nt G$.
Dies bezeichnen wir dann als \textbf{\textit{volles Urbild}}.
\end{enumerate}
\end{sz}

\begin{proof}\
\begin{itemize}
\item $\Ker \varphi$ ist Untergruppe von $G$: Wegen $\varphi(1_G) = 1_H$ gilt $1_G \in \Ker \varphi$.
Sei nun $g,h \in \Ker \varphi$.
Mit
\begin{align*}
\varphi(g^{-1}) =\varphi(g)^{-1} = 1_H^{-1} = 1_H &\Rightarrow g^{-1} \in \Ker \varphi\\
\varphi(g\cdot h) = \varphi(g) \cdot \varphi(h) = 1_H  &\Rightarrow gh \in \Ker \varphi 
\end{align*}
erhalten wir die anderen Untergruppeneigenschaften.
\item $\Ker \varphi$ ist Normalteiler von $G$: 
Sei $x \in G$ und $g \in \Ker \varphi$.
Dann gilt
\begin{align*}
\varphi(x^{-1} g x ) = \varphi(x^{-1}) \varphi(g) \varphi(x) = \varphi(x)^{-1} 1_H \varphi(x) = 1_H,
\end{align*}
womit $\Ker \varphi$ unter Konjugation invariant ist. Also folgt $\Ker \varphi \nt G$. 
\item $\Ker \varphi = 1 \Leftrightarrow \varphi \ \text{injektiv}$: 
Für die Hinrichtung nehmen wir an, dass $\varphi$ nicht injektiv ist.
Damit existieren $g,h \in G$ mit $g \neq h$ und $\varphi(g) = \varphi(h)$.
Wir erhalten durch 
\begin{align*}
\varphi(g^{-1}h) = \varphi(g)^{-1} \cdot \varphi(h) = 1_H
\end{align*}
eine Widerspruch zu $\Ker \varphi = 1$. Für die Rückrichtung nehmen wir an, dass $\Ker \varphi \neq \lbrace 1_G \rbrace$ gilt. Dann würde aber ein $g \neq 1_G$ existieren mit $\varphi(g) = 1_H$.
Dies ist ein Widerspruch zur Injektivität.
\end{itemize}
Die Aussagen aus \textbf{(2)} und \textbf{(3)} können wir durch ähnliches Vorgehen beweisen.
\end{proof}

\begin{genericdf}{Bemerkung}\label{3.6} \index{Homomorphismus!natürlich}
	Sei $ G $ eine Gruppe, $ N \nt G$ und $ G/N $ die zugehörige Faktorgruppe. Dann ist 
	\begin{align*}
	\kappa : G \to G/N, \ g \mapsto gN
	\end{align*}
	ein Gruppenhomomorphismus. Diesen nennen wir \textit{\textbf{natürlicher Homomorphismus}}.
	Für den Kern von $ \kappa $ gilt
	\begin{align*}
	\kappa(g) = N \Leftrightarrow g \in \Ker \kappa \Leftrightarrow g \in N,
	\end{align*}
	womit $ \Ker \kappa  = N $ und $ \Bild \kappa = G/ N $.
	
\end{genericdf}

\begin{genericdf}{Beispiel} \label{skript:3.7} \
	\begin{enumerate}
		\item[\textbf{(1)}]
		Sei $ K $ ein Körper. Wir betrachten die Determinante $ \det : \Gl_n(K) \to K^\ast $. Für deren Kern gilt
		\begin{align*}
		\Ker \det = \Sl_n(K))= \lbrace A \in \Gl_n(K) \ | \ \det(A) = 1 \rbrace,
		\end{align*}
		womit $ \Sl_n(K) \nt \Gl_n(K)$ folgt. Außerdem gilt $ \Bild \det = K^\ast $.
		\item[\textbf{(2)}] 
		Wir betrachten die Abbildung $ \exp : (\R,+) \to (\R^\ast, \cdot) , \ x \mapsto e^x$. Wegen
		\begin{align*}
		\exp(x+y) = e^{x+y} = e^x \cdot e^y = \exp(x) + \exp(y)
		\end{align*}
		ist $ \exp $ ein Homomorphismus. Dieser ist injektiv und es gilt $ \Bild \exp = \R^\ast_{\geq 0} $
	\end{enumerate}
\end{genericdf}

\begin{genericthm}{Lemma}\label{skript:3.8}
	Sei $ X $ eine nicht-leere $ G $ Menge. Dann ist 
	\begin{align*}
	\mu_g : X \to X, \ x \mapsto g.x
	\end{align*}
	für ein festes $ g \in G $ eine Bijektion. Die Abbildung 
	\begin{align*}
	\varphi: G \to \Sym(X), \ g \mapsto \mu_g
	\end{align*}
	ist ein Gruppenhomomorphismus mit 
	\begin{align*}
	\Ker \varphi = \lbrace g \in G \ | \ \forall x \in X: \ g.x = x \rbrace = \bigcap\limits_{x\in X} \Stab_G(x).
	\end{align*}
\end{genericthm}

\begin{proof}\
	\begin{itemize}
		\item $ \mu_g $ ist bijektiv: Sei $ g \in G $ fest. Dann können wir durch
		\begin{align*}
		\mu_{g^{-1}} : X \to X, \ x \to g^{-1}.x 
		\end{align*}
		die Umkehrabbildung explizit angeben. Damit gilt $ \mu_g \in \Sym(X) $ für alle $ g \in G $.
		\item $ \varphi $ ist ein Gruppenhomomorphismus: Aufgrund von
		\begin{align*}
		\varphi(gh) = \mu_{gh}: \ &x \mapsto (gh).x \\
		\varphi(g) \circ \varphi(h) = \mu_g \circ \mu_h : \  &x \mapsto g.(h.x)
		\end{align*}
		erhalten wir die Homomorphismuseigenschaft aus Axiom \textbf{(2)} aus \ref{skript:2.1}.
	\end{itemize}
\end{proof}

\begin{genericthm}{Satz von Cayley} \label{skript:3.9} \index{Satz!Cayley}
	Sei $ G $ eine endliche Gruppe mit $ |G| = n $. Dann gibt es einen injektiven
	Gruppenhomomorphismus $ \varphi : G \to S_n $. Damit ist $ G $ isomorph zu einer Untergruppe von $ S_n $.
\end{genericthm}

\begin{proof}
	Sei $ X = G $, dann ist $ X $ eine $ G $-Menge mit $ g.x = g \cdot x $. 
	Aus \ref{skript:3.8} erhalten wir dann
	\begin{align*}
	\varphi : G \to \Sym(X) \cong S_{|G|}
	\end{align*}
	als Gruppenhomomorphismus. Weiter ist $ g.x = x $ für alle $ x \in X $ nur für $ g = 1_G $ erfüllt.
	Damit gilt $ \Ker \varphi = 1 $ und somit ist $ \varphi $ injektiv.
\end{proof}

\begin{generic_no_num}{Bemerkung}
	Im Allgemeinen ist $ |G| = n $ aus \ref{skript:3.9} bestmöglich.
	Hat $ G $ eine Untergruppe $ U $ vom Index $ m $ mit 
	\begin{align*}
	\bigcap \limits_{g \in G} U^g = \bigcap \limits_{g \in G} g^{-1} U g ,
	\end{align*}
	so lässt sich \ref{skript:3.8} auf $ X = G/U $ mit Linksmultiplikation anwenden.
	Wir erhalten $ \varphi $ wie in \ref{skript:3.8}. 
	Es gilt $ \Ker \varphi \nt G $, $ \Ker \varphi \leq  U = \Stab_G(U) $ und in $ U $ 
	liegt kein nichttrivialer Normalteiler.
 mit \end{generic_no_num}

\begin{df} \label{skript:3.10} \index{Permutation!Matrix}
	Sei $ K $ ein Körper, $ n \geq 1 $ und $ \sigma \in S_n $.
	Die zu $ \sigma $-gehörige \textbf{\textit{Permutaionsmatrix}} $ A(\sigma) $ ist definiert durch
	\begin{align*}
	A(\sigma) := (a_{ij}) \in M_n(K)
	\end{align*}
	mit
	\begin{align*}
	a_{ij} = 
	\begin{cases}
	1 & \text{ falls } i=\sigma(j) \\
	0 & \text{ falls}
	\end{cases}.
	\end{align*}
	Außerdem gilt $ A( \sigma ) \cdot e_i = e_{\sigma(i)} $.
\end{df}

\begin{generic_no_num}{Beispiel}
	Für $ (123),(12) \in S_3 $ gilt
	\begin{align*}
	A((123)) =  
	\begin{pmatrix}
	0 & 0 & 1 \\
	1 & 0 & 0 \\
	0 & 1 & 0
	\end{pmatrix}
	\ \text{und} \ 
	A((12)) = 
	\begin{pmatrix}
	0 & 1 & 0 \\
	1 & 0 & 0 \\
	0 & 0 & 1
	\end{pmatrix}.
	\end{align*}
\end{generic_no_num}

\begin{genericthm}{Lemma} \label{skript:3.11}
	Die Abbildung $ \rho : S_n \to \Gl_n(K), \ \sigma \mapsto A(\sigma) $
	ist ein injektiver Gruppenhomomorphismus.
\end{genericthm}

\begin{proof}
	Sei $ \lbrace e_1,...,e_n \rbrace $ die kanonische Basis  von $ K^n $.
	Es gilt $ A ( \sigma )\cdot e_i = e_{\sigma(i)} $ und $ A(\sigma)  $ ist invertierbar.
	Wir wählen nun $ \sigma, \tau \in S_n$ beliebig, aber fest. Dann gilt 
	\begin{align*}
	(A(\tau) \cdot A(\sigma)) \cdot e_i = A(\tau) \cdot e_{\sigma(i)} = e_{\tau \circ \sigma (i)} = A(\tau \circ \sigma ) \cdot e_i
	\end{align*}
	für alle $ i \in \lbrace 1,...,n \rbrace $. Damit folgt $ \rho(\tau) \cdot \rho(\sigma) = \rho(\tau \circ \sigma) $, womit 
	$ \rho  $ ein Gruppenhomomorphismus ist.
	Die Injektivität folgt direkt, da $ \rho(\sigma) = \id $ nur für $ \sigma = \id  $ erfüllt ist.
\end{proof}

\begin{genericthm}{Folgerung} \label{skript:3.12}
	Sei $ G $ eine endliche Gruppe mit $ |G|= n $ und $ K $ ein Körper.
	Dann ist $ G $ isomorph zu einer Untergruppe von $ \Gl_n(K) $.
\end{genericthm}

\begin{proof}
	Die Aussage erhalten wir direkt durch Anwenden von \ref{skript:3.9} und \ref{skript:3.11}.
\end{proof}

\begin{generic_no_num}{Bemerkung}
	Wir können also das Rechnen auf endlichen Gruppen auf Matrizen übertragen.	
\end{generic_no_num}

\begin{sz} \label{skript:3.13} \index{Gruppe!alternierend} \index{Signumsfunktion}
	Sei $ K = \Q $. Die Abbildung
	\begin{align*}
	\varepsilon : S_n \to \Q^\ast , \ \sigma \mapsto \det(A(\sigma))
	\end{align*}
	ist ein Gruppenhomomorphismus. Falls $ \tau  $ eine Transposition ist gilt $ \varepsilon(\tau) = -1 $ und für das Bild
	gilt $ \Bild \varepsilon = \lbrace -1, 1 \rbrace $.
	Wir nennen $ \varepsilon $ \textbf{\textit{Signumsfunktion}} und kürzen diese mit $ \sgn $ ab.
	Der Kern $ \Ker \varepsilon $ heißt \textbf{\textit{alternierende Gruppe}} vom Grad $ n $. Für diese führen wir die Abkürzung
	$ A_n $ ein.
	Außerdem gilt $ A_n \nt S_n $ und $ \nicefrac{|S_n|}{|A_n|} = 2$.
\end{sz}

\begin{proof}
	Die Abbildung $ \varepsilon = \det \circ \rho $ ist eine Komposition von Gruppenhomomorphismen, also auch ein Gruppenhomomorphismus. Für $ \tau = (ij) $ entsteht $ A(\tau) $ durch Vertauschen der $ i $-ten und $ j $-ten Spalte.
	Damit folgt $ \det(A(\tau)) = -1 $. Insbesondere lässt sich jede Permutation als Produkt von Transpositionen schreiben, womit $ \varepsilon(\sigma) ) = \pm 1 $ für alle $ \sigma \in S_n  $ folgt. 
	Darüber hinaus erhalten wir die Äquivalenz
	\begin{align*}
	\sigma \in A_n \Leftrightarrow \sigma \ \text{ist gerades Produkt von Transpositionen}
	\end{align*}
	und mit \ref{skript:3.5} \textbf{(1)} gilt $ A_n = \Ker \varepsilon \nt S_n $.
	Weiter gilt
	\begin{align*}
	\sigma \in A_n &\Rightarrow (12)\sigma \notin A_n\\
	\sigma \notin A_n &\Rightarrow (12) \sigma \in A_n,
	\end{align*}
	woraus $ S_n = A_n \stackrel{.}{\cup} (12)A_n $ resultiert. Es gibt also zwei disjunkte Nebenklassen, damit ist der Index von $ A_n $ gleich $ 2 $.
\end{proof}

\begin{genericdf}{Bemerkung} \label{skript:3.14}
	Sei $ \sigma $ ein $ k $-Zykel. Dann gilt $ \sigma \in A_n \Leftrightarrow k \ \text{ungerade} $.
\end{genericdf}
\subsection{Aufgaben zum Abschnitt 3}

\begin{exe}\label{aufgabe:3.1} 
	\begin{enumerate}\
		\item[a)]
		Sei $ G $ eine Gruppe und sei $ h : G \to H $ ein Gruppenhomomorphismus.
		Zeigen Sie, dass $ h $ ein Isomorphismus ist genau dann, wenn
		es einen Gruppenhomomorphismus $ h^\prime: H \to G $ gibt,
		sodass $ h \circ h^\prime  = \id_H $ und $ h^\prime \circ h = \id_G $ ist.
		
		\item[b)] 
		Sei $ G $ eine Gruppe,
		$ H \nt G $ und $ K \nt H $.
		Gilt dann auch $ K \nt G $?.
	\end{enumerate}
	\hyperlink{loes:3.1}{Lösung}
\end{exe}

\begin{exe}\label{aufgabe:3.2}
	Seien $ G ,H$ Gruppen und $ S $ eine Teilmenge von $ G $, die $ G $ erzeugt.
	\begin{enumerate}
		\item[a)]
		Seien $ \phi,\psi $ Gruppenhomomorphismen von $ G $ nach $ H $.
		Zeigen Sie:
		\begin{align*}
		\forall s \in S: \ \phi(s) = \psi(s)
		\Rightarrow
		\phi = \psi
		\end{align*}  
		
		\item[b)]
		Wie viele Gruppenhomomorphismen gibt es von $ S_3 $ nach $ C_2 $? 
	\end{enumerate}
	\hyperlink{loes:3.2}{Lösung}
\end{exe}

\begin{exe}\label{aufgabe:3.3}
	Sei $ N $ eine normale Untergruppe von $ G $
	und $ L $ eine normale Untergruppe von $ G/N $.
	\begin{enumerate}
		\item[a)]
		Zeigen Sie,
		dass es eine normale Untergruppe $ K $ von $ G $ gibt
		mit $ N \subseteq K $, sodass
		$ L = K/N = \lbrace kN \ | \ k \in K \rbrace  $ gilt.
		
		\item[b)] 
		Zeigen Sie,
		dass die Abbildung
		\begin{align*}
		\Phi : G/N \to G/K, \ gN \mapsto gK
		\end{align*}
		für $ g \in G $ ein wohldefinierter Gruppenhomomorphismus ist
		und bestimmen sie den Kern von $ \Phi $.
	\end{enumerate}
	\hyperlink{loes:3.3}{Lösung}
\end{exe}

\newpage
\section{Auflösbare und einfache Gruppen}

\begin{df}\label{skript:4.1} \index{Kommutator} \index{Erzeugnis!Kommutator}
	Sei $ G $ eine Gruppe. Dann nennen wir 
	\begin{align*}
	[g,h] := g^{-1}h^{-1}gh \in G
	\end{align*}
	den \textbf{\textit{Kommutator}} von $ g $ und $ h $. Mit 
	\begin{align*}
	G^\prime := < [g,h] \ | \ g,h \in G >
	\end{align*}
	bezeichnen wir das \textbf{\textit{Erzeugnis aller Kommutatoren}}, dieses ist per Definition eine Untergruppe von $ G $.
	
\end{df}

\begin{generic_no_num}{Bemerkung} \index{Kommutator!Gruppe} \index{Gruppe!abgeleitet}
	Wegen
	\begin{align*}
	G^\prime = 1 &\Leftrightarrow \forall g,h \in G : \ [g,h]= 1 
	\Leftrightarrow \forall g,h \in G : \ g^{-1}h^{-1}gh = 1_G
	\Leftrightarrow \forall g,h \in G : \ gh = hg \\
	&\Leftrightarrow G \text{ ist abelsch}
	\end{align*}
	können wir die Größe von $ G^\prime $ als eine Art \glqq Maß\grqq \ betrachten, wie sehr nicht abelsch $ G $ ist.
	Wir nennen $ G^\prime $ die \textbf{\textit{Kommutatorgruppe}} von $ G $ oder auch \bi{abgeleitete Gruppe}.
\end{generic_no_num}

\begin{sz}\label{skript:4.2} \
	\begin{enumerate}
		\item[\textbf{(1)}] $ G^\prime \nt G $ und $ G / G^\prime $ ist abelsch.
		\item[\textbf{(2)}] Ist $ N \nt G $ und $ G / N $ abelsch, so folgt $ G^\prime  \subseteq N $.
		\item[\textbf{(3)}] Ist $ \varphi : G \to H $ ein surjektiver Gruppenhomomorphismus, dann gilt $ \varphi(G^\prime)  = H^\prime$.
	\end{enumerate}
\end{sz}

\begin{proof} \
	\begin{enumerate}
		\item[\textbf{(3)}]
		Sei $ g,k \in G $ beliebig. Dann folgt mit 
		\begin{align*}
		\varphi( g^{-1} k^{-1} gk) = \varphi(g)^{-1} \cdot \varphi(k)^{-1} \cdot \varphi(g) \cdot \varphi(k) \in H^\prime,
		\end{align*}
		dass $ \varphi(G^\prime) \subseteq H^\prime $.
		Seien nun $ h_1,h_2 \in H $ beliebig. Da $ \varphi $ surjektiv ist, existieren $ g_1, g_2 \in G $, sodass 
		$ \varphi(g_1) = h_1 $ und $ \varphi(g_2) = h_2 $ gilt. Damit gilt
		\begin{align*}
		\varphi([g_1,g_2]) = \varphi(g_1)^{-1} \cdot \varphi(g_2)^{-1} \cdot \varphi(g_1) \cdot \varphi(g_2) = [h_1,h_2],
		\end{align*}
		weshalb jeder Kommutator von $ H $ im Bild von $ G^\prime $ liegt. Somit gilt auch $ H^\prime \subseteq \varphi(G^\prime) $.
		\item[\textbf{(1)}]
		Sei $ g \in G $ beliebig. Der innere Automorphismus
		\begin{align*}
		\gamma_g : G \to G, \ x \mapsto g^{-1}x g
		\end{align*}
		ist surjektiv. Mit \textbf{(3)} gilt $ G^\prime = \gamma_g(G^\prime) = g^{-1} G g $ für alle $ g \in G $.
		Somit ist $ G^\prime $ invariant unter allen inneren Automorphismen und wir erhalten $ G^\prime \nt G $.
		Nun ist $ \pi : G \to G / G^\prime $ auch ein surjektiver Homomorphismus.
		Zunächst gilt 
		\begin{align*}
		\pi([g,h]) = [g,h] G^\prime = G^\prime = 1_{G/G^\prime}
		\end{align*}
		für alle $ g,h \in G $. Damit folgt $ \pi( G^\prime) = (G / G^\prime )^\prime = 1 $, womit aus unserer Bemerkung folgt, dass $ G /G^\prime  $ abelsch ist.
		\item[\textbf{(2)}] 
		Wir betrachten die Faktorabbildung $ \pi : G \to G/N $. Diese ist wiederum surjektiv, womit
		\begin{align*}
		\pi ( G^\prime) = (G/N)^\prime = 1
		\end{align*}
		folgt, da $ G/N $ abelsch ist. Damit folgt $ G^\prime \subseteq \Ker \pi = N $ und wir sind fertig.
	\end{enumerate}
\end{proof}

\begin{df} \label{skript:4.3} \index{Gruppe!auflösbar} \index{Kommutator!Gruppe!i-te}
	Wir definieren durch
	\begin{align*}
	G^{(0)} := G, \ G^{(1)} := G^\prime, \ ... \,G^{(i)} := (G^{(i-1)})^\prime
	\end{align*}
	rekursiv Untergruppen von $ G $. Es gilt also $ G^{(i)}  \leq G$ für alle $ i \in \N $.
	Wir nennen $ G^{i} $ die \bi{$ i $-te Kommutatorgruppe} und $ G $ heißt \bi{auflösbar}, falls ein $ d \in N $ existiert mit $ G^{(d)} = 1 $.
\end{df}

\begin{generic_no_num}{Bemerkungen} \index{Gruppe!perfekt} \
	\begin{enumerate}
		\item[\textbf{(1)}]
		Falls $ G $ abelsch ist, so gilt $ G^{(1)} = G^\prime = 1 $. Also ist $ G $ auflösbar.
		\item[\textbf{(2)}] 
		Wir nennen $ G $ \bi{perfekt}, falls $ G = G^\prime $ gilt.
	\end{enumerate}
\end{generic_no_num}

\begin{lemma} \label{skript:4.4}
	Sei $ G $ auflösbar. Dann ist jede Untergruppe von $ G $ auflösbar.
	Außerdem ist jede Faktorgruppe auflösbar.
	%Möglicherweise das mit dem Bild ergänzen
\end{lemma}

\begin{proof}
	Sei $ U \leq G $, dann ist jeder Kommutator in $ U $ auch ein Kommutator in $ G $. Also gilt $ U^\prime \leq G^\prime $ und durch Iteration erhalten wir $ U^{(i)} \leq G^{(i)} $.
	Da $ G $ auflösbar ist, existiert ein $ d \in \N $ mit $ G^{(d)} =1 $. Also folgt $ U^{(d)} = 1 $, womit $ U $ auflösbar ist.
	Den zweiten Teil erhalten wir durch analoges Vorgehen, indem wir \ref{skript:4.2} \textbf{(3)} anwenden. Durch Iteration sehen wir $ G^{(d)} = 1 $, woraus $ (G/N)^{(d)} =1  $ folgt.
\end{proof}

\begin{lemma} \label{skript:4.5}
	Eine Gruppe $ G $ ist \textbf{genau dann} auflösbar, \textbf{wenn} eine Folge von Untergruppen der Form
	\begin{align*}
	1 = U_0 \leq U_1 \leq ... \leq U_{r-1} \leq U_r = G, \ r \in \N
	\end{align*}
	existiert, wobei $ U_{i-1} \nt U_i$ und $ U_i / U_{i-1} $ ist abelsch für alle $ i \in N $ gelten muss.
\end{lemma}

\begin{proof}
	Für die Hinrichtung sei $ G $ auflösbar. Wir setzen
	\begin{align*}
	1 = U_0 = G^{(d)},...,U_{d-1}= G^\prime = G^{(1)},U_d = G
	\end{align*}
	und mit \ref{skript:4.2} folgt iterativ $ U_{i-1} \nt U_{i} $ und $ U_{i} / U_{i-1} $ ist abelsch.
	Für die Rückrichtung setzen wir die Folge
	\begin{align*}
	1 = U_0 \leq U_1\leq ... \leq U_r = G
	\end{align*}
	mit den geforderten Eigenschaften voraus. Es gilt $ U_{r-1} \nt G $ und $ G/ U_{r-1} $ ist abelsch. Also folgt mit \ref{skript:4.2} $ G^\prime \subseteq U_{r-1} $. Analog folgt mit $ U_{r-2} \nt U_{r-1} $ und $ U_{r-1} /U_{r-2} $ abelsch, dass $ U_{r-1}^\prime \subseteq U_{r-2} $ gilt. Nun gilt auch 
	\begin{align*}
	G^{(2)} = (G^\prime)^\prime \subseteq U_{r-1}^\prime \subseteq U_{r-2}.
	\end{align*}
	Setzen wir dieses Verfahren nun fort, erhalten wir $ G^{i} \subseteq U_{r-i} $ für $ 1 \leq i \leq r $. 
	Zu Schluss gilt dann auch $ G^{r} \subseteq U_0  = 1 $, womit $ G $ auflösbar ist.
\end{proof}

\begin{genericdf}{Beispiele} \label{skript:4.6} \ \index{kleinsche Vierergruppe}
	\begin{enumerate}
		\item[\textbf{(1)}]
		Sei $ G=S_3 $, dann gilt $ <(123)> = A_3 \nt S_3$ und $ S_3  / A_3$ ist abelsch.
	    Aus \ref{skript:4.2} \textbf{(2)} folgt dann $ S_3^\prime  \subseteq A_3 $.
	    Nun nehmen wir an, dass $ S_3^\prime $ eine echte Teilmenge von $ A_3 $ ist.
	    Dann müsste aber $ S_3^\prime = 1 $ gelten, womit $ S_3 $ abelsch wäre. Dies ist ein Widerspruch.
	    Nur wie kommen wir auf diesen Widerspruch? Mit unserer Annahme gilt $ S_3^\prime \subset A_3 $.
	    Da $ S_3^\prime \leq S_3 $ folgt $ S_3^\prime < A_3 $. Nach dem Satz von Lagrange muss nun $ |S_3^\prime|  $ die Ordnung von $ A_3 $ teilen, es bleibt also nur noch die $ 1 $ als Möglichkeit übrig.
	    \item[\textbf{(2)}] 
	    Sei $ G= A_4 $, dann ist
		\begin{align*}
		U = \lbrace \id, (12)(34), (13)(24), (14)(23) \rbrace \nt A_4,
		\end{align*}
		den alle anderen $ 8 $ Elemente der $ A_4 $ sind $ 3 $-Zykel.
		Weiter gilt $ |A_4/U| = 3 $, womit $ A_4/U \cong C_3 $ gilt. Da $ A_4/U $ damit abelsch ist folgt mit \ref{skript:4.2} \textbf{(2)}, dass $ A_4^\prime \subseteq  U$ gilt.
		Es kann nicht $ A_4^\prime = 1 $ gelten, ansonsten wäre $ A_4 $ abelsch. 
		Nun wissen wir, dass $ A_4^\prime \nt A_4 $ gelten muss. Dafür untersuchen wir nun die Untergruppen von $ U $.
		Wegen $ (123)(12)(34)(132) \notin A_4 $ ist $ <(12)(34)>  $ kein Normalteiler von $ A_4 $. Wir gehen analog für die anderen Untergruppen der Ordnung $ 2 $ vor und erhalten damit $ A_4^\prime = U $.
		Übrigens ist $ U $ isomorph zur \bi{kleinschen-Vierergruppe}. 
		\item[\textbf{(3)}] 
		Wir sehen in \textbf{(1)} und \textbf{(2)}, dass $ S_3^\prime $ und $ A_4^\prime $ abelsch sind, denn es gilt
		$ |S_3^\prime |= 3 $ und $ |A_4^\prime | = 2^2 $. Außerdem gilt in beiden Fällen $ S_3^{(2)} = 1 $ und $ A_4^{(2)} =1  $, womit die Kommutatorreihen die Länge $ 2 $ haben. Also sind die beiden Gruppen auflösbar.
  	\end{enumerate}
\end{genericdf}

\begin{df} \label{skript:4.7} \index{Gruppe!einfach} 
	Wir nennen eine Gruppe \bi{einfach}, wenn $ G \neq 1 $ und $ 1 $ und $ G $ die einzigen Normalteiler von $ G $ sind.
\end{df}

\begin{genericdf}{Bemerkung und Definition} \label{skript:4.8} \index{Kompositions!-reihe} \index{Kompositions!-faktoren} \index{Normalteiler!maximal} \index{Satz!Jordan-Hölder}
	Sei $ \kappa : G \to H \neq 1 $ ein surjektiver Gruppenhomomorphismus.
	Wir nehmen an, dass $ H $ nicht einfach ist. Dann existiert $ N \nt H $ mit $ 1 \neq N \neq H $ und es folgt
	$ \kappa(N)^{-1} \nt G $. Aufgrund der Surjektivität gilt außerdem $ \Ker \kappa \subsetneq \kappa(N)^{-1} \subsetneq G $. 
	Ein Normalteiler $ M $ heißt \bi{maximal}, falls für $ K \nt G $ und $ M \subseteq K \subseteq G $ gilt,
	dass $ M=N   $ oder $ N=G $ und $ M \neq G $ ist.
	Sei nun $ M $ ein maximaler Normalteiler von $ G $, wobei $ G \neq 1 $ ist. Dann ist $ G/M $ einfach.
	Falls $ G $ eine endliche Gruppe ist, können wir diesen Prozess wiederholen. Wir erhalten
	\begin{align*}
	M_0 = G \trianglerighteq M_1 \trianglerighteq M_2 \trianglerighteq ...\trianglerighteq M_k = 1, \ k \in \N ,
	\end{align*} 
	wobei $ M_i $ ein maximaler Normalteiler von $ M_{i-1} $ für alle $ i \in \N $ ist.
	Nach obiger Bemerkung gilt dann $ M_{i-1} / M_i $ ist einfach. Eine Folge wie in oben angegeben nennen wir 
	\bi{Kompositionsreihe} von $ G $ und die Gruppen $ M_{i-1} / M_i $ \bi{Kompositionsfaktoren}
	(Satz von \bi{Jordan-Hölder}).
\end{genericdf}

\begin{genericdf}{Lemma} \label{skript:4.9}
	Sei $ G $ einfach und auflösbar. Dann ist $ G $ isomorph zu einer zyklischen Gruppe von Primzahlordnung ($ G \cong C_p $ ).
\end{genericdf}

\begin{proof}
	Da $ G $ auflösbar ist folgt, dass $ G \neq G^\prime $ und $ G^\prime \nt G $ ist.
	Nun ist $ G $ einfach, es gilt also $ G^\prime = 1 $. Damit ist $ G $ abelsch.
	Wir wählen  $ g\in G $ mit $ g \neq 1 $. 
	Sei nun $ p  $ eine Primzahl mit $ p | \ord(g) $, dass heißt $ \ord(g) = p \cdot a , \ a \geq 1$.
	Nun hat $ g^a $ die Ordnung $ p $, womit folgt $ U := <g^a> \leq G $. Da aber auch $ U \nt G $ gilt, folgt $ G=U $.
\end{proof}

\begin{sz}\label{skript:4.10} \
	\begin{enumerate}
		\item[\textbf{(1)}]
		Für $ n \geq 5 $ ist $ A_n $ nicht auflösbar.
		\item[\textbf{(2)}]
		$ A_5 $ ist einfach.
	\end{enumerate}
\end{sz}

\begin{proof}\
	\begin{enumerate}
		\item[(1)] 
		Sei $ n \geq 5 $ beliebig. Wir werden nun eine andere Variante als in der Vorlesung fahren und 
		verwenden, dass sich $ \sigma \in A_n$ für $ n \geq 5 $ als Produkt von Dreierzykeln darstellen lässt.
		Es gilt bereits $ A_5^\prime \subseteq A_5 $. Es bleibt also die andere Teilmengenbeziehung zu zeigen.
		Sei nun $ i,j,k,l,m \in \lbrace 1,...,n \rbrace $ beliebig aber paarweise verschieden. 
		Dann erhalten wir mit 
		\begin{align*}
		(ijk) = (mjk)(ljk)(ikm)(ijl),
		\end{align*}
		dass jeder Dreierzykel ein Produkt aus Dreierzykeln ist. Damit ist jedes $ \sigma \in A_n $ ein Produkt aus Kommutatoren. Also gilt $ A_n \subseteq A_n^\prime $.		
		\item[(2)] 
		Diese Aussage werden wir im nächsten Kapitel beweisen.
	\end{enumerate}
\end{proof}

\begin{generic_no_num}{Bemerkung}
	Es gilt sogar $ A_n$ ist einfach für $ n \geq 5 $.
\end{generic_no_num}

\begin{genericthm}{Verschärfung} \label{skript:4.11}
	Sei $ G $ eine Gruppe und $ N \nt G $. Dann gilt
	\begin{align*}
	G \ \text{auflösbar} \Leftrightarrow N \text{ und } G/N \text{ auflösbar}.
	\end{align*}
\end{genericthm}

\begin{proof}
	Die Hinrichtung ist gerade \ref{skript:4.4}. Wir müssen also nur noch die Rückrichtung beweisen.
	Da $ G/N $ auflösbar ist, existiert ein $ k \in \N $ mit $ (G/N)^{(k)} = 1 $. Mit der Faktorabbildung
	$ \kappa : G \to G/N $ folgt dann
	\begin{align*}
	\kappa(G^{(k)}) = G/N^{(k)} = 1,
	\end{align*}
	womit $ G^{k} $ im Kern von $ \kappa  $ liegt. Wir erinnern uns, dass $ \ker \kappa = N $ gilt.
	Da nun $ N $ auflösbar ist, existiert ein $ l \in \N $ mit $ N^{(l)} = 1$.
	Mit der selben Argumentation wie in \ref{skript:4.4} folgt die Auflösbarkeit von $ G $.
\end{proof}
\subsection{Aufgaben zu Abschnitt 4}

\begin{exe}\label{aufgabe:4.1} 
	Sei $ G $ ein Gruppe.
	Bestimmen Sie alle höheren Kommutatorgruppen, wenn
	\begin{enumerate}
		\item[a)]
		$ G = D_8  $, die Symmetriegruppe eines Quadrats ist.
		
		\item[b)]
		$ G = Q_8 $ die Quaternionengruppe ist 
	\end{enumerate}
	Sind diese Gruppen auflösbar?
	\hyperlink{loes:4.1}{Lösung}
\end{exe}



\begin{exe}
	Sei $ n \geq 3 $. Zeigen Sie:
	\begin{enumerate}
		\item[a)]
		Alle $ \sigma \in S_n $ sind ein Produkt von Transpositionen
		\item[b)] 
		Alle $ \sigma \in A_n $ sind Produkt von 3-Zykeln. 
	\end{enumerate}
	%\hyperlink{loes:4.2}{Lösung}
\end{exe}



\newpage
\section*{TODO}
Beweis zu \ref{skript:5.7} fehlt.
Beweis zu \ref{skript:5.9} fehlt.
\section{Homomorphiesatz mit Anwendungen auf endlichen (abelschen) p-Gruppen}

\begin{genericthm}{Homomorphiesatz}\label{skript:5.1} \index{Satz!Homomorphie}

Sei $ \varphi : G \to H$ ein surjektiver Gruppenhomomorphismus und \\
$N= \Ker \varphi$.
Dann gibt es genau einen Gruppenhomomorphismus $\overline{\varphi}: G/N \rightarrow H$ mit
\begin{align*}
\varphi = \overline{ \varphi  } \circ \pi,
\end{align*}
wobei $  \pi : G \to G/N$ die Faktorabbildung ist. 
Durch $\overline{\varphi }$ ist ein Isomorphismus gegeben und wir schreiben $H \cong G/N$. 
%Wir können die gegebene Situation durch
%%\begin{figure}[H]
%%	\centering
%%	\begin{tikzcd}
%%	G \arrow{r}{\varphi} \arrow{rd}{\pi}  &   H \\
%%		{ }			&	G/N \arrow{u}{\overline{ \varphi }}
%%	\end{tikzcd}
%%\end{figure}
%ein kommutatives Diagramm veranschaulichen.
\end{genericthm}

\begin{proof}
	Nach \ref{skript:3.3} ist
	\begin{align*}
	\overline{\varphi} : G/N \to H, \ gN \to \varphi(g)
	\end{align*}
	ein wohldefinierter Gruppenhomomorphismus.
	Nun gilt
	\begin{align*}
	\overline{\varphi}(\pi(g)) = \overline{\varphi}(gN) = \varphi(g)
	\end{align*}
	für jedes $ g \in G $. Also gilt $ \overline{\varphi} \circ \pi = \varphi $.
	Für diese Identität kann $ \overline{\varphi} $  nicht anders definiert sein, woraus die Eindeutigkeit folgt. 
	Da $ \varphi $ surjektiv ist, muss dies auch für $ \overline{\varphi} $ gelten.
	Es gilt $ N = \Ker \varphi $ und das neutrale Element von $ G/N $ ist $ N $.
	Mit  
	\begin{align*}
	\overline{\varphi}(gN) = 1_H = \varphi(g)
	\end{align*}
	muss $ g \in N $ sein. Also ist $ \overline{\varphi} $ injektiv.
\end{proof}

\begin{genericdf}{Beispiel}\label{skript:5.2}
	Wir betrachten die Signumsabbildung
	\begin{align*}
	\varepsilon : S_n \to \Q^\ast , \ \sigma \mapsto \det(A(\sigma))
	\end{align*}
	mit $ \Bild \varepsilon  = \lbrace \pm 1 \rbrace$. Mit der Multiplikation haben wir hier $ C_2 $.
	Nun ist
	\begin{align*}
	\tilde{\varepsilon} : S_n \to \lbrace \pm 1 \rbrace , \ \sigma \to \varepsilon(\sigma)
	\end{align*}
	surjektiv und mit \ref{skript:5.1} folgt
	\begin{align*}
	S_n / \Ker \tilde{\varepsilon} = S_n / A_n \cong C_2.
	\end{align*}
\end{genericdf}

\begin{genericdf}{Beispiel}\label{skript:5.3}
	Wir betrachten eine zyklische Gruppe $ G = \langle g \rangle $. Die Abbildung
	\begin{align*}
	\varphi : \Z \to \langle g \rangle, \ k \mapsto g^k
	\end{align*}
	ist ein surjektiver Gruppenhomomorphismus mit $ \Ker \varphi = m\Z  $ für ein $ m \geq 0 $.
	Damit gilt \\
	$ G \cong \Z / m \Z $.
\end{genericdf}

\begin{genericthm}{Korrespondezsatz} \label{skript:5.4} \index{Satz!Korrespondez} \index{Normalteiler!Korrespondenz}
	Sei $ \varphi : G \to H $ ein surjektiver Gruppenhomomorphismus.
	Seien $ U \leq G $, $ V \leq H $ und $ \Ker \varphi \leq U $.
	Dann gelten:
	\begin{enumerate}
		\item[\textbf{(1)}]
		$ V = \varphi(U) \Leftrightarrow U = \varphi^{-1}(V) $
		\item[\textbf{(2)}] 
		Wenn $ V = \varphi(U) $ ist, so folgt
		\begin{align*}
		U \nt G \Leftrightarrow V \nt H.
		\end{align*}
		Dies nennen wir dann auch die \bi{Korrespondenz von Normalteilern}.
		\item[\textbf{(3)}] 
		Wenn $ V = \varphi(U) $ ist, existiert eine Bijektion $ \sigma: G/U  \to H/V$.
		Falls $ U \nt G $ gilt, gilt sogar $ G/U \cong H/V $.
	\end{enumerate}
\end{genericthm}

\begin{proof} \
	\begin{enumerate}
		\item[\textbf{(1)}]
		Wir setzen $ U =  \varphi^{-1}(V)$ voraus.
		Da $ \varphi $ surjektiv ist, gilt $ \varphi(\varphi^{-1}(V)) = V $. 
		Mit $ U = \varphi^{-1}(V) $ erhalten wir $ \varphi(U) = V $.
		Sei nun umgekehrt $ \varphi(U)  = V $, dann gilt $ U \leq \varphi^{-1}(V) $.
		Wir wählen ein beliebiges $ g \in \varphi^{-1}(V) $, woraus $ \varphi(g) \in V $ folgt.
		Wegen $ \varphi(U )  = V$ existiert dann ein $ u \in U $ mit $ \varphi(g) = \varphi(u) $.
		Dann ist $ gu^{-1} \in \Ker \varphi \leq U$, denn es gilt $ \varphi(gu^{-1}) = 1_H $.
		Damit gilt dann auch $ g \in U $, womit $ \varphi^{-1}(V) \subseteq U $ gilt.
		Insgesamt erhalten wir $ \varphi^{-1}(V) = U $.
		
		\item[\textbf{(2)}] 
		Dies erhalten wir sofort durch Anwendung von \ref{skript:3.5}.
		Volle Urbilder von Normalteilern sind Normalteiler.
		Bilder von Normalteileren unter surjektiven Homomorphismen sind Normalteiler.
		
		\item[\textbf{(3)}] 
		Zunächst definieren wir die Abbildung
		\begin{align*}
		\sigma : G/ U \to H/V, \ gU \mapsto \varphi(g) V
		\end{align*}
		und zeigen deren Wohldefiniertheit.
		Sei $ g U = \tilde{g } U $. Dann existiert ein $ \tilde{u} \in U $ mit $ g = \tilde{g} \tilde{u} $.
		Wegen $ \varphi(U) = V $ gilt $ \varphi(\tilde{u}) \in V $ und es folgt $ g U = \tilde{g} U $.
		Aufgrund der Surjektivität von $ \varphi $ müssen wir uns bei $ \sigma $ keine Gedanken machen.
		Wir kommen nun zur Injektivität.
		Sei $ \varphi(g)V = \varphi(\tilde{g}) V$. Wegen $ \varphi(U) = V $ existiert ein $ u \in U $, sodass 
		$ \varphi(g) \cdot \varphi(u) = \varphi(\tilde{g})$ gilt.
		Damit existiert dann auch ein $ k \in \Ker \varphi $ mit $ g u = \tilde{g} k $.
		Dies ist äquivalent zu $ g u k^{-1} = \tilde{g} $ und durch $ u, k \in U $ folgt $ g U = \tilde{g} U$.
		Gilt nun $ U \nt G $, dann folgt aus \textbf{(2)} $ V \nt H $.
		Wir sehen durch schnelles Nachrechnen, dass $ \sigma  $ dann auch ein Homomorphismus ist. 
	\end{enumerate}
\end{proof}

\begin{sz} \label{skript:5.5}
	Sei $ G $ eine Gruppe und $ p $ eine Primzahl mit $ |G| = p^n $ für ein $ n \in \N $. 
	Dann existiert
	\begin{enumerate}
		\item[\textbf{(1)}]
		ein Normalteiler $ N $ von $ G $ mit $ G / N  \cong C_p$.
		
		\item[\textbf{(2)}] 
		eine Folge von Untergruppen
		\begin{align*}
		1 = U_0 < U_1 < ... < U_n = G
		\end{align*}
		mit $ U_{i-1 }  \nt U_i$ und $ U_i / U_{i-1} \cong C_p$ für $ 1 \leq i \leq n $.
	\end{enumerate}
	Insbesondere sind $ p $-Gruppen auflösbar.
\end{sz}

\begin{proof}\
	\begin{enumerate}
		\item[\textbf{(1)}]
		Wir führen eine Induktion nach $ n $ durch.
		Für $ n = 1 $ ist nichts zu zeigen.
		Nach \ref{skript:2.10} gilt $ Z(G) \neq 1 $.
		Sei nun $ g \in Z(G) \setminus 1 $, dann gilt $ o(g) = p^k $ für ein $ k \in \lbrace 1,...,n \rbrace $. Damit existiert ein $ m \in \N $ mit $ o(g^m) = p  $.
		Wir setzen $ C := \langle g^m  \rangle$. Mit $ C \leq Z(G)  $ folgt $ C \nt G $.
		Wir betrachten nun $ G/ C $ und $ \kappa : G \to G/C:=Q $. Mit Lagrange gilt $ |Q| = p^{n-1} $.
		Nach Induktion existiert nun ein $ N \nt Q $ mit $ |Q/N| = p $.
		Damit folgt mit \ref{skript:5.4}  $ \kappa^{-1}(N) \nt G $ und $ G/\kappa^{-1}(N) \cong Q/N \cong C_p $. 
		
		\item[\textbf{(2)}] 
		Wir erhalten die Aussage direkt aus \textbf{(1)} durch Iteration. Hierfür sei
		$ U_n = G $. Nach \textbf{(1)} existiert ein Normalteiler $ N $ von $ G $ mit 
		$ G / N \cong C_p $. Wir setzen $ U_{n-1} = N $ und durch Wiederholung erhalten wir die Folge
		\begin{align*}
		G = U_n \unrhd U_{n-1} \unrhd ...\unrhd U_0 = 1.
		\end{align*}
		Die Auflösbarkeit folgt dann sofort mit \ref{skript:4.5} oder mit \ref{skript:4.11}.
	\end{enumerate}
\end{proof}

\begin{genericdf}{Beispiel} \label{skript:5.6}
	Sei $ G $ eine zyklische Gruppe mit $ |G| = p^m $.
	Dann gibt es zu jedem $ i \in \lbrace 0,...,m \rbrace $ genau eine Untergruppe mit $ |U_i| = p^i $, d.h. wir erhalten
	\begin{align*}
	1=U_0 \lhd U_1 \lhd ...\lhd U_m = G
	\end{align*}
	als Folge. Es gibt also genau eine Folge wie in \ref{skript:5.5}.
\end{genericdf}

\begin{generic_no_num}{Bemerkung}
	Die Folge aus \ref{skript:5.5} \textbf{(2)} ist eine Kompositionsreihe.
\end{generic_no_num}

\begin{lemma}\label{skript:5.7}
	Sei $ G $ eine endliche, nicht zyklische und abelsche $ p $-Gruppe.
	Außerdem sei $ a $ ein Element größter Ordnung.
	Dann existiert $ U \leq G $, sodass $ G \cong \langle a \rangle \times U $ gilt. 
\end{lemma}

\begin{proof}
	Muss noch ergänzt werden.
\end{proof}

\begin{sz} \label{skript:5.8}
	\
	\begin{enumerate}
		\item[\textbf{(1)}]
		Sei $ p $ eine Primzahl und $ G $ eine endliche abelsche $ p $-Gruppe.
		Dann gilt
		\begin{align*}
		G \cong C_{p^{\alpha_1}} \times \ ... \ C_{p^{\alpha_k}},
		\end{align*}
		wobei $ C_{p^i} $ die zyklische Gruppe der Ordnung $ p^i $ ist und
		\begin{align*}
		p^{\alpha_1 } \geq \ ... \ \geq p^{\alpha_k}
		\end{align*}
		gilt.
		
		\item[\textbf{(2)}] 
		Sei $ G $ eine endliche abelsche Gruppe.
		Dann gilt
		\begin{align*}
		G \cong C_{n_1} \times \ ... \ \times C_ {n_k},
		\end{align*}
		wobei $ n_{i+1}  \mid n_i$ gilt. 
		Dies nennen wir \bi{Elementarteilerzerlegung} und die $ n_i $ bezeichnen wir als \bi{Elementarteiler}.
		\index{Elementarteiler} \index{Elementarteiler!Zerlegung}
	\end{enumerate}
\end{sz}

\begin{proof}
	Die \textbf{(1)} folgt sofort aus \ref{skript:5.7} durch Induktion nach Gruppenordnung.
	Für die \textbf{(2)} benötigen wir \ref{skript:6.9} und ein
	\begin{genericthm}{Hilfslemma} \label{skript:5.9}
		Seien $ m,n \in \N $ mit $ \ggT(m,n) = 1 $.
		Dann folgt
		\begin{align*}
		C_m \times C_n \cong C_{mn}.
		\end{align*}
	\end{genericthm}
	Nun können wir den Beweis fortsetzen.
	Sei 
	\begin{align*}
	|G| = p_1^{\beta_1}\cdots p_k^{\beta_k}
	\end{align*}
	die Primfaktorzerlegung von $ |G| $. Wenn $ P_i $ die $ p_i $- Sylowgruppe für $ 1 \leq i \leq k $ ist, dann gilt nach \ref{skript:6.9}
	\begin{align*}
	G \cong P_1 \times \cdots \times P_k.
	\end{align*}
	Nach \textbf{(1)} erhalten wir nun 
	\begin{align*}
	P_i \cong C_{p_i^{\alpha_{i1}}} \times \cdots \times C_{p_i^{\alpha_{i l_i}}}
	\end{align*}
	und damit folgt dann
	\begin{align*}
	G \cong C_{p_1^{\alpha_{11}}} \times \cdots \times C_{p_1^{\alpha_{1 l_1}}}
	\times \cdots \times C_{p_k^{\alpha_{k1}}} \times \cdots \times C_{p_k^{\alpha_{k l_k}}}.
	\end{align*}
	Wir setzen nun 
	\begin{align*}
	n_1 := p_1^{\alpha_1} \cdot p_2^{\alpha_2} \cdots p_k^{\alpha_k}
	\end{align*}
	und es gilt mit dem Hilfslemma
	\begin{align*}
	C_{n_1} \cong C_{p_1^{\alpha_{11}}} \times \cdots \times C_{p_k^{\alpha^{k1}}}.
	\end{align*}
	Nach Konstruktion ist $ n_1 $ die größtmögliche Ordnung eines Elements in $ G $ und es gilt
	\begin{align*}
	G \cong C_{n_1} \times \ \text{Rest} .
	\end{align*}
	Setzen wir dieses Verfahren iterativ fort, so erhalten wir die gewünschte Aussage.
\end{proof}

\begin{genericdf}{Beispiele} \label{skript:5.10}
	Sei $ G $ eine abelsche Gruppe.
	\begin{enumerate}
		\item[\textbf{(1)}]
		Falls $ |G| = p^3 $, so gibt es 
		\begin{align*}
		G &\cong C_p \times C_p \times C_p \\
		G &\cong C_{p^2} \times C_p \\
		G &\cong C_{p^3}
		\end{align*}
		als Möglichkeiten. Somit gibt es drei verschiedene Isomorphietypen in diesem Fall.
		
		\item[\textbf{(2)}] 
		Wir betrachten
		\begin{align*}
		G = C_9 \times C_6 \times C_{10}.
		\end{align*}
		Dies ist keine Elementarteilerzerlegung und auch keine Sylowzerlegung. Zunächst folgt mit \ref{skript:5.9}, dass
		\begin{align*}
		G \cong C_9 \times C_3 \times C_2 \times C_5 \times C_2
		\end{align*}
		gilt. Durch umsortieren erhalten wir
		\begin{align*}
		G \cong C_2 \times C_2 \times C_9 \times C_3 \times C_5
		\cong P_2 \times P_3 \times P_5
		\end{align*}
		die Sylowzerlegung. Gehen wir nun so vor wie im Beweis zu \ref{skript:5.8} \textbf{(2)} erhalten wir mit
		\begin{align*}
		G \cong C_{90} \times C_6
		\end{align*}
		die Elementarteilerzerlegung.
	\end{enumerate}
 \end{genericdf}




\subsection{Aufgaben zu Abschnitt 5}

\begin{exe}\label{aufgabe:5.1}
	Sie $ G $ eine abelsche Gruppe mit $ 8 $ Elementen von Ordnung $ 3 $,
	$ 18 $ Elementen von Ordnung $ 9 $ und keine anderen Elemente außer der Identität.
	Beschreiben Sie $ G $ als direktes Produkt von zyklischen Gruppen.
	\hyperlink{loes:5.1}{Lösung}
\end{exe}

\begin{exe}\label{aufgabe:5.2}
	Sei $ p $ eine Primzahl und $ G = ( \Z / p^m \Z,+) $.
	Für $ 0 \leq i \leq m $ sei $ U_i = \langle p^i + p^m \Z \rangle $.
	Zeigen Sie, dass die $ U_i $ die einzigen Untergruppen von $ G $ sind 
	und dass $ G $ genau eine Kompositionsreihe besitzt.
	\hyperlink{loes:5.2}{Lösung}
\end{exe}

\begin{exe}\label{aufgabe:5.3}\ 
	Sei $ G $ eine Gruppe der Ordnung $ 72 $ und sei $ N $ ein Normalteiler von $ G $
	der Ordnung $ 8 $.
	Zeigen Sie, dass $ G $ eine Untergruppe der Ordnung $ 24 $ besitzt.
	Ist diese normal?
	\hyperlink{loes:5.3}{Lösung}
\end{exe}

\begin{exe}\label{aufgabe:5.4}
	Geben Sie bis auf Isomorphie alle abelschen Gruppen der Ordnung $ 200 $ an.
	\hyperlink{loes:5.3}{Lösung}
\end{exe}


\newpage
\section*{TODO}
In \ref{skript:6.6} \textbf{(1)} müsste das doch ein entweder oder sein?
\section{Sylowuntergruppen}
In diesem Abschnitt betrachten wir $ G $ durchweg als endliche Gruppe.

\begin{df}\label{skript:6.1} \index{Sylowuntergruppe} 
	Sei $ p $ eine Primzahl und $ a \geq 0 $. 
	Weiter sei $ G $ eine Gruppe mit $ |G| = p^a \cdot m $ und $ \ggT(p,m) = 1 $.
	Dann nennen wir $ P \leq G $ \bi{$ p $-Sylowuntergruppe} von $ G $, falls $ |P| = p^a $ gilt.
	Mit $ \Syl_p(G) $ bezeichnen wir die \bi{Menge aller $ p $- Sylowuntergruppen } von $ G $.
\end{df}

\begin{genericdf}{Beispiele}\label{skript:6.2} \ 
	\begin{enumerate}
		\item[\textbf{(1)}]
		Sei $ G = S_4 $. Dann gilt $ |G| = 24 = 2^3 \cdot 3 $.
		Damit ist $ C_3 \cong \langle (123) \rangle \in \Syl_3(G)$.
		Die Symmetriegruppe des Quadrats hat Ordnung $ 8 $, 
		womit $ \langle(1234),(13) \rangle \in \Syl_2(G)$ gilt.
		
		\item[\textbf{(2)}]
		Sei $ G  = \Gl_n(\Z/ 3\Z)$. Wir haben in der Vortragsübung gezeigt, dass
		\begin{align*}
		|G| = p^{\frac{n(n-1)}{2}} \cdot (p^{n-1} -1 ) \cdot (p^{n-1} -1 ) \quad \cdots  \quad p = p \cdot m
		\end{align*}
		für das entsprechende $ m $ gilt. Des weiteren finden wir den Beweis auch unter Seite 7 Beispiel 2.6 in dem Buch von Herrn Geck.
		Sei nun $ U_n(\Z / p \Z) $ die Menge der oberen Dreiecksmatrizen mit $ 1 $ auf den Diagonaleinträgen.
		Dann gilt
		\begin{align*}
		| U_n(\Z / p \Z) |  = p^{n-1} \cdot p^{n-2} \quad \cdots  \quad p = p^{\sum_{i=1}^n i} = p^{\frac{n(n-1)}{2}}, 
		\end{align*}
		womit $ U_n(\Z / p \Z)  \in \Syl_p(\Gl_n(\Z / p\Z))$ folgt.
	\end{enumerate}
\end{genericdf}

\begin{genericthm}{Lemma} \label{skript:6.3}
	Sei $U \leq G$. Dann gilt
	\begin{align*}
	\Syl_p(G)  \neq  \emptyset \quad \Rightarrow \quad \Syl_p(U) \neq \emptyset.
	\end{align*}
	Außerdem existiert für $P \in \Syl_p(G)$ ein $g \in G$ mit $g P g^{-1} \cap U \in \Syl_p(U)$.
\end{genericthm}

\begin{proof}
	Sei $P \in \Syl_P(G)$. Dann operiert $U$ auf $G/P$ durch Linksmultiplikation und es gilt
	\begin{align*}
	| G/P| = |O_1| + ... + |O_k|, \quad k \in \N
	\end{align*}
	nach dem Bahnensatz.
	Da $P \in \Syl_p(G)$ ist, kann $|G/P|$ nach Lagrange nicht durch $p$ teilbar.
	Demnach muss mindestens eine Bahn $O_i$ mit $p \nmid | O_i |$ existieren. 
	Damit gilt nach dem Bahnensatz
	\begin{align*}
	|O_i| = \frac{|U|}{|\Stab_{U}(gP)|}
	\end{align*}
	für $gP \in O_i$, wobei $g \in G$ ist. 
	Somit erhalten wir $\Stab_U(gP) \in \Syl_p(U)$
	Sei nun $h \in \Stab_U(gP)$ beliebig.
	Mit 
	\begin{align*}
	hgP = g P
	\Leftrightarrow g^{-1}hg P = P
	\Leftrightarrow g^{-1}hg \in P
	\Leftrightarrow h \in g P g^{-1}
	\end{align*}
	folgt $\Stab_U(gP) = U \cap g P g^{-1}$ und wir sind fertig.
\end{proof}

\begin{genericthm}{Sylowsätze} \label{skript:6.4} \index{Satz!Sylow}
	Sei $p$ eine Primzahl und $| G | = p^a \cdot m$ mit $a \geq 0$ und $p \nmid m$.
	\begin{enumerate}
		\item[\textbf{(1)}]
		Es gilt $\Syl_p(G) \neq \emptyset$.
		
		\item[\textbf{(2)}]
		Sei $U \leq G$ mit $|U| = p^b$ für $1 \leq b \leq a$.
		Dann existiert ein $S \in \Syl_p(G)$ mit $U \leq S$.
		Außerdem sind verschiedene Sylowuntergruppen aus $\Syl_p(G)$ zueinander konjugiert.
		
		\item[\textbf{(3)}]
		Sei $n_p(G) := | \Syl_p(G)|$.
		Dann gilt $n_p(G) \equiv 1 \mod p$ und $n_p(G) \mid m$.
	\end{enumerate}
\end{genericthm}

\begin{proof}\
	\begin{enumerate}
		\item[\textbf{(1)}]
		Sei $K = \Z / p \Z$. Mit \ref{skript:3.12} exisitiert ein $n \geq 1$ und injektiver Gruppenhomomorphismus 
		$\varphi : G \to \Gl_n(K)$. Damit erhalten wir $G \cong \varphi(G)$ und $\varphi(G) \leq \Gl_n(K)$.
		Nach \ref{skript:6.2} \textbf{(2)} ist $\Syl_p(\Gl_n(K)) \neq \emptyset$. 
		Mit \ref{skript:6.3} folgt dann $\Syl_p(G) \neq \emptyset$.		
		
		\item[\textbf{(2)}]
		Sei $P \in \Syl_p(G)$.
		Wir wenden \ref{skript:6.3} auf $U$ an.
		Es existiert also ein $g \in G$ mit \\ 
		$g P g^{-1} \cap U \in \Syl_p(U)$.
		Nun gilt jedoch $\Syl_p(U) = \lbrace U \rbrace$, womit $g P g^{-1} \cap U = U $ folgt.
		Es folgt weiter $P \leq Q:= g P g^{-1} \in \Syl_p(G)$.
		Wenn $U \in \Syl_p(G) $ ist gilt $U = Q$, womit zwei $p$-Sylowgruppen zueinander konjugiert sind. 
		
		\item[\textbf{(3)}] 
		In \textbf{(2)} haben wir gesehen, dass $ G $ auf $ \Syl_p(G) $ durch Konjugation operiert.
		Diese Operation ist transitiv.
		Sei nun $ S \in \Syl_p(G) $ beliebig aber fest. Dann ist nach \ref{skript:2.11} $ N_G(S)  $ der Stabilisator dieser Operation.
		Nun erhalten wir 
		\begin{align*}
		|\Syl_p(G)| = \frac{|G|}{|N_G(S)|}
		\end{align*}
		mit dem Bahnensatz.
		Weiter folgt mit $ S \leq N_G(S) $ und Lagrange, dass $ m := |G/S|$ von $ | \Syl_p(G) |$ geteilt wird.
		Wir schränken die Operation nun auf $ S $ ein, also $ S $ operiert durch Konjugation auf $ \Syl_p(G) $. Dies ist nun nicht mehr transitiv.
		Also gilt mit dem Bahnensatz 
		\begin{align*}
		\Syl_p(G) = O_1 \stackrel{.}{\cup} \  ...  \ \stackrel{.}{\cup} O_k
		\end{align*}
		und $ |O_i| = |S / N_S(S_i) | $ für $ 1 \leq i \leq k $.
		Für $ S_1 = S $ folgt $ |O_1| = 1 $.
		Wir betrachten nun ein beliebiges $ i \geq 2 $.
		Angenommen es gilt auch $ |O_i| = p^0 = 1 $.
		Damit gilt $ g^{-1} S_i g = S_i $ für alle $ g \in S $, womit $ S \subseteq N_G(S_i) $ gilt.
		Also folgt $ S,S_i \in \Syl_p(N_G(S_i)) $ und mit \textbf{(2)} erhalten wir
		\begin{align*}
		P = h P_i h^{-1} = P_i
		\end{align*}
		für ein $ h\in N_G(S_i) $.
		Dies ist ein Widerspruch, da $ P $ bereits in $ O_1 $ ist.
		Also muss $ |O_i| = p^{k_i} $ mit $ k_i \geq 1 $ gelten und es folgt
		\begin{align*}
		|\Syl_p(G)| = |O_1| + \sum \limits_{i=2}^k |O_i| \equiv 1 \mod p.
		\end{align*}
	\end{enumerate}
\end{proof}

\begin{genericthm}{Folgerung} \label{skript:6.5}
	Sei $ S \in \Syl_p(G) $, dann gilt
	\begin{align*}
	S \nt G \Leftrightarrow \Syl_p(G) = \lbrace S \rbrace.
	\end{align*}
\end{genericthm}

\begin{proof}
	Diese Aussage erhalten wir direkt aus \ref{skript:6.4} \textbf{(2)}.
\end{proof}

\begin{genericthm}{Satz von Cauchy} \label{skript:6.6} \index{Satz!Cauchy}
	Sei $ p  $ eine Primzahl und $ G $ eine Gruppe.
	Falls $ p $ ein Teiler von $ |G| $ ist, so existieren Elemente der Ordnung $ p $ in $ G $.
\end{genericthm}

\begin{proof}
	Sei $ S \in \Syl_p(G) $. Mit \ref{skript:6.4} wissen wir, dass dies existiert.
	Nach \ref{skript:5.5} existiert eine Folge von Untergruppen 
	\begin{align*}
	1 = U_0 < U_1 < \ ... \ < U_n = S
	\end{align*}
	mit $ U_i / U_{i-1} \cong C_p $.
	Also gilt $ |U_1| = p $, womit jedes nichttriviale Element in $ U_1 $ die Ordnung $ p $ besitzt. 
\end{proof}

\begin{genericdf}{Beispiele} \label{skript:6.7}
	\
	\begin{enumerate}
		\item[\textbf{(1)}]
		Sei $ G $ eine Gruppe mit $ |G| = 12 = 2^2 \cdot 3 $.
		Nach \ref{skript:6.4} \textbf{(3)} gilt $ n_3(G) \mid 2^2  $ und 
		$ n_3(G) \equiv 1 \mod 3 $.
		Damit erhalten wir die Möglichkeiten $ n_3(G) = 1 $ oder $ n_3(G)=4 $.
		Nun nehmen wir an, dass $ n_3(G)  = | Syl_3(G)|= 4 $ gilt.
		Jede $ 3 $-Sylowgruppe von $ G $ ist isomorph zu $ C_3 $.
		Es existieren also vier $ 3 $-Sylowgruppen deren Schnitt paarweise disjunkt ist.
		Also existieren acht Elemente der Ordnung $ 3 $. 
		Es kann also nur noch vier Elemente  von Zweierpotenzordnung geben, wobei wir das neutrale Element mit $ 2^0 $ mitzählen. 
		Damit folgt dann $ n_2(G)  =1  $.
		Insgesamt gilt nun $ n_2(G) = 1 $ oder $ n_3(G) = 1 $.
		
		\item[\text{(2)}] 
		Sei $ G $ eine Gruppe mit $ |G| = 30 = 2\cdot 3 \cdot5 $.
		Nach den Sylowsätzen muss $ n_3(G) \equiv 1 \mod 3 $
		und $ n_3(G) \mid 10 = \nicefrac{30}{3} $ gelten.
		Dadurch erhalten wir $ n_3(G) = 1 $ oder $ n_3(G) = 10 $.
		Sollte $ n_3(G) = 10 $ sein, so gibt es $20 $ Elemente der Ordnung $ 3 $
		und es bleiben noch $ 10 $ Elemente für die anderen Ordnungen übrig.
		Wenn $ n_3(G) = 1 $ ist, ist die $ 3 $-Sylowgruppe normal in $ G $.
		Nun muss $ n_5(G) \equiv 1 \mod 5$ und $ n_5(G) \mid 6 = \nicefrac{30}{5} $ gelten.
		Damit gilt $ n_5(G) = 1  $ oder $ n_5(G) =6 $.
		Sollte $ n_5(G) = 6  $ sein, muss es $ 24  $ Elemente der Ordnung 5 geben.
		Damit kann $ n_5(G) = 6 $ und $ n_3(G) =10 $ aufgrund der Kardinalität von $ G $
		nicht gleichzeitig erfüllt sein.
		Also hat $ G $ eine normale $ 3 $-Sylowgruppe oder eine normale $ 5 $-Sylowgruppe.
		Es gilt also $ |G/S| = 10 $ oder $ |G/T| = 6 $, falls $ S $ eine normale $ 5 $-Sylowgruppe 
		bzw. $ T $ eine normale $ 3 $-Sylowgruppe ist.
		Bei Gruppen dieser Ordnung können wir durch schnelles Nachrechnen die Auflösbarkeit  überprüfen.
		Da $ S $ und $ T $ aufgrund ihrer Ordnung auflösbar sind, erhalten wir mit \ref{skript:4.11}
		die Auflösbarkeit von $ G $.
	\end{enumerate}
\end{genericdf}

\begin{genericdf}{Bemerkung} \label{skript:6.8}
	Seien $ p,q $ und $ r $ paarweise verschiedene Primzahlen.
	Mithilfe der Sylowsätze können wir zeigen, dass Gruppen der Ordnung
	$ p\cdot q , p\cdot q\cdot r$ und $p^2\cdot q$ 
	auflösbar sind.
	Die $ 60 = 2^2 \cdot 3 \cdot 5 $ ist die kleinste Zahl, welche nicht von dieser Form ist.
	Somit sind alle Gruppen mit Ordnung $ < 60 $ auflösbar.
	Die alternierende Gruppe $ A_5 $ ist einfach, perfekt und besitzt die Ordnung $ 60 $.
\end{genericdf}

\begin{genericthm_no_num}{Satz von Burnside}  \index{Satz!Burnside}
	Seien $ a,b \in \N_0 $.
	Dann sind Gruppen der Ordnung $ p^a \cdot q^b $ auflösbar.
\end{genericthm_no_num}

\begin{proof}
	Dieser Beweis verwendet gewöhnliche Darstellungstheorie von Gruppen und wird wahrscheinlich in weiterführenden Algebravorlesungen geführt.
\end{proof}

\begin{generic_no_num}{Bemerkung}
	Als weitere Anwendung von den Sylowsätzen \ref{skript:6.4} erhalten wir die Klassifikation der endlichen abelschen Gruppen.
\end{generic_no_num}

\begin{genericdf}{Bemerkung}\label{skript:6.9}
	Sei $ G $ eine endliche abelsche Gruppe, dann sind alle Untergruppen Normalteiler.
	Damit sind auch alle Sylowuntergruppen normal. Es gilt $ n_p(G) = 1 $ für eine passende Primzahl $ p $. 
	Wir betrachten mit 
	\begin{align*}
	|G| = p_1^{d_1} \ \cdots p_k^{d_k}
	\end{align*}
	die bis auf Reihenfolge eindeutige Primfaktorzerlegung von $ |G| $.
	Dann hat die $ p_i $-Sylowgruppe $ P_i $ die Ordnung $ p_i^{d_i} $ für $ 1 \leq i \leq k $. 
	Nun gilt
	\begin{align*}
	G = \langle P_1 , P_2,...,P_k \rangle := P_1 \ \cdots \ P_k
	\end{align*}
	und 
	\begin{align*}
	P_1 \cap \langle P_2, ..., P_k \rangle = 1
	\end{align*}
	mit $ |\langle P_2, ..., P_k \rangle | = p_2^{d_2} \cdots P_k^{d_k} $.
	In der Vortragsübung wurde gezeigt, dass dann $ G \cong P_1 \times \langle P_2, ..., P_k \rangle  $ gilt.
	Setzen wir dies induktiv fort erhalten wir
	\begin{align*}
	G \cong P_1 \times \langle P_2, ..., P_k \rangle \cong \ ... \ \cong P_1 \times \cdots \times P_k 
	\end{align*}
	als Resultat und nennen dies \bi{Sylowzerlegung}. \index{Sylowzerlegung}
	Endliche abelsche Gruppen sind also ein endliches Produkt ihrer Sylowgruppen.
	Um die Struktur von $ G $ zu verstehen, genügt es uns die Struktur von abelschen $ p $-Gruppen zu bestimmen.
\end{genericdf}
\subsection{Aufgaben zu Abschnitt 6}

\begin{exe}\label{aufgabe:6.1}
	Zeigen Sie, dass es keine einfache Gruppe der Ordnung $ 80  $ gibt.
	\hyperlink{loes:6.1}{Lösung}
\end{exe}

\begin{exe}\label{aufgabe:6.2}
	Sei $ G $ eine Gruppe der Ordnung $ p \cdot q $, wobei $ p $ und $ q $
	zwei verschiedene Primzahlen sind.
	\begin{enumerate}
		\item[a)]
		Zeigen Sie, dass $ G $ nicht einfach ist.
		
		\item[b)]
		 Zeigen Sie, wenn
		 \begin{align*}
		 p \not\equiv 1 \mod q \quad 
		 \wedge
		 \quad q \not\equiv 1 \mod p
		 \end{align*}
		 gilt, ist $ G $ zyklisch.
	\end{enumerate}
	\hyperlink{loes:6.2}{Lösung}
\end{exe}

\begin{exe}\label{aufgabe:6.3}
	Sei $ G $ eine Gruppe der Ordnung $ p^2q $, wobei $ p $ und $ q $ verschiedene Primzahlen sind.
	Zeigen Sie, dass $ G $ entweder eine normale $ p $-Sylowuntergruppe oder eine normale $ q $
	Sylowuntergruppe besitzt.
	Ist $ G $ auflösbar?
	\hyperlink{loes:6.3}{Lösung}
\end{exe}

\begin{exe}\label{aufgabe:6.4}
	Sei $ G $ eine Gruppe der Ordnung $ 6125 = 5^3 \cdot 7^2  $.
	\begin{enumerate}
		\item[a)]
		Bestimmen Sie die Anzahl der $ 5 $-Sylowuntergruppen und der $ 7 $-Sylowuntergruppen.
		
		\item[b)] 
		Ist eine Gruppe dieser Ordnung immer auflösbar?		
	\end{enumerate}
	\hyperlink{loes:6.4}{Lösung}
\end{exe}

\begin{exe}
	Sei $ p $ eine Primzahl, $ G $ eine endliche Gruppe und $ N \nt G $ ein Normalteiler von $ G $, sodass $ p $ kein Teiler von $ [G:N] $ ist.
	\begin{enumerate}
		\item[a)]
		Zeigen Sie:
		\begin{align*}
		P \in \Syl_p(N) \wedge Q \in \Syl_p(G) 
		\Rightarrow
		 |P | = |Q| 
		\end{align*} 
		
		\item[b)]
		Zeigen Sie:
		$ \Syl_p(N) = \Syl_p(G) $
	\end{enumerate}
	\hyperlink{loes:6.5}{Lösung}
\end{exe}
\section*{TODO}
Zusammenfassung erstes Kapitel


\chapter{Ringtheorie}

\setcounter{section}{6}
\section*{TODO}
Das mit den Primzahlen in \ref{skript:7.5} ist noch zu ergänzen.
\section{Beispiele und Grundbegriffe zu Ringen}

\begin{df}\label{skript:7.1}
	Wir nennen eine nichtleere Menge $ R $ mit den Verknüpfungen
	$ + : R \times R \to R $ und $ \cdot : R \times R \to R $
	einen \bi{Ring}, falls 			\index{Ring}
	\begin{enumerate}
		\item[\textbf{(1)}]
		$ (R,+) $ eine abelsche Gruppe ist. Wir bezeichnen das neutrale Element mit $ 0 $.
		
		\item[\textbf{(2)}]
		die Multiplikation $ \cdot $ assoziativ ist die Distributivgesetze  gelten.
		Das heißt es gilt
		\begin{align*}
		a \cdot ( b+c) = a\cdot b + a \cdot c 
		\quad
		(a+b) \cdot c = a \cdot c + b \cdot c
		\end{align*}
		für alle $ a,b,c \in R $.
		
		\item[\textbf{(3)}] 
		ein Einselement $ 1 $ bezüglich der Multiplikation existiert. Es gilt also
		\begin{align*}
		1 \cdot a = a \cdot 1 = a
		\end{align*}
		für alle $ a \in R $.
		In der Literatur wird dieser Punkt meist weggelassen und separat definiert.
		In unserem Fall sprechen wir von einem \bi{Ring ohne $ 1 $}, falls dieser Punkt nicht erfüllt ist.
	\end{enumerate}
	Wir nennen $ R $ \bi{kommutativ}, falls			\index{Ring!kommutativ}
	\begin{align*}
	r \cdot s = s \cdot r
	\end{align*}
	für alle $ r,s \in R $ erfüllt ist.
	Eine Teilmenge $ S \subseteq R $ heißt \bi{Teilring} von $ R $, wenn $ S $ mit den beiden Verknüpfungen ein Ring ist und die $ 1 $ von $ S $ der von $ R $ entspricht.		\index{Teilring}
\end{df}

\begin{genericdf}{Bemerkungen} \label{skript:7.2}
	Sei $ R $ ein Ring.
 	\begin{enumerate}
		\item[\textbf{(1)}]
		In einem Ring gilt
		\begin{align*}
		a \cdot 0 =0 \cdot a = 0 
		\end{align*}
		für alle $ a \in R $. Dies erhalten wir sofort durch Anwendung der Gruppenaxiome.
		Außerdem kann in $ R $ gelten, dass $ 1 = 0  $ ist.
		Damit gilt dann jedoch
		\begin{align*}
		a = a \cdot 1 = a \cdot 0 = 0,
		\end{align*}
		woraus $ R = \lbrace 0 \rbrace  $ folgt.
		
		\item[\textbf{(2)}]
		Die Menge 
		\begin{align*}
		R^\ast := \lbrace a \in R \ | \ \exists b \in R : ab = ba = 1 \rbrace
		\end{align*}
		nennen wir die \bi{Einheiten} von $ R $. 
		Diese ist bezüglich der Multiplikation eine Gruppe.
		\index{Einheiten} 
		
		\item[\textbf{(3)}]
		Falls $ R^\ast = R \setminus \lbrace 0 \rbrace $ gilt, 
		so nennen wir $ R $ einen \bi{Schiefkörper} oder \bi{Divisionsring}.
		\index{Schiefkörper} \index{Divisionsring}
		Einen kommutativen Schiefkörper nennen wir \bi{Körper}. \index{Körper}
	\end{enumerate}
\end{genericdf}

\begin{genericdf}{Beispiele} \label{skript:7.3} \
	\begin{enumerate}
		\item[\textbf{(1)}]
		Die ganzen Zahlen $ (\Z,+, \cdot) $ sind ein kommutativer Ring und es gilt
		$ \Z^\ast = \lbrace \pm 1 \rbrace $.
		Die geraden Zahlen $ 2 \Z $ bilden nur einen kommutativen Ring ohne $ 1 $.
		
		\item[\textbf{(2)}]
		Sei $ K $ ein Körper.
		Dann ist $ M_n(K) $ der $ n \times n  $ \bi{Matrizenring} über $ K $.
		\index{Ring!Matrizen}
		Für $ n \geq 2  $ ist $ M_n(K) $ nicht kommutativ
		und es gilt $ M_n(K)^\ast = \Gl_n(K) $.
		Die Menge 
		\begin{align*}
		\mathbb{H} = \Bigg\lbrace \begin{pmatrix}
		u & v \\ \
		\overline{v} & \overline{u}
		\end{pmatrix} 
		\ | \ u, v \in \C \Bigg\rbrace
		\subsetneq
		M_2(\C)
		\end{align*}
		bildet den sogenannten \bi{Quaternionenschiefkörper} und für die
		\index{Schiefkörper!Quaternionen} 
		Quaterionengruppe gilt $ Q_8 \leq H^\ast $.
		Hierfür vergleiche mit der Vortragsübung und Übung.
	\end{enumerate}	
\end{genericdf}

\begin{df}\label{skript:7.4}
	Sei $ R $ ein kommutativer Ring mit $ 1 \neq 0 $ und $ a,b \in R $.
	Wir sagen \bi{$ a $ teilt $ b $}, falls es ein $ c \in R $ gibt
	mit $ b = a \cdot c $ und schreiben hierfür $ a \mid b $.
	\index{Teilbarkeit}
	Ein Element heißt \bi{Nullteiler},
	wenn es ein $  0 \neq c \in R $ gibt mit $ a \cdot c = 0 $.
	\index{Nullteiler}
	Wir nennen $ R $ \bi{nullteilerfrei} oder \bi{Integritätsring}, falls es außer $ 0 $ keinen anderen Nullteiler gibt. In solchen Mengen gilt
	\index{Nullteiler!frei}\index{Integritätsring}
	\begin{align*}
	a \cdot b = 0 \quad \Rightarrow a= 0 \vee b = 0.
	\end{align*}
\end{df}

\begin{genericdf}{Beispiel}\label{skript:7.5}
	Die Menge
	\begin{align*}
	\Z[i] := \lbrace a + b \cdot i \ | \ a,b \in \Z \rbrace \subseteq \C 
	\end{align*}
	nennen wir \bi{Gaußsche Zahlen}.
	\index{Gaußsche Zahlen} 
	Diese lässt sich mit 
	\begin{align*}
	\Z[\sqrt{d}] := \lbrace a + b \cdot \sqrt{d} \ | \ a,b \in \Z \rbrace
	\end{align*}
	für $ \sqrt{d} \notin \Q $ und $ d \in \Z $ allgemeiner definieren.
\end{genericdf}

\begin{genericdf}{Restklassenringe}\label{skript:7.6}\
	\begin{enumerate}
		\item[\textbf{(1)}]
		Wir betrachten $(\Z  / m \Z, + ) $ für ein festes $m \in \N$.
		Sei $a \in \Z$, dann schreiben wir für die Restklasse $a + m \Z$ bzw. $\overline{a}$.
		Mit 
		\begin{align*}
		(a + m \Z) \cdot (b + m \Z ) = (ab + m \Z) = \overline{ab} = \overline{a} \cdot \overline{b}
		\end{align*}
		wird auf $\Z m / \Z$ eine Multiplikation definiert.
		Die Wohldefiniertheit erhalten wir durch direktes Nachrechnen.
		Somit wird $\Z / m \Z$ zu einem kommutativen Ring und wir nennen diesen
		\bi{Restklassenring modulo $m$}.
		\index{Restklassenring}
		
		\item[\textbf{(2)}]
		Nun interessieren wir uns für die Einheiten in $\Z / m \Z$.
		Sei $\overline{a} \in \Z / m \Z$ eine Einheit.
		Dann existiert ein $\overline{b } \in \Z / m \Z$, sodass
		\begin{align*}
		\overline{a} \cdot \overline{b} = (a + m \Z) \cdot (b + m \Z) = \overline{1}
		\Leftrightarrow
		\exists c \in \Z : 1 = ab + mc
		\Leftrightarrow
		\ggT(a,m) = 1
		\end{align*}
		gilt. Damit erhalten wir
		\begin{align*}
		(\Z / m \Z)^\ast = \lbrace \overline{a} \in \Z / m \Z \ | 0 \neq a \in \Z : \ggT(a,m)  = 1 \rbrace
		\end{align*}
		als Einheitengruppe.
		Sei nun $p$ eine Primzahl. Gilt $m = p $ so ist $\Z / p \Z$ ein Körper.
		Wenn $m$ keine Primzahl ist, so ist $\Z / m \Z$ kein Körper, denn mit
		\begin{align*}
		m = ab 
		\Rightarrow 
		\overline{0} = \overline{m} = \overline{a} \cdot \overline{b}
		\end{align*}
		erhalten wir die Existenz von Nullteilern ungleich $\overline{0}$.
		Die Abbildung
		\begin{align*}
		\Phi : \N \to \N, \ m \mapsto | \lbrace a \in \N \ : \ 1 \leq a \leq m, \ \ggT(a,m) = 1 \rbrace | 
		\end{align*}
		nennen wir  \bi{Eulersche $\Phi$-Funktion} und es gilt
		\index{Eulersche Phi-Funktion}
		\begin{align*}
		| ( \Z / m\Z)^\ast | = \Phi(m). 
		\end{align*}
		
		\item[\textbf{(3)}]
		Es gilt
		\begin{align*}
		a^{\Phi(m)} \equiv 1 \mod m
		\end{align*}
		für $a \in \Z$ mit $\ggT(a,m)=1$.
		\begin{proof}
			Es gilt $\overline{a} \in ( \Z / m \Z )^\ast$.
			Mit Lagrange folgt dann $\ord(\overline{a}) \mid \Phi(m)$ und es gilt
			\begin{align*}
			\overline{a^{\Phi(m)}} = \overline{1}.
			\end{align*}
		\end{proof}
		
		\item[\textbf{(4)}]
		\bi{\underline{Kleiner Satz von Fermat}}:
		\index{Satz!kleiner Fermat}
		Sei $p$ eine Primzahl. Dann gilt
		\begin{align*}
		a^p \equiv a \mod p
		\end{align*}
		für alle $a \in \Z$.
		\begin{proof}
			Es gilt
			\begin{align*}
			a^{p-1} \equiv 1 \mod p 
			\Rightarrow
			a^p \equiv a \mod p
			\end{align*}
			für alle $a \in \Z$ mit $\ggT(a,p) = 1$ und für
			$\overline{a} = \overline{0}$
			\begin{align*}
			a^p \equiv 0 \mod p.
			\end{align*}
		\end{proof}
	\end{enumerate}		
		
\end{genericdf}

\begin{genericdf}{Polynomringe}\label{skript:7.7}
	Sei $R$ ein kommutativer Ring und $x$ eine Unbekannte über $R$.
	Dann bezeichnen wir mit $R[x]$ den \bi{Ring der Polynome} in einer Unbestimmten mit Koeffizienten in $R$.
	\index{Polynomring}
	Ein Element $0 \neq f \in R[x]$ lässt sich durch
	\begin{align*}
	f = a_0 + a_1 \cdot x + a_2 \cdot x^2 + \ \dots \ + a_n \cdot x^n
	\end{align*}
	mit $n \geq 0 $, $a_i \in R$ und $a_n \neq 0$ eindeutig beschreiben. 
	Für das Nullpolynom schreiben wir schlicht $0 $.
	Den \bi{Polynomgrad} $n$ kürzen wir mit $\Grad(f)$ ab und $a_n $ nennen wir den \bi{Leitkoeffizient}.
	\index{Polynom}\index{Polynom!Grad}\index{Polynom!Leitkoeffizient}\index{Polynom!normiert}
	Sollte dieser $1$ sein, so nennen wir $f$ \bi{normiert}.
	Für $f = 0$ setzen wir 
	\begin{align*}
	\Grad(f) = - \infty.
	\end{align*}	 
	Nun nehmen wir ohne Beschränkung der Allgemeinheit an, dass $m \leq n$ und betrachten
	\begin{align*}
	g = b_0 + b_1 \cdot x + \dots + b_m \cdot x^m \in R[x].
	\end{align*}
	Dann gilt
	\begin{align*}
	f+g = (a_0 + b_0) + (a_1 + b_1) \cdot x + \dots + (a_m + b_m) \cdot x^m
		= a_{m+1} \cdot x^{m+1} + \dots + a_n \cdot x^n
	\end{align*}
	und
	\begin{align*}
	f \cdot g = a_0 \cdot b_0 + (a_1 \cdot b_0 + a_0 \cdot b_1 ) \cdot x 
				+ \dots + a_n \cdot b_m \cdot x^{n+m}.
	\end{align*}
	Falls $a_n \cdot b_m \neq 0$ ist, so folgt $\Grad(f+g) = m+n$.
	Dies gilt immer wenn $R$ ein Integritätsring ist, insbesondere ist dann auch $R[x]$ ein Integritätsring.
	Wenn $R$ ein Integritätsring ist, dann sind Polynome $f$ mit $\Grad(f) \geq 1$ nicht invertierbar bezüglich der Multiplikation.
	Es gilt also $R[x]^\ast = R^\ast$.
\end{genericdf}

\begin{genericdf}{Bemerkung} \label{skirpt:7.8}
	Sei $R$ ein Integritätsring.
	Wir definieren analog zur Analysis durch
	\begin{align*}
	D: R[x] \to R[x], 
	\ \sum \limits_{i=0}^n a_i \cdot x^i \mapsto \sum \limits_{i=1}^n \underbrace{i \cdot a_i}_{a_i+ \dots +a_i} \cdot x^i
	\end{align*}		
	eine \bi{formale Ableitung}.
	\index{Formale Ableitung}
	Seien $a,b \in R$ und $f,g \in R[x]$ beliebig.
	Dann gelten:
	\begin{enumerate}
		\item[\textbf{(1)}] \bi{$R$-Linearität:}\\
		$D(a\cdot f + b\cdot g) = a \cdot D(f) + b \cdot D(g)$
		
		\item[\textbf{(2)}] \bi{Produktregel:}\\
		$D(f\cdot g) = f \cdot D(g) + D(f) \cdot g$
		
		\item[\textbf{(3)}]
		$D(f^n) = n \cdot D(f) \cdot f^{n-1}$
	\end{enumerate}
\end{genericdf}

\begin{proof}
	Der Beweis wird in der Übung geführt.
\end{proof}

\begin{generic_no_num}{Anwendung}
	Seien $0 \neq f,g \in R[x]$ mit $g$ teilt $f$.
	Wir nennen $g$ einen \bi{mehrfachen Faktor}, falls $g^2 \mid f$ gilt.
	\index{Faktor!mehrfach}
\end{generic_no_num}

\begin{genericthm_no_num}{Behauptung}
	Wenn $f = g^2 \cdot h$ ist, so gilt $D(f) = g \cdot \tilde{h}$.
	Dies bedeutet $g \mid D(f)$.
\end{genericthm_no_num}

\begin{proof}
	Es gilt
	\begin{align*}
	f = g^2 \cdot h
	\Rightarrow 
	D(f) = D(g^2 \cdot h) &= D(g^2) \cdot h + g^2 \cdot D(h)
	= 2 \cdot D(g) \cdot g\cdot h + g^2 \cdot D(h)\\
	&= g \cdot (\underbrace{2 \cdot D(g) \cdot h + g \cdot D(h)}_{\tilde{h}})
	\end{align*}
\end{proof}

\begin{generic_no_num}{Beispiel}
	Sei $m \geq 1$ und $f(x) = x^m - 1 \in \Z[x]$.
	Dann gilt $D(f) = m \cdot x^{m-1}$ und $\ggT(f,D(f)) = 1$.
	Nun haben $f$ und $D(f)$ keine gemeinsamen Faktoren außer $\pm 1$.
	Damit hat $f$ keine mehrfachen Nullstellen.
\end{generic_no_num}

\begin{df} \label{skript:7.9}
	Ein Integritätsring $R$ heißt \bi{euklidischer Ring}, wenn eine \bi{Normfunktion} bzw. \bi{Gradfunktion}
	\index{euklidischer Ring} \index{Ring!euklidisch} \index{Normfunktion} \index{Gradfunktion}
	\begin{align*}
	\nu : R \setminus \lbrace 0 \rbrace \to \N_0
	\end{align*}
	mit der folgenden Eigenschaft existiert:
	Zu $ a,b \in R $ mit $ b \neq 0 $ gibt es $ q,r \in R $ mit $ a = b \cdot q +r $,
	wobei $ r = 0 $ oder $ r \neq 0 $ und $ \nu(r) < \nu(b) $ gilt.
\end{df}

\begin{sz} \label{skript:7.10} \
	\begin{enumerate}
		\item[\textbf{(1)}]
		Der Ring $ \Z $ ist euklidisch mit $ \nu(n) = | n |  $ für $ n \in \Z $.
		
		\item[\textbf{(2)}]
		Sei $ K $ ein Körper. Dann ist der Polynomring $ K[x] $ euklidisch mit $ \nu(f) = \Grad(f) $ für
		$ f \in K[x] $.
		
		\item[\textbf{(3)}]  
		Die Gaußschen Zahlen $ \Z[i] $ sind euklidisch mit $ \nu(a+ib) = a^2 +b^2 $ für $ a,b \in \Z $.
	\end{enumerate}
\end{sz}

\begin{proof} \
	\begin{enumerate}
		\item[\textbf{(1)}]
		Folgt sofort durch den Euklidischen Algorithmus und Division mit Rest.
		
		\item[\textbf{(2)}]
		Seien $ f,g \in K[x] $ mit $ g \neq 0 $. Dann gilt
		\begin{align*}
		f(x) = \sum \limits_{i=0}^n a_i \cdot x^i, \quad g(x) = \sum \limits_{i=0}^m b_i \cdot x^i.
		\end{align*}
		Wenn $ n < m $ ist, so folgt $ f = 0 \cdot g + f $, also ist $ q =0 $ und $ r = f $.
		Sollte $ n \geq m $ sein, so setzen wir
		\begin{align*}
		\tilde{f} = f - a_n\cdot b_m^{-1} \cdot x^{n-m} \cdot g
		\end{align*}
		und erhalten für $ \tilde{f}  = 0$
		\begin{align*}
		f = a_n \cdot b_m^{-1} \cdot x^{n-m} \cdot g
 		\end{align*}
 		mit $ r = 0 $.
 		Falls $ \tilde{f}  \neq 0$ ist, so gilt
 		\begin{align*}
 		f = \underbrace{a_n \cdot b_m^{-1} \cdot x^{n-m}}_{h^\prime}\cdot g + \tilde{f} 
 		\end{align*}
 		mit $ \Grad(\tilde{f}) < \Grad (f) $.
 		Wir werden nun eine Induktion nach dem Grad von $ f $ durchführen.
 		Unsere Induktionsvoraussetzung ist
 		\begin{align*}
 		\tilde{f} = g \cdot \tilde{h} + r 
 		\end{align*}
 		mit $ r = 0  $ oder $ r \neq 0 $ und $ \Grad(r) < \Grad(g) $.
 		Dann folgt
 		\begin{align*}
 		f = h^\prime \cdot g + g \cdot \tilde{h} + r = g \cdot ( h^\prime + \tilde{h}) + r
 		\end{align*}
 		und falls $ r \neq 0 $ erhalten wir mit der Induktionsvoraussetzung sofort 
 		$ \Grad(r) < \Grad(g) $. Sollte $ r = 0 $ sein sind wir direkt fertig.
		\item[\textbf{(3)}]  
		Diesen Teil behandeln wir in der Übung.
	\end{enumerate}
\end{proof}

\begin{genericdf}{Bemerkungen} \label{skript:7.11}
	\
	\begin{enumerate}
		\item[\textbf{(1)}]
		Sei $R$ ein kommutativer Ring.
 		Die Division mit Rest im letzten Beweis funktioniert auch wenn $b_m \in R^\ast$.
 		
 		\item[\textbf{(2)}]
 		Die Polynome $q$ und $r$ aus dem letzen Beweis sind eindeutig bestimmt.
 		Sei 
 		\begin{align*}
 		f = g \cdot q + r = g \cdot g^\prime + r^\prime,
 		\end{align*}
 		und ohne Beschränkung der Allgemeinheit $\Grad(r) \leq \Grad(r^\prime)$.
 		Dann folgt 
 		\begin{align*}
 		g \cdot (q - q^\prime) = r^\prime -r 
 		\Rightarrow g \mid (r^\prime -r ).
 		\end{align*}
 		und für $r \neq r^\prime$ sein erhalten wir mit
 		\begin{align*}
 		\Grad(r^\prime- r) \leq \Grad(r) < \Grad(g)
 		\end{align*}
 		einen Widerspruch.
 		Also muss auch $q = q^\prime$ gelten.
 		
 		\item[\textit{(3)}]
	 	Sei $R$ ein euklidischer Ring.
	 	Die endliche iterative Anwendung der Division mit Rest
	 	liefert zu $a,b \in R \setminus \lbrace 0 \rbrace$ die Existenz von 
	 	$d, r , s \in R$ mit $\ggT(a,b) = d = r \cdot a + s \cdot b$. 		
 		
	\end{enumerate}
\end{genericdf}
\subsection{Aufgaben zu Abschnitt 7}

\begin{exe}\label{aufgabe:7.1}
	Zeigen Sie:
	Ein endlicher Integritätsring ist ein Körper.
	\hyperlink{loes:7.1}{Lösung}
\end{exe}

\begin{exe}\label{aufgabe:7.2}
	Zeigen Sie, dass
	\begin{align*}
	\Z[\i] := \lbrace n + m \cdot \i \ | \ n,m \in \Z \rbrace
	\end{align*}
	ein euklidischer Ring ist und bestimmen Sie alle Einheiten von $ \Z [\i] $.
	\hyperlink{loes:7.2}{Lösung}
\end{exe}


\newpage
\section*{TODO}
Im Beweis zu \ref{skript:8.5} befinden sich indirekte Übungsaufgaben -> ergänzen
\section{Ideale und Ringhomomorphismen}

\begin{df}\label{skript:8.1}
	\
	\begin{enumerate}
		\item[\textbf{(1)}]
			Sei $R$ ein Ring $I$ eine Untergruppe von $(R,+)$.
			Dann heißt $I$
			\begin{itemize}
			\item
			\bi{Linksideal} von $R$, falls $a\cdot b \in I$ für alle $a \in R$ und $b \in I$ gilt.
			\index{Linksideal}
			Wir schreiben dann $ I \unlhd_L R $.
			\item 
			\bi{Rechtsideal} von $R$, falls $b \cdot a \in I$ für alle $a \in R$ und $b \in I$ gilt.
			\index{Rechtsideal}
			Wir schreiben dann $ I \unlhd_R R $.
		
			\item
			\bi{(zweiseitiges) Ideal} von $R$,
			falls $a \cdot b \in I$ und $b \cdot a \in I$ für alle $a \in R$ und $b \in I$ gilt. 		
			\index{Ideal}
			Wir schreiben dann $ I \unlhd R $.
		\end{itemize}
		Falls $R$ kommutativ ist, fallen diese Begriffe zusammen.		
		
		\item[\textbf{(2)}]
		Ist $I$ ein Ideal von $R$, dann ist $(R/ I, + )$ eine abelsche Gruppe.
		Diese wird durch
		\begin{align*}
		(a + I) \cdot (b+ I) := a \cdot b + I
		\end{align*}
		zu einem Ring und wird auch \bi{Faktorring} genannt.
		Um eine Klasse in $R / I $ schreiben wir auch $\overline{a}$ statt $a + I$.
 		\index{Faktorring}
		Nun müssen wir noch zeigen, dass diese Verknüpfung wohldefiniert ist.
		\begin{proof}
		Sei $\overline{a} = \overline{a^\prime}$ und $\overline{b} = \overline{b^\prime}$.
		Dann gilt
		\begin{align*}
		a &= a^\prime +x \\
		b &= b^\prime + y 
		\end{align*}
		für $x,y \in I$ und es folgt
		\begin{align*}
		(a \cdot b) = (a^\prime +x ) \cdot ( b^\prime + y) 
		= a^\prime \cdot b^\prime + \underbrace{x \cdot b^\prime +a^\prime \cdot y + x \cdot y}_{ \in I},
		\end{align*}
		womit $\overline{a \cdot b} = \overline{a^\prime \cdot b^\prime}$.
		\end{proof}
		
	\end{enumerate}		
\end{df}

\begin{genericdf}{Beispiele}\label{skript:8.2}
	\
	\begin{enumerate}
		\item[\textbf{(1)}]
		Sei $R = \Z$, dann ist $I = m \Z$ ein Ideal und $R/I = \Z / m \Z$ der zugehörige Faktorring.
		
		\item[\textbf{(2)}]
		Sei $R $ ein kommutativer Ring und $a \in R$ fest gewählt.
		Dann ist
		\begin{align*}
		(a) := aR = Ra = \lbrace ba \ | \ b \in R \rbrace \unlhd R
		\end{align*}
		das von $a$ \bi{erzeugte Ideal}.
		\index{Ideal!erzeugt}
		Ein solchs Ideal nennen wir auch \bi{Hauptideal}.
		\index{Hauptideal}
		Falls jedes Ideal von $R$ ein Hauptideal ist, nennen wir $R$ einen \bi{Hauptidealring}.
		\index{Hauptideal!ring}
		
		\item[\textbf{(3)}]
		Der Ring $\Z$ ist ein Hauptidealring.
		Alle Körper $K$ sind Hauptidealringe mit den zwei Idealen $(0)$ und $(1) = K$.
		
		\item[\textbf{(4)}]
		Sei $R$ ein kommutativer Ring und $a_1, \dots , a_m \in R$.
		Dann lässt sich durch
		\begin{align*}
		(a_1, \dots, a_m) := \left\lbrace   \sum \limits_{i=1}^m f_i \cdot a_i \ | \ f_i \in R \right\rbrace 
		\end{align*}
		das Hauptideal allgemeiner definieren.
	\end{enumerate}
\end{genericdf}

\begin{sz} \label{skript:8.3}
	Sei $ R $ ein euklidischer Ring.
	Dann ist $ R $ ein Hauptidealring.
\end{sz}

\begin{proof}
	Sei $ I \unlhd R $.
	Das Ideal $ I = 0 $ ist ein Hauptideal, denn $ (0) = 0 $.
	Wir betrachten also $ I \neq 0 $ und setzen
	\begin{align*}
	d = \min \lbrace \nu(a) \ | \ 0 \neq a \in I \rbrace,
	\end{align*}
	wobei $ \nu $ die Norm von $ R $ ist.
	Das Minimum muss existiert, da Mengen aus natürlichen Zahlen immer ein kleinstes Element besitzen.
	Sei nun $ a_0 \in I $ mit $ \nu(a_0) = d$.
	Wir behaupten, dass $ I = (a_0) $ gilt.
	Um dies zu zeigen, wählen wir ein beliebiges $ b \in I $.
	Dann gilt durch Division mit Rest
	\begin{align*}
	b = q \cdot a_0 + r
	\end{align*}
	für $ q,r \in R $ mit $ r = 0 $ oder $ r \neq 0 $ mit $ \nu(r) < \nu(a_0) $.
	Angenommen $ r \neq 0 $, dann gilt
	\begin{align*}
	r = b - q\cdot a_0 \in I
	\end{align*}
	und es ist $ \nu(r) < \nu(a_0) $.
	Dies ist ein Widerspruch zur Minimalität von $ a_0 $.
	Also muss $ r = 0 $ gelten, womit $ b \in (a_0) $ folgt. 
\end{proof}


\begin{df}\label{skript:8.4}
	Seien $R$ und $S$ Ringe.
	Eine Abbildung 
	\begin{align*}
	\varphi : R \to S
	\end{align*}
	heißt \bi{Ringhomomorphismus}, wenn $\varphi$ ein Gruppenhomomorphismus von $(R,+)$ nach $(S,+)$ ist
	\index{Homomorphismus!Ring}	
	und
	\begin{enumerate}
		\item[\textbf{(1)}]
		$\varphi(r_1 \cdot r_2 ) = \varphi(r_1) \cdot \varphi(r_2)$
		
		\item[\textbf{(2)}]
		$ \varphi(1_R) = 1_S$
	\end{enumerate}
	gelten. Der Punkt muss gelten, da bei uns Ringe immer die $1$ enthalten.
	Wenn $\varphi$ bijektiv ist, heißt $\varphi$ \bi{Isomorphismus}.
	\index{Homomorphismus!Isomorphismus}\index{Homomorphismus!isomorph} 
	Wir sagen dann $R$ und $S$ sind \bi{isomorph} und schreiben dafür $R \cong S$.
	Einen Isomorphismus von $R$ nach $R$ nennen wir \bi{Automorphismus}.
	\index{Homomorphismus!Automorphismus}
	Einen Ringhomomorphismus von $R$ nach $R$ nennen wir \bi{Epimorphismus}.
	\index{Homomorphismus!Epimorphismus}
\end{df}

\begin{sz}\label{skript:8.5}
	Seien $R,S$ Ringe und $\varphi :R \to S$ ein Ringhomomorphismus.
	\begin{enumerate}
		\item[\textbf{(1)}]
		Es gelten $\Ker \varphi = \lbrace a \in R \ | \ \varphi(a) = 0 \rbrace \unlhd R$
		und $\Bild \varphi $ ist ein Teilring von $S$.
		
		\item[\textbf{(2)}]\bi{Homomorphiesatz}:\index{Satz!Homomorphie}
		Sei $I = \Ker \varphi$ Dann existiert ein eindeutiger Ringhomomorphismus
		\begin{align*}
		\overline{\varphi} : R/I \to S
		\end{align*}
		mit $ \varphi = \overline{\varphi} \circ \pi$, wobei 
		\begin{align*}
		\pi : R \to R/I, \ a \mapsto \overline{a}
		\end{align*}
		ein surjektiver Ringhomomorphismus ist.
		Außerdem gilt $\Ker \overline{\varphi} = \lbrace \overline{0} \rbrace $ und $\overline{\varphi}$ ist injektiv.
		Damit folgt $R/I \cong \Bild \varphi$. Falls $\varphi$ surjektiv ist, gilt sogar $R/I \cong S$.
	\end{enumerate}
\end{sz}

\begin{proof} \
	\begin{enumerate}
		\item[\textbf{(1)}]
		Zunächst ist uns aus der Gruppentheorie bekannt, dass $(\Ker \varphi, + ) \leq (R,+)$ gilt.
		Sei nun $a \in \Ker \varphi$ und $r \in R$. Wegen
		\begin{align*}
		\varphi(r \cdot a) &= \varphi(r) \cdot \varphi(r) = 0 \\
		\varphi(a \cdot r) &= \varphi(a) \cdot \varphi(r) = 0
		\end{align*}
		folgt $\Ker \varphi \unlhd R$.
		Der zweite Teil befindet sich in den Übungsaufgaben.
		
		\item[\textbf{(2)}]
		In der Gruppentheorie(\ref{skript:5.1}) haben wir alle nötigen Eigenschaften für $\overline{\varphi}$ gezeigt.
		Übrig bleibt die Wohldefiniertheit, was wir als zusätzliche Übungsaufgabe ansehen.
	\end{enumerate}
\end{proof}

\begin{genericdf}{Charakteristik eines kommutativen Rings}\label{skript:8.6}
	Sei $R$ ein kommutativer Ring, $a \in R$ und $m \in \Z$.
	Ist $m > 0 $, dann definieren wir 
	\begin{align*}
	m \cdot a := \underbrace{a + \dots + a}_{m-\text{mal}}
	\end{align*}
	und für $m=0$ dann $0 \cdot a = 0$.
	Sollte $m < 0 $ sein, so definieren wir $m \cdot a := -(-m) \cdot a$.
	Die Abbildung 
	\begin{align*}
	\varphi : \Z \to R, m \mapsto m \cdot 1_R
	\end{align*}
	ist ein Ringhomomorphismus.
	Da $\Z$ ein Hauptidealring ist existiert ein $n \in \Z$, sodass
	\begin{align*}
	\Ker \varphi = (n) = n \Z
	\end{align*}
	gilt.
	Wir nennen $|n|$ die \bi{Charakteristik} von $R$ und schreiben hierfür $\Char R$.\index{Charakteristik}
	Diese ist eindeutig bestimmt, denn $n$ und $-n$ erzeugen dasselbe Hauptideal.
	Nun betrachten wir ein paar Eigenschaften:
	\begin{enumerate}
		\item[\textbf{(1)}]
		Falls $\Char R = 0$ gilt ist $\varphi$ injektiv und es ist $m \cdot 1 \neq 0$ für alle $m \neq 0$.
		Also hat $R$ einen zu $\Z$ isomorphen Teilring.
		\item[\textbf{(2)}]
		
		Ist $\Char R > 0$, dann gilt
		\begin{align*}
		\underbrace{1+ \dots + 1}_{\Char R- \text{mal}}
		\end{align*}
		und $\Char R$ ist die kleinste positive Zahl mit dieser Eigenschaft.
		Nach \ref{skript:8.5} ist 
		\begin{align*}
		\overline{\varphi} : \Z / \Char R \Z \to R
		\end{align*}				
		injektiv, womit $R$ einen zu $\Z / \Char R \Z$ isomorphen Teilring besitzt.
						
		\item[\textbf{(3)}]
		Sei $R$ ein Integritäsring mit $\Char R > 0$.
		Dann ist $p := \Char R $ eine Primzahl.
		Insbesondere gilt dies auch für Körper.
		\begin{proof}
			Angenommen $\Char R = m_1 \cdot m_2$ mit $0 < m_1,m_2 < \Char R$.
			Dann gilt
			\begin{align*}
			\varphi(\Char R\cdot 1) 
			= \varphi(m_1) \cdot \varphi(m_2) 
			\end{align*}
			und es folgt $\varphi(m_1) = 0$ oder $\varphi(m_2) = 0$.
			Dies ist ein Widerspruch zur Minimalität von $\Char R$.
			Damit muss $\Char R$ prim sein.		
		\end{proof}
	\end{enumerate}
\end{genericdf}

\begin{lemma}\label{skript:8.7}
	Sei $R$ ein Integritätsring und $\Char R =: p > 0$.
	Dann ist
	\begin{align*}
	F : R \to R, \ a \mapsto a^p
	\end{align*}
	ein injektiver Ringhomomorphismus und wir nennen $F$ \bi{Frobeniusendomorphismus}. \index{Endomorphismus!Frobenius}
\end{lemma}

\begin{proof}
	Zunächst zeigen wir die Eigenschschaften für einen Ringhomomorphismus.
	Seien $a,b \in R$ beliebig.
	Da $R$ kommutativ ist gilt
	\begin{align*}
	F(a) \cdot F(b) = a^p \cdot b^p = (a\cdot b)^p = F(a \cdot b),
	\end{align*}
	womit die multiplikative Eigenschaft gezeigt ist.
	Nun folgt wieder mit der Kommutativität
	\begin{align*}
	F(a+b) = (a+b)^p
	= \sum \limits_{i=0}^p \binom{p}{i} \cdot a^i \cdot b^{p-i}
	\end{align*}
	Nun ist der Binomialkoeffizent für $ 1 \leq i \leq p-1$ durch $p$ teilbar.
	Da $\Char R = p$ gilt folgt somit
	\begin{align*}
	F(a+b)  = a^p + a^p = F(a) + F(b),
	\end{align*}
	womit $\varphi$ ein Ringhomomorphismus ist.
	Da $R$ ein Integritätsring ist, gilt auch
	\begin{align*}
	\Ker \varphi = \lbrace a \in R \ | \ a^p = 0 \rbrace
	\end{align*}
	und somit ist $\varphi$ injektiv.			
\end{proof}

\begin{genericthm}{Universelle Eigenschaft von Polynomringen}\label{skript:8.8}
	\index{Polynomring!Universelle Eigenschaft}
	Sei $ R $ ein kommutativer Ring, $ R[x] $ der zugehörige Polynomring,
	$ \varphi : R \to S $ ein Ringhomomorphismus und $ s \in S $ fest gewählt.
	Dann existiert ein eindeutiger Ringhomomorphismus
	\begin{align*}
	\varphi_s : R[x] \to S \quad \text{mit} \quad \varphi_s \Big|_R = \varphi \quad
	\text{und} \quad \varphi_s(x) = s.
 	\end{align*}
 	Hierbei identifizieren wir $ R $ mit den Polynomen vom Grad $ 0 $ und $ - \infty $.
	Wir schreiben für $ f \in R[x] $ statt $ \varphi_s(f) $ einfach $ f(s) $.
	Daher kommt dann der Name \bi{Einsetzungshomomorphismus}.\index{Homomorphismus!Einsetzung}
\end{genericthm}

\begin{proof}
	Sei $ f = a_0 +a_1 \cdot x + \dots + a_n \cdot x^n \in R[x]$.
	Wir definieren $ \varphi_s(f) $ durch \\
	$ \varphi(a_0) + \varphi(a_1) \cdot s + \dots + \varphi(a_n) \cdot s^n $.
	Durch schnelles Nachrechnen erkennen wir, dass die Ringhomomorphismuseigenschaften erfüllt sind.
	Um die Eindeutigkeit zu zeigen wählen wir einen anderen Ringhomomorphismus $ \tilde{\varphi_s} $
	mit $ \tilde{\varphi_s}(x) = s $.
	Dann gilt
	\begin{align*}
	\tilde{\varphi_s}(x^m) = (\tilde{\varphi_s}(x))^m = s^m = (\varphi_s(x))^m = \varphi(x^m)
	\end{align*}
	und jedes Element aus $ R[x] $ ist eine $ R $-Linearkombination von
	$ \lbrace x^m \ | \ m \in \N \rbrace$. 
	Also folgt\\
	$ \tilde{\varphi_s} = \varphi_s $.
\end{proof}

\begin{lemma} \label{skript:8.9}
	Sei $ K $ ein Körper und $ f \in K[x] $ mit $ \Grad (f) = n \geq 0 $.
	Dann besitzt $ f $ höchstens $ n $ Nullstellen in $ K $.
\end{lemma}

\begin{proof}
	Für den Fall das $ \Grad(f) = 0 $ ist folgt die Aussage sofort, denn $ f $ ist konstant und ungleich $ 0 $.
	Also betrachten wir den Fall, dass $ \Grad(f) > 0 $ ist.
	Sei $ a \in K  $ eine Nullstelle von $ f $, dann gilt $ f = (x-a) \cdot g $.
	Dies müssen wir jedoch noch zeigen.
	Die Division mit Rest liefert uns 
	\begin{align*}
	f = q \cdot (x-a) + r
	\end{align*}
	mit $ \Grad(r) < \Grad(q) $. Mit dem Einsetzungshomomorphismus aus \ref{skript:8.8} erhalten wir
	\begin{align*}
	0 = f(a) = q(a) \cdot \underbrace{(a-a)}_{=0} + r(a),
	\end{align*}
	woraus $ r(a) = 0 $ folgt.
	Also ist $ a $ eine Nullstelle von $ r $ und es gilt $ \Grad(r) < \Grad(f) $.
	Dann folgt mit Induktion nach dem Grad $ r = (x-a) \cdot \tilde{q} $, womit weiter
	\begin{align*}
	f = (x-a)  \cdot q + (x-a) \cdot \tilde{q}
	= (x-a) \cdot \underbrace{(q + \tilde{q})}_{=g}
	\end{align*}
	folgt.
	Sei nun $ f = (x-a) \cdot g $.
	Dann gilt $ \Grad(g) = n -1  $ und es folgt induktiv, dass $ g $ höchstens $ n-1 $ Nullstellen besitzt.
	Wir betrachten nun eine weitere Nullstelle $ b \neq a $ von $ f $.
	Da $ R $ eine Integritätsring ist, folgt mit
	\begin{align*}
	0 = f(b) = \underbrace{(b-a)}_{\neq 0} \cdot g(b),
	\end{align*}
	dass $ g(b) = 0 $ ist.
	Also hat $ f $ höchstens $ n $ Nullstellen.
\end{proof} 

\begin{generic_no_num}{Bemerkung}
	Wenn $ K $ kein Körper ist, dann ist dieses Lemma im Allgemeinen falsch.
	Das Polynom $ x^2-1 $ besitzt beispielsweise über $ \Z / 8 \Z $ vier Nullstellen.
\end{generic_no_num}

\begin{sz} \label{skript:8.10}
	Sei $ K $ ein Körper und $ G $ eine endliche Untergruppe von $ K^\ast $.
	Dann ist $ G $ zyklisch.
	Sollte $ K $ endlich sein, ist somit $ K^\ast $ zyklisch.
\end{sz}

\begin{proof}
	Sei $ g \in G $ mit maximaler Ordnung $ \ord(g) = n $.
	Da Körper kommutativ bezügliche beiden Verknüpfungen sind muss $ G $ abelsch sein.
	In \ref{skript:5.8} haben wir gezeigt, dass $ \ord(h) | n $ für alle $ h \in G $ gilt.
	Damit sind alle $ h \in G $ eine Nullstelle von $ x^n -1 \in K[x] $.
	Nach \ref{skript:8.9} hat dieses Polynom höchstens $ n $ Nullstellen.
	Also gilt $ |G| \leq n $.
	Jedoch gilt auch $ \ord(g) = n$, womit $ G = \langle g \rangle $ ist.
	Somit ist $ G $ zyklisch.
\end{proof}

\begin{genericthm}{Chinesischer Restsatz}\label{skript:8.11}
	Seien $ n,m \in \N $ und $ \ggT(n,m) = 1 $.
	Dann gilt
	\begin{align*}
	\Z / (mn) \Z \cong \Z /m \Z \times \Z /n \Z.
	\end{align*}
\end{genericthm}

\begin{proof}
	Dieser Beweis knüpft direkt an den von \ref{skript:5.9} an.
	Die Aussage 
	\begin{align*}
	(\Z / mn \Z , +) \cong (\Z / m \Z,+) \times (\Z / n \Z,+) 
	\end{align*}
	wurde in dort schon gezeigt, wobei wir nun die additive Schreibweise gewählt haben.
	Die Abbildung 
	\begin{align*}
	\varphi : k + mn\Z \mapsto (k + m \Z, k + n \Z)
	\end{align*}
	ist der passende Ringhomomorphismus.
	Wir müssen nur noch die Multiplikation überprüfen, was uns aber zur Übung freigestellt ist.
\end{proof}

\begin{generic_no_num}{Beispiel}
	Wir wollen wissen was $ 2^{16}  \mod 15$ ergibt.
	Wenden wir $ \varphi $ aus dem letzten Beweis darauf an, so erhalten wir 
	$(2^{16} \mod 3, 2^{16} \mod 5)$.
	Nun gilt
	\begin{align*}
	2^{16} \mod 3 &= (-1)^{16} \mod 3 \\
	2^{16} \mod 5	&= 4^8 \mod 5 = (-1)^8 \mod 5,
	\end{align*}
	womit
	\begin{align*}
	(2^{16} \mod 3, 2^{16} \mod 5) = (1 \mod 3, 1 \mod 5)
	\end{align*}
	gilt. Das Urbild bezüglich $ \varphi $ ist dann $ 1 \mod 15 $.
	Es gilt als $ 2^{16} \mod 15  = 1$.
\end{generic_no_num}

\begin{lemma}\label{skript:8.12}
	Sei $ \varphi : R \to S $ ein surjektiver Ringhomomorphismus.
	Dann gelten:
	\begin{enumerate}
		\item[\textbf{(1)}]
		Ist $ I \unlhd R $, dann auch $ \varphi(I) \unlhd S $.
		\item[\textbf{(2)}]
		Für $ J \unlhd S $ gilt $ \varphi^{-1}(J) \unlhd R $. 
	\end{enumerate}
\end{lemma}

\begin{proof}
	Den ersten Teil haben wir in den Übungen bewiesen.
	Deswegen kommen wir direkt zu dem zweiten Teil.
	Es gilt $ \varphi^{-1}(J) \leq R $, denn es gilt $(J,+) \leq (S,+)  $.
	Seien nun $ x \in \varphi^{-1}(J) $ und $ r \in R $.
	Wegen $ J \unlhd_L S $ ist
	\begin{align*}
	\varphi(r \cdot x) = \underbrace{\varphi(r) }_{\in S} \cdot \underbrace{\varphi(x)}_{\in J},
	\end{align*}
	womit $ r \cdot x \in \varphi^{-1}(J) $ gilt.
	Durch analoges Vorgehen mit $ J \unlhd_R S $ folgt $ x \cdot r \in \varphi^{-1}(J) $.
\end{proof}

\begin{generic_no_num}{Bemerkung}
	Die zweite Teil des letzten Lemmas gilt auch wenn $ \varphi $ nicht surjektiv ist.
	Die Surjektivität wird nur für den ersten Teil benötigt.
\end{generic_no_num}

\begin{df} \label{skript:8.13}
	Sei $ R $ ein Ring.
	\begin{enumerate}
		\item[\textbf{(1)}]
		Hier ist $ R $ noch zusätzlich kommutativ.
		Sei $ I,J \unlhd R $, dann definieren wir 
		\begin{align*}
		I\cdot J = 
		\lbrace x_1 \cdot y_1 + \dots + x_n \cdot y_n\ | \ x_i \in I, y_i \in J, n \in \N \rbrace 
		\end{align*}
		und durch schnelles Nachrechnen sehen wir $ I \cdot J \unlhd R $.
		Aus diesem Grund nennen wir dies \bi{Produktideal}.
		\index{Ideal!Produkt}
		Wir gehen analog für $ I + J $ vor und nennen dies \bi{Summenideal}.
		\index{Ideal!Summe}
		
		\item[\textbf{(2)}]
		Sei $ I \unlhd R $ mit $ I \neq R $.
		Wir nennen $ I $ \bi{Primideal}, falls für beliebige Ideale
		$ J,K \unlhd R $ \index{Ideal!prim}
		\begin{align*}
		J \cdot K \subseteq I \Rightarrow J \subseteq I \vee K \subseteq I 
		\end{align*}
		gilt.
		
		\item[\textbf{(3)}]
		Wir nennen $ I \unlhd R $ \bi{maximales Ideal},\index{Ideal!maximal}
		falls für $I \neq R$ und $ J \unlhd R $
		\begin{align*}
		I \subseteq J \subseteq R
		\Rightarrow
		I = J \vee J = R
		\end{align*}
		gilt.
	\end{enumerate}
\end{df}

\begin{lemma}\label{skript:8.14}
	Sei $R$ ein kommutativer Ring und $I \unlhd R$, dann gelten:
	\begin{enumerate}
		\item[\textbf{(1)}]
		$R/I \ \text{Integritätsring} \quad \Leftrightarrow \quad  I \ \text{Primideal} $
		
		\item[\textbf{(2)}]
		$R/I \ \text{Körper} \qquad  \qquad \Leftrightarrow \quad  I \ \text{maximales Ideal} $  
	\end{enumerate}
\end{lemma}

\begin{proof}
	Den ersten Teil werden wir in den Übungen zeigen.
	Deswegen gehen wir direkt zum zweiten Teil über.
	Angenommen $R/ I$ ist ein Körper und sei $I \subseteq J \subseteq R$ und $J \unlhd R$.
	Sei $\pi : R \to R/I$ die Quotientenabbildung.
	Da $\pi$ surjektiv ist, gilt $\pi(J) \unlhd R/I $ und aufgrund der Körpereigenschaft
	folgt
	\begin{align*}
	\pi(J) = 0 \vee \pi(J) = R/I
	\Rightarrow
	J = I \vee J = R.
	\end{align*}
	Also ist $I$ ein maximales Ideal.
	Umgekehrt sei $I$ ein maximales Ideal von $R$ und $a \in R/I$.
	Damit ist auch $(a)$ ein Ideal von $R/I$, womit nach \ref{skript:8.2} $\pi^{-1}((a)) \unlhd R $ gilt.
	Wegen $I \subseteq \pi^{-1}(\pi(I)) \subseteq \pi^{-1}((a))$ folgt 
	\begin{align*}
	\pi^{-1}((a)) = I \vee \pi^{-1}((a)) = R,
	\end{align*}
	da $I$ ein maximales Ideal ist. 
	Wir erhalten nun
	\begin{align*}
	\pi^{-1}((a)) &= I \Rightarrow (a) = 0\\
	\pi^{-1}((a)) &= R \Rightarrow (a) = (1) = R/I,
	\end{align*}
	womit $a$ invertierbar ist.
	Dies gilt, da ein $b \in R/I$ existiert mit $a \cdot b = 1$.
	Aufgrund der Kommutativität von $R$ ist jedes Element von $R/I$ invertierbar und $R/I$ ist kommutativ.
	Also ist $R/I$ ein Körper.
\end{proof}

\begin{genericdf}{Beispiele und Bemerkung}\label{skript:8.15} \
	\begin{enumerate}
		\item[\textbf{(1)}]
		Sei $R = \Z$ und $I = p \Z$ mit $p$ prim.
		Dann ist $R/I$ ein Körper und $I$ somit ein maximales Ideal.
		
		\item[\textbf{(2)}]
		Sei $m = m_1 \cdot m_2$ mit $1 \leq m_1, m_2 \leq m$.
		Dann ist $\Z / m \Z$ kein Körper und besitzt sogar Nullteiler.
		Damit is $m \Z$ weder maximal noch ein Primideal.
		
		\item[\textbf{(3)}]
		Jedes maximale Ideal ist ein Primideal, da Körper Integritätsringe sind.
		
		\item[\textbf{(4)}]	
		Das Ideal $0 \Z$ ist ein Primideal, denn $\Z / 0 \Z \cong \Z$ ist ein Integritätsring.	
	\end{enumerate}
\end{genericdf}



\subsection{Aufgaben zu Abschnitt 8}

\begin{exe}\label{aufgabe:8.1}
	Bestimmen Sie alle zweiseitigen Ideale von
	\begin{enumerate}
		\item[a)] $ \Z $.
		
		\item[b)] $ K[x] $, wobei $ K $ ein Körper ist.
		
		\item[c)] $ M_n(K) $, wobei $ K $ ein Körper ist.
		
		\item[d)] $ M_n(R) $, wobei $ R $ ein Ring mit $ 1 $ ist.
	\end{enumerate}
	\hyperlink{loes:8.1}{Lösung}
\end{exe}

\begin{exe}\label{aufgabe:8.2}
	Sei $ R $ ein kommutativer Ring mit $ 1 $.
	Zeigen Sie:
	\begin{align*}
	I \ \text{Primideal} \quad \Leftrightarrow \quad R/I \ \text{Integritätsring}
	\end{align*}
	\hyperlink{loes:8.2}{Lösung}
\end{exe}

\begin{exe}\label{aufgabe:8.3}
	Sei $ R $ ein Integritätsring.
	Zeigen Sie:\\
	Für den Einsetzungshomomorphismus mit $ \varphi = \id_R $
	\begin{align*}
	\varphi_r : R[x] \to R, \ f \mapsto f(r)
	\end{align*}
	ist $ \Ker \varphi_r $ für alle $ r \in R $ ein Primideal.
	\hyperlink{loes:8.3}{Lösung}
\end{exe}

\begin{exe}\label{aufgabe:8.4}
	Bestimmen Sie die letzte Ziffer von $ 17^{68} $
	und die letzten beiden Ziffern von $ 14^{200} $.
	\hyperlink{loes:8.4}{Lösung}
\end{exe}

\newpage
\section*{TODO}
Die letzten zwei Punkte überarbeiten.
\section{Faktorielle Ringe}

\begin{df}\label{skript:9.1}
	Sei $R$ ein Integritätsring, $p \in R \setminus \lbrace 0 \rbrace$ und $p \notin R^\ast$.
	
	\begin{enumerate}
		\item[\textbf{(1)}]
		Wir nennen $p$ \bi{irreduzibel},\index{Irreduziblität}
		falls aus $p = a \cdot b$ für $a,b \in R$ folgt,
		dass $a \in R^\ast $ oder $b \in R^\ast$ gilt.
		
		\item[\textbf{(2)}]
		Wir nennen $p$ \bi{prim} oder \bi{Primelement},\index{Primelement}
		falls aus $p \ | \ a \cdot b$ für $a,b \in R$ folgt,
		dass $p \ | \ a$ oder $p \ | \ b$ gilt.
	\end{enumerate}
\end{df}

\begin{lemma}\label{skript:9.2}
	Sei $R$ ein Integritätsring.
	\begin{enumerate}
		\item[\textbf{(1)}]
		Sei $p \in R$ prim, dann folgt $p$ ist irreduzibel.
		
		\item[\textbf{(2)}]
		Es gilt
		\begin{align*}
		p \in R \ \text{prim}
		\quad \Leftrightarrow \quad
		R/ (p) \ \text{Integritätsring} 
		\quad \Leftrightarrow \quad
		(p) \ \text{Primideal}.
		\end{align*}				
	\end{enumerate}
\end{lemma}

\begin{proof}\
	\begin{enumerate}
		\item[\textbf{(1)}]
		Sei $p = a \cdot b$, dann gilt $p$ teilt $a \cdot b$.
		Da $p$ prim ist gilt $p$ teilt $a$ oder $p$ teilt $b$.
		Angenommen $p$ teilt $a$, dann exisitiert ein $c \in R$ mit $a = p \cdot c$
		und es gilt 
		\begin{align*}
		p = a \cdot b = p \cdot c \cdot b
		\Rightarrow
		p \cdot (1 - b\cdot c) = 0.
		\end{align*}
		Da nun $R$ ein Integritätsring und $p \neq 0$ ist gilt
		\begin{align*}
		(1 -c \cdot b ) = 0 
		\Rightarrow
		1 = c \cdot b. 
		\end{align*}
		Damit gilt $b \in R^\ast$.
		Für den Fall $p$ teilt $b$ gehen wir analog vor und erhalten, dass $p$ irreduzibel ist.
		
		\item[\textbf{(2)}]
		Sei $p$ prim. 
		Wir wählen $\overline{a}, \overline{b} \in  R / (p)$ mit
		$\overline{a} \cdot \overline{b} = \overline{a \cdot b} = 0$.
		Damit gilt $a \cdot b \in (p)$, womit 
		\begin{align*}
		a \cdot b = p \cdot x 
		\end{align*}
		für ein $x \in R$ folgt. Damit gilt $p$ teilt $a \cdot b$.
		Da $p$ prim ist, gilt $p$ teilt $a$ oder $p$ teilt $b$.
		Also muss $\overline{a} = 0$ oder $\overline{b} = 0$ sein,
		womit $R / (p)$ ein Integritätsring ist.
		Sei nun umgekehrt $R/ (p)$ ein Integritätsring und $p$ teilt $a \cdot b$.
		Dann gilt $a \cdot b = p \cdot x$ für ein $x \in R$.
		Mit der Quotientenabbildung $\pi : R \to R / (p)$ erhalten wir 
		\begin{align*}
		\overline{a} \cdot \overline{b} 
		= \overline{a \cdot b}
		= \overline{p \cdot x}
		= \overline{p} \cdot \overline{x}
		= 0
		\end{align*}
		und mit der Integritätsringeigenschaft von $R/ (p)$
		folgt
		\begin{align*}
		\overline{a} = 0 \vee  \overline{b} = 0
		\Rightarrow
		a \in (p) \vee b \in (p),
		\end{align*}
		womit $p $ teilt $a$ oder $p$ teilt $b$ gilt.
		Also ist $p$ prim.
	\end{enumerate}		
\end{proof}

\begin{genericdf}{Beispiel}\label{skript:9.3} \
	\begin{enumerate}
		\item[\textbf{(1)}]
		Sei $R = \Z$. Primzahlen sind nach Definition irreduzible Elemente.
		Es ist zunächst nicht klar, ob Primzahlen prim sind.
		
		\item[\textbf{(2)}]
		Sei $R = \Z[\sqrt{-5}]$.
		Dann gilt 
		\begin{align*}
		2 \cdot 3 = (1 + \sqrt{-5}) \cdot (1 - \sqrt{-5}).
		\end{align*}
		Also sind $2$, $3$, $1 + \sqrt{-5}$ und $1 - \sqrt{-5}$ irreduzibel, aber nicht prim.
	\end{enumerate}
\end{genericdf}

\begin{df}\label{skript:9.4} \
	\begin{enumerate}
		\item[\textbf{(1)}]
		Ein Integritätsring $ R $ heißt \bi{faktoriell}, wenn\index{Ring!faktoriell} \index{Faktoriell}
		\begin{itemize}
			\item
			jedes $ 0 \neq a \in R $ entweder eine Einheit ist
			oder sich als endliches Produkt von irreduziblen Elementen schreiben lässt.
			Das heißt
			\begin{align*}
			a = p_1\cdots p_i \cdots p_k
			\end{align*}
			mit $ p_i $ irreduzibel für $ 1 \leq i \leq k $.
			
			\item
			jedes irreduzible Element aus $ R $ ein Primelement ist.	
		\end{itemize}
		
		\item[\textbf{(2)}]
		Sei $ R $ faktoriell.
		Wir definieren für $ 0 \neq a \in R $ durch
		 \begin{align*}
		 l(a) := \min \lbrace r \geq 1 \ | \ a= p_1 \cdots p_r 
		 \ \text{mit} \ p_i \ \text{irreduzibel} \rbrace
		 \end{align*}
		 für $ a \notin R^\ast $ und $ l(a) = 0 $ für $ a \in R^\ast $.
		 Dies bezeichnen wir als \bi{Länge} von $ a $.
		 \index{Faktoriell!Länge}
	\end{enumerate}
\end{df}

\begin{sz}\label{skript:9.5}
	Sei $ R $ ein faktorieller Ring und 
	$ 0 \neq a \in R $ mit $ l(a) = k \geq 1 $.
	Also ist $ a = p_1 \cdots p_k $ mit $ p_i $ irreduzibel.
	Für
	\begin{align*}
	a = q_1 \cdots q_l
	\end{align*}
	mit $ q_j $ irreduzibel für $ 1 \leq j \leq l $ gilt $ k = l $.
	Außerdem existiert ein $ \pi \in S_n $ und $ u_i \in R^\ast$
	für $ 1 \leq i \leq k$, sodass 
	\begin{align*}
	u_i \cdot p_i = p_{\pi(i)}
	\end{align*}
	gilt.
	Nun gilt noch $ l(a \cdot b) = l(a) + l(b) $ für alle
	$ a \in R \setminus \lbrace 0 \rbrace $.
\end{sz}

\begin{proof}
	Angenommen $ k = 1 $, dann folgt
	$ p_1 = q_1 \cdots q_l $ mit $ q_i $ irreduzibel.
	Da $ p_1 $ irreduzibel ist, kann nur $ l = 1 $ gelten.
	Sei nun $ k \geq 2 $. Dann gilt
	\begin{align*}
	p_1 \cdots p_k = q_1 \cdots q_l
	\end{align*}
	und weil $ R $ faktoriell ist folgt
	\begin{align*}
	q_1 \mid p_1 \vee \dots \vee q_1 \mid p_k,
	\end{align*}
	denn irreduzible Elemente sind prim.
	Ohne Beschränkung der Allgemeinheit nehmen wir an, dass $ q_1 | p_1 $ gilt.
	Da $ p_1 $ irreduzibel ist folgt $ p = c \cdot q_1 $ mit $ c \in R^\ast $.
	Damit erhalten wir
	\begin{align*}
	c \cdot q_1 \cdot p_2 \cdots p_k = q_1 \cdot q_2 \cdots q_l
	\Leftrightarrow
	c \cdot p_2 \cdots p_k =  q_2 \cdots q_l
	\end{align*}
	und mit der Induktionsvoraussetzung $ k-1 = l-1 $ folgt $ k = l $.
	Weiter existiert ein $ \tilde{\pi} \in S_n $ mit
	\begin{align*}
	q_i = p_{\tilde{\pi}(i)}
	\end{align*}
	und $ u_i \in R^\ast $ für $ 2 \leq i \leq k $.
	Wir setzen $ \pi(1) = 1 $ und $ \pi(i) = \tilde{\pi}(i) $ und haben somit unsere gesuchte Permutation.
	Seien nun $ 0 \neq a,b \in R $.
	Für $ a \in R^\ast $ gilt $ l(a \cdot b) = l(a) $ und analog gilt dies auch für $ b \in R^\ast $.
	Deswegen seien $ a $ und $ b $ keine Einheiten. Dann gilt
	\begin{align*}
	a &= p_1 \cdots p_k\\
	b &= q_1 \cdots q_l\\
	a \cdot b &= p_1 \cdots p_k \cdot q_1 \cdots q_l,
	\end{align*}
	womit $ a \cdot b $ wieder ein Produkt aus irreduziblen Elementen ist.
	Die Anzahl der Faktoren ist wie oben gezeigt immer minimal, womit
	$ l(a \cdot b ) = l(a) + l(b)  $ gilt. 
\end{proof}

\begin{generic_no_num}{Bemerkung}
	Ist $ p $ irreduzibel und $ a \in R^\ast $, so ist $a \cdot p  $
	irreduzibel.
\end{generic_no_num}

\begin{lemma} \label{skript:9.6}
	Sei $ R $ ein Integritäts-und Hauptidealring,
	dann gilt irreduzibel gleich prim. 
	Einen solchen Ring nennen wir auch \bi{Hauptidealbereich}.
	\index{Hauptideal!Bereich}
\end{lemma}

\begin{proof}
	Wir müssen zeigen, dass unter den Voraussetzungen aus irreduzibel prim folgt.
	Die andere Richtung haben wir bereits in \ref{skript:9.2} gezeigt.
	Sei $ q $ irreduzibel mit $ q $ teilt $ a \cdot b $.
	Da $ R $ ein Hauptidealring ist, gibt es ein $ d \in R $, sodass
	\begin{align*}
	(q) + (a) = (d)
	\end{align*}
	gilt.
	Damit folgt $ q = s \cdot d $.
	Da $ q $ irreduzibel ist, folgt $ s \in R^\ast $ oder $ r \in R^\ast $.
	Falls $ s \in R^\ast $ ist, folgt $ d = s^{-1}  \cdot q$ und $ q $ teilt $ d $.
	Wegen
	\begin{align*}
	q | d \wedge d | a
	\end{align*}
	gilt $ q $ teilt $ a $.
	Nun kommen wir zu dem Fall, dass $ d \in R^\ast $ ist.
	Damit gilt $ (d) = (1)  = R$, womit
	\begin{align*}
	1 = r_1 \cdot q + r_2 \cdot a
	\Rightarrow 
	b = b \cdot r_1 \cdot q +b \cdot r_2 \cdot a
	\end{align*}
	für passende $ r_1  $ und $ r_2 $ folgt.
	Mit $ q $ teilt $ a \cdot b $ folgt
	\begin{align*}
	q \mid b \cdot r_2 \cdot a
	\Rightarrow 
	q | b.
	\end{align*} 
	Also ist $ q $ prim.
\end{proof}

\begin{sz}\label{skript:9.7}
	Sei $ K $ ein Körper.
	Die Ringe $ \Z $ und $ K[x] $ sind faktoriell.
\end{sz}

\begin{proof}
	Die Ringe $ \Z $ und $ K[x] $ sind euklidisch und somit auch Hauptidealringe.
	Außerdem sind diese auch Integritätsringe, womit prim gleich irreduzibel ist.
	Sei $ a \in R \setminus R^\ast $.
	Wir müssen noch zeigen, dass $ a $ ein endliches Produkt von irreduziblen Elementen ist.
	Zunächst betrachten wir $ R = \Z $.
	Für $ a = b \cdot c $ folgt $ |a| = |b| \cdot |c| $ mit $ |b|, |c| < |a| $.
	Da für $ 0 \neq a \notin R^\ast $ immer $ |a| \geq 2 $ gilt, folgt mit der Beschränktheit nach unten von $ \N $ die Aussage.
	Nun kommen wir zu $ R = K[x] $.
	Das Vorgehen ist analog, wir verwenden die Gradfunktion.
	Es gilt
	\begin{align*}
	f \in R^\ast &\Leftrightarrow \Grad(f) = 0 \\
	0 \neq f \notin R^\ast &\Leftrightarrow \Grad(f) \geq 1
	\end{align*}
	und für $ a = b \cdot c $ folgt
	\begin{align*}
	\Grad(a) = \Grad(b ) + \Grad(c)
	\Rightarrow
	\Grad(b) < \Grad(a) \wedge \Grad(c) < \Grad(a),
	\end{align*}
	falls $ 0 \neq a,b,c \notin R^\ast $.
\end{proof}

\begin{genericdf}{Bemerkungen}\label{skript:9.8}
	\
	\begin{enumerate}
		\item[\textbf{(1)}]
		Satz \ref{skript:9.7} gilt allgemein für jeden Hauptidealbereich.
		
		\item[\textbf{(2)}]
		Insbesondere also für euklidische Ringe.
		Sei $ R $ euklidisch und $ \nu $ die zugehörige Normfunktion.
		Dann können wir zeigen, dass
		\begin{align*}
		\mu : R \setminus \lbrace 0 \rbrace \to \N_0, \ 
		a \mapsto \min \lbrace \nu(c \cdot a) \ | \ c \in R \setminus\lbrace 0 \rbrace \rbrace
		\end{align*}
		auch ein Normfunktion ist und $ \mu(a \cdot b) \geq \mu(a) $
		für alle $ 0 \neq a,b \in R $ gilt.
		Hiermit können wir konstruktiv zeigen, dass R faktoriell ist.
		
		\item[\textbf{(3)}] 
		Die Sätze \ref{skript:9.5} und \ref{skript:9.7} 
		ergeben zusammen den sogenannten
		\bi{Hauptsatz der elementare Zahlentheorie}, also die bis auf die Reihenfolge eindeutige Zerlegung von ganzen Zahlen in Primfaktoren.
		Insbesondere wissen wir jetzt auch, dass Primzahlen prim sind.
	\end{enumerate}
\end{genericdf}

\begin{lemma} \label{skript:9.9}
	Sei $ R $ ein Hauptidealbereich und $ p \in R $ irreduzibel,
	dann ist $ (p) $ ein maximales Ideal.
	Also ist $ R / (p) $ ein Körper.
\end{lemma}

\begin{proof}
	Sei $ a \in R $ und $ \overline{a} = a + (p) $ mit $ \overline{a} \neq 0 $.
	Es gilt also $ p \nshortmid a $.
	Nun betrachten wir das Ideal $ (p,a) \unlhd R $.
	Da $ R $ ein Hauptidealbereich ist, existiert ein $ d \in R $ mit
	 \begin{align*}
	 (p,a) = (d) 
	 \Rightarrow
	 d = r \cdot p + s \cdot a
	 \end{align*}
	 für passende $ r, s \in R $.
	 Nun gilt $ p \in (d) $, woraus
	 \begin{align*}
	 p = d \cdot c 
	 \Rightarrow
	 d \in R^\ast \vee c \in R^\ast
	 \end{align*}
	 folgt.
	 Nun nehmen wir an, dass $ c \in R^\ast  $ ist.
	 Dann gilt
	 \begin{align*}
	 d = c^{-1} \cdot p
	 \Rightarrow
	 p \mid d
	 \Rightarrow 
	 p \mid a,
	 \end{align*}
	 was ein Widerspruch zu unserer Annahme $ \overline{a}  \neq 0$ ist.
	 Also muss $ d \in R^\ast $ gelten, womit $ (d) = R $ und $ (p) $ maximal ist.
	 Nach \ref{skript:8.14} ist auch $ R/(p) $ ein Körper.
\end{proof}

\begin{genericdf}{Konstruktion von Erweiterungkörpern}\label{skript:9.10}
	\index{Erweiterungskörper!Konstruktion}
	Sei $ K $ ein Körper, $ R  = K[x]$ und $ f $ ein irreduzibles Polynom.
	Nach \ref{skript:9.6} bzw. \ref{skript:9.7} ist $ f $ prim.
	Da $ K[x] $ ein Hauptidealbereich ist, ist $ K[x] / (f) $ nach 
	\ref{skript:9.9} ein Körper.
	Wir betrachten nun $ n = \Grad(f) \geq 1$ und die Quotientenabbildung
	\begin{align*}
	\pi : K[x] \to K[x] / (f), g \mapsto \overline{g} = g + (f).
	\end{align*}
	Wir identifizieren $ K $ mit den Polynomen vom Grad $ \leq 1 $ in $ K[x] $.
%	\begin{figure}[H]
%		\centering
%		\begin{tikzcd}
%			G \arrow{r}{\pi}   &   K[x] / (p) \\
%			K \arrow[u,hook] \arrow{r}{\pi \mid_{K} }  &   \pi(K) \arrow[u,hook]
%		\end{tikzcd}
%	\end{figure}
	Außerdem ist $ \pi $ eingeschränkt auf $ K $ injektiv, damit existiert in
	$ K[x] / (f) $ ein zu $ K $ isomorpher Teilkörper, welchen wir auch mit $ K $
	identifizieren.
	Nun gilt also $ K \subset K[x] / (f) $,
	womit $ K[x] / (f) $ eine Erweiterung von $ K $ ist.
	\begin{generic_no_num}{Bemerkung}
		Sind $ K,L $ Körper mit $ K \subset L $, dann wird $ L $ zu einem Vektorraum über $ K $.
		Die Skalarmultiplikation definieren wir einfach über die Multiplikation in $ L $.
	\end{generic_no_num}
	Nun kommen wir zu der 
	\begin{genericthm_no_num}{Behauptung}
		Sei $ \alpha = \pi(x) $, dann gilt
		\begin{align*}
		K[x] / (f) = \left\lbrace \sum \limits_{i=0}^{n-1}  k_i \cdot \alpha^i 
		\ | \ k_i \in K , n = \Grad(f)\right\rbrace
		\end{align*}
		und
		\begin{align*}
		\lbrace 1, \alpha, \alpha^2, \dots , \alpha^{n-1} \rbrace
		\end{align*}
		ist eine $ K $-Basis von $ K[x] / (f) $.
		Außerdem gilt $ f (\alpha) = \overline{f} = \overline{0} $.
	\end{genericthm_no_num}
	
	\begin{proof}
		Dass jede Restklasse modulo $ (f) $ einen Repräsentanten der Form
		\begin{align*}
		\sum \limits_{i=0}^{n-1} k_i \cdot \alpha_i
		\end{align*}
		besitzt erhalten wir aus Division mit Rest nach $ (f) $.
		Nun nehmen wir an, dass 
		$ \lbrace 1 , \alpha , \dots , \alpha^{n-1} \rbrace $
		linear abhängig ist.
		Damit gilt für
		\begin{align*}
		g:= \sum \limits_{i= 0}^{n-1} k_i \cdot \alpha^i = 0,
		\end{align*}
		dass nicht alle $ k_i = 0  $ sind.
		Nun folgt $ \Grad(g) \leq n-1 $ oder $ g = a \cdot \tilde{g} $ mit $ \Grad(\tilde{g}) \leq n -1 $.
		Weiter gilt $ g(\alpha) = 0 $ bzw. $ \tilde{g}(a) = 0 $.
		Da $ \Ker \pi  = (f)$ ist, müssen $ g $ bzw. $ \tilde{g} $ Vielfache von $ f $ sein.
		Damit gilt $ \Grad g \geq n $ bzw. $ \Grad \tilde{g} \geq n $.
		Dies ist ein Widerspruch zu unserer Annahme.
	\end{proof}
\end{genericdf}

\begin{genericdf}{Beispiel}\label{skript:9.11}
	Sei $ K = \R $ und $ f = x^2 + 1 $.
	Dann ist $ f $ irreduzibel, da es in $ \R $ keine Nullstellen hat und es gilt
	\begin{align*}
	\R[x] / (f) \cong \C
	\end{align*}
\end{genericdf}
\subsection{Aufgaben zu Abschnitt 9}

\begin{exe}\label{aufgabe:9.1}
	Sei $ R $ ein Hauptidealbereich.
	Zeigen Sie:
	\begin{align*}
	\lbrace 0 \rbrace \neq 	I \ \text{Primideal} \quad 
	\Rightarrow \quad
	I \ \text{maximales Ideal}
	\end{align*}
	\hyperlink{loes:9.1}{Lösung}
\end{exe}

\begin{exe}\label{aufgabe:9.2}
	Sei $ R $ ein Integritätsring. Zeigen Sie:
	\begin{align*}
	R[x] \ \text{Hauptidealring}
	\quad
	\Leftrightarrow
	\quad
	R \ \text{Körper}
	\end{align*}
	\hyperlink{loes:9.2}{Lösung}
\end{exe}

\begin{exe}\label{aufgabe:9.3}
	Zeigen Sie:
	\begin{enumerate}
		\item[a)]
		$ \Z[\i] / 2\Z[\i] $ ist kein Körper.
		
		\item[b)] 
		$ \Z[\i] / 3\Z[\i] $ ist ein Körper mit 9 Elementen.
		
		\item[c)] 
		$ \Z[\i] / n\Z[\i] $ ist ein Körper genau dann,
		wenn $ n $ eine Primzahl ist und $ n \neq a^2 + b^2  $ für $ a,b \in \Z $ gilt.
	\end{enumerate}
	\hyperlink{loes:9.3}{Lösung}
\end{exe}

\begin{exe}\label{aufgabe:9.4}
	\begin{enumerate}
		\item[a)]
		Zeigen Sie:
		$ 2 $, $ 3 $, $ 1+ \sqrt{-5} $, $ 1 - \sqrt{-5} $ 
		sind irreduzible Elemente von $ \Z[\sqrt{-5}] $, aber nicht prim.
		Zeigen sie hierfür, dass
		\begin{align*}
		\beta : \Z[\sqrt{-5}] \to \N_0, \ z \mapsto z \cdot \overline{z}
		\end{align*}
		multiplikativ ist.
		
		\item[b)] 
		Bestimmen Sie die Einheitengruppe von $ \Z[\sqrt{-5}] $.
		
		\item[c)]
		Ist $ \Z[\sqrt{-5}] $ ein euklidischer Ring oder ein Hauptidealring?   
	\end{enumerate}
	\hyperlink{loes:9.4}{Lösung}
\end{exe}



\newpage
\section{Irreduziblität von Polynomen}

\begin{genericdf}{Vorbemerkung}\label{skript:10.1}
	Sei $ R $ ein Integritätsring und $ p \in R $ prim.
	Nach \ref{skript:9.6} ist dann auch $ R / (p) $ ein Integritätsring.
	Die Abbildung
	\begin{align*}
	\pi_p : R[x] \to R/(p)[x], \ 
	f = \sum \limits_{i=0}^{n} a_i \cdot x^i \to \sum \limits_{i=0}^{n} \overline{a_i} \cdot x^i = f^\ast
	\end{align*}
	ist ein surjektiver Ringhomomorphismus mit $ \Ker \pi = (p) $ in $ R[x] $.	
\end{genericdf}

\begin{df}\label{skript:10.2}
	Sei $ R $ ein Integritätsring und $ 0 \neq f \in R[x] $.
	Dann nennen wir $ f $ ein \bi{primitives Polynom},\index{Polynom!primitiv}
	wenn es keine Nichteinheit $ 0 \neq a \in R $ gibt, sodass $ a $ alle Koeffizienten
	von $ f $ teilt.
	Insbesondere sind normierte Polynome primitiv.
\end{df}

\begin{genericthm}{Reduktionskriterium}\label{skript:10.3}\index{Reduktionskriterium}
	Sei $ R $ ein Integritätsring,
	$ 0 \neq f \in R[x] $ sei primitiv mit $ \Grad(f) \geq 1 $ und Leitkoeffizient $ a_n  $.
	Sei nun $ p \in R $ ein Primelement mit $ p \nmid a_n $.
	Ist $ \pi_p(f) = f^\ast $ irreduzibel in $ R/(p)[x] $,
	dann ist $ f $ irreduzibel in $ R[x] $.
\end{genericthm}

\begin{proof}
	Wegen $ p \nmid a_n $ muss $ \Grad(f^\ast) = n$ gelten.
	Wir nehmen an, dass $ f = g \cdot h $ mit $ g,h \in R[x] $ gilt.
	Da $ R $ ein Integritätsring ist, gilt $ n = \Grad(f) = \Grad(g) + \Grad(h) $.
	Außerdem ist $ a_n $ Produkt der Leitkoeffizienten von $ g $ und $ h $.
	Damit werden diese auch nicht von $ p $ geteilt.
	Also folgt $ 0 \neq g^\ast $, $ 0 \neq h^\ast $, $ \Grad(g) = \Grad(g^\ast) $
	und $ \Grad(h)  = \Grad(h^\ast)$.
	Somit gilt mit $ \pi_p $ und der Irreduziblität von $ f^\ast $
	\begin{align*}
	f^\ast = g^\ast \cdot h^\ast
	\Rightarrow
	\Grad(g^\ast) = 0 \vee \Grad(h^\ast ) = 0
	\Rightarrow
	\Grad(g) = 0 \vee \Grad(h) = 0.
	\end{align*}
	Wir nehmen ohne Beschränkung der Allgemeinheit an, dass $ \Grad(g) = 0  $ gilt.
	Dann gilt $ f = r \cdot h  $ mit $ r \in R $ und 
	\begin{align*}
	f = \sum a_i \cdot x^i, \quad h = \sum c_i \cdot x^i.
	\end{align*}
	Hieraus folgt $ a_i = r \cdot c_i $.
	Da $ f $ primitiv ist, muss $ r \in R^\ast  $ gelten, womit $ f $ irreduzibel ist.
\end{proof}

\begin{genericthm}{Eisensteinkriterium}\label{skript:10.4}\index{Eisensteinkriterium}
	Sei $ R $ ein Integritätsring und 
	\begin{align*}
	0 \neq f = \sum \limits_{i = 0}^n a_i \cdot x^i \in R[x]
	\end{align*}
	sei  primitiv mit $ \Grad(f) = n \geq 1  $.
	Wenn ein Primelement $ p \in R $ mit $ p \nmid a_n $, $ p \mid a_i $ für alle $ 0 \leq i \leq n-1 $
	und $ p^2 \nmid a_0 $ existiert, dann ist $ f $ irreduzibel in $ R[x] $.
\end{genericthm}

\begin{proof}
	Wir nehmen an, dass $ f = g \cdot h $ ist.
	Wie im letzten Beweis sehen wir, dass $ f^\ast  $, $ g^\ast  $ und $ h^\ast  $ ungleich null sind.
	Außerdem gilt wieder $ \Grad(f^\ast) = \Grad(f) = n = m + k $
	mit \\
	$ m = \Grad(g)  = \Grad(g^\ast)$ und $ k = \Grad(h) = \Grad(h^\ast) $.
	Nun gilt mit unseren Voraussetzungen
	\begin{align*}
	f^\ast = \overline{a_n} \cdot x^n 
	\Rightarrow
	g^\ast= \overline{b_m} \cdot x^m, \quad h^\ast = \overline{c_k} \cdot x^k.
	\end{align*}
	Angenommen $ m > 0 $ und $ k > 0  $.
	Dann sind $ b_0 $ und $ c_0 $ durch $ p $ teilbar, womit $ p^2 \mid a_0 $ gilt.
	Dies ist ein Widerspruch zu unseren Voraussetzungen.
\end{proof}

\begin{genericdf}{Beispiele}\label{skript:10.5}\
	\begin{enumerate}
		\item[\textbf{(1)}]
		Wir betrachten $ f = x^{24} - 18 \cdot x^{7} + 15 \in \Z[x] $,
		dann folgt mit Eisenstein und $ p =3 $, dass $ f $ irreduzibel in $ \Z[x] $ ist.
		
		\item[\textbf{(2)}]
		Sei $ f = x^3-5 \cdot x + 3 \in \Z[x] $.
		Wir wählen $ p = 2 $, womit
		$ f^\ast = x^3 + x + \overline{1}  \in \Z / 2 \Z[x]$ gilt.
		Da $ f^\ast $ keine Nullstellen über $ \Z / 2 \Z $ besitzt, ist $ f^\ast $ irreduzibel.
		Nun ist $ f $ primitiv. 
		Mit dem Reduktionskriterium folgt, dass $ f $ irreduzibel in $ \Z[x] $ ist.
		
		\item[\textbf{(3)}]
		Wir betrachten $ h = x^4 + 1 \in \Z[x]$.
		Aus LAAG ist bekannt, dass Nullstellen das Absolutglied teilen.
		Wegen $ h(\pm 1) = 2 $ besitzt $ h $ keine Nullstellen.
		Wir führen nun das \bi{Verfahren von Kronecker} exemplarisch durch.\index{Kroneckerverfahren}
		Angenommen $ h = h_1 \cdot h_2 $ mit $ \Grad(h_1) = 2  $ und $ \Grad(h_2) = 2 $.
		Sei $ h_1 = a \cdot x^2 + b \cdot x + c$.
		Da $ h $ normiert ist, können wir annehmen, dass $ h_1 $ und $ h_2 $ normiert sind.
		Somit folgt $ h_1 = x^2 + b \cdot x + c $.
		Für ein $ k \in \Z $ erhalten wir $ h(k) = h_1(k) \cdot h_2(k) $, womit $ h_1(k) \mid h(k) $ gelten muss. Nun gilt
		\begin{align*}
		k = 0 		  &\Rightarrow h(0) = 1 \Rightarrow h_1(0) = \pm 1 \\
		k = 1 		  &\Rightarrow h(1) = 2 \Rightarrow h_1(1) = \pm 1, \pm 2\\
					  &\Rightarrow c = \pm 1, \quad 1 + b + c = \pm 1 , \pm 2\\
		\bullet c = \ 1 &\Rightarrow b + 2 = \pm 1 , \pm 2 \Rightarrow b = -4,-3,-1, 0\\
		\bullet c =-1 &\Rightarrow b = \pm 1 , \pm 2 
		\end{align*}
		und für diese insgesamt 8 Möglichkeiten von $ h_1 $ müssen wir jetzt untersuchen, 
		ob diese Teiler von $ h $ sind.
		Jedoch gilt dies für keines, womit $ h $ irreduzibel ist.
		Alternativ wählen wir $ p = 2 $, dann ist $ x^2 + x + 1$ das einzige in $ \Z / 2 \Z[x] $ irreduzible
		Polynom.
		Wir nehmen an, dass $ h $ reduzibel ist und betrachten wieder $ h = h_1 \cdot h_2 $
		mit $ \Grad(h_1) = \Grad(h_2) = 2 $. Nach \ref{skript:9.7} ist $ \Z / 2 \Z[x] $ faktoriell, womit 
		$ h^\ast $ Produkt von irreduziblen Polynomen sein muss.
		Dies ist wegen
		\begin{align*}
		h^\ast = x^4 + \overline{1} \neq (x^2  + x + \overline{1})^2
		\end{align*}
		nicht erfüllt. Damit ist $ h$ irreduzibel.
		
		\item[\textbf{(4)}]
		Sei $ p $ eine Primzahl. Wir betrachten
		\begin{align*}
		f(x) = 1 + x + x^2 + \dots + x^{p-1}
		\end{align*}
		mit der Abbildung
		\begin{align*}
		\varphi : \Z[x] \to \Z[x], f(x) \mapsto f(x+1),
		\end{align*}
		welche nach \ref{skript:8.8} ein Ringhomomorphismus ist.
		Falls $ f = g \cdot h $ ist, folgt $ f(x+1) = g(x+1) \cdot h(x+1) $.
		Wenn also $ f(x+1) $ irreduzibel ist, so ist es auch $ f $.
		Nun gilt 
		\begin{align*}
		f(x+1) = 1 + (x+1) + (x+1)^2 + \dots + (x+1)^{p-1},
		\end{align*}
		womit wir aber nicht zufrieden sind und deswegen
		\begin{align*}
		x^p  - 1 = (x-1) \cdot (1 + x + \dots + x^{p-1}) = (x-1) \cdot f(x)
		\stackrel{\varphi}{\Rightarrow}
		(x+1)^p -1 = x \cdot f(x+1)
		\end{align*}
		betrachten. Weiter gilt
		\begin{align*}
		x \cdot f(x+1) &= x^p + \binom{p}{1} \cdot x^{p-1} + \dots + \binom{p}{p-1} \cdot x + 1 - 1 \\
				&= x \left( x^{p-1} + \binom{p}{1} \cdot x^{p-2} + \dots + \binom{p}{p-1} \right)\\
		\Rightarrow
		f(x+1) &= x^{p-1} + \binom{p}{1} \cdot x^{p-2} + \dots + \binom{p}{p-1} 
		\end{align*}
		und mit Eisenstein folgt, dass $ f(x+1) $ irreduzibel ist.
		Hierfür schaut euch nochmal den Beweis zu \ref{skript:8.7} an. 
		Damit ist auch $ f $ irreduzibel.
		
		\item[\textbf{(5)}]
		Sei $ p  $ eine Primzahl. Für
		\begin{align*}
		(p \cdot x + 1) \cdot (x-1) = p \cdot x^2 + (1-p) \cdot x - 1
		\end{align*}
		gilt über $ \Z / p \Z $, dass $ f^\ast  = \overline{x} - \overline{1} $ ist.
		Dies ist irreduzibel, jedoch $ f $ nicht.
		Dies zeigt, dass $ p \nmid a_n $ notwendig für $ \ref{skript:10.3} $ ist.
		Im weiteren Verlauf werden wir uns der folgende Frage beantworten:
		Falls $ f \in \Z[x] $ irreduzibel ist, gilt dies dann auch über $ \Q[x] $ ?
	\end{enumerate}
\end{genericdf}

\begin{lemma}\label{skript:10.6}
	Sei $ R $ ein Integritätsring und $ p \in R $ prim.
	Dann ist $ p $ prim in $ R[x] $.
\end{lemma}

\begin{proof}
	Mit \ref{skript:9.2} folgt aus $ p  $ prim, dass $ R/(p) $ ein Integritätsring ist.
	Dann ist auch $ R/(p)[x] $ ein Integritätsring und mit dem Homomorphiesatz für Ringe gilt
	$ R/(p)[x] \cong R[x]/(p) $.
	Also folgt mit \ref{skript:9.2}, dass $ p $ Primelement von $ R[x] $ ist.
\end{proof}

\begin{genericthm}{Lemma von Gauß}\label{skript:10.7}\index{Satz!Lemma von Gauß}
	Sei $ R $ faktoriell und $ 0 \neq a \in R[x] $ irreduzibel mit $ \Grad(a) \geq 1 $.
	Falls $ a $ teilt $ b \cdot c $ in $ R[x] $ mit $ 0 \neq b \in R[x] $ und $ c \in R[x] $ gilt,
	dann folgt $ a $ teilt $ c $ in $ R[x] $.
\end{genericthm}

\begin{proof}
	Sei $ l $ die Längenfunktion von $ R $.
	Für $ l(b) = 0 $ folgt $ b \in R^\ast $.
	In diesem Fall folgt die Aussage sofort.
	Sei also $ l(b) > 0$.
	Wir werden nun induktiv über diese Länge vorgehen.
	Sei $ 0 \neq p \in R $ irreduzibel mit $ p \mid b $.
	Dann gilt $ b = p \cdot b^\prime $ mit $ b^\prime \in R $ 
	und $ l(b^\prime) = l(b) -1 $.
	Nun gilt nach Voraussetzung $ b \cdot c = a \cdot d $ mit $ d \in R[x] $.
	Damit gilt $ p \mid a\cdot d  $ in $ R[x] $.
	Da $ R $ faktoriell ist, ist $ p $ auch prim.
	Nach \ref{skript:10.6} ist $ p $ prim in $ R[x] $,
	wodurch $ p \mid a $ oder $ p \mid d  $ folgt.
	Aufgrund der Irreduziblität von $ a $ folgt
	\begin{align*}
	p | d 
	\Rightarrow
	d = p \cdot d^\prime
	\end{align*}
	für eine $ d^\prime \in R[X] $.
	Also gilt
	\begin{align*}
	b \cdot c = p \cdot b^\prime \cdot c &= a \cdot d = a \cdot p \cdot d^\prime \\
	\Rightarrow b^\prime \cdot c &= a \cdot d^\prime
	\end{align*}
	und es folgt durch Induktion $ a \mid c $.
\end{proof}

\begin{df}\label{skript:10.8}
	Sei $R$ ein Integritätsring,
	dann können wir mit $R \subset K$ als Teilring
	und $K = \lbrace a \cdot b^{-1} \ | \ a \in R , b \in R \setminus \lbrace 0 \rbrace \rbrace$
	einen Körper konstruieren.
	Offensichtlich ist $K$ bis auf Isomorphie der kleinste Körper, der $R$ enthält. 
	Wir nennen $K$ den \bi{Quotientenkörper} von $R$ und schreiben $K = \Quot(R)$.
	\index{Quotientenkörper}
\end{df}

\begin{generic_no_num}{Bemerkung}
	Besonders leicht sehen wir dies, wenn $R$ bereits in $K$ enthalten ist.
	Wir betrachten zum Beispiel $\Z \subset \R  $, womit $\Quot(\Z) = \Q$ gilt.
\end{generic_no_num}

\begin{genericthm}{Satz von Gauß}\label{skript:10.9} \index{Satz!Gauß}
	Sei $R$ ein faktorieller Ring mit $\Quot(R) = K$.
	\begin{enumerate}
		\item[\textbf{(1)}]
		Sei $0 \neq f \in R[x]$ und $\Grad(f) \geq 1$.
		Falls $f$ irreduzibel $R[x]$ folgt,
		dass $f$ irreduzibel in $\Quot(R)[x]$ ist.
		
		\item[\textbf{(2)}]
		$R[x]$ ist ebenfalls faktoriell.
	\end{enumerate}
\end{genericthm}

\begin{proof}\
	\begin{enumerate}
		\item[\textbf{(1)}]		
		Sei $g \cdot h = f \in R[x]$ mit $g, h \in K[x]$ und $\Grad(g) \geq 1$.
		Die Koeffizenten von $g $ und $h$ sind von der Form
		\begin{align*}
		a \cdot b^{-1} = \frac{a}{b}
		\end{align*}				
		mit $a \in R$ und $b \in R \setminus \lbrace 0 \rbrace$.
		Sei $d$ das Produkt aller Nenner.
		Wir setzen $\tilde{g} = d \cdot g$ und $\tilde{h} = d \cdot h$.
		Damit folgt
		\begin{align*}
		\tilde{g}, \tilde{h} \in R[x]
		\Rightarrow
		d^2 \cdot f = \tilde{g} \cdot \tilde{h}, \quad \Grad(\tilde{g}) \geq 1,
		\end{align*}				
		womit $ \tilde{g} \notin R[x]^\ast$ ist.
		In \textbf{(2)} werden wir sehen, dass 
		\begin{align*}
		\tilde{g} = q_1 \cdot q_2 \cdots q_r
		\end{align*}
		mit $r \geq 1$ und $q_i$ irreduzibel gilt.
		Sei ohne Beschränkung der Allgemeinheit $\Grad(q_1) \geq 1$.
		Dann folgt mit \ref{skript:10.7}
		\begin{align*}
		q_1 \mid d^2 \cdot f 
		\Rightarrow
		q_1 | f
		\end{align*}
		und da $f$ irreduzibel ist gilt $f = \tilde{u} \cdot q_1$ mit $\tilde{u} \in R^\ast$.
		Es folgt
		\begin{align*}
		d^2 \cdot \tilde{u} = q_2 \cdots q_r \cdot \tilde{h},
		\end{align*}
		damit besitzt die linke Seite den Grad $0$, womit
		$\Grad(\tilde{h}) = \Grad(h) = 0$ folgt.
		Also gilt $h \in K[x]^\ast$.
		\item[\textbf{(2)}]	
		Zunächst haben wir zu zeigen:
		Falls $f \in R[x]$, dann ist $f$ eine Einheit oder ein Produkt aus irreduziblen Elementen
		aus $R[x]$.
		Wir setzen die Längenfunktion $l$ von $R$ auf $R[x]$ durch
		\begin{align*}
		l(g) := n + l(a_n)
		\end{align*}
		für $0 \neq g \in R[x]$ mit $\Grad(g) = n \geq 0$ und Leitkoeffizent $a_n \neq 0$ fort.
		Wegen $l(a\cdot b) = l(a)+l(b)$ für $a,b \in R$ 
		und $\Grad(g \cdot h) = \Grad(g) + \Grad(h)$ für $0 \neq g,h \in R[x]$ gelten
		\begin{itemize}
			\item
			$l(f) = 0 \Rightarrow f \in R^\ast$
			
			\item
			$l(g \cdot h) = l(g) + l(h) $
		\end{itemize}
		und damit folgt induktiv die Aussage.
		Nun müssen wir noch zeigen, dass prim gleich irreduzibel ist.
		Sei also $0 \neq f \in R[x]$ irreduzibel.
		Falls $Grad(f) = 0$ ist, dann liegt $f$ in $R$.
		Da $R$ faktoriell ist, ist $f$ prim in $R$ und damit auch $R[x]$.
		Wir betrachten also $\Grad(f) \geq 1 $ 
		mit $f \mid g \cdot h$ in $R[x]$.
		Dies gilt dann auch in $K[x]$.
		In \textbf{(1)} haben wir gezeigt, dass $f$ irreduzibel in $K[x]$ ist.
		Nach \ref{skript:9.7} ist $K[x]$ euklidisch, damit auch faktoriell.
		Also ist $f$ prim in $K[x]$.
		Damit gilt $f \mid g$ oder $f \mid h$ in $K[x]$.
		Wir nehmen an, dass $f \mid g$ gilt.
		Dann folgt $g = a \cdot f  $ mit $a \in K[x]$.
		Sei $d$ das Produkt der Nenner aller Koeffizenten von $a$. 
		Es folgt 
		\begin{align*}
		d \cdot g =   \underbrace{d \cdot a}_{\tilde{a}:=} \cdot f
		\end{align*}
		mit $\tilde{a} \in R[x]$.
		Es folgt weiter $f | \ d \cdot g $ in $R[x]$ und mit \ref{skript:10.7} 
		$f \mid g$ in R[x].
	\end{enumerate}		
\end{proof}

\begin{genericthm}{Folgerung}\label{skript:10.10}
	Sei $R$ faktoriell und $K = \Quot(R)$.
	Falls $0 \neq f \in R[x]$ irreduzibel nach \ref{skript:10.3}
	oder \ref{skript:10.4} ist, ist $f$ auch irreduzibel in $K[x]$.
\end{genericthm}

\begin{genericthm}{Folgerung}\label{skript:10.11}
	Sei $R$ faktoriell, dann ist der Polynomring
	in $n$ kommutierenden Unbestimmten
	$R[X_1,\dots,X_n]$
	auch faktoriell.
\end{genericthm}
\subsection{Aufgaben zu Abschnitt 10}

\begin{exe}\label{aufgabe:10.1}
	Bestimmen Sie ob diese Polynome irreduzibel in $ \Z[x] $ sind,
	andernfalls bestimmen Sie ihre irreduziblen Teiler:
	\begin{enumerate}
		\item[a)]
		$ x^4 -8 \cdot x^3 + 12 \cdot x^2 - 6\cdot x + 2 $
		
		\item[b)]
		$ x^5 - 12 \cdot x^3 + 36 \cdot x - 12 $
		
		\item[c)]
		$ x^3 - 13\cdot x - 2 $
		
		\item[d)] 
		$ x^4 + x + 1 $   
	\end{enumerate}
	\hyperlink{loes:10.1}{Lösung}
\end{exe}

\begin{exe}\label{aufgabe:10.2}
	Welche der folgenden Polynome sind irreduzibel?
	Geben Sie andernfalls eine Zerlegung an.
	\begin{enumerate}
		\item[a)]
		$ x^4 + x^2 +1 \in \F_2[x]  $
		
		\item[b)]
		$ x^5 + 6 \cdot x + 2 \in \Q[x] $
		
		\item[c)]
		$ x^6 + x^3 +1 \in \Z[x] $  
	\end{enumerate}
	\hyperlink{loes:10.2}{Lösung}
\end{exe}

\begin{exe}\label{aufgabe:10.3}
	\begin{enumerate}
		\item[a)]
		Seien $ a_1,\dots,a_2 \in \Z $ paarweise verschieden.
		Zeigen Sie, dass
		\begin{align*}
		\prod \limits_{i=1}^n (x-a_i) - 1
		\end{align*}
		irreduzibel in $ \Z[x] $ ist.
		
		\item[b)]
		Bestimmen Sie die Anzahl der irreduziblen Polynome vom Grad zwei
		in $ \F_p[x] $. 
	\end{enumerate}
	\hyperlink{loes:10.3}{Lösung}
\end{exe}

\begin{exe}\label{aufgabe:10.4}
	\begin{enumerate}
		\item[a)]
		Sei $ R $ ein Hauptidealbereich.
		Zeigen Sie:
		Sind $ a,b \in R $ koprim,
		dann gibt es $ c,d \in R $ mit $ 1 = a\cdot c + b \cdot d $.
		Ein Element $ x \in R $ ist koprim, falls
		\begin{align*}
		x \mid a \wedge x \mid b 
		\Rightarrow 
		x \in R^\ast
		\end{align*}
		gilt.
		
		\item[b)]
		Sei $ R $ ein Hauptidealbereich mit Quotientenkörper $ K $.
		und $ F $ ein Ring mit $ R \subseteq F \subseteq K $.
		Zeigen Sie:
		$ F $ ist ein Hauptidealbereich. 
	\end{enumerate}
	\hyperlink{loes:10.4}{Lösung}
\end{exe}

\newpage
\section{Kreisteilungspolynome}

\begin{df}\label{skript:11.1}
	Sei $K$ ein Körper, $n \geq 1$ und $f(x) = x^n -1 \in K[x]$.
	Wir nennen
	\begin{align*}
	E_n(K) := \lbrace \alpha \in K\ | \ f(\alpha) = 0 \rbrace 
	\end{align*}
	die Menge der \bi{$n$-ten Einheitswurzeln}.\index{n-te Einheitswurzel}
	Mit Multiplikation in $K$ bildet $E_n(K)$ eine endliche,zyklische Untergruppe von $K^\ast$.
	Die Endlichkeit folgt aus \ref{skript:8.9} und mit \ref{skript:8.10} ist $E_n(K)$ zyklisch.
	Wir definieren
	\begin{align*}
	E_n^\ast := \lbrace \alpha \in E_n \ | \ E_n = \langle \alpha \rangle \rbrace
	\end{align*}
	und nennen $\alpha \in E_n^\ast$ \bi{primitve} $n$-te Einheitswurzel.\index{n-te Einheitswurzel!primitiv}
	Nun definieren wir mit
	\begin{align*}
	\Phi_n(K) := \prod_{\alpha \in E_n^\ast} (x - \alpha) \in K[x]
	\end{align*}
	das \bi{$n$-te Kreisteilungspolynom} über $K$.\index{n-te Kreisteilungspolynom}		 
\end{df}

\begin{genericdf}{Beispiele}\label{skript:11.2}
	\begin{enumerate}
		\item[\textbf{(1)}]
		Sei $K = \Q$, dann gilt $E_n(\Q) = \lbrace \pm 1 \rbrace$ für alle geraden $n$.
		
		\item[\textbf{(2)}]
		Sei $K = \Z / p \Z$, dann gilt $E_p(K) = 1$.
		Dies folgt aus den kleinen Satz von Fermat gilt
		\begin{align*}
		a^p \equiv a \mod p 
		\end{align*}
		für alle $a \in \Z$. 
		
		\item[\textbf{(3)}]
		Sei $K = \C$, dann gilt
		\begin{align*}
		E_n(\C) = \left\lbrace \left( e^{\frac{2 \pi i}{n}} \right)^k \ | \ 1 \leq k \leq n \right\rbrace
		\end{align*}
		und $\zeta = e^{2 \pi i k}$ ist eine primitive $n$-te Einheitswurzel.
		Nun gilt
		\begin{align*}
		\zeta^k \ \text{primitiv} 
		\Leftrightarrow
		\ggT(k,n) = 1 
		\Leftrightarrow
		\Grad \Phi_n(\C) = \Phi(n),
		\end{align*}
		wobei $\Phi$ die Eulerfunktion ist.
		Für die Kreisteilungspolynome gilt:
		\begin{align*}
		\Phi_1 &= x -1 \\
		\Phi_2 &= x+1  \\
		\Phi_3 &= 1 + x + x^2 \\
		\Phi_4 &= x^2 + 1 \\
		\Phi_5 &= 1 + x + x^2 + x^3 +x^4 + x^5\\
		\Phi_6 &= x^2 - x +1
		\end{align*}
	\end{enumerate}
\end{genericdf}

\begin{sz}\label{skript:11.3}
	Sei $K = \C$ und $n \geq 1$.
	Dann gilt
	\begin{enumerate}
		\item[\textbf{(1)}]
		$n = \sum \limits_{d | n} \Phi(d)$.
		\item[\textbf{(2)}]
		$\Phi_n \in \Z[x]$ und $x^n - 1 = \prod_{d | n } \Phi_d$.
	\end{enumerate}
	Der Beweis ist eine Übungsaufgabe.
\end{sz}

\begin{sz}\label{skript:11.4}
	Sei $K = \C$.
	Dann ist $\Phi_n \in \Z[x]$ irreduzibel für alle $n \geq 1$.
\end{sz}

\begin{proof}
	Für den Beweis einfach in das Buch von Geck schauen.
\end{proof}

\subsection{Aufgaben zu Abschnitt 11}

\begin{exe}\label{aufgabe:11.1}
	Sei $ \Phi_n $ das $ n $-te Kreisteilungspolynom über $ \C $ und $ n \geq 1 $.
	Zeigen Sie:
	\begin{enumerate}
		\item[a)]
		$ n = \sum \limits_{d \mid n} \Phi(d)$
		
		\item[b)] 
		$\Phi_n \in \Z[x]$ und $x^n - 1 = \prod_{d | n } \Phi_d$
	\end{enumerate}
	\hyperlink{loes:11.1}{Lösung}
\end{exe}

\newpage
\section{Ganze algebraische Zahlen}

\begin{df}\label{skript:12.1}
	Wir nennen eine Zahl $z \in \C$ \bi{ganz algebraisch},\index{algebraisch!ganz}
	wenn $n \geq 1$ und\\
	$a_0, a_1, \dots , a_{n-1} \in \Z$ mit
	\begin{align*}
	z^n + a_{n-1} \cdot z^{n-1} + \dots +  a_1 \cdot z + a_0 = 0
	\end{align*}
	existieren.
	Wir nennen $z \in \C$ \bi{algebraisch}, wenn es eine Nullstelle von irgendeinem
	Polynom in $\Z[x]$ ist.\index{algebraisch}
\end{df}

\begin{genericdf}{Beispiele}\label{skript:12.2} \
	\begin{enumerate}
		\item[\textbf{(1)}]
		Einheitswurzeln sind ganze algebraische Zahlen.
		
		\item[\textbf{(2)}]
		Die $\sqrt{2}$ ist eine ganze algebraische Zahl.
		
		\item[\textbf{(3)}]
		Die $\nicefrac{1}{2}$ ist keine ganze algebraische Zahl.
	\end{enumerate}

\end{genericdf}

\begin{sz}\label{skript:12.3}
	Sei $\mathbb{A} $ die Menge der ganzen algebraischen Zahlen.
	Diese Menge ist ein Teilring von $\C$ und es gilt
	\begin{align*}
	\mathbb{A} \cap \Q = \Z.
	\end{align*}
\end{sz}

\chapter{Körpertheorie}
\setcounter{section}{12}
\section{Körpererweiterungen; algebraische und transzendente Elemente}

\begin{df}\label{skript:13.1}
	Sei $L$ ein Körper und $K$ ein Teilkörper von L.
	Dann nennen wir $L$ eine \bi{Körpererweiterung} von $K$\index{Körper!Erweiterung}
	und schreiben $L \supseteq K$.
	In der Literatur wird auch oft $L | K $ geschrieben.
	Wir nennen 
	\begin{align*}
	| L : K | := \dim_K L
	\end{align*}
	den \bi{Grad} bzw. \bi{Körpergrad} von $L$ über $K$.\index{Körper!Grad}
\end{df}

\begin{genericdf}{Beispiele}\label{skript:13.2}\
	\begin{enumerate}
		\item[\textbf{(1)}]
		Für $\C \supseteq \R$ gilt $| \C : \R | = 2$.
		
		\item[\textbf{(2)}]
		Für $\R \supseteq \Q$ gilt $|\R : \Q| = \infty$.
		
		\item[\textbf{(3)}]
		Sei $f \in K[x]$ irreduzibel.
		Dann ist nach \ref{skript:9.10}
		\begin{align*}
		\underbrace{K[x] / (f) }_{:= L} \supseteq K
		\end{align*}
		eine Körpererweiterung mit $| L : K | = \Grad(f)$.
	\end{enumerate}
\end{genericdf}

\begin{genericdf}{Primkörper}\label{skript:13.3}
	Sei $L$ ein Körper,
	dann bezeichnen wir mit $\langle 1 \rangle$ den kleinsten Teilkörper von $L$,
	der die $1$ enthält.
	Wenn $\Char L = p$ ist, gilt $\langle 1 \rangle \cong \Z / p \Z$
	und wenn $\Char L = 0 $ ist, gilt $\langle 1 \rangle \cong  \Q$.
	Wir nennen $\Z / p \Z$ bzw. $\Q$ \bi{Primkörper}.\index{Körper!prim}
	Falls $|L | < \infty$ ist, so gilt $\Char L = p$.
	Damit folgt $| L : \Z / p \Z | < \infty$ und $|L | = p^m$.
\end{genericdf}

\begin{df}\label{skript:13.4}
	Sei $L \supseteq K$ eine Körpererweiterung und $\alpha \in L$.
	Nach \ref{skript:8.8} existiert ein Einsetzungshomomorphismus
	\begin{align*}
	\sigma_\alpha : K[x] \to L, f \mapsto f(\alpha).
	\end{align*}
	Wir setzen $K[\alpha] = \lbrace f(\alpha) \ | \ f \in K[x] \rbrace = \Bild \sigma_\alpha$.
	Da $K[x]$ ein Hauptidealring ist, gilt
	$\Ker \sigma_\alpha = 0$ oder $\Ker \sigma_\alpha = (\mu_\alpha)$ für ein $0 \neq \mu_\alpha \in K[x]$.
	Falls $\Ker \sigma_\alpha  = 0 $ ist, nennen wir $\alpha$ \bi{transzendent über $K$}.
	\index{transzendent}
	Sollte $\Ker \sigma_\alpha = ( \mu_\alpha) $ sein,
	nennen wir $\alpha$ \bi{algebraisch über $K$}.
	\index{algebraisch}
	Sind alle $\alpha \in L$ algebraisch über $K$, nennen wir $L \supseteq K$ \bi{algebraisch}
	Wenn $\mu_\alpha$ normiert ist, nennen wir $\mu_\alpha $ \bi{Minimalpolynom}.
\end{df}

\begin{generic_no_num}{Bemerkung}
	$K[\alpha]$ ist ein Körper, also ein Teilkörper von $L$.
\end{generic_no_num}

\begin{proof}
	Mit dem Homomorphiesatz folgt $K[\alpha] \cong K[x] / (\mu_\alpha)$.
	Damit ist $K[x] / (\mu_\alpha)$ ein Integritätsring und mit \ref{skript:9.2}
	ist $\mu_\alpha$ irreduzibel.
	Nun ist $K[x]$ ein Hauptidealbereich, also folgt mit \ref{skript:9.9}
	die Körpereigenschaft.
\end{proof}

\begin{genericdf}{Beispiele}\label{skript:13.5}\
	\begin{enumerate}
		\item[\textbf{(1)}]
		Für $\C \supseteq \R$ und $f=x^2+1$ gilt $f(i)=0$. $f$ ist das Minimalpolynom von $\i$, da irreduzibel und normiert.
		\item[\textbf{(2)}]
		Für $\R \supseteq \Q$, $f=x^4-2$ und $\alpha=\sqrt[4]{2}\in\R^+$ gilt $f(\alpha)=0$. $f$ ist das Minimalpolynom von $\alpha$, da irreduzibel (nach dem Eisenstein-Kriterium, $p=2$) und normiert.
		\item[\textbf{(3)}]
		Für $\C \supseteq \R$ und $\alpha=\i+\sqrt{2}$:
		\[\alpha^2=1+2\i\sqrt{2}, \ \alpha^3=5\i-\sqrt{2}, \ \alpha^4=-7+4\i\sqrt{2}=2\alpha^2-9.\]
		Das Polynom $f=x^4-2x^2+9$ ist normiert. $f$ ist irreduzibel, da: In $\Z/2\Z[x]$ ist $f=x^4+1$ irreduzibel (vgl. \ref{skript:10.5} \textbf{(3)}). $f(\alpha)=0$, also ist $f=\mu_\alpha$.
	\end{enumerate}
\end{genericdf}

\begin{genericdf}{Lemma}\label{skript:13.6}\
	Ist $|L:K| <\infty$, dann ist die Körpererweiterung $L\supseteq K$ algebraisch.
\end{genericdf}

\begin{proof}
	Sei $\alpha\in L$, $n=|L:K|$. Dann sind $1, \alpha, \alpha^2, \ldots, \alpha^n$ linear abhängig über $K$. Somit existieren $a_i\in K$, sodass
	\[a_0+a_1\alpha+\ldots+a_n\alpha^n=0\]
	und mindestens ein $a_i\neq0$. D. h. es gibt ein Polynom $f\in K[x]$, $f\neq0$ mit $f(\alpha)=0$, also ist $\Ker \sigma_\alpha\neq0$ und $\sigma_\alpha$ nicht injektiv. Also ist $\alpha$ algebraisch.
\end{proof}

\begin{genericdf}{Beispiel}\label{skript:13.7}\
	Sei $|L:K|=2$ und $\Char K\neq2$. Dann ist $L\supseteq K$ algebraisch und es existiert ein $\beta\in L$ mit $\beta^2\in K$ und $\beta\notin K$, denn:\\
	Es gibt ein $\alpha\in L$, sodass $\{1,\alpha\}$ eine $K$-Basis von $L$ ist. D. h. $L=K[\alpha]$ und
	\[\alpha^2=a\alpha+b\]
	mit $a,b\in K$. Da $\Char K\neq2$ ist, gibt es ein $c\in K$ mit $a=2c$. Also folgt
	\[\alpha^2=2c\alpha+b\]
	und daraus
	\[(\alpha-c)^2=b+c^2.\]
	Setzt man $\beta:=\alpha-c$, folgt: $\{1,\beta\}$ ist $K$-Basis von $L$, $L=K[\beta]$, $\beta^2\in K$, $\beta\notin K$.
\end{genericdf}

\begin{genericdf}{Bemerkung (Existenz transzendenter Zahlen über $\Q$)}\label{skript:13.8}\
	Die Menge der Nullstellen der Polynome $f\in\Q[x]$ mit $\Grad f\geq 1$ ist abzählbar: Sei
	\[P_n=\{f\in\Q[x]|\Grad f=n\geq1\}.\]
	Dann lässt sich jedes Polynom $f\in P_n$ über seine Koeffizienten $a_0,\ldots,a_n$ genau einem Element aus $\{(x_0,\ldots,x_n)\in\Q^{n+1}|x_n\neq0\}$ zuordnen und umgekehrt. Diese Menge ist abzählbar. Damit ist auch die Menge
	\[\bigcup_{n\geq1}P_n\]
	der nicht konstanten Polynome von $\Q[x]$ abzählbar und somit auch die Menge ihrer Nullstellen. Aus der Analysis ist bekannt, dass $\R$ überabzählbar ist. Es gibt also unendlich viele $\alpha\in\R$, die über $\Q$ transzendent sind.\\
	(Hermite 1873: $e$ ist transzendent, Lindemann 1882: $\pi$ ist transzendent)
\end{genericdf}

\begin{genericdf}{Satz (Gradformel)}\label{skript:13.9}\
	Seien $L\supseteq M$ und $M\supseteq K$ Körpererweiterungen mit $|L:M|<\infty$ und $|M:K|<\infty$. Dann ist auch $|L:K|<\infty$ und es gilt
	\[|L:K|=|L:M|\cdot|M:K|.\]
\end{genericdf}

\begin{proof}
	Seien $B_1=\{x_i|1\leq i\leq m\}$ $K$-Basis von $M$ und $B_2=\{y_j|1\leq j\leq n\}$ $M$-Basis von $L$. Dann ist $B=\{x_iy_j|x_i\in B_1, y_j\in B_2\}$ eine $K$-Basis von $L$ mit $|B|=n\cdot m$, denn:\\
	$B$ ist Erzeugendensystem: Sei $z\in L$. Dann gilt
	\[z=\sum_{j=1}^n m_jy_j, \ m_j\in M, \ m_j=\sum_{i=1}^m k_{ij}x_i, \ k_{ij}\in K.\]
	Daraus folgt
	\[z=\sum_{i=1}^m\sum_{j=1}^n k_{ij}x_iy_j.\]
	$B$ ist linear unabhängig: Seien $k_{ij}\in K$ und
	\[\sum_{i=1}^m\sum_{j=1}^n k_{ij}x_iy_j=0.\]
	Da $B_2$ linear unabhängig ist, folgt für $1\leq j\leq n$
	\[\sum_{i=1}^m k_{ij}x_i=0\]
	und aus der linearen Unabhängigkeit von $B_1$ schließlich $k_{ij}=0$ für $1\leq i\leq m$, $1\leq j\leq n$.
\end{proof}

\begin{genericdf}{Definition, Bemerkung, Beispiele}\label{skript:13.10}\
	Für eine Teilmenge $S$ von $L$ bezeichnet $K(S)$ den kleinsten Teilkörper von $L$, der $K\cup S$ enthält, d. h. den Schnitt aller Teilkörper von $L$, die $K\cup S$ enthalten. Ist $S=\{\alpha_1,\ldots,\alpha_n\}$, dann schreibt man $K(S)=:K(\alpha_1,\ldots,\alpha_n)$.
	\begin{enumerate}
		\item[\textbf{(1)}]
		Ist $\alpha\in L$ transzendent, dann ist
		\[K(\alpha)=\left\{\frac{f(\alpha)}{g(\alpha)}|f,g\in K[x], g\neq0\right\}.\]
		\item[\textbf{(2)}]
		Ist $\alpha\in L$ algebraisch, dann ist $K[\alpha]=K(\alpha)$. Es folgt:
		\[|K(\alpha):K|=\dim K[x]/(\mu_\alpha)=\Grad \mu_\alpha\]
		\item[\textbf{(3)}]
		Sind $\alpha_1,\ldots,\alpha_r\in L$ algebraisch über $K$, dann gilt
		\[K(\alpha_1)=K[\alpha_1], \ K(\alpha_1,\alpha_2)=K(\alpha_1)[\alpha_2], \ \ldots, \ K(\alpha_1,\ldots,\alpha_r)=K(\alpha_1,\ldots,\alpha_{r-1})[\alpha_r].\]
		Mit \ref{skript:13.9} folgt
		\[K(\alpha_1,\ldots,\alpha_r):K|<\infty\]
		und daraus nach \ref{skript:13.6}, dass $K(\alpha_1,\ldots,\alpha_r)$ algebraisch ist.
	\end{enumerate}
\end{genericdf}

\begin{genericdf}{Folgerung}\label{skript:13.11}\
	Seien $L\supseteq M$ und $M\supseteq K$ algebraische Körpererweiterungen. Dann ist auch $L\supseteq K$ algebraisch.
\end{genericdf}

\begin{proof}
	Sei $\alpha\in L$ und $\mu_\alpha=\beta_0+\beta_1 x+\ldots+\beta_m x^m$ das Minimalpolynom von $\alpha$ über $M$ mit $\beta_i\in M$ für $1\leq i\leq m$. Da nach Voraussetzung $M\supseteq K$ algebraisch ist, sind auch $\beta_0,\ldots,\beta_m$ algebraisch über $K$. Setze $K'=K(\beta_0,\ldots,\beta_m)$. Nach \ref{skript:13.10} \textbf{(3)} ist der Erweiterungsgrad $|K':K|<\infty$. Da $\mu_\alpha$ ein Polynom mit Koeffizienten aus $K'$ ist, ist $\alpha$ algebraisch über $K'$. D. h. es gilt $|K'(\alpha):K'|<\infty$. Mit der Gradformel folgt nun
	\[|K'(\alpha):K|=|K'(\alpha):K'|\cdot|K':K|<\infty\]
	und somit ist $\alpha$ algebraisch über $K$.
\end{proof}

\begin{genericdf}{Folgerung}\label{skript:13.12}\
	Sei $L\supseteq K$ beliebige Körpererweiterung. Definiere
	\[M:=\{\alpha\in L|\alpha\text{ ist algebraisch über }K\}.\]
	Dann ist $M$ ein Teilkörper von $L$.
\end{genericdf}

\begin{proof}
	Seien $\alpha,\beta\in M\backslash\{0\}$. Dann sind $\alpha\pm\beta,\alpha\beta,\alpha\beta^{-1}\in K(\alpha,\beta)$. Nach \ref{skript:13.10} \textbf{(3)} sind $\alpha\pm\beta,\alpha\beta,\alpha\beta^{-1}$ algebraisch über $K$ und somit Elemente von $M$. Also ist $M$ ein Teilkörper von $L$.
\end{proof}

\begin{genericdf}{Beispiel}\label{skript:13.13}\
	Seien $L=\Q(\sqrt{2},\i)$, $M=\Q(\sqrt{2})$. $x^2+1\in M[x]$ ist das Minimalpolynom von $\i$ über $M$.
	\[\underbrace{|L:M|}_{2}\cdot\underbrace{|M:\Q|}_{2}=\underbrace{|L:\Q|}_{4}\]
	$\mu_{\i+\sqrt{2}}$ besitzt den Grad 4. Also ist $\Q(\sqrt{2},\i)=\Q(\i+\sqrt{2})$.
	\begin{center}
	\begin{tikzpicture}
		\node at (0,0) {$\Q$};
		\node at (-1,1) {$\Q(\sqrt{2})$};
		\node at (1,1) {$\Q(\i)$};
		\node at (0,2) {$\Q(\sqrt{2},\i)$};
		\draw (-0.1,0.25) -- (-1,0.75);
		\draw (0.1,0.25) -- (1,0.75);
		\draw (-1,1.25) -- (-0.1,1.75);
		\draw (1,1.25) -- (0.1,1.75);
	\end{tikzpicture}
	\end{center}
	Frage: Gibt es noch weitere Teilkörper von $\Q(\sqrt{2},\i)$?
\end{genericdf}

\begin{df}\label{skript:13.14}
	Seien $L\supseteq K$, $f\in K[x]$, $f\neq0$ und $\Grad f\geq 1$. Man sagt, dass $f$ über $L$ in Linearfaktoren zerfällt, falls
	\[f=c\cdot(x-\alpha_1)\cdot(x-\alpha_2)\cdot\ldots\cdot(x-\alpha_n)\]
	mit $c\in K$ und $\alpha_1,\ldots,\alpha_n\in L$.\\
	Beachte: Jede Wurzel $\alpha_i$ ist algebraisch über $K$, denn $f(\alpha_i)=0$. Gilt zusätzlich, dass
	\[L=K(\alpha_1,\ldots,\alpha_n),\]
	dann nennt man $L$ einen Zerfällungskörper von $K$. Wegen $|L:K|<\infty$ (vgl. \ref{skript:13.10}) ist dann der Zerfällungskörper $L\supseteq K$ algebraisch. Gibt es $i\neq j$ mit $\alpha_i=\alpha_j$, dann nennt man $\alpha_i$ eine mehrfache Nullstelle von $f$.
\end{df}

\begin{sz}\label{skript:13.15}\
	Sei $K$ ein Körper, $f\in K[x]$ mit $\Grad f=n\geq1$. Dann existiert ein Zerfällungskörper $L\supseteq K$ von $f$ und es gilt:
	\[|L:K|\leq n!\]
\end{sz}

\begin{proof}
	Induktion nach $n$:\\
	Falls $n=1$, setze $L=K$.\\
	Sei $n>1$, $f_1\in K[x]$ irreduzibel und $f_1$ teile $f$. Nach \ref{skript:13.2} ist $K_1=K[x]/(f_1)\supseteq K$ eine Körpererweiterung mit
	\[|K_1:K|=\Grad f_1\leq n.\]
	Ferner gilt $K_1=K(\alpha_1)$ (vgl. \ref{skript:13.10} \textbf{(2)}), wobei $\alpha_1$ eine Nullstelle von $f_1$ ist. Es folgt
	\[f=(x-\alpha_1)\cdot g\text{ in }K_1[x],\]
	also ist auch $g\in K_1[x]$ und $\Grad g=n-1$. Nach Induktionsvoraussetzung existiert ein Zerfällungskörper $L\supseteq K_1$ von $g$ mit
	\[|L:K_1|\leq(n-1)!\]
	Seien $\alpha_2,\ldots,\alpha_n$ die Nullstellen von $g$. Es gilt $L=K_1(\alpha_2,\ldots,\alpha_n)$, also auch $L=K(\alpha_1,\ldots,\alpha_n)$. Somit ist $L$ Zerfällungskörper von $f$ und es gilt:
	\[|L:K|=|L:K_1|\cdot|K_1:K|\leq (n-1)!\cdot n=n!\]
\end{proof}

\begin{genericdf}{Beispiel}\label{skript:13.16}\
	$f=x^4-2\in\Q[x]$. Gesucht ist der Zerfällungskörper von $f$ über $\Q$. $\alpha=\sqrt[4]{2}$ ist eine Nullstelle von $f$. Die Linearfaktorzerlegung von $f$ ist
	\[f(x)=(x-\alpha)(x+\alpha)(x-\i\alpha)(x+\i\alpha).\]
	Dann ist
	\[L=\Q(\alpha,-\alpha,\i\alpha,-\i\alpha)=\Q(\alpha,\i\alpha)=\Q(\alpha,\i)\]
	ein Zerfällungskörper von $f$. Es gilt $|\Q(\alpha):\Q|=4$ (vgl. \ref{skript:13.5} \textbf{(2)}) und $\i\notin\Q(\alpha)$. Das Minimalpolynom von $\i$ über $\Q(\alpha)$ ist $\mu_\i=x^2+1$. Mit der Gradformel folgt schließlich:
	\[|L:\Q|=|L:\Q(\alpha)|\cdot|\Q(\alpha):\Q|=2\cdot4=8\]
\end{genericdf}

\begin{df}\label{skript:13.17}
	Ein Körper $K$ heißt algebraisch abgeschlossen, wenn jedes Polynom $f\in K[x]$ mit $\Grad f\geq 1$ über $K$ in Linearfaktoren zerfällt. $L\supseteq K$ heißt algebraischer Abschluss von $K$, wenn $L$ algebraisch abgeschlossen ist und $L\supseteq K$ algebraisch ist.
\end{df}

\begin{genericdf}{Bemerkung und Beispiele}\label{skript:13.18}\
	\begin{enumerate}
		\item[\textbf{(1)}]
		Aus der komplexen Analysis ist bekannt, dass $\C$ algebraisch abgeschlossen ist.
		\item[\textbf{(2)}]
		Ein algebraischer Abschluss von $\Q$ ist gegeben durch
		\[\overline{\Q}:=\{\alpha\in\C|\alpha\text{ ist algebraisch über }\Q\}.\]
		$\overline{\Q}$ ist nach Folgerung \ref{skript:13.12} ein Teilkörper von $\C$ (vgl. §12). Ist $f\in\overline{\Q}[x]$ mit $\Grad f=n\geq 1$, dann ist
		\[f=c\cdot\prod_{i=1}^n (x-\alpha_i)\]
		mit $c\in\overline{\Q}$, $\alpha_i\in\C$ für $1\leq i\leq n$, da $\C$ algebraisch abgeschlossen ist und
		\[f=\sum_{i=0}^n\tilde{\alpha}_ix^i\]
		mit $\tilde{\alpha}_i\in\overline{\Q}$. Setze $L=\Q(\tilde{\alpha}_0,\ldots,\tilde{\alpha}_n)$. Aus $\tilde{\alpha}_i\in\overline{\Q}$ folgt, dass $\tilde{\alpha}_i$ algebraisch über $\Q$ sind. Mit \ref{skript:13.10} \textbf{(3)} ist auch $L$ algebraisch über $\Q$. Es gilt $f(\alpha_i)=0$, also sind die $\alpha_i$ algebraisch über $L$ und $L(\alpha_i)$ ist algebraisch über $L$. Nach \ref{skript:13.11} ist nun $L(\alpha_i)$ algebraisch über $\Q$ und damit auch $\alpha_i$. Also folgt $\alpha_i\in\overline{\Q}$. D. h. $f$ zerfällt über $\overline{\Q}$ in Linearfaktoren, also ist $\Q$ algebraisch abgeschlossen.
		\item[\textbf{(3)}]
		Allgemein gilt: Jeder Körper $K$ hat einen algebraischen Abschluss. (Der Beweis verwendet das Lemma von Zorn, ist also nicht konstruktiv).
	\end{enumerate}
\end{genericdf}
\subsection{Aufgaben zu Abschnitt 13}

\begin{exe}\label{aufgabe:13.1}
	\begin{enumerate}
		\item[a)]
		Sei $ K $ ein Teilkörper des Körpers $ L $.
		Zeigen Sie, dass dann $ L $ zu einem $ K $- Vektorraum wird,
		indem die Skalarmultiplikation durch die Multiplikation in $ L $
		definiert wird.
		
		\item[b)]
		Konstruieren Sie Körper mit $ 4$, $ 8 $ und $ 16 $ Elementen
		mit Hilfe geeigneter irreduzibler Polynome von $ \Z / 2 \Z $.
		
		\item[c)]
		Zeigen Sie, dass ein endlicher Körper Primzahlpotenzordnung besitzt.   
	\end{enumerate}
	\hyperlink{loes:13.1}{Lösung}
\end{exe}

\newpage
\section{Körpererweiterungen; Körper-Automorphismen}

\begin{genericdf}{Vorbemerkung}\label{skript:14.1}
	Seien $K,K'$ Körper,  $\sigma:K \to K'$ ein Ringhomomorphismus. Dann ist $\sigma$ injektiv.
	\begin{proof}
	Es gilt $\Ker\sigma\nt K$. Da $K$ ein Körper ist, gibt es nur zwei Möglichkeiten: Entweder ist $\Ker\sigma=0$ oder $\Ker\sigma=K$. Da $\sigma(1)=1$, kann $\Ker(	\sigma)=K$ nicht sein. Also muss $\Ker\sigma=0$ gelten und damit $\sigma$ injektiv sein.
	\end{proof}
	Ein Ringhomomorphismus zwischen Körpern heißt auch \bi{Körperhomomorphismus}. Ist $\sigma: K \to K'$ ein Körperhomomorphismus, dann kann man $\sigma$ zu dem Ringhomomorphismus 
	\[\hat{\sigma}: K[x] \to K'[x]: f=\sum_{i=1}^n a_i x^{i} \mapsto \hat{\sigma}(f)=\sum_{i=1}^n \sigma (a_i) x^{i}\]
	erweitern. Es gilt dann
	\[f=\prod_{i=1}^n (x-z_i) \Rightarrow \hat{\sigma}(f)=\prod_{i=1}^n (x-\sigma(z_i))\]
	und
	\[\sigma(f(z))= \hat{\sigma}(f)(\sigma(z))\]
	für alle $f \in K[x]$ und $z \in K$.
\end{genericdf}

\begin{df}\label{skript:14.2} 
	Sei $L\supseteq K$ eine Körpererweiterung. Wir setzen
	\[\Aut(L,K):=\{ \sigma \in \Aut(L) \ | \  \forall k \in K: \ \sigma(k)=k \}.\]
	$\Aut(L,K)$ ist mit der Komposition als Verknüpfung eine Gruppe. Durch
	\[\sigma.y = \sigma(y)\]
	für $\sigma \in \Aut(L,K)$ und $y \in L$ operiert $\Aut(L,K)$ auf $L$.
\end{df}

\begin{lemma}\label{skript:14.3} 
	Seien $\sigma \in G:=\Aut (L,K)$, $f \in K[x]$, $z \in L$ mit $ f(z)=0$. Dann folgt $f(\sigma(z))=0$, d. h. $\sigma$ operiert auf den Nullstellen von $f$. Ist $|L:K| < \infty$, dann gilt auch $|\Aut (L,K)| < \infty$.
\end{lemma}
\begin{proof}
	Sei
	\[f=\sum_{i=1}^n a_iz^i\]
	mit $a_i\in K$. Dann gilt $\sigma(a_i)=a_i$, da $\sigma\in\Aut(L,K)$. Daraus folgt:
	\[0=\sigma(0)=\sigma(f(z))=\sigma(\sum_{i=1}^n a_iz^i)=\sum_{i=1}^n \sigma(a_i) \sigma(z)^i=\sum_{i=1}^n a_i \sigma(z)^i \Rightarrow f(\sigma(z))=0.\]
	Betrachte nun $L$ als $K$-Vektorraum. Sei $\sigma \in G$, dann ist $\sigma$ $K$-linear, denn
	\[\sigma(kx)=\sigma(k) \sigma(x)=k \sigma(x)\]
	und
	\[\sigma(x+y)=\sigma(x)+\sigma(y).\]
	Sei $B=\{b_1,...,b_m\}$ eine $K$-Basis von $L$. Da $|L:K|< \infty$, ist jedes $b_i$ algebraisch über $K$. Somit hat jedes $b_i$ ein Minimalpolynom $f_i \in K[x]$ mit $f_i(b_i)=0$, also nach dem ersten Teil auch $f_i(\sigma(b_i))=0$. Da $f_i$ nur endlich viele Nullstellen hat, gibt es nur endlich viele Möglichkeiten für $\sigma(b_i)$. Daraus folgt, dass $G$ endlich ist, denn $\sigma$ ist als $K$-lineare Abbildung durch die Bilder der Basis $B$ festgelegt.
\end{proof}

\begin{genericdf}{Fortsetzungslemma}\label{skript:14.4}
	Seien $K,K'$ Körper, $\sigma: K \to K'$ ein Körperisomorphismus, $f \in K[x]$ irreduzibel, $L\supseteq K$ eine Körpererweiterung und $z \in L $ mit $f(z)=0$. Setze
	\[f'=\hat{\sigma}(f)\]
	(vgl. \ref{skript:14.1}). Sei nun $z'\in L'\supseteq K'$ mit $f'(z')=0$. Dann gibt es genau einen Körperisomorphismus
	\[\sigma_1: K(z)\to K'(z')\]
	mit $z\mapsto z'$ und $\forall k\in K: \sigma_1(k)=\sigma(k)$.
\end{genericdf}
\begin{proof}
	Verwende hierzu den Homomorphiesatz (\ref{skript:8.5} \textbf{(2)})
	\begin{center}
	\begin{tikzpicture}
		\node at (0,2) {$(f)$};
		\node at (0,0) {$K[x]$};
		\node at (0,-2) {$K[x]/(f)$};
		\node at (1.5,-2) {$\cong K(z)$};
		\node at (5,0) {$K'[x]$};
		\node at (5,-2) {$K(z')$};
		\draw [right hook-to] (0,1.75) -- (0,0.25) node [midway,right] {$\iota$};
		\draw [->] (0,-0.25) -- (0,-1.75) node [midway,right] {$\kappa$};
		\draw [->] (0.5,0) -- (4.5,0) node [midway,above] {$\hat{\sigma}$};
		\draw [->] (5,-0.25) -- (5,-1.75) node [midway,right] {$\kappa'$};
		\draw [dashed,->] (2.25,-2) -- (4.5,-2) node [midway,above] {$\exists!\sigma_1$};
	\end{tikzpicture}
	\end{center}
	mit $\sigma_1\circ\kappa=\kappa'\circ\hat{\sigma}$. $\sigma_1$ ist ein Körperisomorphismus. Für das Polynom $x$ gilt:
	\[\kappa(x)=z,\ \kappa'(x)=z'.\]
	Daraus folgt
	\[\sigma_1(z)=(\sigma_1\circ\kappa)(x)=(\kappa'\circ\hat{\sigma})(x)=\kappa'(x)=z'.\]
	Analog sieht man für $k\in K$:
	\[\sigma_1(k)=(\sigma_1\circ\kappa)(k)=(\kappa'\circ\hat{\sigma})(k)=\kappa'(\sigma(k))=\sigma(k).\]
\end{proof}

\begin{genericdf}{Beispiele}\label{skript:14.5}
	\begin{itemize}
		\item[\textbf{(1)}]
		Primkörper haben nur triviale Automorphismen:\\
		Sei $K$ ein Körper mit $K\supseteq\Q$, dann ist $\Aut (K, \Q)=\Aut (K)$.
		Sei $p$ eine Primzahl, $K$ ein Körper mit $\Char K=p$ und $K\supseteq\F_p=\Z/p\Z$, dann ist $\Aut(K,\F_p)=\Aut(K)$.
		\item[\textbf{(2)}]
		Sei $L_n = \Q (\zeta_n)$ mit $\zeta_n = e^{\frac{2 \pi \i}{n}}$. $L_n$ ist Zerfallungskörper zu $\Phi_n$, da alle Nullstellen von $\Phi_n$ Potenzen von $\zeta_n$ sind. Es gilt
		\[|L_n : \Q | = \phi(n).\]
		Sei nun $\sigma \in \Aut(L_n, \Q)$, dann ist $\sigma$ festgelegt durch $\sigma (\zeta_n)=\zeta_n^k$ mit $\ggT(n,k)=1$. Daraus folgt, dass
		\[|\Aut (L_n, \Q)| \leq \phi (n).\]
		Nach Lemma 14.4 gilt nun mit
		\[L'=L=L_n,\ z=\zeta_n,\ z'= \zeta_n^{k}\ (\ggT(n,k)=1),\ f=\Phi_n=f',\ K'=K=\Q,\]
		dass genau ein $\sigma_k: L_n\to L_n$ existiert mit $\zeta_n\mapsto\zeta_n^k$. Es gilt also
		\[|\Aut (L_n, \Q)| = \phi (n).\]
		Ferner ist die Abbildung
		\[\tau: \Aut(L_n,\Q)\to(\Z/n\Z)^\ast: \sigma_k\mapsto\bar{k}\]
		offensichtlich ein Gruppenisomorphismus.
		\[\Aut(L_n)=\Aut(L_n,\Q)\cong(\Z/n\Z)^\ast\]
	\end{itemize}
\end{genericdf}

\begin{genericdf}{Fortsetzungssatz für Körperisomorphismen}\label{skript:14.6}\
	Seien $K,K'$ Körper und $\sigma: K \to K'$ ein Körperisomorphismus. $L\supseteq K$ sei Zerfällungskörper von $K$ bezüglich $f \in K[x]$ und $L'\supseteq K'$ ein Zerfällungskörper von $K'$ bezüglich $\hat{\sigma}(f)=f' \in K'[x]$. Dann gelten:
	\begin{itemize}
		\item[\textbf{(1)}]
		Es existiert ein Körperisomorphismus $\tau: L\to L'$ mit $\tau\big|_K=\sigma$, d. h. $\tau$ setzt $\sigma$ fort.
		\item[\textbf{(2)}]
		Hat $f$ in $L$ keine mehrfachen Nullstellen, dann existieren mindestens $|L:K|$ solche Fortsetzungen $\tau: L\to L'$.
	\end{itemize}
\end{genericdf}
\begin{proof}
Beweis durch Induktion nach $n=|L:K|$:\\
Falls $n=1$ ist, $L=K$ und die Aussage ist trivial.\\
Sei nun $n>1$. Dann gibt es ein $g \in K[x]$ irreduzibel mit $\Grad(g) \geq 2 $ und $g$ teilt $f$. Sei
\[g' = \hat{\sigma} (g) \in K'[x].\]
Dann teilt auch $g'$ $f'$, da $\hat{\sigma}$ ein Ringhomomorphismus ist. $f$ zerfällt über $L$ in Linearfaktoren, somit auch $g$. Analog gilt dies für $f',g'$ über $L'$. Seien nun $\{z_1,\ldots,z_d\}$ die Nullstellen von $g$ in $L$. Dann ist $d \leq \Grad(g)$. Sei $z' \in L'$ eine Nullstelle von $g'$. Nach \ref{skript:14.4} existiert nun für $1 \leq i \leq d$ ein Isomorphismus
\[\sigma_i : K(z_i) \to K'(z')\]
mit $\sigma_i (z_i) =z'$ und $\sigma_i\big|_K=\sigma$. Setze $K_i:=K(z_i)$, $K_1'=K'(z')$. Es gilt $f \in K_i[x]$, da $K_i\supseteq K$ ist, sowie $f' \in K_1'[x]$. Nach dem Gradsatz \ref{skript:13.9} gilt:
\[|L:K|= |L:K_i|\cdot\underbrace{|K_i:K|}_{\geq2} > |L:K_i|.\]
Nach der Induktionsvoraussetzung, wobei $L$ und $L'$ die Zerfällungskörper bleiben, existiert ein $\tau: L\to L'$ mit
\[\tau\big|_{K_i}=\sigma_i.\]
Also gilt auch
\[\tau\big|_K=\sigma_i\big|_K=\sigma,\]
woraus \textbf{(1)} folgt.
Es fehlt noch zu zeigen, dass, wenn $f$ keine mehrfachen Nullstellen in $L$ hat, mindestens $|L:K|$ viele solche Fortsetzungen existieren. Nach Induktion gibt es mindestens $|L:K_i|$ Isomorphismen $L\to L'$, die $\sigma_i$ fortsetzen. Da es nun $d$ solche $\sigma_i$ gibt, existieren mindestens $d\cdot|L:K_i|$ Fortsetzungen.
Da nach Voraussetzungen $d = \Grad (g)$ ist, folgt
\[|L:K|=\underbrace{|K_i:K|}_{=d}\cdot|L:K_i|\]
und es gibt somit mindestens $|L:K|$ Fortsetzungen.
\end{proof}

\begin{df}\label{skript:14.7}
	Sei $f\in\Q[x]$, $f\neq0$ und $n=\Grad f\geq1$. Sei $L$ der Zerfällungskörper von $f$ über $\C$. Dann wird die Gruppe
	\[\Aut(L,\Q)=\Aut(L)\]
	die \bi{Galoisgruppe} von $f$ genannt. Schreibweise: $\Gal(f,\Q)$.
\end{df}

\begin{sz}\label{skript:14.8}
	Sei $L\supseteq K$ Zerfällungskörper von $f\in K[x]$, $f\neq0$. Dann operiert $G=\Aut(L,K)$ auf der Menge
	\[X:=\{z\in L|f(z)=0\}\]
	und der Homomorphismus $\rho: G\to\Sym(X)$ ist injektiv. $G$ ist also isomorph zu einer Untergruppe von $S_n$ mit $n=\Grad f$. Ist $f$ irreduzibel, dann operiert $G$ transitiv auf $X$, d. h. $X$ besteht nur aus einer $G$-Bahn.
\end{sz}
\begin{proof}
	Sei $X=\{z_1,\ldots,z_m\}$. Beachte hierbei $m\leq n$. Nach früheren Rechnungen operiert $G$ auf $X$ (vgl. \ref{skript:14.3} \textbf{(1)}). Nach §3 erhalten wir einen Gruppenhomomorphismus
	\[\rho: G\to\Sym(X)\cong S_m\]
	mit
	\[\Ker\rho=\{\sigma\in G\ |\ \forall i\in\{1,\ldots,m\}: \sigma(z_i)=z_i\}.\]
	Da als Zerfällungskörper $L=K(z_1,\ldots,z_m)$, folgt aus $\sigma(z_i)=z_i$ für $1\leq i\leq m$, dass $\sigma=\id$ ist. Somit folgt $\Ker\rho=0$ und damit, dass $\rho$ injektiv ist.\\
	Ist $f$ irreduzibel, dann existiert zu jedem $i$ nach Lemma $\ref{skript:14.4}$ ein Isomorphismus
	\[\tau_i: K(z_i)\to K(z_1)\]
	mit $\tau_i(z_i)=z_1$ und $\tau_i\big|_K=\id$.
	Nach \ref{skript:14.6} setzt sich $\tau_i$ fort zu einem Automorphismus $\sigma_i\in\Aut(L,K)$. Alle Nullstellen $z_i$ liegen also in einer Bahn der Operation, d. h. $G$ operiert transitiv.
\end{proof}

\begin{genericdf}{Beispiel}\label{skript:14.9}
	Sei $L=\Q(\alpha)$ mit $\alpha=\sqrt[4]{2}\in\R^+$. Betrachte zunächst $\Aut(L,\Q)$.
	\[f=x^4-2\]
	ist irreduzibel mit den Nullstellen $\alpha,-\alpha,\i\alpha,-\i\alpha$. Da $L\subseteq\R$, kann ein $\sigma\in\Aut(L,\Q)$ die Nullstelle $\alpha$ nur auf $\pm\alpha$ abbilden. Nach \ref{skript:14.4} existiert genau ein $\sigma: L\to L$ mit $\alpha\mapsto-\alpha$. Zusammen mit der Identität ergibt dies
	\[\Aut(L,\Q)\cong C_2.\]
	Es gibt auch einen Körperisomorphismus $\sigma_2: L\to\Q(\i\alpha)$ mit $\alpha\mapsto\i\alpha$ (ebenfalls nach \ref{skript:14.4} mit $\sigma:\Q\to\Q, f'=f,z=\alpha,z'=\i\alpha$). Der Körper
	\[\tilde{L}:=\Q(\alpha,\i\alpha)\]
	ist Zerfällungskörper von $f$. Nach \ref{skript:13.16} gilt $|\tilde{L}:\Q|=8$. Sei nun $\sigma\in\Aut(\tilde{L},\Q)$. Dann operiert nach \ref{skript:14.8} $\Aut(\tilde{L},\Q)$ transitiv auf
	\[X=\{\alpha,-\alpha,\i\alpha,-\i\alpha\}\hat{=}\{1,2,3,4\}.\]
	Beachte: $\alpha\mapsto-\alpha\Rightarrow\i\alpha\mapsto\pm\i\alpha,-\alpha\mapsto\alpha$. Daraus ergeben sich die Permutationen
	\[\id, (1,2), (1,2)(3,4).\]
	Komposition liefert noch
	\[(3,4).\]
	Ebenso sieht man, dass die Permutationen
	\[(1,3)(2,4), (1,4)(2,3)\]
	ebenfalls von Automorphismen kommen. Erneute Komposition liefert
	\[(1,2)(1,4)(2,3)=(1,3,2,4)\]
	und damit auch das Inverse
	\[(1,4,2,3)\]
	als Automorphismus.\\
	Die Permutation $(1,2,3,4)$ kann nicht von einem Automorphismus kommen, da $\alpha\mapsto-\alpha\mapsto\i\alpha$ unmöglich ist. Obige Rechnungen zeigen mit dem Erweiterungssatz und dem Erweiterungslemma, dass es 8 Automorphismen gibt.\\
	Da $\Aut(\tilde{L},\Q)\cong S_4$ unmöglich ist, folgt mit Lagrange
	\[|\Aut(\tilde{L},\Q)|=8,\]
	da es keine größeren Untergruppen der $S_4$ gibt. Somit ist $\Aut(\tilde{L},\Q)$ isomorph zu einer 2-Sylowuntergruppe von $S_4$, also isomorph zur Diedergruppe der Ordnung 8.
\end{genericdf}

\begin{lemma}\label{skript:14.10}
	Sei $f\in K[x]$ und $\Grad f\geq1$, $L\supseteq K$ eine Körpererweiterung.
	\begin{itemize}
		\item[\textbf{(1)}]
		Falls $f$ und $D(f)$ keinen gemeinsamen Teiler vom Grad mindestens 1 in $K[x]$ haben, dann hat $f$ keine mehrfachen Nullstellen in $L$.
		\item[\textbf{(2)}]
		Ist $f$ irreduzibel und $D(f)\neq0$, dann besitzt $f$ ebenfalls keine mehrfachen Nullstellen in $L$.
	\end{itemize}
\end{lemma}
\begin{proof}
	\begin{itemize}
		\item[\textbf{(1)}]
		Der Euklidsche Algorithmus liefert ein $d\in K[x]$, welches sowohl $f$ als auch $D(f)$ teilt und
		\[d=g\cdot f+h\cdot D(f)\]
		mit $g,h\in K[x]$. Angenommen, es gelte
		\[f=(x-\alpha)^2 r\]
		mit $r\in L[x]$. Dann folgt aus der Produktregel, dass
		\[D(f)=2(x-\alpha)r+(x-\alpha)^2D(r),\]
		also teilt $(x-\alpha)$ sowohl $f$ als auch $D(f)$ in $L[x]$. Damit muss aber $(x-\alpha)$ auch ein Teiler von $d$ sein und somit der Grad von $d$ mindestens 1.
		\item[\textbf{(2)}]
		Sei $D(f)\neq0$ und $f$ irreduzibel, d. h. $f$ besitzt nur Teiler vom Grad 0 oder vom gleichen Grad wie $f$. Wenn $\Grad f=1$ oder $\Grad D(f)=0$, dann folgt die Aussage sofort aus \textbf{(1)}. Ist $\Grad D(f)\geq1$ und $g$ ein nicht konstanter Teiler von $f$ und $D(f)$, dann gilt
		\[1\leq\Grad g\leq\Grad D(f)<\Grad f.\]
		Dies ist aber ein Widerspruch zur Irreduzibilität von $f$. Also gibt es kein solches $g$ und die Aussage folgt mit \textbf{(1)}.
	\end{itemize}
\end{proof}
\begin{genericdf}{Bemerkung}\label{skript:14.11}
	$f\in K[x]$ sei irreduzibel und $\Char K=0$. Dann ist $D(f)\neq0$. Falls aber $\Char K=p\neq0$ ist, kann $D(f)=0$ sein: z. B. sei
	\[K=\F_p(t)=\Quot(\F_p[t])\]
	und $f\in K[x]$ mit $f=x^p-t$. Dann ist
	\[D(f)=px^{p-1}=0.\]
	Beachte: $t$ ist prim in $\F_p[t]$. Nach Eisenstein ist $f$ irreduzibel in $\F_p[t]$ und nach Gauß auch in $\F_p(t)$. Sei $\alpha$ eine Nullstelle von $f$. Im Zerfällungskörper $L$ von $f$ gilt dann, da $t=\alpha^p$:
	\[(x-\alpha)^p=x^p-\alpha^p=x^p-t.\]
	Also ist $\alpha$ eine mehrfache Nullstelle!
\end{genericdf}

\begin{genericdf}{Hauptsatz für endliche Körper}\label{skript:14.12}
	Sei $p$ eine Primzahl, $\F_p=\Z/p\Z$. Für $n\geq1$ sei $f=x^{p^n}-x\in\F_p[x]$ und es sei $K\supseteq\F_p$ ein Zerfällungskörper von $f$. Dann gelten:
	\begin{itemize}
		\item[\textbf{(1)}]
		$|K|=p^n$ und es existiert ein $g\in\F_p[x]$, $g\neq0$, welches irreduzibel ist und
		\[K\cong\F_p[x]/(g).\]
		\item[\textbf{(2)}]
		Ist $K'$ ein Körper mit $|K'|=p^n$, dann ist $K'$ ein Zerfällungskörper von $f$ und $K'\cong K$.
		\item[\textbf{(3)}]
		$\Aut(K,\F_p)=\langle F\rangle$, wobei $F$ der Frobeniusautomorphismus ist, gegeben durch
		\[F: K\to K: y\mapsto y^p.\]
		Außerdem gilt $|\Aut(K,\F_p)|=n$.
	\end{itemize}
\end{genericdf}
\begin{proof}
	Zunächst: Da $K$ ein Zerfällungskörper ist, gilt $|K:\F_p|<\infty$ und damit auch $|K|<\infty$. Nach Lemma \ref{skript:8.7} ist $F$ injektiv. Wegen $|K|<\infty$ folgt, dass $F$ auch surjektiv ist. Somit ist $F\in\Aut(K,\F_p)$.
	\begin{itemize}
		\item[\textbf{(1)}]
		Es gilt
		\[D(f)=p^nx^{p^n-1}-1=-1\neq0.\]
		$f$ hat also keine mehrfachen Nullstellen in $K$. Sei $X$ die Menge der Nullstellen von $f$ in $K$, dann ist $|X|=p^n=\Grad f$. Es ist
		\[X=\{y\in K\ |\ y^{p^n}=y\}=\{y\in K\ |\ F^n(y)=y\}.\]
		Damit ist zunächst $F^n\big|_X=\id$. Da $F$ ein Körperautomorphismus ist, gilt nun $F^n=\id$ auf ganz $K$ und somit $X=K$. Nach Satz \ref{skript:8.10} ist $K^\ast$ zyklisch, d. h. es gibt ein $z_0$ mit $K^\ast=\langle z_0\rangle$. Daraus folgt
		\[K=\F_p(z_0).\]
		Sei $g=\mu_{z_0}$ das Minimalpolynom. Da $|K:\F_p|=n$, gilt $\Grad g=n$. Betrachte
		\[\varphi: \F_p[x]\to K: h\mapsto h(z_0).\]
		$\varphi$ ist surjektiv und $g\in\Ker\varphi$. Nach dem Homomorphiesatz ist
		\[\F_p[x]/\Ker\varphi\cong K.\]
		Nun ist aber $(g)\subseteq\Ker\varphi$ und $(g)$ ist maximal, da $g$ irreduzibel ist. Daraus folgt $(g)=\Ker\varphi$ und \textbf{(1)} ist bewiesen.
		\item[\textbf{(2)}]
		$\F_p$ ist auch der Primkörper von $K'$. Da $|K'|=p^n$ und $|K'^\ast|=p^n-1$, ist $K'$ ein Zerfällungskörper von $f$. \textbf{(2)} wird dann durch \ref{skript:14.6} gezeigt.
		\item[\textbf{(3)}]
		Sei $K^\ast=\langle z_0\rangle$, $g$ ein irreduzibles Polynom wie in \textbf{(1)} und $\sigma\in\Aut(K,\F_p)$. Dann ist $\sigma$ durch $z_0$ eindeutig bestimmt, da $K=\F_p(z_0)$. Es ist
		\[g(\sigma(z_0))=\sigma(g(z_0))=\sigma(0)=0.\]
		Da $\Grad g=n$ ist, existieren höchstens $n$ Möglichkeiten für $\sigma(z_0)$, d. h. es gibt höchstens $n$ verschiedene $\sigma$. Andererseits sei $m\geq1$ die Ordnung von $F$ in $\Aut(K,\F_p)$. Dann ist $z_0=F^m(z_0)$, also $z_0^{p-1}=1$. Die Ordnung von $z_0$ ist $p^n-1$, also ist $m\geq n$. Damit liefern die Potenzen $n$ verschiedene Automorphismen, d. h. $|\Aut(K,\F_p)|=n$ und folglich
		\[\Aut(K,\F_p)=\langle F\rangle.\]
	\end{itemize}
\end{proof}

\begin{generic_no_num}{Abschließende Bemerkungen zu §14}
	In vielen Büchern zur Algebra finden sich die Begriffe normal und separabel für Körpererweiterungen. Sei $L\supseteq K$ eine algebraische Körpererweiterung.
	\begin{itemize}
		\item[\textbf{(1)}]
		Man nennt $L\supseteq K$ \bi{separabel}, falls für jedes $\alpha\in L$ das Minimalpolynom $\mu_\alpha$ keine mehrfachen Nullstellen im algebraischen Abschluss $\overline{K}$ besitzt.
		\item[\textbf{(2)}]
		Man nennt $L\supseteq K$ \bi{normal}, wenn für jeden Körperhomomorphismus $\sigma: L\to\overline{L}$ mit $\sigma\big|_K=\id$ gilt, dass $\sigma(L)=L$, d. h. $\sigma\in\Aut(L,K)$. Ist $|L:K|<\infty$, dann gilt: $L\supseteq K$ ist genau dann normal, wenn $L$ Zerfällungskörper eines Polynoms $f\in K[x]$ ist.
	\end{itemize}
\end{generic_no_num}

\newpage
\section{Galoiserweiterungen}

Ab jetzt wird durchweg folgende Notation bzw. Situation betrachtet: $L\supseteq K$ sei eine Körpererweiterung mit $|L:K|<\infty$ und $G:=\Aut(L,K)$. Nach \ref{skript:14.3} gilt $|G|<\infty$.

\begin{lemma}\label{skript:15.1}
	Seien $M_1,\ldots,M_n$ Zwischenkörper von $L\supseteq K$, d. h. $K\subseteq M_i\subseteq L$. Sei
	\[L=\bigcup_{i=1}^n M_i,\]
	dann existiert ein $i_0$ mit $M_{i_0}=L$.
\end{lemma}
\begin{proof}
	Sei $|K|<\infty$, dann ist auch $|L|<\infty$. Nach Satz \ref{skript:8.10} existiert ein $z_0\in L$ mit $L^\ast=\langle z_0\rangle$. Andererseits gilt nach Voraussetzung
	\[L=\bigcup_{i=1}^n M_i.\]
	Dann muss es ein $i_0$ mit $z_0\in M_i$ geben. Daraus folgt $L^\ast\subseteq M_{i_0}$ und somit $L=M_{i_0}$.\\
	Sei nun $|K|=\infty$. Der Beweis wird durch Induktion nach $n$ geführt: Für $n=1$ ist die Aussage trivialerweise richtig. Sei nun $n>1$ und $M_1\neq L$. Sei $z\in M_1$ und $w\in L\backslash M_1$. Für jedes $k\in K^\ast$ ist dann
	\[z+kw\notin M_1\Rightarrow\exists i\geq2: z+kw\in M_i.\]
	Da $|K|=\infty$ und es nur endlich viele Zwischenkörper gibt, existieren $k_1,k_2\in K$, sodass für ein geeignetes $j\geq2$
	\[k_1+k_2\in M_j,\ z+k_1w\in M_j,\ z+k_2w\in M_j\]
	gelten. Damit ist auch
	\[0\neq k_1-k_2\in M_j,\ (z+k_1w)-(z-k_2w)=(k_1-k_2)w\in M_j\]
	und somit $w\in M_j$. Also ist auch für jedes $k\in K$ $kw\in M_j$ und damit auch $z\in M_j$. $z$ war aber beliebig aus $M_1$, d. h.
	\[M_1\subseteq\bigcup_{i=2}^n M_i\]
	und damit
	\[\bigcup_{i=2}^n=L.\]
	Mit der Induktionsvoraussetzung folgt nun, dass ein $i_0\geq 2$ existiert mit $M_{i_0}=L$.
\end{proof}

\begin{lemma}\label{skript:15.2}
	Sei $z\in L$ und $\mu_z$ das Minimalpolynom von $z$ über $K$. Sei
	\[B=\{\sigma(z)\ |\ \sigma\in G\}\]
	die Bahn von $z$ unter der Operation von $G$ auf $L$. Dann ist $|B|<\infty$ und
	\[\forall y\in B: \mu_z(y)=0.\]
	Ferner ist
	\[|L:K|\geq|K(z):K|=\Grad(\mu_z)\geq |B|.\]
\end{lemma}
\begin{proof}
	Da $|G|<\infty$, ist auch $|B|<\infty$. Nach \ref{skript:14.1} gilt für jedes $\sigma\in G$
	\[\mu_z(\sigma(z))=\sigma(\mu_z(z))=\sigma(0)=0.\]
	Somit kann $|B|$ höchstens gleich der Anzahl der Nullstellen von $\mu_z$ in einem Zerfällungskörper sein, was gerade $\Grad \mu_z$ ist. Da $K(z)\subseteq L$, gilt auch $|L:K|\geq|K(z):K|$.
\end{proof}

\begin{lemma}\label{skript:15.3}
	Es gilt stets $|G|\leq|L:K|$ und es gibt ein $z_0\in L$ mit
	\[\Stab_G(z_0)=1.\]
	Falls $|G|=|L:K|$, dann gilt $L=K(z_0)$.
\end{lemma}
\begin{proof}
	Für $G=1$ ist nichts zu zeigen (trivial). Sei nun $G\neq1$ und $\sigma\in G$. Dann setze
	\[M_\sigma:=\{y\in L\ |\ \sigma(y)=y\}.\]
	$M_\sigma$ ist offensichtlich ein Teilkörper von $L$. Ist $\sigma\neq\id$, dann ist $M_\sigma\subsetneq L$. Da $G\neq1$, ist auch $K\subsetneq L$. Nach Lemma \ref{skript:15.1} gilt dann:
	\[L\neq\bigcup_{\sigma\in G\backslash\{\id\}} M_\sigma\]
	Dann existiert $z_0\in L$, sodass für jedes $\sigma\neq\id$
	\[\sigma(z_0)\neq z_0\]
	ist. Damit besteht $\Stab_G(z_0)$ nur aus der Identität. Wenn $B$ die Bahn mit $z_0\in B$ ist, dann hat $B$ also genau $|G|$ Elemente. Nach Lemma \ref{skript:15.2} gilt:
	\[|L:K|\geq|K(z_0):K|\geq|B|=|G|\]
	Falls $|L:K|=|G|$, dann gilt oben überall Gleichheit. Insbesondere ist $|L:K|=|K(z_0):K|$, also $L=K(z_0)$.
\end{proof}

\begin{df}\label{skript:15.4}
	Ist $|G|=|L:K|$, dann heißt $L\supseteq K$ \bi{Galoiserweiterung}. Man spricht bei Galoiserweiterungen dann auch von der \bi{Galoisgruppe} $G$ von $L$.
\end{df}

\begin{genericdf}{Beispiele}\label{skript:15.5}
	\begin{itemize}
		\item[\textbf{(1)}]
		$\C\supseteq\R$ ist eine Galoiserweiterung: $|\C:\R|=2$, $G=\langle z\mapsto\overline{z}\rangle\cong C_2$.
		\item[\textbf{(2)}]
		$L=\Q(\zeta_n)\supseteq\Q$ ist eine Galoiserweiterung: $|L:\Q|=\phi(n)$, $G\cong(\Z/n\Z)^\ast$ (vgl. \ref{skript:14.5})
		\item[\textbf{(3)}]
		$K=\F_{p^n}\supseteq\F_p$ ist eine Galoiserweiterung: $|K:\F_p|=n$. Nach Satz \ref{skript:14.12} ist $G\cong C_n$.
		\item[\textbf{(4)}]
		$K=\Q(\sqrt[4]{2})\supseteq\Q$ ist keine Galoiserweiterung: $|K:\Q|=4$, aber $G=\Aut(K,\Q)\cong C_2$ (vgl. \ref{skript:14.9}).
	\end{itemize}
\end{genericdf}

\begin{genericthm}{Charakterisierung von Galoiserweiterungen}\label{skript:15.6}
	Sei $L\supseteq K$ eine Körpererweiterung mit $|L:K|<\infty$ und $G=\Aut(L,K)$. Dann sind folgende Aussagen äquivalent:
	\begin{itemize}
		\item[\textbf{(i)}]
		$|L:K|=|G|$, d. h. $L\supseteq K$ ist eine Galoiserweiterung.
		\item[\textbf{(ii)}]
		Es existiert ein $z_0\in L$ mit $L=K(z_0)$, sodass $\mu_{z_0}\in K[x]$ über $L$ in Linearfaktoren zerfällt und keine mehrfachen Nullstellen besitzt.
		\item[\textbf{(iii)}]
		$L$ ist Zerfällungskörper eines $f\in K[x]$ mit $\Grad f\geq 1$ und $f$ besitzt keine mehrfachen Nullstellen.
		\item[\textbf{(iv)}]
		Es gilt: $K=\{y\in L\ |\ \forall\sigma\in G:\sigma(y)=y\}$.
	\end{itemize}
\end{genericthm}

\chapter{Lösungen}
\section{Lösungen zum Abschnitt 1}

\begin{loes}\hypertarget{loes:1.1}
Seien $U_1$ und $U_2$ zwei echte Untergruppen von $G$.
Nun nehmen wir an, dass 
\begin{align*}
G = U_1 \cup U_2
\end{align*}
gilt.
Zwischen $U_1$ und $U_2 $ kann es keine Teilmengenbeziehung geben, ansonsten wäre $U_1$ oder $U_2$ keine echte Untergruppe.
Nun wählen wir $g \in U_1 \setminus U_2$ und $h \in U_2 \setminus U_1$.
Es gilt $g \ast h \in G$ und mit unserer Annahme muss $g \ast h \in U_1 \cup U_2$ erfüllt sein.
Mit 
\begin{align*}
g \ast h \in U_1 \Rightarrow h \in U_1 \quad \text{und} \quad g \ast h \in U_2 \Rightarrow g \in U_2 
\end{align*}
erhalten wir einen Widerspruch zu unserer Wahl der Elemente. Damit war unsere Annahme falsch und das Gegenteil ist richtig.
\end{loes}

\begin{loes}\hypertarget{loes:1.2}
Die Aussage ist äquivalent zu
\begin{align*}
|G| = \infty \quad \Rightarrow \quad \text{es gibt unendlich viele Untergruppen von } G.
\end{align*}
Für den Beweis müssen wir zwei Fälle unterscheiden.
Der \textit{erste Fall} ist, dass alle Gruppenelemente endliche Ordnung besitzen.
Das heißt für alle $g \in G$ gilt $\ord(g) < \infty$.
Aus der Identität 
\begin{align*}
G = \bigcup\limits_{g \in G} \langle g \rangle
\end{align*}
folgt das es unendlich viele Untergruppen geben muss, ansonsten wäre $G$ endlich.
Der \textit{zweite Fall} ist, dass mindestens ein Gruppenelement mit unendlicher Ordnung existiert.
Sei $g \in G$ dieses Gruppenelement. Nun gilt
\begin{align*}
\langle g \rangle &\cong \Z \\
\langle g^2 \rangle &\cong 2\Z\\
&\vdots
\end{align*}
und somit haben wir auch in diesem Fall unendlich viele Untergruppen.
\end{loes}

\begin{loes}\hypertarget{loes:1.3}
Zunächst zeigen wir, dass $U$ eine Untergruppe von $G$ ist. 
Es gilt $E \in U$, wobei $E$ die Einheitsmatrix bezeichnet.
Mit
\begin{align*}
\begin{pmatrix}
1 & a & b \\
0 & 1 & c \\
0 & 0 & 1
\end{pmatrix}
\cdot
\begin{pmatrix}
1 & d & e \\
0 & 1 & f \\
0 & 0 & 1
\end{pmatrix}
= 
\begin{pmatrix}
1 & a+d & b +af + e \\
0 & 1 & c+f \\
0 & 0 & 1
\end{pmatrix}
\end{align*}
für $a,b,c,d,e,f \in \Z / 3\Z$ liegt auch das Produkt in $U$.
Wenden wir nun auf den ersten Faktor im obigen Produkt den Gauß-Algorithmus an, so erhalten wir 
\begin{align*}
\begin{pmatrix}
1 & -a & ac-b \\
0 & 1 & -c \\
0 & 0 & 1
\end{pmatrix}
\end{align*}
als Inverse. Damit ist auch die letzte Untergruppenbedingung erfüllt.
Für die Anzahl betrachten wir 
\begin{align*}
\begin{pmatrix}
1 & \ast & \ast \\
0 & 1 & \ast \\
0 & 0 & 1
\end{pmatrix}
\end{align*} 
und mit $| \lbrace 0,1,2 \rbrace^3 | = 27$ erhalten wir $|U| = 27$. 
Da die Ordnung eines Elements die Gruppenordnung teilt, haben wir die möglichen Ordnungen $1,3 ,9$ und $27$.
Es gilt offensichtlich $\ord(E) = 1$. Sei nun $A \in U \setminus \lbrace E \rbrace$.
Dann gilt
\begin{align*}
A &=
\begin{pmatrix}
1 & a & b \\
0 & 1 & c \\
0 & 0 & 1
\end{pmatrix}\\
A^2 &=
\begin{pmatrix}
1 & 2a & 2b+ac \\
0 & 1 & 2c \\
0 & 0 & 1
\end{pmatrix}\\
A^2 &=
\begin{pmatrix}
1 & 3a & 3b+3ac \\
0 & 1 & 3c \\
0 & 0 & 1
\end{pmatrix} 
= E
\end{align*}
für passende $a,b$ und $c$.
Damit gilt $\ord(A) = 3$ für alle $A \in U \setminus \lbrace E \rbrace$.
\end{loes}

\begin{loes}\hypertarget{loes:1.4} \
\begin{enumerate}
\item[a)]
Wir bemerken das $ g^2 = 1_G$ für alle $g \in G$ gilt.
Sei nun $g,h \in G$ beliebig. Dann folgt mit
\begin{align*}
(gh)^2 = 1_G
\Leftrightarrow
ghgh = 1 
\Leftrightarrow
gh = (gh)^{-1} = h^{-1} g^{-1} = h g
\end{align*}
die Kommutativität der Gruppe.
\item[b)]
Wegen $[G:U] = 2 $ wählen wir $g \in G \setminus U $ beliebig. Da $G$ eine disjunkte Vereinigung von Nebenklassen ist gilt
\begin{align*}
G = U   \overset{.}{\cup} g U = U  \overset{.}{\cup} U g
\end{align*}
und somit auch 
\begin{align*}
g U = U g \Leftrightarrow g^{-1} U g = U
\end{align*}
für alle $g \in G \setminus U$. Für alle anderen Gruppenelemente ist diese Identität so oder so erfüllt.
Insgesamt folgt $U \nt G$.
\item[c)]
Hierfür können wir ein Gegenbeispiel angeben.
Wir betrachten $S_3 $ mit $U = \langle (12) \rangle$. 
Mit Lagrange gilt
\begin{align*}
[S_3 : U ] = \frac{|S_3|}{|U|} = 3.
\end{align*}
Nun wählen wir $(23) \in S_3$ und erhalten mit
\begin{align*}
(23) \id &= \id (23)  \\
(231) = (23)(12) &= (12)(23) = (13)
\end{align*}
das Gegenbeispiel.
\item[d)]
Wir verwenden die Aufgabe 1.3 als Gegenbeispiel.
Es gilt
\begin{align*}
\begin{pmatrix}
1 & a & b \\
0 & 1 & c \\
0 & 0 & 1
\end{pmatrix}
\cdot
\begin{pmatrix}
1 & d & e \\
0 & 1 & f \\
0 & 0 & 1
\end{pmatrix}
&= 
\begin{pmatrix}
1 & a+d & b +af + e \\
0 & 1 & c+f \\
0 & 0 & 1
\end{pmatrix}\\
\begin{pmatrix}
1 & d & e \\
0 & 1 & f \\
0 & 0 & 1
\end{pmatrix}
\cdot
\begin{pmatrix}
1 & a & b \\
0 & 1 & c \\
0 & 0 & 1
\end{pmatrix}
&=
\begin{pmatrix}
1 & a+d & b +d c + e \\
0 & 1 & c+f \\
0 & 0 & 1
\end{pmatrix}
\end{align*}
und mit der Wahl $a = f = 1$ und $0$ sonst erhalten wir das Gegenbeispiel.
\end{enumerate}


\end{loes}

\begin{loes}\hypertarget{loes:1.5} \
\begin{enumerate}
\item[a)]
Sei $g \in G$ mit $\ord(g) = n \in \N$. Mit der Homomorphismuseigenschaft erhalten wir
\begin{align*}
\varphi(g)^n = \varphi(g^n) = \varphi(1_G) = 1_H.
\end{align*}
Damit muss $\ord(\varphi(g)) \leq n$ sein. Sei $m $ diese Ordnung.
Mit $n =  k \cdot m + r$ für $k \in \N$ mit $0 \leq r < m$ erhalten wir
\begin{align*}
1_H = \varphi(g)^n = \varphi(g)^{k \cdot m + r} = 1_H \cdot \varphi(g)^r =  \varphi(g)^r,
\end{align*}
womit $r = 0 $ sein muss. Dies folgt sofort, da $r < m $ ist. 
Also ist $m$ Teiler von $n$.
\item[b)]
Zunächst müssen wir die Untergruppeneigenschaft für $ \varphi^{-1}(M) $ zeigen.
Wegen $ \varphi(1_G)  = 1_H$ ist $ 1_G \in \varphi^{-1}(M) $.
Sei nun $ a,b \in \varphi^{-1}(M)   $. Dann folgt aus $ M \leq H $ durch
\begin{align*}
\varphi(a \ast b) = \varphi(a) \cdot \varphi(b) \in M \quad
\Rightarrow \quad
a \ast b \in \varphi^{-1}(M) 
\end{align*}
die Abgeschlossenheit unter der Verknüpfung aus $ G $. Durch eine ähnliche Argumentation
folgt auch
\begin{align*}
a \in \varphi^{-1}(M) \quad \Rightarrow \quad a^{-1} \in \varphi^{-1}(M) .
\end{align*}
Für die Normalteilereigenschaft wählen wir $ g \in \varphi^{-1}(M)  $ und $ x \in G $ beliebig.
Aufgrund $ M \nt H $ folgt mit
\begin{align*}
\varphi(x^{-1} g x) = \varphi(x)^{-1} \cdot \varphi(g) \cdot \varphi(x) \in M,
\end{align*}
dass $ x^{-1} g x \in \varphi^{-1}(M)  $ gilt.
Durch die beliebige Wahl gilt $ \varphi^{-1}(M)  \nt G $.
\item[c)] 
Hierfür haben wir ein Gegenbeispiel.
Sei $ G = S_2 $, $ H = S_3 $ und $ \varphi : G \to H $ die Einbettung von $ S_2$ in $ S_3 $.
Es gilt also $ \varphi(S_2) \cong S_2 $. 
Nun gilt $ S_2 \nt S_2 $, jedoch folgt mit Aufgabe 4 c), dass $ S_2 \ntrianglelefteq S_3 $ gilt.
\item[d)] 
Es gilt $ \varphi(U) \leq H $. Nun ist $ \varphi(U)  $ invariant unter allen Automorphismen von $ H $.
Damit gilt insbesondere auch
\begin{align*}
\gamma_x(\varphi(U)) = x^{-1} \varphi(U) x = \varphi(U)
\end{align*}
für alle $ x \in H $. Daraus folgt $ \varphi(U) \nt H $.
\end{enumerate}


\end{loes}
\newpage
\section{Lösungen zum Abschnitt 2}

\begin{loes}\hypertarget{loes:2.1}
	Sei $a,b,c,d \in \R $.
	Dann erhalten wir durch
	\begin{align*}
	\begin{pmatrix}
	a & b \\
	c & d
	\end{pmatrix}
	\cdot
	\begin{pmatrix}
	1 & 0 \\
	0 & -1
	\end{pmatrix}
	&=
	\begin{pmatrix}
	a & -b \\
	c & -d
	\end{pmatrix} \\
	\begin{pmatrix}
	1 & 0 \\
	0 & -1
	\end{pmatrix}
	\cdot
	\begin{pmatrix}
	a & b \\
	c & d
	\end{pmatrix}
	&=
	\begin{pmatrix}
	a & b \\
	-c & -d
	\end{pmatrix},
	\end{align*}
	dass $ b = c = 0 $ gelten muss. Da unsere Matrix in $ \Gl_2(\R) $ enthalten sein muss, sollte dann $ a, d \neq 0 $ gelten. Also erhalten wir 
	\begin{align*}
	C_{\Gl_2(\R)}(S) := \lbrace \begin{pmatrix}
	a & 0 \\
	0 & d
	\end{pmatrix} \ | \ a,d \neq 0 \rbrace
	\end{align*}
	als Zentralisator. In diesem befinden sich alle invertierbaren Diagonalmatrizen und das Produkt von Diagonalmatrizen ist wieder eine Diagonalmatrix.
	Also ist unser Zentralisator eine Untergruppe. Dieser ist jedoch kein Normalteiler, denn für 
	$ a \neq b $ gilt
	\begin{align*}
	\begin{pmatrix}
	1 & 1 \\
	0 & 1
	\end{pmatrix}
	\cdot
	\begin{pmatrix}
	a & 0 \\
	0 & b
	\end{pmatrix}
	&= 
	\begin{pmatrix}
	a & b \\
	0 & b
	\end{pmatrix}
	\\
	\begin{pmatrix}
	a & 0 \\
	0 & b
	\end{pmatrix}
	\cdot
	\begin{pmatrix}
	1 & 1 \\
	0 & 1
	\end{pmatrix}
	&= 
	\begin{pmatrix}
	a & a \\
	0 & b
	\end{pmatrix}.
	\end{align*}
\end{loes}

\begin{loes}\hypertarget{loes:2.2}
	\
	\begin{enumerate}
		\item[a)]
		Für die Lösung dieser Aufgabe betrachten wir zunächst ein Bild.
		\begin{figure}[H]
			\centering
			\begin{tikzpicture}
			\node (A) at (-1,-1)  {$1$};
			\node (B) at (1,-1) {$2$};
			\node (C) at (1,1) {$3$};
			\node (D) at (-1,1) {$4$};
			
			
			%\node (X) at (-0.5,0.69) {$ a $};
			%\node (Y) at (-0.7,0.19) {$ b $};
			%\node (Z) at (-0.7,-0.50) {$ c $};
			
			\draw[-] (A) to (B);
			\draw[-] (B) to (C);
			\draw[-] (C) to (D);
			\draw[-] (D) to (A);
			
			\draw[-,red] (-0.9,0) to (0.9,0);
			\draw[-,red] (A) to (C);
			\draw[-,red] (B) to (D);
			\draw[-,red] (0,-0.9) to (0,0.9);
			\end{tikzpicture}
		\end{figure}
		Mit einem Quadrat können wir vier Rotationen um $ 90^\circ $ ausführen und kommen dann an der Ausgangsstellung.
		Weiter können wir das Quadrat an den rot eingezeichneten Achsen spiegeln.
		Damit ergeben sich für die Rotationen 
		\begin{align*}
		r  &= (1234) \\
		r^2 &= (13)(24) \\
		r^3 &= (1432) \\
		r^4 &= \id
		\end{align*}
		und für die Spiegelungen
		\begin{align*}
		s_1 &= (14)(23) =: s\\
		s_2 &= (24)		= sr\\
		s_3 &= (12)(34)  = sr^2\\
		s_4 &= (13)		= sr^3.
		\end{align*}
		Damit haben wir alle Möglichkeiten aufgeführt.
		Wir bezeichnen diese Gruppe als \bi{Diedergruppe} vierten Grades und schreiben hierfür $ D_4 $.
		\index{Diedergruppe}
		Für diese gilt $ |D_4 | = 2 \cdot 4 = 8 $.
		
		\item[b)] 
		Nun interessieren wir uns für das Zentrum von $ G $.
		Ohne Zweifel gilt $ \id \in Z(G) $. Durch Ausprobieren erhalten wir noch
		$ (13)(24) \in Z(G) $.
		Dies lässt sich leicht mit Bildern verdeutlichen.
		
		\item[c)] 
		Wir wissen, dass $ |D_4 | = 8  $ gilt. Also erhalten wir mit dem Satz von Lagrange
		$ 1 ,2 ,3 $ und $ 8 $ als mögliche Untergruppenordnungen.
		Diese arbeiten wir nun nacheinander ab.
		Für die Ordnung $ 1 $ ist $ \lbrace \id \rbrace \leq D_4 $.
		Die zyklischen Untergruppen der Ordnung $ 2 $ sind
		\begin{align*}
		\langle (13) \rangle ,\langle(23) \rangle, \langle (13)(24) \rangle, \langle (12)(34) \rangle,
		\langle (14)(23) \rangle.
		\end{align*}
		und durch 
		\begin{align*}
		\langle (1234) \rangle &= \langle (1432) \rangle\\
		\langle (12)(34),(13)(24) \rangle &= \langle (14)(23), (13)(24) \rangle\\
		\langle (13),(24) \rangle &
		\end{align*}
		erhalten wir die Untergruppen der Ordnung 4.
		Mit
		\begin{figure}[H]
			\centering
			\begin{tikzpicture}
			\node (A) at (0,0)  {$\lbrace \id \rbrace$};
			
			\node (B) at (-4,1) {$\langle (13)(24) \rangle$};
			\node (C) at (-2,1) {$\langle (12)(34) \rangle$};
			\node (D) at (0,1) {$\langle (14)(23) \rangle$};
			\node (E) at (2,1) {$\langle (13) \rangle$};
			\node (F) at (4,1) {$\langle (24) \rangle$};
			
			\node (G) at (-3,4) {$\langle (1234) \rangle$};
			\node (H) at (0,4) {$\langle (12)(34),(13)(24) \rangle$};
			\node (I) at (3,4) {$\langle (13),(24) \rangle$};
			
			\node (J) at (0,5) {$D_4$};
			
			\draw[-] (A) to (B);
			\draw[-] (A) to (C);
			\draw[-] (A) to (D);
			\draw[-] (A) to (E);
			\draw[-] (A) to (F);
			
			\draw[-] (B) to (G);
			\draw[-] (B) to (H);
			\draw[-] (B) to (I);
			
			\draw[-] (C) to (H);
			\draw[-] (D) to (H);
			
			\draw[-] (E) to (I);
			\draw[-] (F) to (I);
			
			\draw[-] (J) to (G);
			\draw[-] (J) to (H);
			\draw[-] (J) to (I);
			\end{tikzpicture}
		\end{figure}
		sollen die Teilmengenbeziehungen dargestellt werden. Das Diagramm ist von unten nach oben zu lesen.
	\end{enumerate}
\end{loes}

\begin{loes}\hypertarget{loes:2.3}
	Sei $ H \leq G = \langle g \rangle $. Falls $ H  $ die triviale Untergruppe ist haben wir nichts zu zeigen.
	Sei also $ H \neq \lbrace e \rbrace $.
	Dann enthält $ H $ ein Element mit $ g^l \neq e , \ l \in \Z$. Dann gilt aufgrund der Untergruppeneigenschaft
	$ g^{-l} \in H $.
	Aus diesem Grund nehmen wir an, dass $ l > 0  $ ist und wir setzen
	$ k := \min \lbrace l > 0 \ | \ g^l \in H \rbrace $.
	Wir wollen nun zeigen, dass $ H = \langle g^k \rangle $ gilt.
	Sei $ a = g^j $ ein beliebiges Element aus $ G $. 
	Durch Division mit Rest gilt
	\begin{align*}
	j = q \cdot k + r
	\end{align*}
	für $ 1 \leq r < k $.
	Ist nun $ a \in H $, so muss aufgrund der Untergruppeneigenschaft
	\begin{align*}
	g^r = (g^k)^{-q} \cdot a \in H
	\end{align*}
	gelten. Aus der Minimalität folgt dann $ r = 0 $.
\end{loes}

\begin{loes}\hypertarget{loes:2.4}\
	\begin{enumerate}
		\item[a)]
		Sei $ h \in G $ beliebig, aber fest.
		Mit
		\begin{align*}
		\gamma_h(g_1 g_2) = h^{-1} g_1 g_2 h 
		=h^{-1} g_1 h h^{-1} g_2 h = \gamma_h(g_1) \cdot \gamma_h(g_2) 
		\end{align*}
		für alle $ g_1,g_2 \in G $ erhalten wir die Homomorphismuseigenschaft für $ \gamma_h $.
		Durch schnelles Nachrechnen erhalten wir als Umkehrabbildung $ \gamma_{h^{-1} }$.
		Damit ist $ \gamma_h $ bijektiv.
		
		\item[b)]
		Sei $ \operatorname{Inn}(G) $ die Menge der inneren Automorphismen.
		Nun sei $ \varphi \in \Aut(G) $ und $ g,h \in G $ beliebig.
		Dann folgt mit
		\begin{align*}
		(\varphi \circ \gamma_h \circ \varphi^{-1})(g)
		= \varphi \circ \gamma_h(\varphi^{-1}(g))
		= \varphi(h^{-1} \varphi^{-1}(g) h)
		= \varphi(h)^{-1} \cdot g \cdot \varphi(h)
		= \gamma_{\varphi(h)}(g),
		\end{align*}
		dass $ \operatorname{Inn}(G) \nt \Aut(G) $ gilt.
		Die Untergruppeneigenschaften stehen nach dem Aufschreiben sofort da.
		
	\end{enumerate}
\end{loes}

\begin{loes}\hypertarget{loes:2.5}\
	\begin{enumerate}
		\item[a)]
		Es gilt
		\begin{align*}
		N_G(H) = \lbrace g \in G \ | \ g^{-1} H g = H \rbrace
		\end{align*}
		und mit $ H \nt K $ folgt sofort $ H \subset N_G(H) $.
		
		\item[b)]
		Zuerst müssen wir zeigen, dass $ K \cdot H \leq G $ gilt.
		Zunächst gilt $ K \leq N_G(H) $, das heißt
		\begin{align*}
		k^{-1} H k = H
		\end{align*}
		für alle $ k \in K $.
		Nun wenden wir das Untergruppenkriterium an.
		Seien $ k_1 , k_2  \in K$ und $ h_1, h_2 \in H $ beliebig.
		Dann folgt mit
		\begin{align*}
		k_1 h_1(k_2 h_2)^{-1}
		= k_1 h_1 h_2^{-1} k_2^{-1}
		= k_1 h_1 k_2^{-1}\underbrace{k_2 h_2^{-1} k_2^{-1}}_{=: \tilde{h}}
		= \underbrace{k_1 k_2^{-1}}_{\in K } \underbrace{k_2 h_1 k_2^{-1} \tilde{h}}_{\in H} \in K \cdot H
		\end{align*}
		die Untergruppeneigenschaft.
		Nun fehlt noch $ H \nt K \cdot H $.
		Dies erhalten wir durch
		\begin{align*}
		(kh)^{-1} H kh
		= h^{-1} k^{-1} H k h
		= H
		\end{align*}
		für $ k \in K  $ und $ h \in H  $ sofort.
	\end{enumerate}
\end{loes}

\begin{loes}\hypertarget{loes:2.6}\
	\begin{enumerate}
		\item[a)]
		Wir müssen zwei Fälle betrachten.
		Sei zunächst $ v = 0 $.
		Dann gilt $ A \cdot v = 0  $ für alle $ A \in \Sl_n(\R) $
		und folgt $ v \in O_0 $.
		Nun betrachten wir $ v \neq 0  $ und ergänzen $ v $ mithilfe 
		des Basisergänzungssatz zu einer Basis von $ \R^n $.
		Diese sei $ \lbrace v , b_2 , \dots , b_n  \rbrace$ und mit
		\begin{align*}
		M := \begin{pmatrix}
		v & b_2 & \hdots & b_n
		\end{pmatrix}
		\end{align*}
		erhalten wir eine invertierbare Matrix mit $ M \cdot e_1 = v $ und $ \det(M) \neq 0 $.
		Mit 
		\begin{align*}
		\tilde{M} = \begin{pmatrix}
		v & \frac{1}{\det(M)} \cdot b_2 & \hdots & b_n
		\end{pmatrix}
		\end{align*}
		gilt noch $ \tilde{M} \in \Sl_n(\R) $, womit $ v \in O_{e_1} $ folgt.
		Insgesamt gibt es also die zwei Bahnen $ O_0 $ und $ O_{e_1} $.
		
		\item[b)] 
		Der Fall, dass $ v = 0 $ ist, sollte nach a) klar sein.
		Nun betrachten wir $ || v || =  \lambda $ mit $ \lambda > 0 $.
		Mit dem Basisergänzungssatz und Gram-Schmidt-Verfahren erhalten wir eine ONB.
		In $ A $ seien die Spalten dieser ONB.
		Dann gilt 
		\begin{align*}
		A \cdot e_1 = \frac{1}{\lambda} \cdot v
		\Leftrightarrow
		A \cdot (\lambda e_1) = v
		\end{align*}
		und wir erhalten mit
		\begin{align*}
		\R^n = O_0 \cup \bigcup \limits_{\lambda > 0} O_{\lambda e_1}
		\end{align*}
		alle Bahnen.
		
		\item[c)]
		Wir betrachten 
		\begin{align*}
		\begin{pmatrix}
		\lambda_1 & \hdots & 0\\
		\vdots  &\ddots & \vdots \\
		0  & \hdots & \lambda_n
		\end{pmatrix}
		\cdot
		\begin{pmatrix}
		\bullet \\
		\vdots \\
		\bullet
		\end{pmatrix}
		= 
		\begin{pmatrix}
		\lambda_1\cdot \bullet \\
		\vdots \\
		 \lambda_n \cdot\bullet
		\end{pmatrix}
		\end{align*}
		und erkennen, dass es nur relevant ist, welche Punkte ungleich null sind.
		Zur Vereinfachung vereinbaren wir, dass die Punkte nur die Werte $ 0 $ oder $ 1 $
		annehmen können.
		Dadurch erhalten wir ein kombinatorisches Problem.
		Durch 
		\begin{align*}
		\sum \limits_{k=0}^n \binom{n}{k}
		=
		\sum \limits_{k=0}^n \binom{n}{k} \cdot 1^{n-k} \cdot 1^k
		=2^n
		\end{align*}
		erhalten wir die Anzahl der möglichen Bahnen.
		
		\item[d)]
		Für $ v = 0  $ erhalten wir wieder eine Bahn.
		Nun gilt für eine invertierbare oberer Dreiecksmatrix $ A $ und 
		$ 1 \leq k \leq n $
		\begin{align*}
		A \cdot e_k = 
		\begin{pmatrix}
		a_{k1}\\
		\vdots\\
		a_{k(k-1)}\\
		a_{kk}
		\end{pmatrix}
		\end{align*}
		mit $ a_{kk} \neq 0 $.
		Damit erhalten wir mit
		\begin{align*}
		O_{e_1} &= \left\lbrace 
		\begin{pmatrix}
		x_1 \\
		0\\
		\vdots\\
		0	
		\end{pmatrix}
		\ | \ x_1 \in \R \wedge x_1 \neq 0 \right\rbrace \\
		&\vdots\\
		O_{e_n} &= \left\lbrace 
		\begin{pmatrix}
		x_1 \\
		x_2\\
		\vdots\\
		x_n	
		\end{pmatrix}
		\ | \ x_1,\dots ,x_n \in \R \wedge x_n \neq 0 \right\rbrace 
		\end{align*}
		alle Bahnen.
	\end{enumerate}
\end{loes}

\begin{loes}\hypertarget{loes:2.7}\
	\begin{enumerate}
		\item[a)]
		 Sei $ G $ eine Gruppe mit zwei disjunkten Konjugationsklassen.
		 Dann muss
		 \begin{align*}
		 G = \lbrace e \rbrace \cup \lbrace h^{-1} g h \ | \ h \in G \rbrace
		 \end{align*}
		 für $ g \in G \setminus \lbrace e \rbrace $ gelten.
		 Mit der Klassengleichung erhalten wir:
		 \begin{align*}
		 n := |G| = \frac{|G|}{|C_G(e)|} + \frac{|G|}{|C_G(g)|}
		 = 1 + \frac{|G|}{|C_G(g)|}
		 \Leftrightarrow
		 n-1 = \frac{|G|}{|C_G(g)|}
		 \Leftrightarrow
		 |C_G(g)| = \frac{n}{n-1} \in \N
		 \end{align*}
		 Damit kommt für $ n $ nur die $ 2 $ infrage, womit $ G \cong C_2 $ gilt.
		 
		 \item[b)]
		 Sei $ G $ eine $ p$- Gruppe mit drei disjunkten Konjugationsklassen.
		 Dann gilt
		 \begin{align*}
		 | G | &= | C_e | + | C_2 | + | C_3 |\\
		 \Leftrightarrow
		 p^k &= 1 + p^l + p^m
		 \end{align*}
		 mit $ l, m < k $.
		 Falls $ l = 0  $ und $ r = 0 $ ist, gilt
		 \begin{align*}
		 |G| = 3 \Rightarrow G \cong C_3.
		 \end{align*}
		 Gilt $ l = 0  $ und $ m \neq 0 $, so folgt
		 \begin{align*}
		 p^k = 2 + p^m
		 \Rightarrow p^m \mid p^k \wedge 2 \mid p^k
		 \Rightarrow p^2
		 \end{align*}
		 und mit 
		 \begin{align*}
		 p^k - p^m = 2 
		 \Leftrightarrow
		 1 = 2^{m-1} \cdot (2^{k-m} - 1 ) 
		 \end{align*}
		 folgt weiter, dass $ m = 1$ und $ k = 2 $ sein muss.
		 Mit Aufgabe 4 auf Blatt 3 ist $ G $ somit abelsch.
		 Damit besitzt $ G $ vier Konjugationsklassen.
		 Dies ist ein Widerspruch.
		 Nun fehlt noch $ l \neq 0 $ und $ m \neq 0 $.
		 Hier folgt direkt mit
		 \begin{align*}
		 p^k = 1 + p^l + p^m
		 \Leftrightarrow 
		 p^k - p^l - p^m  = 1
		 \Rightarrow 
		 p \mid 1
		 \end{align*}
		 ein Widerspruch.
	\end{enumerate}
\end{loes}

\begin{loes}\hypertarget{loes:2.8}\
	\begin{enumerate}
		\item[a)]
		Wir nehmen an der Index von $ Z(G) $ ist gleich $ p $.
		Nun folgt mit dem guten alten Lagrange, dass die
		Faktorgruppe $ | G/Z(G) | = p  $ zyklisch ist.
		Damit ist diese auch abelsch
		und es folgt $ G = Z(G) $.
		Aber dann müsste der Index von $ Z(G) $ gleich $ 1 $ sein.
		Somit war unsere Annahme falsch.
		
		\item[b)]
		Nach Lagrange muss $ |Z(G) |  $ teilt $ | G| $ gelten.
		Damit haben wir $ 1 $, $ p $ oder $ p^2 $ als Möglichkeiten.
		Nach  \ref{skript:2.10}  fällt die erste davon raus.
		Die zweite geht aufgrund der a) nicht.
		Für die dritte folgt $ Z(G) = G $, womit $ G $ abelsch ist. 
	\end{enumerate}
\end{loes}

\begin{loes}\hypertarget{loes:2.9}\
	\begin{enumerate}
		\item[a)]
		Es folgt direkt, dass $ \id.x = x $ für alle $ x \in X $ gilt.
		Seien nun $ \sigma, \tau \in S_5 $ und $ x =(y_1,y_2) $ beliebig.
		Dann gilt
		\begin{align*}
		\sigma.(\tau.(y_1, y_2))
		= \sigma.(\tau(y_1),\tau(y_2)
		= (\sigma \circ \tau(y_1),\sigma \circ \tau(y_2))
		=(\omega \circ \tau).(y_1,y_2),
		\end{align*} 
		womit dies eine Gruppenoperation von $ G $ auf $ X $ ist.
		
		\item[b)]
		Wir wollen zeigen, dass es zwei Bahnen gibt.
		Sei $ i \in Y $ beliebig.
		Für alle $ j \in Y $ existiert mindestens ein $ \sigma \in S_5 $ 
		mit $ \sigma(i) = j $.
		Da Permutationen bijektiv sind, folgt
		\begin{align*}
		i \neq j &\Rightarrow \sigma(i) \neq \sigma(j)\\
		i = j &\Rightarrow \sigma(i) = \sigma(j)
		\end{align*}  
		für alle $ \sigma \in S_5 $.
		Damit sind 
		\begin{align*}
		O_{(1,1)} &= \lbrace (i,i) \ | \ i \in Y \rbrace\\
		O_{(1,2)} &= \lbrace (i,j) \ | \ i,j \in Y \wedge i \neq j\rbrace
		\end{align*}
		die gesuchten Bahnen.
		
		\item[c)]
		Für $ x_1 =(5,5) $ und $ x_2 = (4,5) $ gelten
		\begin{align*}
		\Stab_G(x_1) \cong S_4 \quad 
		\Stab_G(x_2) \cong S_3.
		\end{align*}
		Dies steht direkt nach dem Aufschreiben da.
		Wer lustig drauf ist, kann sich natürlich die Isomorphismen aufschreiben.
		Der gesuchte Zusammenhang ist 
		\begin{align*}
		|O_x| = \frac{|G|}{|\Stab_G(x)|}
		\end{align*}
		für $ x \in X $.
		Damit ergeben sich auch
		\begin{align*}
		|O_{(1,1)}| = 5 \quad \wedge \quad |O_{(1,2)}| = 25,
		\end{align*}
		was uns die Existenz von zwei Bahnen nochmal bestätigt.
	\end{enumerate}
\end{loes}
\newpage
\section{Lösungen zum Abschnitt 3}

\begin{loes}\hypertarget{loes:3.1}\
	\begin{enumerate}
		\item[a)]
		 Angenommen $ h $ ist ein Isomorphismus.
		 Dann ist $ h $ bijektiv mit Umkehrabbildung $ h^{-1} $.
		 Wir müssen also nur noch zeigen, dass $ h^{-1} $
		 ein Gruppenhomomorphismus ist.
		 Sei $ a,b \in H $ beliebig.
		 Da $ h $ bijektiv ist, finden wir eindeutige $ x,y \in G $, sodass 
		 $ h(x) = a  $ und $ h(y) = b $ gilt.
		 Damit erhalten wir mit
		 \begin{align*}
		 h^{-1}(a \cdot b)
		 = h^{-1}(h(x) \cdot h(y))
		 =h^{-1}(h(x \cdot y))
		 = x \cdot y
		 = h^{-1}(a) \cdot h^{-1}(b)
		 \end{align*}
		 die Homomorphismuseigenschaft.
		 
		 Nun nehmen wir an, dass ein $ h^\prime $ mit den geforderten Eigenschaften existiert und müssen die Bijektivität von $ h $ zeigen.
		 \begin{itemize}
		 	\item $ h $ ist injektiv:
		 	Sei $ h(x)  = h(y)$.
		 	Mit $ h^\prime \circ h = \id_G $ folgt durch
		 	\begin{align*}
		 	h^\prime \circ h(x) = h^\prime \circ h(y)
		 	\Rightarrow 
		 	x = y
		 	\end{align*}
		 	die Injektivität.
		 	
		 	\item $ h $ ist surjektiv:
		 	Sei $ y \in H  $ beliebig.
		 	Wir müssen zeigen, dass ein $ x \in G $ mit $ h(x) = y $ existiert.
		 	Dies erhalten wir sofort durch
		 	\begin{align*}
		 	\id_H(y) = h (\underbrace{h^\prime(y)}_{x:=}) = h(x) = y
		 	\end{align*}
		 	und sind fertig.
		 \end{itemize}
		 Also ist $ h $ bijektiv und somit ein Isomorphismus.
		 
		 \item[b)] 
		 Wir werden hier zwei Gegenbeispiele angeben.
		 Zuerst betrachten wir 
		 \begin{align*}
		 G &= A_4\\
		 H &= \lbrace \id , (12)(34),(13)(24),(14)(23) \rbrace
		 K &= \lbrace \id , (12)(34) \rbrace
		 \end{align*}
		 und mit \ref{skript:4.6} gilt $ H \nt G $.
		 Wegen $ [K:H] = 2 $ folgt auch $ K \nt H $ und mit
		 \begin{align*}
		 (123)(12)(34)(132) = (13) (24) \notin K
		 \end{align*}
		 folgt $ K \ntrianglelefteq G $.
		 Nun betrachten wir 
		 \begin{align*}
		 G  = D_4 &= \langle r,s \rangle\\
		 H   &= \langle r^2,s \rangle\\
		 K &= \langle s \rangle
		 \end{align*}
		 und es gelten
		 $ [G : K] = 2 $ bzw. $ [K:H] = 2 $.
		 Also folgt $ G \nt K $ und $ K \nt H $.
		 Jedoch gilt
		 \begin{align*}
		 r K &= \lbrace r, rs \rbrace\\
		 K r &= \lbrace r , sr \rbrace,
		 \end{align*}
		 womit $ K \ntrianglelefteq G  $ folgt.
		 
	\end{enumerate}

\end{loes}

\begin{loes}\hypertarget{loes:3.2}\
	\begin{enumerate}
		\item[a)]
		Wir wählen $ g \in G $ beliebig.
		Dann gilt
		\begin{align*}
		g = s_1 \cdots s_r
		\end{align*}
		für $ r \geq 1 $, $ 1 \leq i \leq r $ und $ s_i \in S \cup S^{-1} $.
		Mit unseren Voraussetzungen folgt direkt
		\begin{align*}
		\phi(g)
		= \phi(s_1) \cdots \phi(s_r)
		= \psi(s_1) \cdots \psi(s_r)
		= \psi(g)
		\end{align*}
		und wir sind fertig.
		Wir sollten nur kurz aufpassen falls $ r \in S^{-1} $ ist.
		Dann existiert ein $ s \in S $ mit $ s^{-1} = r $ und wir beheben mit 
		\begin{align*}
		\phi(r) = \phi(s^{-1}) = \phi(s)^{-1} = \psi(s)^{-1} = \psi(r)
		\end{align*}
		das Problem.
		
		\item[b)]
		Es gibt hier zwei Möglichkeiten, einmal den trivialen Homomorphismus und
		\begin{align*}
		\varphi : S_3 \to C_2 = \lbrace e , g \rbrace, \
		\sigma \mapsto
		\begin{cases}
		e &  \text{, falls} \ \ord(\sigma) \neq 2\\
		g &  \text{, falls} \ \ord(\sigma) = 2
		\end{cases}.
		\end{align*}
		Sei $ \alpha \in S_3 $ mit $ \ord(\alpha) = 3 $ und $ \psi $ ein beliebiger Homomorphismus von $ S_3 $ nach $ C_2 $.
		Die Elemente der Ordnung $ 3 $ müssen immer auf $ e $ abgebildet werden, ansonsten würde man mit 
		\begin{align*}
		\psi(\sigma^2) = g\\
		\psi(\sigma) \cdot \varphi(\sigma) = e
		\end{align*}
		einen Widerspruch erhalten.
		Die Elemente der Ordnung $ 2 $ müssen entweder alle auf $ e  $ oder alle auf $ g $ abgebildet werden, denn das Produkt von $ 2 $-Zykeln ist entweder ein $ 3 $-Zykel oder die Identität.
	\end{enumerate}
\end{loes}

\begin{loes}\hypertarget{loes:3.3}\
	\begin{enumerate}
		\item[a)]
		Zuerst betrachten wir den natürlichen Homomorphismus
		\begin{align*}
		\varphi : G \to G/N , \  g \mapsto gN 
		\end{align*}
		und erhalten mit \ref{skript:3.5}, dass $ \varphi^{-1}(L) \nt G  $ gilt.
		Wegen $ \Ker \varphi = N \subseteq \varphi^{-1}(L) $ setzen wir
		\begin{align*}
		K := \varphi^{-1}(L)
		\end{align*}
		und müssen noch zeigen, dass $ L = K/N = \lbrace kN \ | \ k \in K \rbrace $.
		Aufgrund 
		\begin{align*}
		gN \in L 
		\Leftrightarrow 
		g \in \varphi^{-1}(L) = K 
		\Leftrightarrow
		gN \in K/N
		\end{align*}
		folgt dies jedoch sofort.
		
		\item[b)]
		Nun müssen wir zeigen, dass
		\begin{align*}
		\Phi : G/N \to G/K, \ gN \mapsto gK
		\end{align*}
		wohldefiniert ist.
		Sei $ g_1 N  = g_2 N $, dann existiert ein $ n \in N  $
		mit $ g_1 = g_2 \cdot n $ und 
		\begin{align*}
		\Phi(g_1N) = g_1 K = (g_2 n ) K = g_2 K = \Phi(g_2N)
		\end{align*}
		folgt durch $ N \subseteq K $ die Wohldefiniertheit.
		Wir behaupten nun, dass 
		\begin{align*}
		\Ker \Phi = L 
		\end{align*}
		gilt. Die Beziehung $ L \subseteq \Ker \Phi $ steht direkt da.
		Mit 
		\begin{align*}
		gN \in \Ker \Phi 
		\Rightarrow
		\Phi(gN) = gK = K
		\Rightarrow 
		g \in K 
		\Rightarrow
		gN \in L
		\end{align*}
		folgt die andere Richtung.
	\end{enumerate}
\end{loes}
\newpage
\section{Lösungen zum Abschnitt 4}

\begin{loes}\hypertarget{loes:4.1}\
	\begin{enumerate}\
		\item[a)]
		 Wir wissen bereits, dass 
		 \begin{align*}
		 D_8 = \lbrace \id , r , r^2, r^3, s , sr , sr^2, sr^3 \rbrace
		 \end{align*}
		 und $ r^i s = s r^{-i} $ für $ i \in \lbrace 0 , 1 , 2 ,3 \rbrace $ gilt.
		 Durch nerviges Ausrechnen von
		 \begin{align*}
		 &[\id, \id]\\
		 &[r^i,r^j]\\
		 &[r^i,sr^j]\\
		 &[sr^i, sr^j]
		 \end{align*}
		 erhalten wir $ D_8^\prime = \lbrace \id , r^2 \rbrace = \langle r^2 \rangle $.
		 Diese Kommutatorgruppe ist abelsch, damit ist $ D^{(2)}_8 = 1 $.
		 Also ist $ D_8 $ auflösbar.
		 
		 \item[b)] 
		 Wir müssen analog zu a) alle Möglichkeiten für 
		 \begin{align*}
		 Q_8 = \lbrace \pm 1 , \pm i , \pm j , \pm k  \rbrace
		 \end{align*}
		 mit $ i^2 = j^2 = k^2 = i \cdot j \cdot k = -1 $
		 ausprobieren.
		 Damit erhalten wir $ D^\prime_8 = \lbrace \pm 1 \rbrace $.
		 Diese Gruppe ist abelsch, damit ist  $ D_8 $ auflösbar.
		  
		 
	\end{enumerate}
\end{loes}


\newpage
\section{Lösungen zum Abschnitt 5}

\begin{loes}\hypertarget{loes:5.1}\
	Es gilt 
	\begin{align*}
	|G| = 1 + 8 + 18 = 27,
	\end{align*}
	womit $ G $ eine abelsche $ p $-Gruppe ist.
	Damit können wir \ref{skript:5.8} und \ref{skript:5.10}
	anwenden.
	Somit wir erhalten als Möglichkeiten
	\begin{align*}
	G &\cong C_{27} \\
	G &\cong C_3 \times C_3 \times C_3\\
	G &\cong C_9 \times C_3.	
	\end{align*}
	Die ersten beiden können nicht erfüllt sein.
	Die erste enthält Elemente der Ordnung $ 27 $ und
	die zweite nur Elemente der Ordnung $ 3 $.
\end{loes}

\begin{loes}\hypertarget{loes:5.2}\
	Sei $ \overline{a} \in G $ beliebig.
	Dann ist
	\begin{align*}
	\langle \overline{a} \rangle
	= \langle a + p^m \Z \rangle 
	\end{align*}	
	die kleinste Untergruppe, die $ \overline{a} $ enthält.
	Für $ 0 \leq i \leq m $ gilt
	\begin{align*}
	\ggT(a,p^m) = p^i,
	\end{align*}
	falls $ a = k \cdot p^i $, $ p \nmid k $ und $ k \in \Z $.
	Mit dem Lemma von Bezout existieren $ z_1 , z_2 \in \Z  $, sodass 
	\begin{align*}
	z_1 \cdot a + z_2 \cdot p^m  = \ggT(a,p^m) = p^i
	\end{align*} 
	gilt.
	Damit folgt $ p^i \in \langle a + p^m \Z \rangle$ und wegen $ a = k \cdot p^i $
	auch $ \overline{a} \in \langle p^i + p^m \Z  \rangle$.
	Insgesamt erhalten wir also
	\begin{align*}
	\langle a + p^m \Z \rangle = \langle p^i + p^m \Z \rangle.
	\end{align*}
	Wir wissen, dass
	\begin{align*}
	G = U_0 \unrhd U_1 \unrhd \dots \unrhd U_m = 1
	\end{align*}
	mit $ U_{i-1} / U_i \cong C_p $ einfach und $ U_i \nt U_{i-1} $ maximal
	eine Kompositionsreihe ist.
	Nun werden wir zeigen, dass es keine andere gibt.
	Sei $ j > i $, dann folgt mit $ U_i \cong \Z / p^{m-i} \Z $
	\begin{align*}
	U_i / U_j 
	\cong
	(\Z / p^{m-i} \Z) / (\Z / p^{m - j } \Z )
	\cong \Z / p^{j-i} \Z
	\cong U_{m-j+i} \ \text{einfach}. 
	\end{align*}
	Damit hat $ U_{m-j+i} $ nur sich selbst und die $ 1 $ als Normalteiler.
	Es folgt
	\begin{align*}
	m - j + i = m - 1 
	\Leftrightarrow
	j = i +1 ,
	\end{align*}
	womit es keine weitere Kompositionsreihe gibt.
\end{loes}

\begin{loes}\hypertarget{loes:5.2}\
	Es gilt
	\begin{align*}
	[G:N] = \frac{72}{8}  = 9,
	\end{align*}
	womit $ G/N $ eine $ p^2 $ Gruppe ist.
	Nach \ref{aufgabe:2.8} ist diese abelsch.
	Es existiert eine Untergruppe $ K \leq G/N $ mit Ordnung $ 3 $.
	Da $ G/N $ abelsch ist, ist diese auch ein Normalteiler.
	Nun betrachten wir den surjektiven Gruppenhomomorphismus
	\begin{align*}
	\varphi : G \to G/N , \ g \mapsto gN
	\end{align*}
	und wenden den Korresponedzsatz \ref{skript:5.4} an.
	Wir setzen
	\begin{align*}
	U := \varphi^{-1}(K) \leq G
	\end{align*}
	und durch eine kurze Überlegung erhält man auch $ \Ker \varphi \leq U $.
	Da nun $ \varphi(U) = K $ gilt, folgt $ U \nt G $ und
	\begin{align*}
	G/U \cong (G/N) / K 
	\Rightarrow 
	[G : U ] = 3
	\Rightarrow 
	|  U | = 24.
	\end{align*}
	Damit existiert eine solche Gruppe und diese ist normal.
\end{loes}

\begin{loes}\hypertarget{loes:5.3}\
	Sei $ G $ eine abelsche Gruppe der Ordnung $ 200 = 2^3 \cdot 5^2$.
	Da $ G $ abelsch ist, gibt es für $ 2 $ und $ 3 $ jeweils nur ein Sylowuntergruppe.
	Mit \ref{skript:6.9} erhalten wir dann
	\begin{align*}
	G \cong P \times Q
	\end{align*}
	für $ P \in \Syl_5(G)  $ und $ Q \in \Syl_2(G) $.
	Nun gibt es nach \ref{skript:5.8}
	\begin{align*}
	P \cong
	\begin{cases}
	C_5 \times C_5 \\
	C_{5^2} 
	\end{cases}
	\end{align*}
	zwei Isomorphietypen für $ P $.
	Analog erhalten wir mit
	\begin{align*}
	Q \cong
	\begin{cases}
	C_2 \times C_2 \times C_2 \\
	C_{2^2} \times C_2\\
	C_{2^3} 
	\end{cases}
	\end{align*}
	drei Isomorphietypen für $ Q $.
	Daraus erhalten wir mit 
	\begin{align*}
	C_5 \ &\times C_5 \times C_2 \times C_2 \times C_2 \\
	C_5 \  &\times C_5 \times C_{2^2} \times C_2 \\
	C_5 \ &\times C_5 \times C_{2^3}\\
	C_{5^2} &\times C_2 \times C_2 \times C_2 \\
	C_{5^2} &\times C_{2^2} \times C_2 \\
	C_{5^2} &\times C_{2^3}
	\end{align*}
	sechs mögliche Isomorphietypen für $ G $.
	Wenn wir lustig sind, können wir daraus auch die Elementarteilerzerlegungen konstruieren. 
\end{loes}


\newpage
\section{Lösungen zum Abschnitt 6}

\begin{loes}\hypertarget{loes:6.1}\
	Es gilt
	\begin{align*}
	|G| = 80 = 2^4 \cdot 5^1
	\end{align*}
	und nach den Sylowsätzen \ref{skript:6.4} müssen
	\begin{align*}
	n_2(G) &\equiv 1 \mod 2 \quad  \wedge \quad n_2(G) \mid 5\\
	n_5(G) &\equiv 1 \mod 5 \quad  \wedge \quad  n_5(G) \mid 16
	\end{align*}
	erfüllt.
	Angenommen es gibt $ 16  $ verschiedene $ 5 $-Sylowuntergruppen.
	Da Untergruppen der Ordnung $ 5 $ zyklisch sind, besitzen die Gruppen
	in $ \Syl_5(G) $ nur den trivialen Schnitt.
	Damit erhalten wir $ 16 \cdot 4  +1 = 65  $ Elemente.
	Also kann es nur eine $ 2 $-Sylowuntergruppe mit Ordnung $ 16 $ geben, womit
	\begin{align*}
	\Syl_2(G) = \lbrace P \rbrace 
	\Leftrightarrow
	P \nt G
	\end{align*}
	gilt.
	Also ist $ G $ nicht einfach.
 \end{loes}
 
 \begin{loes}\hypertarget{loes:6.2}\
 	\begin{enumerate}
 		\item[a)]
 		Wir nehmen ohne Beschränkung der Allgemeinheit an, dass $ p < q $ gilt.
 		Wegen
 		\begin{align*}
 		n_q(G) \equiv 1 \mod q \quad \wedge \quad n_q(G)  \mid p
 		\end{align*}
 		kann nur $ n_q(G) = 1 $ erfüllt sein.
 		Damit gilt nach \ref{skript:6.5}
 		\begin{align*}
 		\Syl_q(G) = \lbrace Q \rbrace 
 		\Leftrightarrow
 		Q \nt G,
 		\end{align*}
 		womit $ G $ nicht einfach ist.
 		
 		\item[b)]
 		Mit den Voraussetzungen erhalten wir
 		\begin{align*}
 		p \notin 1 + q \Z \quad \wedge \quad q \notin 1 + p \Z,
 		\end{align*}
 		womit $ p \neq n_q(G) $ und $ q \neq n_p(G) $ gilt.
 		Da $ p $ und $ q $ Primzahlen sind, 
 		kann nur
 		\begin{align*}
 		n_p(G) = 1  \quad \wedge \quad n_q(G) = 1 
 		\end{align*}
 		erfüllt sein.
 		Damit folgt dann
 		\begin{align*}
 		\Syl_p(G) = \lbrace P \rbrace &\Leftrightarrow P \nt G\\
 		\Syl_q(G) = \lbrace Q \rbrace &\Leftrightarrow Q \nt G
 		\end{align*}
 		und $ P,Q $ sind aufgrund ihrer Ordnungen zyklisch.
 		Nun gilt noch $ P \cap Q = 1  $ und $ P \cdot Q = G $.
 		Damit folgt dann
 		\begin{align*}
 		G \cong P \times Q \cong C_p \times C_q \cong C_{pq},
 		\end{align*}
 		womit $ G $ zyklisch ist.
 		Eine andere Variante wäre es über den Kommutator zu gehen.
 		Da $ P \cap Q  = 1$ und $ P,Q \nt G $ muss 
 		\begin{align*}
 		[p,q] = 1
 		\end{align*}
 		für beliebige $ p \in P $ und $ q \in Q $ gelten.
 		Wegen $ G = P \cdot Q $ ist $ G $ somit abelsch.
 		Damit folgt dann sofort
 		\begin{align*}
 		(pq)^{\ord(p) \cdot \ord(g)} = 1
 		\end{align*}
 		und wir sind fertig.
 	\end{enumerate}
 \end{loes}
 
 \begin{loes}\hypertarget{loes:6.3}\
 	Wir wissen zunächst, dass
 	\begin{align*}
 	n_p(G) \equiv 1 &\mod p \quad \wedge \quad n_p(G) \mid q \\
 	n_q(G) \equiv 1 &\mod q \quad \wedge \quad n_q(G) \mid p^2
 	\end{align*}
 	erfüllt sein müssen.
 	Nun müssen wir drei Fälle betrachten:
 	\begin{itemize}
 		\item $ q < p $:
 		 In diesem Fall folgt sofort, dass $ n_p(G) = 1 $ sein muss.
 		 Damit gilt 
 		 \begin{align*}
 		 \Syl_p(G) = \lbrace P \rbrace \Leftrightarrow P \nt G,
 		 \end{align*}
 		 womit $ G $ nicht einfach ist und eine normale $ p $-Sylowuntergruppe existiert.
 		 Nach \ref{skript:5.5} sind $ P $ und $ G/P $ auflösbar 
 		 und mit $ \ref{skript:4.11}  $ auch $ G $.
 		 
 		 \item $ q > p^2 $:
 		 Hier folgt wieder sofort, dass $ n_q(G) = 1 $ sein muss.
 		 Mit der selben Argumentation wie im ersten Fall
 		 existiert eine $ q $-Sylowuntergruppe und $ G $ ist auflösbar.
 		 
 		 \item $ p < q < p^2 $:
 		 Aufgrund der Voraussetzungen wissen wir, dass 
 		 $ n_q(G)  \in \lbrace 1 , p , p^2 \rbrace$
 		 gilt und erhalten schon wieder drei Fälle:
 		 \begin{itemize}
 		 	\item $ n_q(G) = 1 $:
 		 	Hierfür einfach die zwei ersten Punkte anschauen.
 		 	
 		 	\item $ n_q(G) = p $:
 		 	Dieser Fall kann wegen
 		 	\begin{align*}
 		 	n_q(G) \equiv 1 \mod q \quad \wedge \quad p<q
 		 	\end{align*}
 		 	nicht erfüllt sein.
 		 	
 		 	\item $ n_q(G) = p^2 $:
 		 	In diesem Fall gibt es $ p^2 $ Untergruppen der Ordnung $ q $.
 		 	Also erhalten wir $ p^2 \cdot (q-1) $ Elemente der Ordnung $ q $
 		 	und es bleiben noch $ p^2 $ Elemente die fehlen.
 		 	Somit ist $ n_p(G) = 1 $.
 		 	Den Rest kennen wir ja schon.
 		 \end{itemize} 
 	\end{itemize}
 	Insgesamt finden wir immer entweder eine normale $ p $-Sylowgruppe oder eine normale $ q $-Sylowgruppe und $ G $ ist auflösbar.
 \end{loes}
 
\begin{loes}\hypertarget{loes:6.4}\
	Wenn wir uns die Eigenschaft \textbf{(3)} aus \ref{skript:6.4} aufschreiben,
	erhalten wir sofort
	\begin{align*}
	n_5(G) = 1 \quad \wedge \quad n_7(G) = 1. 
	\end{align*}
	Damit folgt mit \ref{skript:4.11} und \ref{skript:5.5} auch direkt die Auflösbarkeit.
	Also sind Gruppen der Ordnung $ 6125 $ immer auflösbar.
\end{loes}

\begin{loes}\hypertarget{loes:6.4}\
	\begin{enumerate}
		\item[a)]
		Sei $ |G| = p^a \cdot m $ mit $ \ggT(p, m ) = 1 $.
		Da $p$ kein Teiler von $ [G:N] $ ist, muss
		\begin{align*}
		|N| = p^a \cdot \tilde{m}
		\end{align*}
		mit $ \ggT(p,\tilde{m}) = 1 $ und $ m = c \cdot \tilde{m} $ gelten.
		Also erhalten wir mit
		\begin{align*}
		P \in Syl_p(N) \Rightarrow |P| = p^a
		\quad \wedge \quad
		Q \in Syl_p(G) \Rightarrow |G| = p^a
		\end{align*}
		die gesuchte Aussage.
		
		\item[b)]
		Sei $ P \in \Syl_p(N)$.
		Dann gilt insbesondere auch $ P \leq G $ mit $ |P| = p^a $.
		Nach den Sylowsätzen existiert dann ein $ S \in \Syl_p(G)$
		mit $ P \leq S $.
		Aufgrund von a) folgt somit $ P = S $.
		Nun wählen wir ein $ Q \in \Syl_p(G) $ und wollen zeigen,
		dass $ Q $ in $ \Syl_p(N) $ liegt.
		Da die Menge der Sylowuntergruppen nicht leer ist,
		existiert ein $ P \in \Syl_p(N) $.
		Mit der selben Argumentation wie in der Hinrichtung liegt $ P $ in $ \Syl_p(G) $.
		Also existiert ein $ g \in G $ mit $ g^{-1} P g = Q $.
		Wegen $ N \nt G $ gilt
		\begin{align*}
		g^{-1} P g \in \Syl_p(N)
		\end{align*}
		für alle $ g \in G $, womit auch $ Q $ in $ \Syl_p(N) $ liegt.
		Insgesamt erhalten wir $ \Syl_p(N) = \Syl_p(G)$.
	\end{enumerate}
\end{loes}
\newpage
\section{Lösungen zum Abschnitt 7}

\begin{loes}\hypertarget{loes:7.1}\
	Sei $ R $ ein endlicher Integritätsring.
	Also ist $ R $ ein endlicher, kommutativer und nullteilerfreier Ring mit $ 1 $.
	Wir wählen $ a \in R \setminus \lbrace 0 \rbrace $ beliebig aber fest und wollen zeigen, dass ein multiplikatives Inverses existiert.
	Dafür betrachten wir die Abbildung
	\begin{align*}
	\varphi : R \to R , \ x \mapsto a \cdot x.
	\end{align*} 
	Wegen
	\begin{align*}
	\varphi(x) = \varphi(y)
	\Leftrightarrow 
	a \cdot x = a \cdot y
	\Leftrightarrow
	a \cdot (x - y ) = 0
	\Rightarrow x = y
	\end{align*}
	ist $ \varphi $ injektiv und aufgrund der Endlichkeit von $ R $
	auch surjektiv.
	Also existiert ein $ b \neq 0 $ mit
	\begin{align*}
	\varphi(b) = a \cdot b = 1
	\end{align*}
	und wir sind fertig.
\end{loes}

\begin{loes}\hypertarget{loes:7.2}\
	Zunächst ist $ \Z[\i] $ ein Teilring von $ \C $, womit
	$ \Z[\i] $ ein Integritätsring ist.
	Wir wählen 
	\begin{align*}
	N(n + \i \cdot m) = n^2 + m^2
	\end{align*}
	als Normfunktion.
	Diese ist multiplikativ, das heißt es gilt
	\begin{align*}
	N(a \cdot b) = N(a) \cdot N(b)
	\end{align*}
	für alle $ a , b \in \Z[\i] $.
	Wer Lust und Laune hat kann dies gerne nachrechnen.
	Nun wählen wir $ a := x + \i \cdot y \in \Z[i] $
	und $ 0 \neq b := s + i \cdot t \in \Z[\i] $.
	Damit erhalten wir 
	\begin{align*}
	\frac{a}{b}
	= \dots 
	= \underbrace{\frac{xs +yt}{s^2 + t^2}}_{u:=} 
	+ \i \cdot \underbrace{\frac{ys - xt}{s^2 + t^2}}_{v:=}
	\end{align*}
	mit $ u,v \in \Q $, woraus $ a = b \cdot(u + \i v ) $ folgt.
	Jetzt wählen wir $ m,n \in \Z $
	mit $ | u - m | \leq \nicefrac{1}{2} $ bzw.
	$ |v -n | \leq \nicefrac{1}{2} $.
	Nun setzen wir $ q := m + \i \cdot n $, $ \alpha := u-m $
	und $ \beta := v - n  $.
	Damit folgt dann
	\begin{align*}
	r = a - q \cdot b = (u+\i \cdot v) \cdot b - (m+ \i \cdot n) \cdot b
	= \dots
	=(\alpha +  \i \cdot  \beta) \cdot b.
	\end{align*}
	Falls $ r = 0  $ ist, folgt aufgrund der Nullteilerfreiheit, dass $ \alpha = \beta = 0 $ ist. Also ist auch $ u = m  $ und $ v = n $.
	Für den Fall $ r \neq 0 $ folgt
	\begin{align*}
	N(r) = N((\alpha + \i \cdot\beta) \cdot b )
	= N(\alpha + \i \cdot\beta) \cdot N(b)
	= (\alpha^2 + \beta^2) \cdot N(b)
	\leq \frac{1}{2} \cdot N(b) < N(b).
	\end{align*}
	Insgesamt gilt $ a = q \cdot b + r  $ mit $ r =0  $ oder $ r \neq 0 $ und $ N(r) < N(b) $.
	Es bleiben also noch die Einheiten übrig.
	Für alle $ a \in \Z[\i]^\ast $ gilt
	\begin{align*}
	|a|^2 = N(a) \geq 1 
	\Rightarrow
	|a^{-1} |^2 = |a|^{-2} \leq 1
	\Rightarrow |a| = 1
	\end{align*}
	und durch kurze Überlegungen erhalten wir schnell, dass
	\begin{align*}
	\Z[\i]^\ast = \lbrace \pm 1, \pm \i \rbrace
	\end{align*}
	gilt.
\end{loes}


\newpage
\section{Lösungen zum Abschnitt 8}

\begin{loes}\hypertarget{loes:8.1}\
	\begin{enumerate}
		\item[a)]
		Wir haben bereits gezeigt, dass alle Untergruppen von $ \Z $
		die Form $ m\Z $ besitzen.
		Sei $ m \in \Z $, $ a \in \Z $ und $ b \in m \Z $ beliebig.
		Dann gilt
		\begin{align*}
		a \cdot b = a \cdot k \cdot m \in m \Z
		\end{align*}
		und wir sind aufgrund der Kommutativität von $ \Z $ fertig.
		
		\item[b)]
		Sei $ I \unlhd K[x] $ ein Ideal.
		Wir wählen $ f \in I $ mit $ \Grad(f) $ minimal
		und $ x \in I $ beliebig.
		Nun wenden wir Polynomdivision an und erhalten
		\begin{align*}
		x = \underbrace{q \cdot f}_{\in I} + r , \ \Grad(r) < \Grad(g). 
		\end{align*}
		Also ist auch $ r \in I $ und aus der Minimalität von $ f $ folgt $ r = 0 $.
		Insgesamt gilt $ x = q \cdot f $, womit $ K[x] $ ein Hauptidealring ist.
		
		\item[c)]
		Fehlt noch.
		
		\item[d)]
		Fehlt noch.  
	\end{enumerate}
\end{loes}

\begin{loes}\hypertarget{loes:8.2}\
	Sei $ I $ ein Primideal von $ R $
	und $ a + I,b+I \in R/I $ mit
	\begin{align*}
	0 + I = (a  + I ) \cdot (b + I ) = ab + I.
	\end{align*}
	Daraus erhalten wir mit der Primidealeigenschaft
	\begin{align*}
	a \cdot b \in I
	\Rightarrow
	a \in I \vee b \in I
	\Rightarrow 
	(a + I) = 0 \vee (b + I) = 0
	\end{align*}
	die Nullteilerfreiheit.
	Da $ I \neq R $ ist, ist $ R/I $ ein Integritätsring.
	Umgekehrt nehmen wir nun an, dass $ R/I $ ein Integritätsring ist.
	Sei $ a \cdot b \in I $, dann folgt
	\begin{align*}
	0 + I = ab + I = (a+I)\cdot (b+I)
	\end{align*}
	und aus der Nullteilerfreiheit erhalten wir
	\begin{align*}
	a + I = 0 \vee b + I = 0
	\Rightarrow 
	a \in I \vee b \in I.
	\end{align*} 
	Damit $ I $ ein Primideal.
\end{loes}

\begin{loes}\hypertarget{loes:8.3}\
	Sei $ r \in R $ beliebig
	und $ f,h \in R[x] $ mit $ f \cdot g \in P := \Ker \varphi_r $.
	Dann gilt
	\begin{align*}
	0 = (f \cdot g)(r) = f(r) \cdot g(r)
	\Rightarrow
	f(r) = 0 \vee g(r) = 0
	\Rightarrow
	f \in P \vee g \in P,
	\end{align*}
	womit $ P $ ein Primideal ist.
\end{loes}

\begin{loes}\hypertarget{loes:8.4}\
	Wir verwenden den chinesischen Restsatz.
	Durch 
	\begin{align*}
	17^{68} \mod 10
	\rightarrow
	(17^{68} \mod 2, 17^{68} \mod 5 )
	\rightarrow
	 \dots
	\rightarrow
	(1 \mod 2, 1 \mod 5 )
	\rightarrow
	1 \mod 10
	\end{align*}
	erhalten wir als letzte Ziffer die $ 1 $.
	Nun zum zweiten Teil.
	Hier erhalten wir nach ein wenig herumrechnen
	\begin{align*}
	14^{200} \mod 100
	\rightarrow
	\dots
	\rightarrow
	(0 \mod 4, 1 \mod 25),
	\end{align*}
	wodurch
	\begin{align*}
	1 + 25 \cdot k \equiv 0 \mod 4
	\end{align*}
	erfüllt sein muss.
	Dies tritt zuerst bei $ k= 3 $ auf.
	Damit sind die letzten beiden Ziffern $ 76 $.
\end{loes}
\newpage
\section{Lösungen zum Abschnitt 9}

\begin{loes}\hypertarget{loes:9.1}\
	Sei $ I $ ein Primideal von $ R $.
	Damit gilt $ I = (p) $ für ein Primelement $ p $.
	Wir betrachten nun ein maximales Ideal $ (a) $
	mit $ I \subseteq (a) \subsetneq R $.
	Damit folgt $ p = r \cdot a $  mit $ a \notin R^\ast $ 
	Da in einem Hauptidealbereich alle Primelemente irreduzibel sind, 
	erhalten wir $ r \in R^\ast $.
	Also folgt $ (p) = (a) $, womit $ I $ maximal ist.
\end{loes}

\begin{loes}\hypertarget{loes:9.2}\
	Sei $ R[x] $ ein Hauptidealring.
	Wir betrachten den Einsetzungshomomorphismus $ \varphi_0 $
	für $ \varphi = \id_R $.
	Dann ist $ I := \Ker \varphi_0 $ ein Primideal.
	Da $ R[x] $ ein Hauptidealbereich ist, ist $ I $ auch maximal.
	Mit dem Homomorphiesatz für Ringe und der Surjektivität von $ \varphi_0 $ folgt
	\begin{align*}
	R = \Bild \varphi_0 \cong R/I,
	\end{align*}
	womit $ R $ ein Körper ist.
	Nun nehmen wir an, dass $ R $ ein Körper ist.
	Dann ist $ R[x] $ euklidisch und somit auch ein Hauptidealring.
\end{loes}

\begin{loes}\hypertarget{loes:9.3}\
	\begin{enumerate}
		\item[a)]
		Wegen 
		\begin{align*}
		2 = (1-i) \cdot (1 + i)
		\end{align*}
		ist $ \Z[\i] / 2 \Z[\i] $ nicht nullteilerfrei und somit kein Körper.
		
		\item[b)]
		Zunächst zeigen wir, dass $ 3\Z[\i] = (3) $ ein maximales Ideal ist.
		Wir wissen, dass $ \Z[\i] $ ein euklidischer Ring ist, womit $ \Z[\i] $
		auch Hauptidealring ist.
		Insgesamt ist $ \Z[\i] $ ein Hauptidealbereich.
		Nun betrachten wir ein $ a \notin R^\ast $ mit
		\begin{align*}
		(3) \subseteq (a) \subseteq \Z[i]
		\end{align*}
		und mit $ 3 $ irreduzibel folgt $ (3) = (a) $,
		Also ist $ (3) $ maximal und $ \Z[\i] / 3 \Z[\i] $ ein Körper mit den Elementen
		\begin{align*}
		\lbrace
		\overline{0},
		\overline{1},
		\overline{2},
		\overline{\i},
		\overline{2\i},
		\overline{1+\i},
		\overline{2+\i},
		\overline{1+2\i},
		\overline{2+2\i}
		\rbrace.
		\end{align*}
		Alternativ könnte man zeigen, dass $ \Z[\i] / 3 \Z[\i]^\ast = \Z[\i] / 3 \Z[\i]\setminus \lbrace 0 \rbrace $ gilt, denn $ \Z[\i] / 3 \Z[\i]$ ist bereits kommutativ.
		
		\item[c)]
		Sei $ \Z[\i] / n \Z[\i]  $ ein Körper.
		Wir nehmen an, dass $ n $ keine Primzahl ist oder
		$ n = a^2 + b^2 $ für $ a, b \in \Z $ gilt.
		Beide Fälle können nicht eintreten, da wir ansonsten jeweils einen direkten
		Widerspruch zur Nullteilerfreiheit erhalten.
		Damit ist unsere Annahme falsch und es gilt das Gegenteil.
		Sei nun $ n $ eine Primzahl und $ n \neq a^2 +b^2 $ für $ a,b \in \Z $.
		Wegen $ n \neq a^2 +b^2 $  ist $  \Z[\i] / n \Z[\i]   $ nullteilerfrei.
		Und da $ n $ prim ist, ist $ (n)  $ maximal und $ \Z[\i] / n \Z[\i]  $ somit ein Körper.
		
	\end{enumerate}
\end{loes}

\begin{loes}\hypertarget{loes:9.4}\
	\begin{enumerate}
		\item[a)]
		Wir ersparen uns zu zeigen, dass $ \beta $ multiplikativ ist.
		Wer Lust darauf hat kann dies gerne machen.
		Mit 
		\begin{align*}
		\beta(z) = 1 
		\Leftrightarrow
		z \in \Z[\sqrt{-5}]^\ast = \lbrace \pm 1 \rbrace
		\end{align*}
		erhalten wir die für die Aufgabe wichtige Äquivalenz.
		\begin{itemize}
			\item $ 2 $:
			Sei $ 2 = x \cdot y $ mit $ x,y \in \Z[\sqrt{-5}] $.
			Damit gilt dann
			\begin{align*}
			\beta(2) = 4 = \beta(x) \cdot \beta(y),
			\end{align*} 
			womit $ \beta(x) \in \lbrace 1, 2, 4 \rbrace $ sein kann.
			Falls $ \beta(x) = 1 $ gilt, ist $ x \in \Z[\sqrt{-5}]^\ast $ und $ \beta(y) = 4 $.
			Analog gehen wir vor, falls $ \beta(x) = 4 $ ist.
			Der Fall $ \beta(x) = 2 $ kann nicht erfüllt sein, denn ansonsten müsste
			\begin{align*}
			\beta(x) = 2 = a^2 
			\end{align*}
			gelten.
			Insgesamt ist $ 2 $ irreduzibel in $ \Z[\sqrt{-5}] $.
			Jedoch kann $ 2 $ nicht prim sein.
			Wir betrachten
			\begin{align*}
			2 \mid 6 =\underbrace{(1 + \sqrt{-5})}_{a_1:=}  \cdot \underbrace{(1 - \sqrt{-5})}_{a_2:=}
			\end{align*}
			und es gilt $ \beta(2) = 4 $ und $ \beta(a_1) = \beta(a_2) = 6 $.
			Damit wird $ \beta(a_1) $ und $ \beta(a_2) $ nicht von $ \beta(2)$ geteilt.
			Da $ \beta $ multiplikativ ist, ist $ 2 $ nicht prim.
			
			\item $ 3 $:
			Das Vorgehen ist analog zur $ 2 $.
			Wir müssen nur die Zahlen austauschen.
			
			\item $ 1 + \sqrt{-5} $:
			Sei $ 1 + \sqrt{-5} = x \cdot y $
			Es gilt $ \beta(1 + \sqrt{-5}) = 6 $.
			Damit kann $ \beta \in \lbrace 1, 2, 3 ,6 $ sein.
			Wenn $ \beta(x)  = 1 $ oder $ \beta(x) = 6  $ folgt direkt,
			dass $ x \in \Z[\sqrt{-5}] ^\ast$ ist.
			Die anderen beiden Fälle führen analog zur $ 2 $ zu einem Widerspruch.
			Angenommen $ 1 + \sqrt{-5} $ ist prim.
			Dann gilt
			\begin{align*}
			1 + \sqrt{-5} \mid 2 \cdot 3
			\Rightarrow
			1 + \sqrt{-5} \mid 2  \vee 1 + \sqrt{-5} \mid 3, 
			\end{align*}
			womit ein $ a + b \cdot \sqrt{-5} \in \Z[\sqrt{-5}] $ mit
			\begin{align*}
			(1+ \sqrt{-5}) \cdot a + b \cdot \sqrt{-5}
			= a - 5\cdot b + \sqrt{-5} \cdot (a + b) = 5
			\end{align*}
			existiert.
			Nach kurzem Überlegen erhalten wir, dass dies nicht erfüllt sein kein.
			Das Vorgehen bei $ 1 + \sqrt{-5} \mid 3 $ ist vollkommen analog.
			
			\item $ 1 - \sqrt{-5} $: 
			Analog zu $ 1 + \sqrt{-5} $. 
		\end{itemize}
		
		\item[b)]
		Steht sofort da, dass Ergebnis steht in der in a) aufgeführten Äquivalenz.
		
		\item[c)] 
		Da für einen Hauptidealbereich gelten muss, dass prim irreduzibel entspricht, kann $ \Z[\sqrt{-5}] $ kein Hauptidealring und somit auch kein euklidischer Ring sein.
	\end{enumerate}
\end{loes}
\newpage
\section{Lösungen zum Abschnitt 10}

\begin{loes}\hypertarget{loes:10.1}\
	\begin{enumerate}
		\item[a)]
		Wir wenden Eisenstein für $ p = 2 $ an.
		Der Leitkoeffizient wird nicht von $ 2 $ geteilt,
		alle anderen Koeffizienten jedoch schon und es gilt $ 4 \nmid 2 $.
		Damit ist das Polynom irreduzibel.
		
		\item[b)]
		Dasselbe wie in a), nur für $ p=3 $.
		
		\item[c)]
		Dieses Polynom besitzt keine ganzzahligen Nullstellen.
		Da der Grad des Polynoms gleich drei ist, ist es irreduzibel.  
		
		\item[d)]
		Wir wenden das Reduktionskriterium für $ p=2 $ an.
		Das Polynom
		\begin{align*}
		x^4+x+1
		\end{align*}
		besitzt über $ \Z / 2 \Z $ keine Nullstellen.
		Es könnte also nur noch in Polynome vom Grad zwei zerfallen. 
		Da $ \Z / 2 \Z $ faktoriell ist, gilt dies auch für $ \Z/2 \Z[x] $.
		Also muss das Polynom ein Produkt aus irreduziblen Elementen sein,
		jedoch gilt
		\begin{align*}
		x^4+x+1 \neq (x^2 + x +1)^2 = x^4 +x^2 + 1
		\end{align*}
		und $ x^2 + x +1 $ ist das einzige irreduzible Polynom vom Grad zwei.
		Damit ist unser Polynom irreduzibel.
	\end{enumerate}
\end{loes}

\begin{loes}\hypertarget{loes:10.2}\
	\begin{enumerate}
		\item[a)]
		Es gilt
		\begin{align*}
		x^4+ x^2 +1
		=(x^2 + x +1 )^2 
		\end{align*}
		woraus die Reduziblität folgt.
		
		\item[b)]
		Mit Eisenstein für $ p = 2 $ erhalten wir die Irreduziblität in
		$ \Z[x] $ und nach dem Satz von Gauß folgt diese auch für $ \Q[x] $.
		
		\item[c)]
		Wir wenden das Reduktionskriterium für $ p = 2 $ an.
		Durch schnelles Nachrechnen sehen wir, dass
		\begin{align*}
		 x^6 + x^3 + 1		
		\end{align*} 
		keine Nullstellen besitzt.
		Das Polynom kann also noch zwei irreduzible Polynome vom Grad drei,
		vom Grad zwei und vier oder drei irreduzible Polynome vom Grad zwei
		zerfallen.
		Nach dem Ausprobieren aller Möglichkeiten, erfahren wir, dass dies nicht möglich ist.
		Also ist das Polynom irreduzibel.  
	\end{enumerate}
\end{loes}

\begin{loes}\hypertarget{loes:10.3}\
	\begin{enumerate}
		\item[a)]
		Wir nehmen an, dass
		\begin{align*}
		\prod \limits_{i=1}^n (x-a_i) -1
		\end{align*}
		reduzibel ist.
		Dann folgt 
		\begin{align*}
		f = g \cdot h
		\end{align*}
		mit $ \Grad(f) = n $ und $ \Grad(g),\Grad(h) < n $.
		Weiter erhalten wir 
		\begin{align*}
		f(a_i) = -1
		\Rightarrow
		h(a_i) \cdot g(a_i) = -1
		\Rightarrow
		h(a_i) = -g(a_i) \wedge h(a_i), g(a_i) \in \Z[x]^\ast
		\end{align*}
		für alle $ i \in \lbrace 1,\dots, n \rbrace $.
		Wegen $ n \geq \Grad(g)  + 1$ und $ n \geq \Grad(h)  +  1$
		folgt somit
		\begin{align*}
		h(x) = - g(x)
		\Rightarrow
		f(x) = -h^2(x)
		\Rightarrow
		\Grad(h) = \frac{n}{2}
		\end{align*}
		für alle $ x \in \Z $.
		Nun ist der Leitkoeffizient unseres Polynoms eins.
		Sei $ b $ der Leitkoeffizient von $ h $.
		Durch
		\begin{align*}
		-b^2 \cdot x^{\frac{n}{2} + \frac{n}{2}}
		= 1 \cdot x^n
		\end{align*}
		erhalten wir einen Widerspruch.
		Also war unsere Annahme falsch und unser Polynom ist irreduzibel.
		
		\item[b)]
		Zuerst betrachten wir zerlegbare normierte Polynome zweiten Grades.
		Diese können wir durch
		\begin{align*}
		(x-a)(x-b)
		\end{align*}
		für $ a,b \in \F_p $ zerlegen.
		Wir haben also
		\begin{align*}
		\binom{p+2-1}{2} =\frac{(p+1)p}{2}
		\end{align*}
		mögliche Zerlegungen, denn die Reihenfolge ist egal und wir können zurücklegen.
		Nun gibt es insgesamt $ p^2 $ Polynome der Form
		\begin{align*}
		x^2 + c\cdot x + d
		\end{align*}
		für $ c,d \in \F_p $.
		Damit erhalten wir 
		\begin{align*}
		p^2 - \frac{(p+1)p}{2}
		\end{align*}
		irreduzible Polynome von Grad zwei.
	\end{enumerate}
\end{loes}

\begin{loes}\hypertarget{loes:10.4}\
	\begin{enumerate}
		\item[a)]
		Seien $ a, b \in R $ koprim und $ I := (a,b) \unlhd R $.
		Da $ R $ ein Hauptidealbereich ist, existiert ein $ x \in R $
		mit 
		\begin{align*}
		(x) = (a,b)
		\Rightarrow
		(a) \subseteq (x) \wedge (b) \subseteq (x)
		\Rightarrow
		x \mid a \wedge x \mid b
		\Rightarrow
		x \in R^\ast
		\end{align*}
		und es existieren $ r ,s \in R $, sodass
		\begin{align*}
		x = a \cdot r + b \cdot s
		\end{align*}
		gilt. Da $ R $ kommutativ ist erhalten wir
		\begin{align*}
		1 = a \cdot \underbrace{r \cdot x^{-1}}_{c:=}  + b \cdot 
		\underbrace{s \cdot x^{-1} }_{d:=}
		\end{align*}
		als Resultat.
		
		\item[b)]
		Sei $ I \unlhd F $ mit $ \nicefrac{a}{b}  \in I$ .
		Da $ R $ ein Hauptidealring ist, existiert ein $ x \in R $ mit
		$ (x)_R  =(a,b)_R$ und es folgt
		\begin{align*}
		a = a^\prime \cdot x, \ b = b^\prime \cdot x
		\Rightarrow
		R (a^\prime \cdot x) +R (b^\prime \cdot x) = R x
		\Rightarrow
		R a^\prime + R b^\prime = R
		\Rightarrow
		(a^\prime, b^\prime) = R.
		\end{align*}
		Gilt nun $ x^\prime \mid a^\prime $ und $ x^\prime \mid b^\prime $,
		so folgt
		\begin{align*}
		(a^\prime, b^\prime) \subseteq (x^\prime)
		\Rightarrow 
		(x^\prime) = R
		\Rightarrow
		x^\prime \in R^\ast,
		\end{align*}
		womit $ a^\prime $ und $ b^\prime $ koprim sind.
		Nun gilt 
		\begin{align*}
		\frac{a}{b} = \frac{a^\prime}{b^\prime}
		\end{align*}
		und mit a) existieren $ c,d \in R $, sodass
		\begin{align*}
		1 = a^\prime \cdot c + b^\prime \cdot d
		\Rightarrow
		\frac{1}{b^\prime} = \frac{a^\prime}{b^\prime} \cdot c + d
		\Rightarrow
		\frac{1}{b^\prime} \in F
		\end{align*}
		folgt.
		Damit erhalten wir direkt
		\begin{align*}
		\left(\frac{a^\prime}{b^\prime}\right)_F = (a^\prime)_F.
		\end{align*}
		Diese Aussage wenden wir nun auf
		\begin{align*}
		\left(\frac{a_1}{b_1}\right) 
		\subseteq
		\left(\frac{a_1}{b_1}, \frac{a_2}{b_2}\right) 
		\subseteq
		\dots
		\subseteq
		I \unlhd F
		\end{align*}
		an und erhalten
		\begin{align*}
		(a_1^\prime) 
		\subseteq
		(a_1^\prime, a_2^\prime)
		\subseteq
		\dots
		\subseteq
		I \unlhd F.
		\end{align*}
		Wir betrachten nun $ J \unlhd R $ mit
		\begin{align*}
		J = (a_1\prime)_R \cup (a_1^\prime,a_2^\prime)\cup \dots
		=(z_1)_R \cup (z_2)_R \cup \dots = (z)_R,
		\end{align*}
		womit ein $ i \in \N $ mit $ z \in (z_i) $ existiert.
		Damit folgt dann
		\begin{align*}
		(z_i) = (z_{i+1}) = \dots
		\end{align*}
		und es gilt
		\begin{align*}
		a_i^\prime \in (z)_R \subseteq (z)_F
		\Rightarrow 
		I = (z)_F.
		\end{align*}
	\end{enumerate}
\end{loes}
\newpage
\section{Lösungen zum Abschnitt 11}

\begin{loes}\hypertarget{loes:11.1}\
	\begin{enumerate}
		\item[a)]
		Zunächst bemerken wir, falls $ d $ teilt $ n $ gilt, folgt
		\begin{align*}
		E_d \subseteq E_n \Rightarrow E_d^\ast \subseteq E_d \subseteq E_n.
		\end{align*}
		Nun wählen wir $ \alpha \in E_n $ beliebig
		und setzen $ d = \ord(\alpha) $.
		Damit folgt dann $ \alpha \in E_d^\ast $ und mit Lagrange gilt
		\begin{align*}
		d \mid n 
		\Rightarrow 
		E_n = \biguplus \limits_{d \mid n} E_d^\ast.
		\end{align*}
		Die Vereinigung muss disjunkt sein, sonst gäbe es Elemente mit verschiedenen Ordnungen.
		Nun gilt
		\begin{align*}
		x^n -1 
		= \prod \limits_{\alpha \in E_n} (x - \alpha)
		= \prod \limits_{d \mid n} \cdot \left( \prod \limits_{\alpha \in E_d^\ast} (x- \alpha) \right)
		=\prod \limits_{d \mid n} \Phi_d
		\end{align*}
		und damit folgt
		\begin{align*}
		n = \Grad(x^n - 1)
		=\Grad\left(\prod \limits_{d \mid n} \Phi_d\right)
		= \sum \limits_{d \mid n} \Grad(\Phi_d)
		= \sum \limits_{d \mid n} \Phi_d.
		\end{align*}
		
		\item[b)]
		Die zweite Aussage wurde bereits in a) gezeigt.
		Die Erste werden wir mit Induktion zeigen.
		Für $ n=1 $ folgt $ \Phi_1 = x - 1 \in \Z[x] $.
		Sei nun $ n > 1 $, dann ist nach Induktionsvoraussetzung $ \Phi_d \in \Z[x] $
		für $ d < n $.
		Nun gilt
		\begin{align*}
		x^n - 1 
		=\prod \limits_{d \mid n} \Phi_d
		= \Phi_n \cdot 
		\underbrace{\prod \limits_{d \mid n, d < n} \Phi_d}_{=: g \in \Z[x]}
		= \Phi_n \cdot g 
		\end{align*}
		und da $ x^n -1  $ normiert ist folgt mit Divison mit Rest, dass 
		$ \Phi_n \in \Z[x] $ ist. 
	\end{enumerate}
\end{loes}
\newpage
\section{Lösungen zum Abschnitt 12}
\newpage
\section{Lösungen zum Abschnitt 13}

\begin{loes}\hypertarget{loes:13.1}\
	\begin{enumerate}
		\item[a)]
		Zunächst bemerkten wir das $ (L,+) $ eine abelsche Gruppe ist.
		Die Eigenschaften der Skalarmultiplikation stehen aufgrund der
		Teilkörpereigenschaft von $ K $ direkt da.
		
		\item[b)]
		Sei $ K := \Z /2 \Z  $.
		\begin{itemize}
			\item 4 Elemente:
			Wir wählen das Polynom $ f_1 = x^2 + x + 1 \in K[x] $
			und mit \ref{skript:9.10} ist
			\begin{align*}
			\lbrace 1, \alpha \rbrace
			\end{align*}
			eine Basis von $K_1 := K[x] /(f_1) $, wobei $ \alpha = \pi(x) $ gilt.
			Es folgt $  |K_1 : K |=  \dim K_1 = 2 $, womit $ |K_1| = 4 $ gilt.
			Durch 
			\begin{align*}
			K_1 = \lbrace 0 +(f_1), 1 + (f_1), x + (f_1), (x+1) + (f_1) \rbrace
			\end{align*}
			erhalten wir alle Elemente.
			
			\item 8 Elemente:
			Wir wählen das Polynom $ f_2 = x^3 + x + 1 \in K[x] $ und mit
			\ref{skript:9.10} ist
			\begin{align*}
			\lbrace 1, \alpha , \alpha^2 \rbrace
			\end{align*}
			eine Basis von $ K_2 := K[x] / (f_2) $.
			Es folgt $|K_2 : K |= \dim K_2 = 3 $, womit $ |K_2| = 8 $ gilt.
			
			\item 16 Elemente:
			Wir wählen das Polynom $ f_3 = x^4 + x + 1 \in K[x] $
			und mit \ref{skript:9.10} ist
			\begin{align*}
			\lbrace 1, \alpha , \alpha^2 , \alpha^3 \rbrace
			\end{align*}
			eine Basis von $ K_3 := K[x] /(f_3) $.
			Es folgt $ |K_3 : K | = \dim K_3 = 3 $, womit $ |K_3| = 16 $ gilt.
		\end{itemize}
		
		\item[c)]
		Sei $ L $ ein endlicher Körper.
		Damit gilt $ \Char L = p $ für eine Primzahl $ p $.
		Also ist $ \F_p = \Z /p \Z $ der Primkörper von $ L $.
		In der a) haben wir gezeigt, dass $ L  $ ein $ F_2 $-Vektorraum ist, womit
		wegen der Endlichkeit eine endliche Basis
		\begin{align*}
		\lbrace b_1, \dots,  b_k \rbrace
		\end{align*}
		mit $ b_1,\dots,b_n \in L $ und $ k \in \N $ existiert.
		Damit erhalten wir schon $ |L| = p^k $.
	\end{enumerate}
\end{loes}
\printindex
\end{document}