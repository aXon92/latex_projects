\section{Ideale und Ringhomomorphismen}

\begin{df}\label{skript:8.1}
	\
	\begin{enumerate}
		\item[\textbf{(1)}]
			Sei $R$ ein Ring $I$ eine Untergruppe von $(R,+)$.
			Dann heißt $I$
			\begin{itemize}
			\item
			\bi{Linksideal} von $R$, falls $a\cdot b \in I$ für alle $a \in R$ und $b \in I$ gilt.
			\index{Linksideal}
			Wir schreiben dann $ I \unlhd_L R $.
			\item 
			\bi{Rechtsideal} von $R$, falls $b \cdot a \in I$ für alle $a \in R$ und $b \in I$ gilt.
			\index{Rechtsideal}
			Wir schreiben dann $ I \unlhd_R R $.
		
			\item
			\bi{(zweiseitiges) Ideal} von $R$,
			falls $a \cdot b \in I$ und $b \cdot a \in I$ für alle $a \in R$ und $b \in I$ gilt. 		
			\index{Ideal}
			Wir schreiben dann $ I \unlhd R $.
		\end{itemize}
		Falls $R$ kommutativ ist, fallen diese Begriffe zusammen.		
		
		\item[\textbf{(2)}]
		Ist $I$ ein Ideal von $R$, dann ist $(R/ I, + )$ eine abelsche Gruppe.
		Diese wird durch
		\begin{align*}
		(a + I) \cdot (b+ I) := a \cdot b + I
		\end{align*}
		zu einem Ring und wird auch \bi{Faktorring} genannt.
		Um eine Klasse in $R / I $ schreiben wir auch $\overline{a}$ statt $a + I$.
 		\index{Faktorring}
		Nun müssen wir noch zeigen, dass diese Verknüpfung wohldefiniert ist.
		\begin{proof}
		Sei $\overline{a} = \overline{a^\prime}$ und $\overline{b} = \overline{b^\prime}$.
		Dann gilt
		\begin{align*}
		a &= a^\prime +x \\
		b &= b^\prime + y 
		\end{align*}
		für $x,y \in I$ und es folgt
		\begin{align*}
		(a \cdot b) = (a^\prime +x ) \cdot ( b^\prime + y) 
		= a^\prime \cdot b^\prime + \underbrace{x \cdot b^\prime +a^\prime \cdot y + x \cdot y}_{ \in I},
		\end{align*}
		womit $\overline{a \cdot b} = \overline{a^\prime \cdot b^\prime}$.
		\end{proof}
		
	\end{enumerate}		
\end{df}

\begin{genericdf}{Beispiele}\label{skript:8.2}
	\
	\begin{enumerate}
		\item[\textbf{(1)}]
		Sei $R = \Z$, dann ist $I = m \Z$ ein Ideal und $R/I = \Z / m \Z$ der zugehörige Faktorring.
		
		\item[\textbf{(2)}]
		Sei $R $ ein kommutativer Ring und $a \in R$ fest gewählt.
		Dann ist
		\begin{align*}
		(a) := aR = Ra = \lbrace ba \ | \ b \in R \rbrace \unlhd R
		\end{align*}
		das von $a$ \bi{erzeugte Ideal}.
		\index{Ideal!erzeugt}
		Ein solchs Ideal nennen wir auch \bi{Hauptideal}.
		\index{Hauptideal}
		Falls jedes Ideal von $R$ ein Hauptideal ist, nennen wir $R$ einen \bi{Hauptidealring}.
		\index{Hauptideal!ring}
		
		\item[\textbf{(3)}]
		Der Ring $\Z$ ist ein Hauptidealring.
		Alle Körper $K$ sind Hauptidealringe mit den zwei Idealen $(0)$ und $(1) = K$.
		
		\item[\textbf{(4)}]
		Sei $R$ ein kommutativer Ring und $a_1, \dots , a_m \in R$.
		Dann lässt sich durch
		\begin{align*}
		(a_1, \dots, a_m) := \left\lbrace   \sum \limits_{i=1}^m f_i \cdot a_i \ | \ f_i \in R \right\rbrace 
		\end{align*}
		das Hauptideal allgemeiner definieren.
	\end{enumerate}
\end{genericdf}

\begin{sz} \label{skript:8.3}
	Sei $ R $ ein euklidischer Ring.
	Dann ist $ R $ ein Hauptidealring.
\end{sz}

\begin{proof}
	Sei $ I \unlhd R $.
	Das Ideal $ I = 0 $ ist ein Hauptideal, denn $ (0) = 0 $.
	Wir betrachten also $ I \neq 0 $ und setzen
	\begin{align*}
	d = \min \lbrace \nu(a) \ | \ 0 \neq a \in I \rbrace,
	\end{align*}
	wobei $ \nu $ die Norm von $ R $ ist.
	Das Minimum muss existiert, da Mengen aus natürlichen Zahlen immer ein kleinstes Element besitzen.
	Sei nun $ a_0 \in I $ mit $ \nu(a_0) = d$.
	Wir behaupten, dass $ I = (a_0) $ gilt.
	Um dies zu zeigen, wählen wir ein beliebiges $ b \in I $.
	Dann gilt durch Division mit Rest
	\begin{align*}
	b = q \cdot a_0 + r
	\end{align*}
	für $ q,r \in R $ mit $ r = 0 $ oder $ r \neq 0 $ mit $ \nu(r) < \nu(a_0) $.
	Angenommen $ r \neq 0 $, dann gilt
	\begin{align*}
	r = b - q\cdot a_0 \in I
	\end{align*}
	und es ist $ \nu(r) < \nu(a_0) $.
	Dies ist ein Widerspruch zur Minimalität von $ a_0 $.
	Also muss $ r = 0 $ gelten, womit $ b \in (a_0) $ folgt. 
\end{proof}


\begin{df}\label{skript:8.4}
	Seien $R$ und $S$ Ringe.
	Eine Abbildung 
	\begin{align*}
	\varphi : R \to S
	\end{align*}
	heißt \bi{Ringhomomorphismus}, wenn $\varphi$ ein Gruppenhomomorphismus von $(R,+)$ nach $(S,+)$ ist
	\index{Homomorphismus!Ring}	
	und
	\begin{enumerate}
		\item[\textbf{(1)}]
		$\varphi(r_1 \cdot r_2 ) = \varphi(r_1) \cdot \varphi(r_2)$
		
		\item[\textbf{(2)}]
		$ \varphi(1_R) = 1_S$
	\end{enumerate}
	gelten. Der Punkt muss gelten, da bei uns Ringe immer die $1$ enthalten.
	Wenn $\varphi$ bijektiv ist, heißt $\varphi$ \bi{Isomorphismus}.
	\index{Homomorphismus!Isomorphismus}\index{Homomorphismus!isomorph} 
	Wir sagen dann $R$ und $S$ sind \bi{isomorph} und schreiben dafür $R \cong S$.
	Einen Isomorphismus von $R$ nach $R$ nennen wir \bi{Automorphismus}.
	\index{Homomorphismus!Automorphismus}
	Einen Ringhomomorphismus von $R$ nach $R$ nennen wir \bi{Epimorphismus}.
	\index{Homomorphismus!Epimorphismus}
\end{df}

\begin{sz}\label{skript:8.5}
	Seien $R,S$ Ringe und $\varphi :R \to S$ ein Ringhomomorphismus.
	\begin{enumerate}
		\item[\textbf{(1)}]
		Es gelten $\Ker \varphi = \lbrace a \in R \ | \ \varphi(a) = 0 \rbrace \unlhd R$
		und $\Bild \varphi $ ist ein Teilring von $S$.
		
		\item[\textbf{(2)}]\bi{Homomorphiesatz}:\index{Satz!Homomorphie}
		Sei $I = \Ker \varphi$ Dann existiert ein eindeutiger Ringhomomorphismus
		\begin{align*}
		\overline{\varphi} : R/I \to S
		\end{align*}
		mit $ \varphi = \overline{\varphi} \circ \pi$, wobei 
		\begin{align*}
		\pi : R \to R/I, \ a \mapsto \overline{a}
		\end{align*}
		ein surjektiver Ringhomomorphismus ist.
		Außerdem gilt $\Ker \overline{\varphi} = \lbrace \overline{0} \rbrace $ und $\overline{\varphi}$ ist injektiv.
		Damit folgt $R/I \cong \Bild \varphi$. Falls $\varphi$ surjektiv ist, gilt sogar $R/I \cong S$.
	\end{enumerate}
\end{sz}

\begin{proof} \
	\begin{enumerate}
		\item[\textbf{(1)}]
		Zunächst ist uns aus der Gruppentheorie bekannt, dass $(\Ker \varphi, + ) \leq (R,+)$ gilt.
		Sei nun $a \in \Ker \varphi$ und $r \in R$. Wegen
		\begin{align*}
		\varphi(r \cdot a) &= \varphi(r) \cdot \varphi(r) = 0 \\
		\varphi(a \cdot r) &= \varphi(a) \cdot \varphi(r) = 0
		\end{align*}
		folgt $\Ker \varphi \unlhd R$.
		Der zweite Teil befindet sich in den Übungsaufgaben.
		
		\item[\textbf{(2)}]
		In der Gruppentheorie(\ref{skript:5.1}) haben wir alle nötigen Eigenschaften für $\overline{\varphi}$ gezeigt.
		Übrig bleibt die Wohldefiniertheit, was wir als zusätzliche Übungsaufgabe ansehen.
	\end{enumerate}
\end{proof}

\begin{genericdf}{Charakteristik eines kommutativen Rings}\label{skript:8.6}
	Sei $R$ ein kommutativer Ring, $a \in R$ und $m \in \Z$.
	Ist $m > 0 $, dann definieren wir 
	\begin{align*}
	m \cdot a := \underbrace{a + \dots + a}_{m-\text{mal}}
	\end{align*}
	und für $m=0$ dann $0 \cdot a = 0$.
	Sollte $m < 0 $ sein, so definieren wir $m \cdot a := -(-m) \cdot a$.
	Die Abbildung 
	\begin{align*}
	\varphi : \Z \to R, m \mapsto m \cdot 1_R
	\end{align*}
	ist ein Ringhomomorphismus.
	Da $\Z$ ein Hauptidealring ist existiert ein $n \in \Z$, sodass
	\begin{align*}
	\Ker \varphi = (n) = n \Z
	\end{align*}
	gilt.
	Wir nennen $|n|$ die \bi{Charakteristik} von $R$ und schreiben hierfür $\Char R$.\index{Charakteristik}
	Diese ist eindeutig bestimmt, denn $n$ und $-n$ erzeugen dasselbe Hauptideal.
	Nun betrachten wir ein paar Eigenschaften:
	\begin{enumerate}
		\item[\textbf{(1)}]
		Falls $\Char R = 0$ gilt ist $\varphi$ injektiv und es ist $m \cdot 1 \neq 0$ für alle $m \neq 0$.
		Also hat $R$ einen zu $\Z$ isomorphen Teilring.
		\item[\textbf{(2)}]
		
		Ist $\Char R > 0$, dann gilt
		\begin{align*}
		\underbrace{1+ \dots + 1}_{\Char R- \text{mal}}
		\end{align*}
		und $\Char R$ ist die kleinste positive Zahl mit dieser Eigenschaft.
		Nach \ref{skript:8.5} ist 
		\begin{align*}
		\overline{\varphi} : \Z / \Char R \Z \to R
		\end{align*}				
		injektiv, womit $R$ einen zu $\Z / \Char R \Z$ isomorphen Teilring besitzt.
						
		\item[\textbf{(3)}]
		Sei $R$ ein Integritäsring mit $\Char R > 0$.
		Dann ist $p := \Char R $ eine Primzahl.
		Insbesondere gilt dies auch für Körper.
		\begin{proof}
			Angenommen $\Char R = m_1 \cdot m_2$ mit $0 < m_1,m_2 < \Char R$.
			Dann gilt
			\begin{align*}
			\varphi(\Char R\cdot 1) 
			= \varphi(m_1) \cdot \varphi(m_2) 
			\end{align*}
			und es folgt $\varphi(m_1) = 0$ oder $\varphi(m_2) = 0$.
			Dies ist ein Widerspruch zur Minimalität von $\Char R$.
			Damit muss $\Char R$ prim sein.		
		\end{proof}
	\end{enumerate}
\end{genericdf}

\begin{lemma}\label{skript:8.7}
	Sei $R$ ein Integritätsring und $\Char R =: p > 0$.
	Dann ist
	\begin{align*}
	F : R \to R, \ a \mapsto a^p
	\end{align*}
	ein injektiver Ringhomomorphismus und wir nennen $F$ \bi{Frobeniusendomorphismus}. \index{Endomorphismus!Frobenius}
\end{lemma}

\begin{proof}
	Zunächst zeigen wir die Eigenschschaften für einen Ringhomomorphismus.
	Seien $a,b \in R$ beliebig.
	Da $R$ kommutativ ist gilt
	\begin{align*}
	F(a) \cdot F(b) = a^p \cdot b^p = (a\cdot b)^p = F(a \cdot b),
	\end{align*}
	womit die multiplikative Eigenschaft gezeigt ist.
	Nun folgt wieder mit der Kommutativität
	\begin{align*}
	F(a+b) = (a+b)^p
	= \sum \limits_{i=0}^p \binom{p}{i} \cdot a^i \cdot b^{p-i}
	\end{align*}
	Nun ist der Binomialkoeffizent für $ 1 \leq i \leq p-1$ durch $p$ teilbar.
	Da $\Char R = p$ gilt folgt somit
	\begin{align*}
	F(a+b)  = a^p + a^p = F(a) + F(b),
	\end{align*}
	womit $\varphi$ ein Ringhomomorphismus ist.
	Da $R$ ein Integritätsring ist, gilt auch
	\begin{align*}
	\Ker \varphi = \lbrace a \in R \ | \ a^p = 0 \rbrace
	\end{align*}
	und somit ist $\varphi$ injektiv.			
\end{proof}

\begin{genericthm}{Universelle Eigenschaft von Polynomringen}\label{skript:8.8}
	\index{Polynomring!Universelle Eigenschaft}
	Sei $ R $ ein kommutativer Ring, $ R[x] $ der zugehörige Polynomring,
	$ \varphi : R \to S $ ein Ringhomomorphismus und $ s \in S $ fest gewählt.
	Dann existiert ein eindeutiger Ringhomomorphismus
	\begin{align*}
	\varphi_s : R[x] \to S \quad \text{mit} \quad \varphi_s \Big|_R = \varphi \quad
	\text{und} \quad \varphi_s(x) = s.
 	\end{align*}
 	Hierbei identifizieren wir $ R $ mit den Polynomen vom Grad $ 0 $ und $ - \infty $.
	Wir schreiben für $ f \in R[x] $ statt $ \varphi_s(f) $ einfach $ f(s) $.
	Daher kommt dann der Name \bi{Einsetzungshomomorphismus}.\index{Homomorphismus!Einsetzung}
\end{genericthm}

\begin{proof}
	Sei $ f = a_0 +a_1 \cdot x + \dots + a_n \cdot x^n \in R[x]$.
	Wir definieren $ \varphi_s(f) $ durch \\
	$ \varphi(a_0) + \varphi(a_1) \cdot s + \dots + \varphi(a_n) \cdot s^n $.
	Durch schnelles Nachrechnen erkennen wir, dass die Ringhomomorphismuseigenschaften erfüllt sind.
	Um die Eindeutigkeit zu zeigen wählen wir einen anderen Ringhomomorphismus $ \tilde{\varphi_s} $
	mit $ \tilde{\varphi_s}(x) = s $.
	Dann gilt
	\begin{align*}
	\tilde{\varphi_s}(x^m) = (\tilde{\varphi_s}(x))^m = s^m = (\varphi_s(x))^m = \varphi(x^m)
	\end{align*}
	und jedes Element aus $ R[x] $ ist eine $ R $-Linearkombination von
	$ \lbrace x^m \ | \ m \in \N \rbrace$. 
	Also folgt\\
	$ \tilde{\varphi_s} = \varphi_s $.
\end{proof}

\begin{lemma} \label{skript:8.9}
	Sei $ K $ ein Körper und $ f \in K[x] $ mit $ \Grad (f) = n \geq 0 $.
	Dann besitzt $ f $ höchstens $ n $ Nullstellen in $ K $.
\end{lemma}

\begin{proof}
	Für den Fall das $ \Grad(f) = 0 $ ist folgt die Aussage sofort, denn $ f $ ist konstant und ungleich $ 0 $.
	Also betrachten wir den Fall, dass $ \Grad(f) > 0 $ ist.
	Sei $ a \in K  $ eine Nullstelle von $ f $, dann gilt $ f = (x-a) \cdot g $.
	Dies müssen wir jedoch noch zeigen.
	Die Division mit Rest liefert uns 
	\begin{align*}
	f = q \cdot (x-a) + r
	\end{align*}
	mit $ \Grad(r) < \Grad(q) $. Mit dem Einsetzungshomomorphismus aus \ref{skript:8.8} erhalten wir
	\begin{align*}
	0 = f(a) = q(a) \cdot \underbrace{(a-a)}_{=0} + r(a),
	\end{align*}
	woraus $ r(a) = 0 $ folgt.
	Also ist $ a $ eine Nullstelle von $ r $ und es gilt $ \Grad(r) < \Grad(f) $.
	Dann folgt mit Induktion nach dem Grad $ r = (x-a) \cdot \tilde{q} $, womit weiter
	\begin{align*}
	f = (x-a)  \cdot q + (x-a) \cdot \tilde{q}
	= (x-a) \cdot \underbrace{(q + \tilde{q})}_{=g}
	\end{align*}
	folgt.
	Sei nun $ f = (x-a) \cdot g $.
	Dann gilt $ \Grad(g) = n -1  $ und es folgt induktiv, dass $ g $ höchstens $ n-1 $ Nullstellen besitzt.
	Wir betrachten nun eine weitere Nullstelle $ b \neq a $ von $ f $.
	Da $ R $ eine Integritätsring ist, folgt mit
	\begin{align*}
	0 = f(b) = \underbrace{(b-a)}_{\neq 0} \cdot g(b),
	\end{align*}
	dass $ g(b) = 0 $ ist.
	Also hat $ f $ höchstens $ n $ Nullstellen.
\end{proof} 

\begin{generic_no_num}{Bemerkung}
	Wenn $ K $ kein Körper ist, dann ist dieses Lemma im Allgemeinen falsch.
	Das Polynom $ x^2-1 $ besitzt beispielsweise über $ \Z / 8 \Z $ vier Nullstellen.
\end{generic_no_num}

\begin{sz} \label{skript:8.10}
	Sei $ K $ ein Körper und $ G $ eine endliche Untergruppe von $ K^\ast $.
	Dann ist $ G $ zyklisch.
	Sollte $ K $ endlich sein, ist somit $ K^\ast $ zyklisch.
\end{sz}

\begin{proof}
	Sei $ g \in G $ mit maximaler Ordnung $ \ord(g) = n $.
	Da Körper kommutativ bezügliche beiden Verknüpfungen sind muss $ G $ abelsch sein.
	In \ref{skript:5.8} haben wir gezeigt, dass $ \ord(h) | n $ für alle $ h \in G $ gilt.
	Damit sind alle $ h \in G $ eine Nullstelle von $ x^n -1 \in K[x] $.
	Nach \ref{skript:8.9} hat dieses Polynom höchstens $ n $ Nullstellen.
	Also gilt $ |G| \leq n $.
	Jedoch gilt auch $ \ord(g) = n$, womit $ G = \langle g \rangle $ ist.
	Somit ist $ G $ zyklisch.
\end{proof}

\begin{genericthm}{Chinesischer Restsatz}\label{skript:8.11}
	Seien $ n,m \in \N $ und $ \ggT(n,m) = 1 $.
	Dann gilt
	\begin{align*}
	\Z / (mn) \Z \cong \Z /m \Z \times \Z /n \Z.
	\end{align*}
\end{genericthm}

\begin{proof}
	Dieser Beweis knüpft direkt an den von \ref{skript:5.9} an.
	Die Aussage 
	\begin{align*}
	(\Z / mn \Z , +) \cong (\Z / m \Z,+) \times (\Z / n \Z,+) 
	\end{align*}
	wurde in dort schon gezeigt, wobei wir nun die additive Schreibweise gewählt haben.
	Die Abbildung 
	\begin{align*}
	\varphi : k + mn\Z \mapsto (k + m \Z, k + n \Z)
	\end{align*}
	ist der passende Ringhomomorphismus.
	Wir müssen nur noch die Multiplikation überprüfen, was uns aber zur Übung freigestellt ist.
\end{proof}

\begin{generic_no_num}{Beispiel}
	Wir wollen wissen was $ 2^{16}  \mod 15$ ergibt.
	Wenden wir $ \varphi $ aus dem letzten Beweis darauf an, so erhalten wir 
	$(2^{16} \mod 3, 2^{16} \mod 5)$.
	Nun gilt
	\begin{align*}
	2^{16} \mod 3 &= (-1)^{16} \mod 3 \\
	2^{16} \mod 5	&= 4^8 \mod 5 = (-1)^8 \mod 5,
	\end{align*}
	womit
	\begin{align*}
	(2^{16} \mod 3, 2^{16} \mod 5) = (1 \mod 3, 1 \mod 5)
	\end{align*}
	gilt. Das Urbild bezüglich $ \varphi $ ist dann $ 1 \mod 15 $.
	Es gilt als $ 2^{16} \mod 15  = 1$.
\end{generic_no_num}

\begin{lemma}\label{skript:8.12}
	Sei $ \varphi : R \to S $ ein surjektiver Ringhomomorphismus.
	Dann gelten:
	\begin{enumerate}
		\item[\textbf{(1)}]
		Ist $ I \unlhd R $, dann auch $ \varphi(I) \unlhd S $.
		\item[\textbf{(2)}]
		Für $ J \unlhd S $ gilt $ \varphi^{-1}(J) \unlhd R $. 
	\end{enumerate}
\end{lemma}

\begin{proof}
	Den ersten Teil haben wir in den Übungen bewiesen.
	Deswegen kommen wir direkt zu dem zweiten Teil.
	Es gilt $ \varphi^{-1}(J) \leq R $, denn es gilt $(J,+) \leq (S,+)  $.
	Seien nun $ x \in \varphi^{-1}(J) $ und $ r \in R $.
	Wegen $ J \unlhd_L S $ ist
	\begin{align*}
	\varphi(r \cdot x) = \underbrace{\varphi(r) }_{\in S} \cdot \underbrace{\varphi(x)}_{\in J},
	\end{align*}
	womit $ r \cdot x \in \varphi^{-1}(J) $ gilt.
	Durch analoges Vorgehen mit $ J \unlhd_R S $ folgt $ x \cdot r \in \varphi^{-1}(J) $.
\end{proof}

\begin{generic_no_num}{Bemerkung}
	Die zweite Teil des letzten Lemmas gilt auch wenn $ \varphi $ nicht surjektiv ist.
	Die Surjektivität wird nur für den ersten Teil benötigt.
\end{generic_no_num}

\begin{df} \label{skript:8.13}
	Sei $ R $ ein Ring.
	\begin{enumerate}
		\item[\textbf{(1)}]
		Hier ist $ R $ noch zusätzlich kommutativ.
		Sei $ I,J \unlhd R $, dann definieren wir 
		\begin{align*}
		I\cdot J = 
		\lbrace x_1 \cdot y_1 + \dots + x_n \cdot y_n\ | \ x_i \in I, y_i \in J, n \in \N \rbrace 
		\end{align*}
		und durch schnelles Nachrechnen sehen wir $ I \cdot J \unlhd R $.
		Aus diesem Grund nennen wir dies \bi{Produktideal}.
		\index{Ideal!Produkt}
		Wir gehen analog für $ I + J $ vor und nennen dies \bi{Summenideal}.
		\index{Ideal!Summe}
		
		\item[\textbf{(2)}]
		Sei $ I \unlhd R $ mit $ I \neq R $.
		Wir nennen $ I $ \bi{Primideal}, falls für beliebige Ideale
		$ J,K \unlhd R $ \index{Ideal!prim}
		\begin{align*}
		J \cdot K \subseteq I \Rightarrow J \subseteq I \vee K \subseteq I 
		\end{align*}
		gilt.
		
		\item[\textbf{(3)}]
		Wir nennen $ I \unlhd R $ \bi{maximales Ideal},\index{Ideal!maximal}
		falls für $I \neq R$ und $ J \unlhd R $
		\begin{align*}
		I \subseteq J \subseteq R
		\Rightarrow
		I = J \vee J = R
		\end{align*}
		gilt.
	\end{enumerate}
\end{df}

\begin{lemma}\label{skript:8.14}
	Sei $R$ ein kommutativer Ring und $I \unlhd R$, dann gelten:
	\begin{enumerate}
		\item[\textbf{(1)}]
		$R/I \ \text{Integritätsring} \quad \Leftrightarrow \quad  I \ \text{Primideal} $
		
		\item[\textbf{(2)}]
		$R/I \ \text{Körper} \qquad  \qquad \Leftrightarrow \quad  I \ \text{maximales Ideal} $  
	\end{enumerate}
\end{lemma}

\begin{proof}
	Den ersten Teil werden wir in den Übungen zeigen.
	Deswegen gehen wir direkt zum zweiten Teil über.
	Angenommen $R/ I$ ist ein Körper und sei $I \subseteq J \subseteq R$ und $J \unlhd R$.
	Sei $\pi : R \to R/I$ die Quotientenabbildung.
	Da $\pi$ surjektiv ist, gilt $\pi(J) \unlhd R/I $ und aufgrund der Körpereigenschaft
	folgt
	\begin{align*}
	\pi(J) = 0 \vee \pi(J) = R/I
	\Rightarrow
	J = I \vee J = R.
	\end{align*}
	Also ist $I$ ein maximales Ideal.
	Umgekehrt sei $I$ ein maximales Ideal von $R$ und $a \in R/I$.
	Damit ist auch $(a)$ ein Ideal von $R/I$, womit nach \ref{skript:8.2} $\pi^{-1}((a)) \unlhd R $ gilt.
	Wegen $I \subseteq \pi^{-1}(\pi(I)) \subseteq \pi^{-1}((a))$ folgt 
	\begin{align*}
	\pi^{-1}((a)) = I \vee \pi^{-1}((a)) = R,
	\end{align*}
	da $I$ ein maximales Ideal ist. 
	Wir erhalten nun
	\begin{align*}
	\pi^{-1}((a)) &= I \Rightarrow (a) = 0\\
	\pi^{-1}((a)) &= R \Rightarrow (a) = (1) = R/I,
	\end{align*}
	womit $a$ invertierbar ist.
	Dies gilt, da ein $b \in R/I$ existiert mit $a \cdot b = 1$.
	Aufgrund der Kommutativität von $R$ ist jedes Element von $R/I$ invertierbar und $R/I$ ist kommutativ.
	Also ist $R/I$ ein Körper.
\end{proof}

\begin{genericdf}{Beispiele und Bemerkung}\label{skript:8.15} \
	\begin{enumerate}
		\item[\textbf{(1)}]
		Sei $R = \Z$ und $I = p \Z$ mit $p$ prim.
		Dann ist $R/I$ ein Körper und $I$ somit ein maximales Ideal.
		
		\item[\textbf{(2)}]
		Sei $m = m_1 \cdot m_2$ mit $1 \leq m_1, m_2 \leq m$.
		Dann ist $\Z / m \Z$ kein Körper und besitzt sogar Nullteiler.
		Damit is $m \Z$ weder maximal noch ein Primideal.
		
		\item[\textbf{(3)}]
		Jedes maximale Ideal ist ein Primideal, da Körper Integritätsringe sind.
		
		\item[\textbf{(4)}]	
		Das Ideal $0 \Z$ ist ein Primideal, denn $\Z / 0 \Z \cong \Z$ ist ein Integritätsring.	
	\end{enumerate}
\end{genericdf}


