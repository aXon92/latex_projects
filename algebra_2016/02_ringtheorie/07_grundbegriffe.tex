\section{Beispiele und Grundbegriffe zu Ringen}

\begin{df}\label{skript:7.1}
	Wir nennen eine nichtleere Menge $ R $ mit den Verknüpfungen
	$ + : R \times R \to R $ und $ \cdot : R \times R \to R $
	einen \bi{Ring}, falls 			\index{Ring}
	\begin{enumerate}
		\item[\textbf{(1)}]
		$ (R,+) $ eine abelsche Gruppe ist. Wir bezeichnen das neutrale Element mit $ 0 $.
		
		\item[\textbf{(2)}]
		die Multiplikation $ \cdot $ assoziativ ist die Distributivgesetze  gelten.
		Das heißt es gilt
		\begin{align*}
		a \cdot ( b+c) = a\cdot b + a \cdot c 
		\quad
		(a+b) \cdot c = a \cdot c + b \cdot c
		\end{align*}
		für alle $ a,b,c \in R $.
		
		\item[\textbf{(3)}] 
		ein Einselement $ 1 $ bezüglich der Multiplikation existiert. Es gilt also
		\begin{align*}
		1 \cdot a = a \cdot 1 = a
		\end{align*}
		für alle $ a \in R $.
		In der Literatur wird dieser Punkt meist weggelassen und separat definiert.
		In unserem Fall sprechen wir von einem \bi{Ring ohne $ 1 $}, falls dieser Punkt nicht erfüllt ist.
	\end{enumerate}
	Wir nennen $ R $ \bi{kommutativ}, falls			\index{Ring!kommutativ}
	\begin{align*}
	r \cdot s = s \cdot r
	\end{align*}
	für alle $ r,s \in R $ erfüllt ist.
	Eine Teilmenge $ S \subseteq R $ heißt \bi{Teilring} von $ R $, wenn $ S $ mit den beiden Verknüpfungen ein Ring ist und die $ 1 $ von $ S $ der von $ R $ entspricht.		\index{Teilring}
\end{df}

\begin{genericdf}{Bemerkungen} \label{skript:7.2}
	Sei $ R $ ein Ring.
 	\begin{enumerate}
		\item[\textbf{(1)}]
		In einem Ring gilt
		\begin{align*}
		a \cdot 0 =0 \cdot a = 0 
		\end{align*}
		für alle $ a \in R $. Dies erhalten wir sofort durch Anwendung der Gruppenaxiome.
		Außerdem kann in $ R $ gelten, dass $ 1 = 0  $ ist.
		Damit gilt dann jedoch
		\begin{align*}
		a = a \cdot 1 = a \cdot 0 = 0,
		\end{align*}
		woraus $ R = \lbrace 0 \rbrace  $ folgt.
		
		\item[\textbf{(2)}]
		Die Menge 
		\begin{align*}
		R^\ast := \lbrace a \in R \ | \ \exists b \in R : ab = ba = 1 \rbrace
		\end{align*}
		nennen wir die \bi{Einheiten} von $ R $. 
		Diese ist bezüglich der Multiplikation eine Gruppe.
		\index{Einheiten} 
		
		\item[\textbf{(3)}]
		Falls $ R^\ast = R \setminus \lbrace 0 \rbrace $ gilt, 
		so nennen wir $ R $ einen \bi{Schiefkörper} oder \bi{Divisionsring}.
		\index{Schiefkörper} \index{Divisionsring}
		Einen kommutativen Schiefkörper nennen wir \bi{Körper}. \index{Körper}
	\end{enumerate}
\end{genericdf}

\begin{genericdf}{Beispiele} \label{skript:7.3} \
	\begin{enumerate}
		\item[\textbf{(1)}]
		Die ganzen Zahlen $ (\Z,+, \cdot) $ sind ein kommutativer Ring und es gilt
		$ \Z^\ast = \lbrace \pm 1 \rbrace $.
		Die geraden Zahlen $ 2 \Z $ bilden nur einen kommutativen Ring ohne $ 1 $.
		
		\item[\textbf{(2)}]
		Sei $ K $ ein Körper.
		Dann ist $ M_n(K) $ der $ n \times n  $ \bi{Matrizenring} über $ K $.
		\index{Ring!Matrizen}
		Für $ n \geq 2  $ ist $ M_n(K) $ nicht kommutativ
		und es gilt $ M_n(K)^\ast = \Gl_n(K) $.
		Die Menge 
		\begin{align*}
		\mathbb{H} = \Bigg\lbrace \begin{pmatrix}
		u & v \\ \
		\overline{v} & \overline{u}
		\end{pmatrix} 
		\ | \ u, v \in \C \Bigg\rbrace
		\subsetneq
		M_2(\C)
		\end{align*}
		bildet den sogenannten \bi{Quaternionenschiefkörper} und für die
		\index{Schiefkörper!Quaternionen} 
		Quaterionengruppe gilt $ Q_8 \leq H^\ast $.
		Hierfür vergleiche mit der Vortragsübung und Übung.
	\end{enumerate}	
\end{genericdf}

\begin{df}\label{skript:7.4}
	Sei $ R $ ein kommutativer Ring mit $ 1 \neq 0 $ und $ a,b \in R $.
	Wir sagen \bi{$ a $ teilt $ b $}, falls es ein $ c \in R $ gibt
	mit $ b = a \cdot c $ und schreiben hierfür $ a \mid b $.
	\index{Teilbarkeit}
	Ein Element heißt \bi{Nullteiler},
	wenn es ein $  0 \neq c \in R $ gibt mit $ a \cdot c = 0 $.
	\index{Nullteiler}
	Wir nennen $ R $ \bi{nullteilerfrei} oder \bi{Integritätsring}, falls es außer $ 0 $ keinen anderen Nullteiler gibt. In solchen Mengen gilt
	\index{Nullteiler!frei}\index{Integritätsring}
	\begin{align*}
	a \cdot b = 0 \quad \Rightarrow a= 0 \vee b = 0.
	\end{align*}
\end{df}

\begin{genericdf}{Beispiel}\label{skript:7.5}
	Die Menge
	\begin{align*}
	\Z[i] := \lbrace a + b \cdot i \ | \ a,b \in \Z \rbrace \subseteq \C 
	\end{align*}
	nennen wir \bi{Gaußsche Zahlen}.
	\index{Gaußsche Zahlen} 
	Diese lässt sich mit 
	\begin{align*}
	\Z[\sqrt{d}] := \lbrace a + b \cdot \sqrt{d} \ | \ a,b \in \Z \rbrace
	\end{align*}
	für $ \sqrt{d} \notin \Q $ und $ d \in \Z $ allgemeiner definieren.
\end{genericdf}

\begin{genericdf}{Restklassenringe}\label{skript:7.6}\
	\begin{enumerate}
		\item[\textbf{(1)}]
		Wir betrachten $(\Z  / m \Z, + ) $ für ein festes $m \in \N$.
		Sei $a \in \Z$, dann schreiben wir für die Restklasse $a + m \Z$ bzw. $\overline{a}$.
		Mit 
		\begin{align*}
		(a + m \Z) \cdot (b + m \Z ) = (ab + m \Z) = \overline{ab} = \overline{a} \cdot \overline{b}
		\end{align*}
		wird auf $\Z m / \Z$ eine Multiplikation definiert.
		Die Wohldefiniertheit erhalten wir durch direktes Nachrechnen.
		Somit wird $\Z / m \Z$ zu einem kommutativen Ring und wir nennen diesen
		\bi{Restklassenring modulo $m$}.
		\index{Restklassenring}
		
		\item[\textbf{(2)}]
		Nun interessieren wir uns für die Einheiten in $\Z / m \Z$.
		Sei $\overline{a} \in \Z / m \Z$ eine Einheit.
		Dann existiert ein $\overline{b } \in \Z / m \Z$, sodass
		\begin{align*}
		\overline{a} \cdot \overline{b} = (a + m \Z) \cdot (b + m \Z) = \overline{1}
		\Leftrightarrow
		\exists c \in \Z : 1 = ab + mc
		\Leftrightarrow
		\ggT(a,m) = 1
		\end{align*}
		gilt. Damit erhalten wir
		\begin{align*}
		(\Z / m \Z)^\ast = \lbrace \overline{a} \in \Z / m \Z \ | 0 \neq a \in \Z : \ggT(a,m)  = 1 \rbrace
		\end{align*}
		als Einheitengruppe.
		Sei nun $p$ eine Primzahl. Gilt $m = p $ so ist $\Z / p \Z$ ein Körper.
		Wenn $m$ keine Primzahl ist, so ist $\Z / m \Z$ kein Körper, denn mit
		\begin{align*}
		m = ab 
		\Rightarrow 
		\overline{0} = \overline{m} = \overline{a} \cdot \overline{b}
		\end{align*}
		erhalten wir die Existenz von Nullteilern ungleich $\overline{0}$.
		Die Abbildung
		\begin{align*}
		\Phi : \N \to \N, \ m \mapsto | \lbrace a \in \N \ : \ 1 \leq a \leq m, \ \ggT(a,m) = 1 \rbrace | 
		\end{align*}
		nennen wir  \bi{Eulersche $\Phi$-Funktion} und es gilt
		\index{Eulersche Phi-Funktion}
		\begin{align*}
		| ( \Z / m\Z)^\ast | = \Phi(m). 
		\end{align*}
		
		\item[\textbf{(3)}]
		Es gilt
		\begin{align*}
		a^{\Phi(m)} \equiv 1 \mod m
		\end{align*}
		für $a \in \Z$ mit $\ggT(a,m)=1$.
		\begin{proof}
			Es gilt $\overline{a} \in ( \Z / m \Z )^\ast$.
			Mit Lagrange folgt dann $\ord(\overline{a}) \mid \Phi(m)$ und es gilt
			\begin{align*}
			\overline{a^{\Phi(m)}} = \overline{1}.
			\end{align*}
		\end{proof}
		
		\item[\textbf{(4)}]
		\bi{\underline{Kleiner Satz von Fermat}}:
		\index{Satz!kleiner Fermat}
		Sei $p$ eine Primzahl. Dann gilt
		\begin{align*}
		a^p \equiv a \mod p
		\end{align*}
		für alle $a \in \Z$.
		\begin{proof}
			Es gilt
			\begin{align*}
			a^{p-1} \equiv 1 \mod p 
			\Rightarrow
			a^p \equiv a \mod p
			\end{align*}
			für alle $a \in \Z$ mit $\ggT(a,p) = 1$ und für
			$\overline{a} = \overline{0}$
			\begin{align*}
			a^p \equiv 0 \mod p.
			\end{align*}
		\end{proof}
	\end{enumerate}		
		
\end{genericdf}

\begin{genericdf}{Polynomringe}\label{skript:7.7}
	Sei $R$ ein kommutativer Ring und $x$ eine Unbekannte über $R$.
	Dann bezeichnen wir mit $R[x]$ den \bi{Ring der Polynome} in einer Unbestimmten mit Koeffizienten in $R$.
	\index{Polynomring}
	Ein Element $0 \neq f \in R[x]$ lässt sich durch
	\begin{align*}
	f = a_0 + a_1 \cdot x + a_2 \cdot x^2 + \ \dots \ + a_n \cdot x^n
	\end{align*}
	mit $n \geq 0 $, $a_i \in R$ und $a_n \neq 0$ eindeutig beschreiben. 
	Für das Nullpolynom schreiben wir schlicht $0 $.
	Den \bi{Polynomgrad} $n$ kürzen wir mit $\Grad(f)$ ab und $a_n $ nennen wir den \bi{Leitkoeffizient}.
	\index{Polynom}\index{Polynom!Grad}\index{Polynom!Leitkoeffizient}\index{Polynom!normiert}
	Sollte dieser $1$ sein, so nennen wir $f$ \bi{normiert}.
	Für $f = 0$ setzen wir 
	\begin{align*}
	\Grad(f) = - \infty.
	\end{align*}	 
	Nun nehmen wir ohne Beschränkung der Allgemeinheit an, dass $m \leq n$ und betrachten
	\begin{align*}
	g = b_0 + b_1 \cdot x + \dots + b_m \cdot x^m \in R[x].
	\end{align*}
	Dann gilt
	\begin{align*}
	f+g = (a_0 + b_0) + (a_1 + b_1) \cdot x + \dots + (a_m + b_m) \cdot x^m
		= a_{m+1} \cdot x^{m+1} + \dots + a_n \cdot x^n
	\end{align*}
	und
	\begin{align*}
	f \cdot g = a_0 \cdot b_0 + (a_1 \cdot b_0 + a_0 \cdot b_1 ) \cdot x 
				+ \dots + a_n \cdot b_m \cdot x^{n+m}.
	\end{align*}
	Falls $a_n \cdot b_m \neq 0$ ist, so folgt $\Grad(f+g) = m+n$.
	Dies gilt immer wenn $R$ ein Integritätsring ist, insbesondere ist dann auch $R[x]$ ein Integritätsring.
	Wenn $R$ ein Integritätsring ist, dann sind Polynome $f$ mit $\Grad(f) \geq 1$ nicht invertierbar bezüglich der Multiplikation.
	Es gilt also $R[x]^\ast = R^\ast$.
\end{genericdf}

\begin{genericdf}{Bemerkung} \label{skirpt:7.8}
	Sei $R$ ein Integritätsring.
	Wir definieren analog zur Analysis durch
	\begin{align*}
	D: R[x] \to R[x], 
	\ \sum \limits_{i=0}^n a_i \cdot x^i \mapsto \sum \limits_{i=1}^n \underbrace{i \cdot a_i}_{a_i+ \dots +a_i} \cdot x^i
	\end{align*}		
	eine \bi{formale Ableitung}.
	\index{Formale Ableitung}
	Seien $a,b \in R$ und $f,g \in R[x]$ beliebig.
	Dann gelten:
	\begin{enumerate}
		\item[\textbf{(1)}] \bi{$R$-Linearität:}\\
		$D(a\cdot f + b\cdot g) = a \cdot D(f) + b \cdot D(g)$
		
		\item[\textbf{(2)}] \bi{Produktregel:}\\
		$D(f\cdot g) = f \cdot D(g) + D(f) \cdot g$
		
		\item[\textbf{(3)}]
		$D(f^n) = n \cdot D(f) \cdot f^{n-1}$
	\end{enumerate}
\end{genericdf}

\begin{proof}
	Der Beweis wird in der Übung geführt.
\end{proof}

\begin{generic_no_num}{Anwendung}
	Seien $0 \neq f,g \in R[x]$ mit $g$ teilt $f$.
	Wir nennen $g$ einen \bi{mehrfachen Faktor}, falls $g^2 \mid f$ gilt.
	\index{Faktor!mehrfach}
\end{generic_no_num}

\begin{genericthm_no_num}{Behauptung}
	Wenn $f = g^2 \cdot h$ ist, so gilt $D(f) = g \cdot \tilde{h}$.
	Dies bedeutet $g \mid D(f)$.
\end{genericthm_no_num}

\begin{proof}
	Es gilt
	\begin{align*}
	f = g^2 \cdot h
	\Rightarrow 
	D(f) = D(g^2 \cdot h) &= D(g^2) \cdot h + g^2 \cdot D(h)
	= 2 \cdot D(g) \cdot g\cdot h + g^2 \cdot D(h)\\
	&= g \cdot (\underbrace{2 \cdot D(g) \cdot h + g \cdot D(h)}_{\tilde{h}})
	\end{align*}
\end{proof}

\begin{generic_no_num}{Beispiel}
	Sei $m \geq 1$ und $f(x) = x^m - 1 \in \Z[x]$.
	Dann gilt $D(f) = m \cdot x^{m-1}$ und $\ggT(f,D(f)) = 1$.
	Nun haben $f$ und $D(f)$ keine gemeinsamen Faktoren außer $\pm 1$.
	Damit hat $f$ keine mehrfachen Nullstellen.
\end{generic_no_num}

\begin{df} \label{skript:7.9}
	Ein Integritätsring $R$ heißt \bi{euklidischer Ring}, wenn eine \bi{Normfunktion} bzw. \bi{Gradfunktion}
	\index{euklidischer Ring} \index{Ring!euklidisch} \index{Normfunktion} \index{Gradfunktion}
	\begin{align*}
	\nu : R \setminus \lbrace 0 \rbrace \to \N_0
	\end{align*}
	mit der folgenden Eigenschaft existiert:
	Zu $ a,b \in R $ mit $ b \neq 0 $ gibt es $ q,r \in R $ mit $ a = b \cdot q +r $,
	wobei $ r = 0 $ oder $ r \neq 0 $ und $ \nu(r) < \nu(b) $ gilt.
\end{df}

\begin{sz} \label{skript:7.10} \
	\begin{enumerate}
		\item[\textbf{(1)}]
		Der Ring $ \Z $ ist euklidisch mit $ \nu(n) = | n |  $ für $ n \in \Z $.
		
		\item[\textbf{(2)}]
		Sei $ K $ ein Körper. Dann ist der Polynomring $ K[x] $ euklidisch mit $ \nu(f) = \Grad(f) $ für
		$ f \in K[x] $.
		
		\item[\textbf{(3)}]  
		Die Gaußschen Zahlen $ \Z[i] $ sind euklidisch mit $ \nu(a+ib) = a^2 +b^2 $ für $ a,b \in \Z $.
	\end{enumerate}
\end{sz}

\begin{proof} \
	\begin{enumerate}
		\item[\textbf{(1)}]
		Folgt sofort durch den Euklidischen Algorithmus und Division mit Rest.
		
		\item[\textbf{(2)}]
		Seien $ f,g \in K[x] $ mit $ g \neq 0 $. Dann gilt
		\begin{align*}
		f(x) = \sum \limits_{i=0}^n a_i \cdot x^i, \quad g(x) = \sum \limits_{i=0}^m b_i \cdot x^i.
		\end{align*}
		Wenn $ n < m $ ist, so folgt $ f = 0 \cdot g + f $, also ist $ q =0 $ und $ r = f $.
		Sollte $ n \geq m $ sein, so setzen wir
		\begin{align*}
		\tilde{f} = f - a_n\cdot b_m^{-1} \cdot x^{n-m} \cdot g
		\end{align*}
		und erhalten für $ \tilde{f}  = 0$
		\begin{align*}
		f = a_n \cdot b_m^{-1} \cdot x^{n-m} \cdot g
 		\end{align*}
 		mit $ r = 0 $.
 		Falls $ \tilde{f}  \neq 0$ ist, so gilt
 		\begin{align*}
 		f = \underbrace{a_n \cdot b_m^{-1} \cdot x^{n-m}}_{h^\prime}\cdot g + \tilde{f} 
 		\end{align*}
 		mit $ \Grad(\tilde{f}) < \Grad (f) $.
 		Wir werden nun eine Induktion nach dem Grad von $ f $ durchführen.
 		Unsere Induktionsvoraussetzung ist
 		\begin{align*}
 		\tilde{f} = g \cdot \tilde{h} + r 
 		\end{align*}
 		mit $ r = 0  $ oder $ r \neq 0 $ und $ \Grad(r) < \Grad(g) $.
 		Dann folgt
 		\begin{align*}
 		f = h^\prime \cdot g + g \cdot \tilde{h} + r = g \cdot ( h^\prime + \tilde{h}) + r
 		\end{align*}
 		und falls $ r \neq 0 $ erhalten wir mit der Induktionsvoraussetzung sofort 
 		$ \Grad(r) < \Grad(g) $. Sollte $ r = 0 $ sein sind wir direkt fertig.
		\item[\textbf{(3)}]  
		Diesen Teil behandeln wir in der Übung.
	\end{enumerate}
\end{proof}

\begin{genericdf}{Bemerkungen} \label{skript:7.11}
	\
	\begin{enumerate}
		\item[\textbf{(1)}]
		Sei $R$ ein kommutativer Ring.
 		Die Division mit Rest im letzten Beweis funktioniert auch wenn $b_m \in R^\ast$.
 		
 		\item[\textbf{(2)}]
 		Die Polynome $q$ und $r$ aus dem letzen Beweis sind eindeutig bestimmt.
 		Sei 
 		\begin{align*}
 		f = g \cdot q + r = g \cdot g^\prime + r^\prime,
 		\end{align*}
 		und ohne Beschränkung der Allgemeinheit $\Grad(r) \leq \Grad(r^\prime)$.
 		Dann folgt 
 		\begin{align*}
 		g \cdot (q - q^\prime) = r^\prime -r 
 		\Rightarrow g \mid (r^\prime -r ).
 		\end{align*}
 		und für $r \neq r^\prime$ sein erhalten wir mit
 		\begin{align*}
 		\Grad(r^\prime- r) \leq \Grad(r) < \Grad(g)
 		\end{align*}
 		einen Widerspruch.
 		Also muss auch $q = q^\prime$ gelten.
 		
 		\item[\textit{(3)}]
	 	Sei $R$ ein euklidischer Ring.
	 	Die endliche iterative Anwendung der Division mit Rest
	 	liefert zu $a,b \in R \setminus \lbrace 0 \rbrace$ die Existenz von 
	 	$d, r , s \in R$ mit $\ggT(a,b) = d = r \cdot a + s \cdot b$. 		
 		
	\end{enumerate}
\end{genericdf}