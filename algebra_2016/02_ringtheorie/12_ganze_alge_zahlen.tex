\section{Ganze algebraische Zahlen}

\begin{df}\label{skript:12.1}
	Wir nennen eine Zahl $z \in \C$ \bi{ganz algebraisch},\index{algebraisch!ganz}
	wenn $n \geq 1$ und\\
	$a_0, a_1, \dots , a_{n-1} \in \Z$ mit
	\begin{align*}
	z^n + a_{n-1} \cdot z^{n-1} + \dots +  a_1 \cdot z + a_0 = 0
	\end{align*}
	existieren.
	Wir nennen $z \in \C$ \bi{algebraisch}, wenn es eine Nullstelle von irgendeinem
	Polynom in $\Z[x]$ ist.\index{algebraisch}
\end{df}

\begin{genericdf}{Beispiele}\label{skript:12.2} \
	\begin{enumerate}
		\item[\textbf{(1)}]
		Einheitswurzeln sind ganze algebraische Zahlen.
		
		\item[\textbf{(2)}]
		Die $\sqrt{2}$ ist eine ganze algebraische Zahl.
		
		\item[\textbf{(3)}]
		Die $\nicefrac{1}{2}$ ist keine ganze algebraische Zahl.
	\end{enumerate}

\end{genericdf}

\begin{sz}\label{skript:12.3}
	Sei $\mathbb{A} $ die Menge der ganzen algebraischen Zahlen.
	Diese Menge ist ein Teilring von $\C$ und es gilt
	\begin{align*}
	\mathbb{A} \cap \Q = \Z.
	\end{align*}
\end{sz}