\section{Irreduziblität von Polynomen}

\begin{genericdf}{Vorbemerkung}\label{skript:10.1}
	Sei $ R $ ein Integritätsring und $ p \in R $ prim.
	Nach \ref{skript:9.6} ist dann auch $ R / (p) $ ein Integritätsring.
	Die Abbildung
	\begin{align*}
	\pi_p : R[x] \to R/(p)[x], \ 
	f = \sum \limits_{i=0}^{n} a_i \cdot x^i \to \sum \limits_{i=0}^{n} \overline{a_i} \cdot x^i = f^\ast
	\end{align*}
	ist ein surjektiver Ringhomomorphismus mit $ \Ker \pi = (p) $ in $ R[x] $.	
\end{genericdf}

\begin{df}\label{skript:10.2}
	Sei $ R $ ein Integritätsring und $ 0 \neq f \in R[x] $.
	Dann nennen wir $ f $ ein \bi{primitives Polynom},\index{Polynom!primitiv}
	wenn es keine Nichteinheit $ 0 \neq a \in R $ gibt, sodass $ a $ alle Koeffizienten
	von $ f $ teilt.
	Insbesondere sind normierte Polynome primitiv.
\end{df}

\begin{genericthm}{Reduktionskriterium}\label{skript:10.3}\index{Reduktionskriterium}
	Sei $ R $ ein Integritätsring,
	$ 0 \neq f \in R[x] $ sei primitiv mit $ \Grad(f) \geq 1 $ und Leitkoeffizient $ a_n  $.
	Sei nun $ p \in R $ ein Primelement mit $ p \nmid a_n $.
	Ist $ \pi_p(f) = f^\ast $ irreduzibel in $ R/(p)[x] $,
	dann ist $ f $ irreduzibel in $ R[x] $.
\end{genericthm}

\begin{proof}
	Wegen $ p \nmid a_n $ muss $ \Grad(f^\ast) = n$ gelten.
	Wir nehmen an, dass $ f = g \cdot h $ mit $ g,h \in R[x] $ gilt.
	Da $ R $ ein Integritätsring ist, gilt $ n = \Grad(f) = \Grad(g) + \Grad(h) $.
	Außerdem ist $ a_n $ Produkt der Leitkoeffizienten von $ g $ und $ h $.
	Damit werden diese auch nicht von $ p $ geteilt.
	Also folgt $ 0 \neq g^\ast $, $ 0 \neq h^\ast $, $ \Grad(g) = \Grad(g^\ast) $
	und $ \Grad(h)  = \Grad(h^\ast)$.
	Somit gilt mit $ \pi_p $ und der Irreduziblität von $ f^\ast $
	\begin{align*}
	f^\ast = g^\ast \cdot h^\ast
	\Rightarrow
	\Grad(g^\ast) = 0 \vee \Grad(h^\ast ) = 0
	\Rightarrow
	\Grad(g) = 0 \vee \Grad(h) = 0.
	\end{align*}
	Wir nehmen ohne Beschränkung der Allgemeinheit an, dass $ \Grad(g) = 0  $ gilt.
	Dann gilt $ f = r \cdot h  $ mit $ r \in R $ und 
	\begin{align*}
	f = \sum a_i \cdot x^i, \quad h = \sum c_i \cdot x^i.
	\end{align*}
	Hieraus folgt $ a_i = r \cdot c_i $.
	Da $ f $ primitiv ist, muss $ r \in R^\ast  $ gelten, womit $ f $ irreduzibel ist.
\end{proof}

\begin{genericthm}{Eisensteinkriterium}\label{skript:10.4}\index{Eisensteinkriterium}
	Sei $ R $ ein Integritätsring und 
	\begin{align*}
	0 \neq f = \sum \limits_{i = 0}^n a_i \cdot x^i \in R[x]
	\end{align*}
	sei  primitiv mit $ \Grad(f) = n \geq 1  $.
	Wenn ein Primelement $ p \in R $ mit $ p \nmid a_n $, $ p \mid a_i $ für alle $ 0 \leq i \leq n-1 $
	und $ p^2 \nmid a_0 $ existiert, dann ist $ f $ irreduzibel in $ R[x] $.
\end{genericthm}

\begin{proof}
	Wir nehmen an, dass $ f = g \cdot h $ ist.
	Wie im letzten Beweis sehen wir, dass $ f^\ast  $, $ g^\ast  $ und $ h^\ast  $ ungleich null sind.
	Außerdem gilt wieder $ \Grad(f^\ast) = \Grad(f) = n = m + k $
	mit \\
	$ m = \Grad(g)  = \Grad(g^\ast)$ und $ k = \Grad(h) = \Grad(h^\ast) $.
	Nun gilt mit unseren Voraussetzungen
	\begin{align*}
	f^\ast = \overline{a_n} \cdot x^n 
	\Rightarrow
	g^\ast= \overline{b_m} \cdot x^m, \quad h^\ast = \overline{c_k} \cdot x^k.
	\end{align*}
	Angenommen $ m > 0 $ und $ k > 0  $.
	Dann sind $ b_0 $ und $ c_0 $ durch $ p $ teilbar, womit $ p^2 \mid a_0 $ gilt.
	Dies ist ein Widerspruch zu unseren Voraussetzungen.
\end{proof}

\begin{genericdf}{Beispiele}\label{skript:10.5}\
	\begin{enumerate}
		\item[\textbf{(1)}]
		Wir betrachten $ f = x^{24} - 18 \cdot x^{7} + 15 \in \Z[x] $,
		dann folgt mit Eisenstein und $ p =3 $, dass $ f $ irreduzibel in $ \Z[x] $ ist.
		
		\item[\textbf{(2)}]
		Sei $ f = x^3-5 \cdot x + 3 \in \Z[x] $.
		Wir wählen $ p = 2 $, womit
		$ f^\ast = x^3 + x + \overline{1}  \in \Z / 2 \Z[x]$ gilt.
		Da $ f^\ast $ keine Nullstellen über $ \Z / 2 \Z $ besitzt, ist $ f^\ast $ irreduzibel.
		Nun ist $ f $ primitiv. 
		Mit dem Reduktionskriterium folgt, dass $ f $ irreduzibel in $ \Z[x] $ ist.
		
		\item[\textbf{(3)}]
		Wir betrachten $ h = x^4 + 1 \in \Z[x]$.
		Aus LAAG ist bekannt, dass Nullstellen das Absolutglied teilen.
		Wegen $ h(\pm 1) = 2 $ besitzt $ h $ keine Nullstellen.
		Wir führen nun das \bi{Verfahren von Kronecker} exemplarisch durch.\index{Kroneckerverfahren}
		Angenommen $ h = h_1 \cdot h_2 $ mit $ \Grad(h_1) = 2  $ und $ \Grad(h_2) = 2 $.
		Sei $ h_1 = a \cdot x^2 + b \cdot x + c$.
		Da $ h $ normiert ist, können wir annehmen, dass $ h_1 $ und $ h_2 $ normiert sind.
		Somit folgt $ h_1 = x^2 + b \cdot x + c $.
		Für ein $ k \in \Z $ erhalten wir $ h(k) = h_1(k) \cdot h_2(k) $, womit $ h_1(k) \mid h(k) $ gelten muss. Nun gilt
		\begin{align*}
		k = 0 		  &\Rightarrow h(0) = 1 \Rightarrow h_1(0) = \pm 1 \\
		k = 1 		  &\Rightarrow h(1) = 2 \Rightarrow h_1(1) = \pm 1, \pm 2\\
					  &\Rightarrow c = \pm 1, \quad 1 + b + c = \pm 1 , \pm 2\\
		\bullet c = \ 1 &\Rightarrow b + 2 = \pm 1 , \pm 2 \Rightarrow b = -4,-3,-1, 0\\
		\bullet c =-1 &\Rightarrow b = \pm 1 , \pm 2 
		\end{align*}
		und für diese insgesamt 8 Möglichkeiten von $ h_1 $ müssen wir jetzt untersuchen, 
		ob diese Teiler von $ h $ sind.
		Jedoch gilt dies für keines, womit $ h $ irreduzibel ist.
		Alternativ wählen wir $ p = 2 $, dann ist $ x^2 + x + 1$ das einzige in $ \Z / 2 \Z[x] $ irreduzible
		Polynom.
		Wir nehmen an, dass $ h $ reduzibel ist und betrachten wieder $ h = h_1 \cdot h_2 $
		mit $ \Grad(h_1) = \Grad(h_2) = 2 $. Nach \ref{skript:9.7} ist $ \Z / 2 \Z[x] $ faktoriell, womit 
		$ h^\ast $ Produkt von irreduziblen Polynomen sein muss.
		Dies ist wegen
		\begin{align*}
		h^\ast = x^4 + \overline{1} \neq (x^2  + x + \overline{1})^2
		\end{align*}
		nicht erfüllt. Damit ist $ h$ irreduzibel.
		
		\item[\textbf{(4)}]
		Sei $ p $ eine Primzahl. Wir betrachten
		\begin{align*}
		f(x) = 1 + x + x^2 + \dots + x^{p-1}
		\end{align*}
		mit der Abbildung
		\begin{align*}
		\varphi : \Z[x] \to \Z[x], f(x) \mapsto f(x+1),
		\end{align*}
		welche nach \ref{skript:8.8} ein Ringhomomorphismus ist.
		Falls $ f = g \cdot h $ ist, folgt $ f(x+1) = g(x+1) \cdot h(x+1) $.
		Wenn also $ f(x+1) $ irreduzibel ist, so ist es auch $ f $.
		Nun gilt 
		\begin{align*}
		f(x+1) = 1 + (x+1) + (x+1)^2 + \dots + (x+1)^{p-1},
		\end{align*}
		womit wir aber nicht zufrieden sind und deswegen
		\begin{align*}
		x^p  - 1 = (x-1) \cdot (1 + x + \dots + x^{p-1}) = (x-1) \cdot f(x)
		\stackrel{\varphi}{\Rightarrow}
		(x+1)^p -1 = x \cdot f(x+1)
		\end{align*}
		betrachten. Weiter gilt
		\begin{align*}
		x \cdot f(x+1) &= x^p + \binom{p}{1} \cdot x^{p-1} + \dots + \binom{p}{p-1} \cdot x + 1 - 1 \\
				&= x \left( x^{p-1} + \binom{p}{1} \cdot x^{p-2} + \dots + \binom{p}{p-1} \right)\\
		\Rightarrow
		f(x+1) &= x^{p-1} + \binom{p}{1} \cdot x^{p-2} + \dots + \binom{p}{p-1} 
		\end{align*}
		und mit Eisenstein folgt, dass $ f(x+1) $ irreduzibel ist.
		Hierfür schaut euch nochmal den Beweis zu \ref{skript:8.7} an. 
		Damit ist auch $ f $ irreduzibel.
		
		\item[\textbf{(5)}]
		Sei $ p  $ eine Primzahl. Für
		\begin{align*}
		(p \cdot x + 1) \cdot (x-1) = p \cdot x^2 + (1-p) \cdot x - 1
		\end{align*}
		gilt über $ \Z / p \Z $, dass $ f^\ast  = \overline{x} - \overline{1} $ ist.
		Dies ist irreduzibel, jedoch $ f $ nicht.
		Dies zeigt, dass $ p \nmid a_n $ notwendig für $ \ref{skript:10.3} $ ist.
		Im weiteren Verlauf werden wir uns der folgende Frage beantworten:
		Falls $ f \in \Z[x] $ irreduzibel ist, gilt dies dann auch über $ \Q[x] $ ?
	\end{enumerate}
\end{genericdf}

\begin{lemma}\label{skript:10.6}
	Sei $ R $ ein Integritätsring und $ p \in R $ prim.
	Dann ist $ p $ prim in $ R[x] $.
\end{lemma}

\begin{proof}
	Mit \ref{skript:9.2} folgt aus $ p  $ prim, dass $ R/(p) $ ein Integritätsring ist.
	Dann ist auch $ R/(p)[x] $ ein Integritätsring und mit dem Homomorphiesatz für Ringe gilt
	$ R/(p)[x] \cong R[x]/(p) $.
	Also folgt mit \ref{skript:9.2}, dass $ p $ Primelement von $ R[x] $ ist.
\end{proof}

\begin{genericthm}{Lemma von Gauß}\label{skript:10.7}\index{Satz!Lemma von Gauß}
	Sei $ R $ faktoriell und $ 0 \neq a \in R[x] $ irreduzibel mit $ \Grad(a) \geq 1 $.
	Falls $ a $ teilt $ b \cdot c $ in $ R[x] $ mit $ 0 \neq b \in R[x] $ und $ c \in R[x] $ gilt,
	dann folgt $ a $ teilt $ c $ in $ R[x] $.
\end{genericthm}

\begin{proof}
	Sei $ l $ die Längenfunktion von $ R $.
	Für $ l(b) = 0 $ folgt $ b \in R^\ast $.
	In diesem Fall folgt die Aussage sofort.
	Sei also $ l(b) > 0$.
	Wir werden nun induktiv über diese Länge vorgehen.
	Sei $ 0 \neq p \in R $ irreduzibel mit $ p \mid b $.
	Dann gilt $ b = p \cdot b^\prime $ mit $ b^\prime \in R $ 
	und $ l(b^\prime) = l(b) -1 $.
	Nun gilt nach Voraussetzung $ b \cdot c = a \cdot d $ mit $ d \in R[x] $.
	Damit gilt $ p \mid a\cdot d  $ in $ R[x] $.
	Da $ R $ faktoriell ist, ist $ p $ auch prim.
	Nach \ref{skript:10.6} ist $ p $ prim in $ R[x] $,
	wodurch $ p \mid a $ oder $ p \mid d  $ folgt.
	Aufgrund der Irreduziblität von $ a $ folgt
	\begin{align*}
	p | d 
	\Rightarrow
	d = p \cdot d^\prime
	\end{align*}
	für eine $ d^\prime \in R[X] $.
	Also gilt
	\begin{align*}
	b \cdot c = p \cdot b^\prime \cdot c &= a \cdot d = a \cdot p \cdot d^\prime \\
	\Rightarrow b^\prime \cdot c &= a \cdot d^\prime
	\end{align*}
	und es folgt durch Induktion $ a \mid c $.
\end{proof}

\begin{df}\label{skript:10.8}
	Sei $R$ ein Integritätsring,
	dann können wir mit $R \subset K$ als Teilring
	und $K = \lbrace a \cdot b^{-1} \ | \ a \in R , b \in R \setminus \lbrace 0 \rbrace \rbrace$
	einen Körper konstruieren.
	Offensichtlich ist $K$ bis auf Isomorphie der kleinste Körper, der $R$ enthält. 
	Wir nennen $K$ den \bi{Quotientenkörper} von $R$ und schreiben $K = \Quot(R)$.
	\index{Quotientenkörper}
\end{df}

\begin{generic_no_num}{Bemerkung}
	Besonders leicht sehen wir dies, wenn $R$ bereits in $K$ enthalten ist.
	Wir betrachten zum Beispiel $\Z \subset \R  $, womit $\Quot(\Z) = \Q$ gilt.
\end{generic_no_num}

\begin{genericthm}{Satz von Gauß}\label{skript:10.9} \index{Satz!Gauß}
	Sei $R$ ein faktorieller Ring mit $\Quot(R) = K$.
	\begin{enumerate}
		\item[\textbf{(1)}]
		Sei $0 \neq f \in R[x]$ und $\Grad(f) \geq 1$.
		Falls $f$ irreduzibel $R[x]$ folgt,
		dass $f$ irreduzibel in $\Quot(R)[x]$ ist.
		
		\item[\textbf{(2)}]
		$R[x]$ ist ebenfalls faktoriell.
	\end{enumerate}
\end{genericthm}

\begin{proof}\
	\begin{enumerate}
		\item[\textbf{(1)}]		
		Sei $g \cdot h = f \in R[x]$ mit $g, h \in K[x]$ und $\Grad(g) \geq 1$.
		Die Koeffizenten von $g $ und $h$ sind von der Form
		\begin{align*}
		a \cdot b^{-1} = \frac{a}{b}
		\end{align*}				
		mit $a \in R$ und $b \in R \setminus \lbrace 0 \rbrace$.
		Sei $d$ das Produkt aller Nenner.
		Wir setzen $\tilde{g} = d \cdot g$ und $\tilde{h} = d \cdot h$.
		Damit folgt
		\begin{align*}
		\tilde{g}, \tilde{h} \in R[x]
		\Rightarrow
		d^2 \cdot f = \tilde{g} \cdot \tilde{h}, \quad \Grad(\tilde{g}) \geq 1,
		\end{align*}				
		womit $ \tilde{g} \notin R[x]^\ast$ ist.
		In \textbf{(2)} werden wir sehen, dass 
		\begin{align*}
		\tilde{g} = q_1 \cdot q_2 \cdots q_r
		\end{align*}
		mit $r \geq 1$ und $q_i$ irreduzibel gilt.
		Sei ohne Beschränkung der Allgemeinheit $\Grad(q_1) \geq 1$.
		Dann folgt mit \ref{skript:10.7}
		\begin{align*}
		q_1 \mid d^2 \cdot f 
		\Rightarrow
		q_1 | f
		\end{align*}
		und da $f$ irreduzibel ist gilt $f = \tilde{u} \cdot q_1$ mit $\tilde{u} \in R^\ast$.
		Es folgt
		\begin{align*}
		d^2 \cdot \tilde{u} = q_2 \cdots q_r \cdot \tilde{h},
		\end{align*}
		damit besitzt die linke Seite den Grad $0$, womit
		$\Grad(\tilde{h}) = \Grad(h) = 0$ folgt.
		Also gilt $h \in K[x]^\ast$.
		\item[\textbf{(2)}]	
		Zunächst haben wir zu zeigen:
		Falls $f \in R[x]$, dann ist $f$ eine Einheit oder ein Produkt aus irreduziblen Elementen
		aus $R[x]$.
		Wir setzen die Längenfunktion $l$ von $R$ auf $R[x]$ durch
		\begin{align*}
		l(g) := n + l(a_n)
		\end{align*}
		für $0 \neq g \in R[x]$ mit $\Grad(g) = n \geq 0$ und Leitkoeffizent $a_n \neq 0$ fort.
		Wegen $l(a\cdot b) = l(a)+l(b)$ für $a,b \in R$ 
		und $\Grad(g \cdot h) = \Grad(g) + \Grad(h)$ für $0 \neq g,h \in R[x]$ gelten
		\begin{itemize}
			\item
			$l(f) = 0 \Rightarrow f \in R^\ast$
			
			\item
			$l(g \cdot h) = l(g) + l(h) $
		\end{itemize}
		und damit folgt induktiv die Aussage.
		Nun müssen wir noch zeigen, dass prim gleich irreduzibel ist.
		Sei also $0 \neq f \in R[x]$ irreduzibel.
		Falls $Grad(f) = 0$ ist, dann liegt $f$ in $R$.
		Da $R$ faktoriell ist, ist $f$ prim in $R$ und damit auch $R[x]$.
		Wir betrachten also $\Grad(f) \geq 1 $ 
		mit $f \mid g \cdot h$ in $R[x]$.
		Dies gilt dann auch in $K[x]$.
		In \textbf{(1)} haben wir gezeigt, dass $f$ irreduzibel in $K[x]$ ist.
		Nach \ref{skript:9.7} ist $K[x]$ euklidisch, damit auch faktoriell.
		Also ist $f$ prim in $K[x]$.
		Damit gilt $f \mid g$ oder $f \mid h$ in $K[x]$.
		Wir nehmen an, dass $f \mid g$ gilt.
		Dann folgt $g = a \cdot f  $ mit $a \in K[x]$.
		Sei $d$ das Produkt der Nenner aller Koeffizenten von $a$. 
		Es folgt 
		\begin{align*}
		d \cdot g =   \underbrace{d \cdot a}_{\tilde{a}:=} \cdot f
		\end{align*}
		mit $\tilde{a} \in R[x]$.
		Es folgt weiter $f | \ d \cdot g $ in $R[x]$ und mit \ref{skript:10.7} 
		$f \mid g$ in R[x].
	\end{enumerate}		
\end{proof}

\begin{genericthm}{Folgerung}\label{skript:10.10}
	Sei $R$ faktoriell und $K = \Quot(R)$.
	Falls $0 \neq f \in R[x]$ irreduzibel nach \ref{skript:10.3}
	oder \ref{skript:10.4} ist, ist $f$ auch irreduzibel in $K[x]$.
\end{genericthm}

\begin{genericthm}{Folgerung}\label{skript:10.11}
	Sei $R$ faktoriell, dann ist der Polynomring
	in $n$ kommutierenden Unbestimmten
	$R[X_1,\dots,X_n]$
	auch faktoriell.
\end{genericthm}