\section{Kreisteilungspolynome}

\begin{df}\label{skript:11.1}
	Sei $K$ ein Körper, $n \geq 1$ und $f(x) = x^n -1 \in K[x]$.
	Wir nennen
	\begin{align*}
	E_n(K) := \lbrace \alpha \in K\ | \ f(\alpha) = 0 \rbrace 
	\end{align*}
	die Menge der \bi{$n$-ten Einheitswurzeln}.\index{n-te Einheitswurzel}
	Mit Multiplikation in $K$ bildet $E_n(K)$ eine endliche,zyklische Untergruppe von $K^\ast$.
	Die Endlichkeit folgt aus \ref{skript:8.9} und mit \ref{skript:8.10} ist $E_n(K)$ zyklisch.
	Wir definieren
	\begin{align*}
	E_n^\ast := \lbrace \alpha \in E_n \ | \ E_n = \langle \alpha \rangle \rbrace
	\end{align*}
	und nennen $\alpha \in E_n^\ast$ \bi{primitve} $n$-te Einheitswurzel.\index{n-te Einheitswurzel!primitiv}
	Nun definieren wir mit
	\begin{align*}
	\Phi_n(K) := \prod_{\alpha \in E_n^\ast} (x - \alpha) \in K[x]
	\end{align*}
	das \bi{$n$-te Kreisteilungspolynom} über $K$.\index{n-te Kreisteilungspolynom}		 
\end{df}

\begin{genericdf}{Beispiele}\label{skript:11.2}
	\begin{enumerate}
		\item[\textbf{(1)}]
		Sei $K = \Q$, dann gilt $E_n(\Q) = \lbrace \pm 1 \rbrace$ für alle geraden $n$.
		
		\item[\textbf{(2)}]
		Sei $K = \Z / p \Z$, dann gilt $E_p(K) = 1$.
		Dies folgt aus den kleinen Satz von Fermat gilt
		\begin{align*}
		a^p \equiv a \mod p 
		\end{align*}
		für alle $a \in \Z$. 
		
		\item[\textbf{(3)}]
		Sei $K = \C$, dann gilt
		\begin{align*}
		E_n(\C) = \left\lbrace \left( e^{\frac{2 \pi i}{n}} \right)^k \ | \ 1 \leq k \leq n \right\rbrace
		\end{align*}
		und $\zeta = e^{2 \pi i k}$ ist eine primitive $n$-te Einheitswurzel.
		Nun gilt
		\begin{align*}
		\zeta^k \ \text{primitiv} 
		\Leftrightarrow
		\ggT(k,n) = 1 
		\Leftrightarrow
		\Grad \Phi_n(\C) = \Phi(n),
		\end{align*}
		wobei $\Phi$ die Eulerfunktion ist.
		Für die Kreisteilungspolynome gilt:
		\begin{align*}
		\Phi_1 &= x -1 \\
		\Phi_2 &= x+1  \\
		\Phi_3 &= 1 + x + x^2 \\
		\Phi_4 &= x^2 + 1 \\
		\Phi_5 &= 1 + x + x^2 + x^3 +x^4 + x^5\\
		\Phi_6 &= x^2 - x +1
		\end{align*}
	\end{enumerate}
\end{genericdf}

\begin{sz}\label{skript:11.3}
	Sei $K = \C$ und $n \geq 1$.
	Dann gilt
	\begin{enumerate}
		\item[\textbf{(1)}]
		$n = \sum \limits_{d | n} \Phi(d)$.
		\item[\textbf{(2)}]
		$\Phi_n \in \Z[x]$ und $x^n - 1 = \prod_{d | n } \Phi_d$.
	\end{enumerate}
	Der Beweis ist eine Übungsaufgabe.
\end{sz}

\begin{sz}\label{skript:11.4}
	Sei $K = \C$.
	Dann ist $\Phi_n \in \Z[x]$ irreduzibel für alle $n \geq 1$.
\end{sz}

\begin{proof}
	Für den Beweis einfach in das Buch von Geck schauen.
\end{proof}
