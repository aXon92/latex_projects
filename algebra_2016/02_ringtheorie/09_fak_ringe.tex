\section{Faktorielle Ringe}

\begin{df}\label{skript:9.1}
	Sei $R$ ein Integritätsring, $p \in R \setminus \lbrace 0 \rbrace$ und $p \notin R^\ast$.
	
	\begin{enumerate}
		\item[\textbf{(1)}]
		Wir nennen $p$ \bi{irreduzibel},\index{Irreduziblität}
		falls aus $p = a \cdot b$ für $a,b \in R$ folgt,
		dass $a \in R^\ast $ oder $b \in R^\ast$ gilt.
		
		\item[\textbf{(2)}]
		Wir nennen $p$ \bi{prim} oder \bi{Primelement},\index{Primelement}
		falls aus $p \ | \ a \cdot b$ für $a,b \in R$ folgt,
		dass $p \ | \ a$ oder $p \ | \ b$ gilt.
	\end{enumerate}
\end{df}

\begin{lemma}\label{skript:9.2}
	Sei $R$ ein Integritätsring.
	\begin{enumerate}
		\item[\textbf{(1)}]
		Sei $p \in R$ prim, dann folgt $p$ ist irreduzibel.
		
		\item[\textbf{(2)}]
		Es gilt
		\begin{align*}
		p \in R \ \text{prim}
		\quad \Leftrightarrow \quad
		R/ (p) \ \text{Integritätsring} 
		\quad \Leftrightarrow \quad
		(p) \ \text{Primideal}.
		\end{align*}				
	\end{enumerate}
\end{lemma}

\begin{proof}\
	\begin{enumerate}
		\item[\textbf{(1)}]
		Sei $p = a \cdot b$, dann gilt $p$ teilt $a \cdot b$.
		Da $p$ prim ist gilt $p$ teilt $a$ oder $p$ teilt $b$.
		Angenommen $p$ teilt $a$, dann exisitiert ein $c \in R$ mit $a = p \cdot c$
		und es gilt 
		\begin{align*}
		p = a \cdot b = p \cdot c \cdot b
		\Rightarrow
		p \cdot (1 - b\cdot c) = 0.
		\end{align*}
		Da nun $R$ ein Integritätsring und $p \neq 0$ ist gilt
		\begin{align*}
		(1 -c \cdot b ) = 0 
		\Rightarrow
		1 = c \cdot b. 
		\end{align*}
		Damit gilt $b \in R^\ast$.
		Für den Fall $p$ teilt $b$ gehen wir analog vor und erhalten, dass $p$ irreduzibel ist.
		
		\item[\textbf{(2)}]
		Sei $p$ prim. 
		Wir wählen $\overline{a}, \overline{b} \in  R / (p)$ mit
		$\overline{a} \cdot \overline{b} = \overline{a \cdot b} = 0$.
		Damit gilt $a \cdot b \in (p)$, womit 
		\begin{align*}
		a \cdot b = p \cdot x 
		\end{align*}
		für ein $x \in R$ folgt. Damit gilt $p$ teilt $a \cdot b$.
		Da $p$ prim ist, gilt $p$ teilt $a$ oder $p$ teilt $b$.
		Also muss $\overline{a} = 0$ oder $\overline{b} = 0$ sein,
		womit $R / (p)$ ein Integritätsring ist.
		Sei nun umgekehrt $R/ (p)$ ein Integritätsring und $p$ teilt $a \cdot b$.
		Dann gilt $a \cdot b = p \cdot x$ für ein $x \in R$.
		Mit der Quotientenabbildung $\pi : R \to R / (p)$ erhalten wir 
		\begin{align*}
		\overline{a} \cdot \overline{b} 
		= \overline{a \cdot b}
		= \overline{p \cdot x}
		= \overline{p} \cdot \overline{x}
		= 0
		\end{align*}
		und mit der Integritätsringeigenschaft von $R/ (p)$
		folgt
		\begin{align*}
		\overline{a} = 0 \vee  \overline{b} = 0
		\Rightarrow
		a \in (p) \vee b \in (p),
		\end{align*}
		womit $p $ teilt $a$ oder $p$ teilt $b$ gilt.
		Also ist $p$ prim.
	\end{enumerate}		
\end{proof}

\begin{genericdf}{Beispiel}\label{skript:9.3} \
	\begin{enumerate}
		\item[\textbf{(1)}]
		Sei $R = \Z$. Primzahlen sind nach Definition irreduzible Elemente.
		Es ist zunächst nicht klar, ob Primzahlen prim sind.
		
		\item[\textbf{(2)}]
		Sei $R = \Z[\sqrt{-5}]$.
		Dann gilt 
		\begin{align*}
		2 \cdot 3 = (1 + \sqrt{-5}) \cdot (1 - \sqrt{-5}).
		\end{align*}
		Also sind $2$, $3$, $1 + \sqrt{-5}$ und $1 - \sqrt{-5}$ irreduzibel, aber nicht prim.
	\end{enumerate}
\end{genericdf}

\begin{df}\label{skript:9.4} \
	\begin{enumerate}
		\item[\textbf{(1)}]
		Ein Integritätsring $ R $ heißt \bi{faktoriell}, wenn\index{Ring!faktoriell} \index{Faktoriell}
		\begin{itemize}
			\item
			jedes $ 0 \neq a \in R $ entweder eine Einheit ist
			oder sich als endliches Produkt von irreduziblen Elementen schreiben lässt.
			Das heißt
			\begin{align*}
			a = p_1\cdots p_i \cdots p_k
			\end{align*}
			mit $ p_i $ irreduzibel für $ 1 \leq i \leq k $.
			
			\item
			jedes irreduzible Element aus $ R $ ein Primelement ist.	
		\end{itemize}
		
		\item[\textbf{(2)}]
		Sei $ R $ faktoriell.
		Wir definieren für $ 0 \neq a \in R $ durch
		 \begin{align*}
		 l(a) := \min \lbrace r \geq 1 \ | \ a= p_1 \cdots p_r 
		 \ \text{mit} \ p_i \ \text{irreduzibel} \rbrace
		 \end{align*}
		 für $ a \notin R^\ast $ und $ l(a) = 0 $ für $ a \in R^\ast $.
		 Dies bezeichnen wir als \bi{Länge} von $ a $.
		 \index{Faktoriell!Länge}
	\end{enumerate}
\end{df}

\begin{sz}\label{skript:9.5}
	Sei $ R $ ein faktorieller Ring und 
	$ 0 \neq a \in R $ mit $ l(a) = k \geq 1 $.
	Also ist $ a = p_1 \cdots p_k $ mit $ p_i $ irreduzibel.
	Für
	\begin{align*}
	a = q_1 \cdots q_l
	\end{align*}
	mit $ q_j $ irreduzibel für $ 1 \leq j \leq l $ gilt $ k = l $.
	Außerdem existiert ein $ \pi \in S_n $ und $ u_i \in R^\ast$
	für $ 1 \leq i \leq k$, sodass 
	\begin{align*}
	u_i \cdot p_i = p_{\pi(i)}
	\end{align*}
	gilt.
	Nun gilt noch $ l(a \cdot b) = l(a) + l(b) $ für alle
	$ a \in R \setminus \lbrace 0 \rbrace $.
\end{sz}

\begin{proof}
	Angenommen $ k = 1 $, dann folgt
	$ p_1 = q_1 \cdots q_l $ mit $ q_i $ irreduzibel.
	Da $ p_1 $ irreduzibel ist, kann nur $ l = 1 $ gelten.
	Sei nun $ k \geq 2 $. Dann gilt
	\begin{align*}
	p_1 \cdots p_k = q_1 \cdots q_l
	\end{align*}
	und weil $ R $ faktoriell ist folgt
	\begin{align*}
	q_1 \mid p_1 \vee \dots \vee q_1 \mid p_k,
	\end{align*}
	denn irreduzible Elemente sind prim.
	Ohne Beschränkung der Allgemeinheit nehmen wir an, dass $ q_1 | p_1 $ gilt.
	Da $ p_1 $ irreduzibel ist folgt $ p = c \cdot q_1 $ mit $ c \in R^\ast $.
	Damit erhalten wir
	\begin{align*}
	c \cdot q_1 \cdot p_2 \cdots p_k = q_1 \cdot q_2 \cdots q_l
	\Leftrightarrow
	c \cdot p_2 \cdots p_k =  q_2 \cdots q_l
	\end{align*}
	und mit der Induktionsvoraussetzung $ k-1 = l-1 $ folgt $ k = l $.
	Weiter existiert ein $ \tilde{\pi} \in S_n $ mit
	\begin{align*}
	q_i = p_{\tilde{\pi}(i)}
	\end{align*}
	und $ u_i \in R^\ast $ für $ 2 \leq i \leq k $.
	Wir setzen $ \pi(1) = 1 $ und $ \pi(i) = \tilde{\pi}(i) $ und haben somit unsere gesuchte Permutation.
	Seien nun $ 0 \neq a,b \in R $.
	Für $ a \in R^\ast $ gilt $ l(a \cdot b) = l(a) $ und analog gilt dies auch für $ b \in R^\ast $.
	Deswegen seien $ a $ und $ b $ keine Einheiten. Dann gilt
	\begin{align*}
	a &= p_1 \cdots p_k\\
	b &= q_1 \cdots q_l\\
	a \cdot b &= p_1 \cdots p_k \cdot q_1 \cdots q_l,
	\end{align*}
	womit $ a \cdot b $ wieder ein Produkt aus irreduziblen Elementen ist.
	Die Anzahl der Faktoren ist wie oben gezeigt immer minimal, womit
	$ l(a \cdot b ) = l(a) + l(b)  $ gilt. 
\end{proof}

\begin{generic_no_num}{Bemerkung}
	Ist $ p $ irreduzibel und $ a \in R^\ast $, so ist $a \cdot p  $
	irreduzibel.
\end{generic_no_num}

\begin{lemma} \label{skript:9.6}
	Sei $ R $ ein Integritäts-und Hauptidealring,
	dann gilt irreduzibel gleich prim. 
	Einen solchen Ring nennen wir auch \bi{Hauptidealbereich}.
	\index{Hauptideal!Bereich}
\end{lemma}

\begin{proof}
	Wir müssen zeigen, dass unter den Voraussetzungen aus irreduzibel prim folgt.
	Die andere Richtung haben wir bereits in \ref{skript:9.2} gezeigt.
	Sei $ q $ irreduzibel mit $ q $ teilt $ a \cdot b $.
	Da $ R $ ein Hauptidealring ist, gibt es ein $ d \in R $, sodass
	\begin{align*}
	(q) + (a) = (d)
	\end{align*}
	gilt.
	Damit folgt $ q = s \cdot d $.
	Da $ q $ irreduzibel ist, folgt $ s \in R^\ast $ oder $ r \in R^\ast $.
	Falls $ s \in R^\ast $ ist, folgt $ d = s^{-1}  \cdot q$ und $ q $ teilt $ d $.
	Wegen
	\begin{align*}
	q | d \wedge d | a
	\end{align*}
	gilt $ q $ teilt $ a $.
	Nun kommen wir zu dem Fall, dass $ d \in R^\ast $ ist.
	Damit gilt $ (d) = (1)  = R$, womit
	\begin{align*}
	1 = r_1 \cdot q + r_2 \cdot a
	\Rightarrow 
	b = b \cdot r_1 \cdot q +b \cdot r_2 \cdot a
	\end{align*}
	für passende $ r_1  $ und $ r_2 $ folgt.
	Mit $ q $ teilt $ a \cdot b $ folgt
	\begin{align*}
	q \mid b \cdot r_2 \cdot a
	\Rightarrow 
	q | b.
	\end{align*} 
	Also ist $ q $ prim.
\end{proof}

\begin{sz}\label{skript:9.7}
	Sei $ K $ ein Körper.
	Die Ringe $ \Z $ und $ K[x] $ sind faktoriell.
\end{sz}

\begin{proof}
	Die Ringe $ \Z $ und $ K[x] $ sind euklidisch und somit auch Hauptidealringe.
	Außerdem sind diese auch Integritätsringe, womit prim gleich irreduzibel ist.
	Sei $ a \in R \setminus R^\ast $.
	Wir müssen noch zeigen, dass $ a $ ein endliches Produkt von irreduziblen Elementen ist.
	Zunächst betrachten wir $ R = \Z $.
	Für $ a = b \cdot c $ folgt $ |a| = |b| \cdot |c| $ mit $ |b|, |c| < |a| $.
	Da für $ 0 \neq a \notin R^\ast $ immer $ |a| \geq 2 $ gilt, folgt mit der Beschränktheit nach unten von $ \N $ die Aussage.
	Nun kommen wir zu $ R = K[x] $.
	Das Vorgehen ist analog, wir verwenden die Gradfunktion.
	Es gilt
	\begin{align*}
	f \in R^\ast &\Leftrightarrow \Grad(f) = 0 \\
	0 \neq f \notin R^\ast &\Leftrightarrow \Grad(f) \geq 1
	\end{align*}
	und für $ a = b \cdot c $ folgt
	\begin{align*}
	\Grad(a) = \Grad(b ) + \Grad(c)
	\Rightarrow
	\Grad(b) < \Grad(a) \wedge \Grad(c) < \Grad(a),
	\end{align*}
	falls $ 0 \neq a,b,c \notin R^\ast $.
\end{proof}

\begin{genericdf}{Bemerkungen}\label{skript:9.8}
	\
	\begin{enumerate}
		\item[\textbf{(1)}]
		Satz \ref{skript:9.7} gilt allgemein für jeden Hauptidealbereich.
		
		\item[\textbf{(2)}]
		Insbesondere also für euklidische Ringe.
		Sei $ R $ euklidisch und $ \nu $ die zugehörige Normfunktion.
		Dann können wir zeigen, dass
		\begin{align*}
		\mu : R \setminus \lbrace 0 \rbrace \to \N_0, \ 
		a \mapsto \min \lbrace \nu(c \cdot a) \ | \ c \in R \setminus\lbrace 0 \rbrace \rbrace
		\end{align*}
		auch ein Normfunktion ist und $ \mu(a \cdot b) \geq \mu(a) $
		für alle $ 0 \neq a,b \in R $ gilt.
		Hiermit können wir konstruktiv zeigen, dass R faktoriell ist.
		
		\item[\textbf{(3)}] 
		Die Sätze \ref{skript:9.5} und \ref{skript:9.7} 
		ergeben zusammen den sogenannten
		\bi{Hauptsatz der elementare Zahlentheorie}, also die bis auf die Reihenfolge eindeutige Zerlegung von ganzen Zahlen in Primfaktoren.
		Insbesondere wissen wir jetzt auch, dass Primzahlen prim sind.
	\end{enumerate}
\end{genericdf}

\begin{lemma} \label{skript:9.9}
	Sei $ R $ ein Hauptidealbereich und $ p \in R $ irreduzibel,
	dann ist $ (p) $ ein maximales Ideal.
	Also ist $ R / (p) $ ein Körper.
\end{lemma}

\begin{proof}
	Sei $ a \in R $ und $ \overline{a} = a + (p) $ mit $ \overline{a} \neq 0 $.
	Es gilt also $ p \nshortmid a $.
	Nun betrachten wir das Ideal $ (p,a) \unlhd R $.
	Da $ R $ ein Hauptidealbereich ist, existiert ein $ d \in R $ mit
	 \begin{align*}
	 (p,a) = (d) 
	 \Rightarrow
	 d = r \cdot p + s \cdot a
	 \end{align*}
	 für passende $ r, s \in R $.
	 Nun gilt $ p \in (d) $, woraus
	 \begin{align*}
	 p = d \cdot c 
	 \Rightarrow
	 d \in R^\ast \vee c \in R^\ast
	 \end{align*}
	 folgt.
	 Nun nehmen wir an, dass $ c \in R^\ast  $ ist.
	 Dann gilt
	 \begin{align*}
	 d = c^{-1} \cdot p
	 \Rightarrow
	 p \mid d
	 \Rightarrow 
	 p \mid a,
	 \end{align*}
	 was ein Widerspruch zu unserer Annahme $ \overline{a}  \neq 0$ ist.
	 Also muss $ d \in R^\ast $ gelten, womit $ (d) = R $ und $ (p) $ maximal ist.
	 Nach \ref{skript:8.14} ist auch $ R/(p) $ ein Körper.
\end{proof}

\begin{genericdf}{Konstruktion von Erweiterungkörpern}\label{skript:9.10}
	\index{Erweiterungskörper!Konstruktion}
	Sei $ K $ ein Körper, $ R  = K[x]$ und $ f $ ein irreduzibles Polynom.
	Nach \ref{skript:9.6} bzw. \ref{skript:9.7} ist $ f $ prim.
	Da $ K[x] $ ein Hauptidealbereich ist, ist $ K[x] / (f) $ nach 
	\ref{skript:9.9} ein Körper.
	Wir betrachten nun $ n = \Grad(f) \geq 1$ und die Quotientenabbildung
	\begin{align*}
	\pi : K[x] \to K[x] / (f), g \mapsto \overline{g} = g + (f).
	\end{align*}
	Wir identifizieren $ K $ mit den Polynomen vom Grad $ \leq 1 $ in $ K[x] $.
%	\begin{figure}[H]
%		\centering
%		\begin{tikzcd}
%			G \arrow{r}{\pi}   &   K[x] / (p) \\
%			K \arrow[u,hook] \arrow{r}{\pi \mid_{K} }  &   \pi(K) \arrow[u,hook]
%		\end{tikzcd}
%	\end{figure}
	Außerdem ist $ \pi $ eingeschränkt auf $ K $ injektiv, damit existiert in
	$ K[x] / (f) $ ein zu $ K $ isomorpher Teilkörper, welchen wir auch mit $ K $
	identifizieren.
	Nun gilt also $ K \subset K[x] / (f) $,
	womit $ K[x] / (f) $ eine Erweiterung von $ K $ ist.
	\begin{generic_no_num}{Bemerkung}
		Sind $ K,L $ Körper mit $ K \subset L $, dann wird $ L $ zu einem Vektorraum über $ K $.
		Die Skalarmultiplikation definieren wir einfach über die Multiplikation in $ L $.
	\end{generic_no_num}
	Nun kommen wir zu der 
	\begin{genericthm_no_num}{Behauptung}
		Sei $ \alpha = \pi(x) $, dann gilt
		\begin{align*}
		K[x] / (f) = \left\lbrace \sum \limits_{i=0}^{n-1}  k_i \cdot \alpha^i 
		\ | \ k_i \in K , n = \Grad(f)\right\rbrace
		\end{align*}
		und
		\begin{align*}
		\lbrace 1, \alpha, \alpha^2, \dots , \alpha^{n-1} \rbrace
		\end{align*}
		ist eine $ K $-Basis von $ K[x] / (f) $.
		Außerdem gilt $ f (\alpha) = \overline{f} = \overline{0} $.
	\end{genericthm_no_num}
	
	\begin{proof}
		Dass jede Restklasse modulo $ (f) $ einen Repräsentanten der Form
		\begin{align*}
		\sum \limits_{i=0}^{n-1} k_i \cdot \alpha_i
		\end{align*}
		besitzt erhalten wir aus Division mit Rest nach $ (f) $.
		Nun nehmen wir an, dass 
		$ \lbrace 1 , \alpha , \dots , \alpha^{n-1} \rbrace $
		linear abhängig ist.
		Damit gilt für
		\begin{align*}
		g:= \sum \limits_{i= 0}^{n-1} k_i \cdot \alpha^i = 0,
		\end{align*}
		dass nicht alle $ k_i = 0  $ sind.
		Nun folgt $ \Grad(g) \leq n-1 $ oder $ g = a \cdot \tilde{g} $ mit $ \Grad(\tilde{g}) \leq n -1 $.
		Weiter gilt $ g(\alpha) = 0 $ bzw. $ \tilde{g}(a) = 0 $.
		Da $ \Ker \pi  = (f)$ ist, müssen $ g $ bzw. $ \tilde{g} $ Vielfache von $ f $ sein.
		Damit gilt $ \Grad g \geq n $ bzw. $ \Grad \tilde{g} \geq n $.
		Dies ist ein Widerspruch zu unserer Annahme.
	\end{proof}
\end{genericdf}

\begin{genericdf}{Beispiel}\label{skript:9.11}
	Sei $ K = \R $ und $ f = x^2 + 1 $.
	Dann ist $ f $ irreduzibel, da es in $ \R $ keine Nullstellen hat und es gilt
	\begin{align*}
	\R[x] / (f) \cong \C
	\end{align*}
\end{genericdf}