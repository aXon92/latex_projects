\chapter{Ausblick}


Die Untersuchung bedingt konvergenter Reihen lässt sich weiter fortsetzen. 
Die Folgerung von Chobanyan \ref{th:conclusion_chobanyan} setzt einen Banachraum vom Typ $ p $ voraus. 
Diese Voraussetzung lässt sich auf einen Banachraum vom \textit{Infratyp} $ p $\cite[Ch. 5]{Kadets1997} abschwächen.
Außerdem gibt es für derartige Räume eine offene Vermutung.
\begin{genericthm_no_num}{Satz\cite[Ch. 7]{Kadets1997}}
	Sei $ X $ ein Banachraum vom Infratyp-$ p $ und $ (x_n) $ eine Folge in $ X $
	mit $ \sum \| x_i \|^p < \infty $.
	Dann besitzt $ (x_n) $ die Steinitzeigenschaft.
\end{genericthm_no_num}
%Zu den Räumen mit der Infratyp $ p $-Eigenschaft gibt es noch eine offene Vermutung.
\begin{generic_no_num}{Vermutung\cite[Ch. 7]{Kadets1997}}
	Sei $ X $ ein Banachraum vom Infratyp $ p $.
	Dann ist $ (\|x_n\|^p) $ für jede perfekt unsummierbare Folge $ (x_n) $ unsummierbar.
\end{generic_no_num}
Die Umkehrung dieser Vermutung gilt: 
Sei $ (\|x_n\|^p) $ für jede perfekt unsummierbare Folge $ (x_n) $ unsummierbar. Dann ist $ X $ ein Banachraum vom Infratyp $ p $.


In dieser Arbeit wurde sich auf das Konvergenzverhalten bei Umordnung bezüglich der Norm beschränkt.
Dies lässt sich auf Topologien erweitern, welche nicht durch eine Norm beschreibar sind.
\begin{genericthm_no_num}{Satz von Banaszczyk\cite{Banaszczyk1990}}
	Sei $ (x_n) $ bedingt summierbar in einem metrisierbaren nuklearen Raum.
	Dann erfüllt $ (x_n) $ die Steinitzeigenschaft.
\end{genericthm_no_num}
Wie im ursprünglichen Satz von Steinitz \ref{th:lemma_of_steiniz} genügt hier die bedingte Summierbarkeit als Voraussetzung.

Den Satz von Orlicz wurde für $ \L^p $-Räume gezeigt. 
Für ein geeignetes $ p $ sind $ C $-konvexe Räume\cite[Ch. 5]{Kadets1997} vom $ M $-Kotyp $ p$.
Nach der Äquivalenz \ref{th:orlicz_equi_kotyp} folgt die Aussage von Orlicz.
Allgemeinere Versionen dieses Satzes finden sich in \cite[Ch. 3,Ch. 10, Ch.11]{Diestel1995}.
Genaue Zusammenhänge zwischen den Begriffen Typ $ p $, $ M $-Kotyp und Infratyp $ p $ sind in \cite[Ch. 5]{Kadets1997} und 
\cite[Ch. 11, Ch. 14., Ch. 16]{Diestel1995} dargestellt.
Nachdem in dieser Arbeit absolut summierbare Operatoren skizziert wurden, kann das Studium dieser in \cite[Ch. 8]{Albaic2006} und \cite{Diestel1995} fortgesetzt werden.
Weiterführend von der unbedingten Konvergenz sind Banachräume mit der UMD-Eigenschaft\cite[Ch. 4]{Hytoenen2016}.

%Absolut summierbare Operatoren sind  umfangreich dargestellt.




%In \cite[Ch. 5]{Kadets1997} 
%Dieser Satz gilt auch für $ C $-konvexe Räume\cite[Ch. 5]{Kadets1997}, denn diese Räume sind vom $ M $-Kotyp $ p $(siehe \ref{th:orlicz_equi_kotyp}).




%Die Typ $ p $ Eigenschaft folgt dann daraus.
