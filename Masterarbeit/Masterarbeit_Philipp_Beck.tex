%\documentclass[12pt,twoside,a4paper]{scrartcl}
\documentclass[12pt,twoside,a4paper]{scrbook}
\usepackage[automark,pagestyleset=standard,markcase=used]{scrlayer-scrpage}
\usepackage[utf8]{inputenc}
\usepackage[T1]{fontenc}
\usepackage[ngerman]{babel}
%\usepackage[a4paper,scale=0.8]{geometry}
%\usepackage[top=2cm,bottom=2.5cm]{geometry} 
\usepackage[left=2cm, right=2.7cm, top=2.5cm, bottom=2.5cm]{geometry}
%\usepackage{fancyhdr}
\usepackage{setspace} 
%\usepackage{layout}




%\pagestyle{fancy} %eigener Seitenstil
%\fancyhf{} %alle Kopf- und Fußzeilenfelder bereinigen
%\fancyhead[L]{} %Kopfzeile links
%\fancyhead[C]{} %zentrierte Kopfzeile
%\fancyhead[R]{\leftmark} %Kopfzeile rechts
%\renewcommand{\headrulewidth}{0.4pt} %obere Trennlinie
%\fancyfoot[C]{\thepage} %Seitennummer
%\renewcommand{\footrulewidth}{0.4pt} %untere Trennlinie
%\pagestyle{fancy}
%\fancyhead{}
%\fancyhead[RE]{\textit{\nouppercase{\leftmark}}}
%\fancyhead[LO]{\textit{Umordnung und Klassifizierung von Reihen in Banachräumen}}
%\fancyhead[RO,LE]{\thepage}
%\fancyfoot{}
%\addtolength{\headheight}{0.15cm}

\usepackage{paralist}
\usepackage{enumitem} 

%\usepackage {picins}
\usepackage{color}
\usepackage{float}

% -- Mathe
%\usepackage{gensymb}
%fancyhdr, lastpage, booktabs, xy
\usepackage{graphicx}
\usepackage{amsmath}
\usepackage{amsfonts}
\usepackage{amssymb}
\usepackage{units}
\usepackage{amsthm}



\usepackage{stmaryrd}

\usepackage{subfigure} 
\usepackage{autobreak} 
% bibliography
\usepackage[numbers,sort&compress]{natbib}

% -- Malen
\usepackage{tikz}
\usepackage{tikz-cd}
\usetikzlibrary{patterns, decorations.pathreplacing, decorations.pathmorphing, arrows}

\usepackage{units}

%\usepackage{xcolor}
%\usepackage{titlesec}
%\usepackage{enumerate}



\usepackage{hyperref}


\renewcommand{\theenumi}{\textbf{(\roman{enumi})}}

%\titleformat{\chapter}[display]
%{\normalfont\sffamily\huge\bfseries\color{blue}}
%{\chaptertitlename\ \thechapter}{20pt}{\Huge}
%\titleformat{\section}
%{\normalfont\sffamily\Large\bfseries\color{cyan}}
%{\thesection}{1em}{}


\numberwithin{equation}{chapter}

\author{Philipp Beck}



%\makeatletter
%\renewcommand*{\thesection}{\Roman{\thesection}}
%\renewcommand*{\p@section}{\thechapter.}

%Befehle für bestimmte mathematische Symbole
\newcommand{\id}{\operatorname{Id}}
\newcommand{\sign}{\mathrm{sign}}
\newcommand{\lin}{\mathrm{lin}}
\newcommand{\nt}{\unlhd}
\newcommand{\dx}[1]{\mathrm{d}#1}
\newcommand{\dxS}[1]{ \ \mathrm{d} #1}
\newcommand{\supp}{\mathrm{supp}}
\newcommand{\spn}{\mathrm{span}}
\newcommand{\tr}{\mathrm{tr}}
%\newcommand{\esssupp}{\operatorname{\mathrm{ess \ sup}}}
\DeclareMathOperator*{\esssupp}{ess\,sup}
\newcommand{\Bild}{\mathrm{Bild}}
\newcommand{\Kern}{\mathrm{Kern}}

%Mengensymbole
\newcommand{\N}{\mathbb{N}}
\newcommand{\Z}{\mathbb{Z}}

\newcommand{\Q}{\mathbb{Q}}
\newcommand{\R}{\mathbb{R}}
\newcommand{\C}{\mathbb{C}}
\newcommand{\K}{\mathbb{K}}
\renewcommand{\L}{\mathrm{L}}
\renewcommand{\i}{\textrm{i}}
\newcommand{\M}{\mathrm{M}}
%\renewcommand{\S}{\mathcal{S}}
\newcommand{\F}{\mathcal{F}}
\newcommand{\T}{\mathcal{T}}
\newcommand{\W}{\mathrm{W}}
\renewcommand{\H}{\mathrm{H}}
\renewcommand\qedsymbol{$\blacksquare$}

%Big and italic
\newcommand{\bi}[1]{\textbf{\textit{#1}}}

\newcommand{\TODO}[1]{ \textcolor{red}{\textbf{\underline{TODO:} #1}} }

\swapnumbers

\newtheoremstyle{satz}% name
{10pt}% Space above
{5pt}% Space below
{\itshape}% Body font
{}% Indent amount: Indent amount: empty = no indent, \parindent = normal paragraph indent
{\bfseries}% Theorem head font
{:}% Punctuation after theorem head
{\newline }% Space after theorem head: Space after theorem head: { } = normal interword space; \newline = linebreak
{}% Theorem head spec (can be left empty, meaning `normal')

\newtheoremstyle{def}% name
{10pt}% Space above
{5pt}% Space below
{}% Body font
{}% Indent amount: Indent amount: empty = no indent, \parindent = normal paragraph indent
{\bfseries}% Theorem head font
{:}% Punctuation after theorem head
{\newline }% Space after theorem head: Space after theorem head: { } = normal interword space; \newline = linebreak
{}% Theorem head spec (can be left empty, meaning `normal')







\newcommand{\thistheoremname}{}
\theoremstyle{satz}
%Satz
\newtheorem{sz}{Satz}[chapter]

%Lemma
\newtheorem{lem}[sz]{Lemma}
\newtheorem{kor}[sz]{Korollar}
\newtheorem{gsz}[sz]{\thistheoremname}
\newtheorem*{gthm_no_num}{\thistheoremname}


\theoremstyle{def}
\newtheorem{df}[sz]{Definition}
\newtheorem{bem}[sz]{Bemerkung}
\newtheorem{gd}[sz]{\thistheoremname}

\newtheorem*{gdf_no_num}{\thistheoremname}



%Für Beispiele , Bemerkungen
%Also alles was wie eine Definition aussehen soll, aber keine ist.
%\begin{genericdf}{(Hier den Text einfügen)}
%	Inhalt...
%\end{genericdf}
\newenvironment{genericdf}[1]{\renewcommand{\thistheoremname}{#1}\begin{gd}}{\end{gd}}
%Dasselbe ohne Nummerierung
\newenvironment{generic_no_num}[1]{\renewcommand{\thistheoremname}{#1}\begin{gdf_no_num}}{\end{gdf_no_num}}

%Hier für Sätze
\newenvironment{genericthm}[1]{\renewcommand{\thistheoremname}{#1}\begin{gsz}}{\end{gsz}}
%Analog zu Definitionen
\newenvironment{genericthm_no_num}[1]{\renewcommand{\thistheoremname}{#1}\begin{gthm_no_num}}{\end{gthm_no_num}}
\parindent0pt
%\parskip1ex
\newcommand{\blankpage}{
	\newpage
	\thispagestyle{empty}
	\mbox{}
	\newpage
}



%----Dokument----
%\allowdisplaybreaks 
\onehalfspacing
\begin{document}
%\layout
%Deckblatt
\newgeometry{a4paper,scale=0.8}
\thispagestyle{empty}
\begin{center}
	
	
	\hspace{-1.0cm}
	\includegraphics[width=0.6\textwidth]{uni_logo.png}
	\vspace{-0.6 cm}
	
	\hspace{2,2cm}
	\\[1cm]
	{\Large Institut für Analysis, Dynamik und Modellierung\\(IADM)}
	
	
	\vspace*{3cm}
	
	% Art der Arbeit => (Bachelorarbeit ,Diplomarbeit, Masterarbeit, Seminararbeit)
	\Large{\textbf{Masterarbeit}\\
		
		\vspace{1cm}
		
		% Titel der Arbeit 
		\textbf{\Large{Umordnung und Klassifizierung von Reihen in Banachräumen}}
		
		\vspace*{1mm}
		
		
		\vspace{1.5cm}
		
		% Name des/der Autors/Autoren
		%{\LARGE Alisa Baransegeta}\\[35mm]
		
		% Gutachter, Kontaktdaten und Abgabetermin
		\parbox{120mm}{
			\begin{large}
				\begin{tabbing}
					Betreuer: \hspace{0.7cm} \= Prof. Dr. Guido Schneider\\[4mm]
					Verfasser:\> Philipp Beck\\ % alphabetische Reihenfolge (Nachname)
					Matrikel-Nr.:\> 2800897\\
					Email:\>uni@phil-beck.de\\
					%\textbf{Abgabetermin:} \hspace{0,3cm} \> \textbf{21. April 2019}\\
				\end{tabbing}
			\end{large}
		}}
	\end{center}
%\pagestyle{fancy}
\blankpage
\newgeometry{left=2cm, right=2.7cm, top=2.5cm, bottom=2.5cm}	
\cleardoublepage
\setcounter{page}{1}

\tableofcontents


\blankpage

\section{Einleitung}
Im Jahre 1979 führte Hans-Georg Feichtinger die Feichtingeralgebra $ S_0(G) $ auf lokalkompakten Gruppen ein\cite{Feichtinger1979}.
Der österreichische Mathematiker hat eine Professur an der mathematische Fakultät der 
Universität Wien inne. 
Der Schwerpunkt seiner Arbeit liegt auf dem Gebiet der harmonischen Analysis.
Speziell ist hier die Zeit-Frequenz-Analyse hervorzuheben.
Die Struktur und Notation dieser Ausarbeitung richtet sich nach dem Buch  von Karlheinz Gröchenig\cite{noauthor2009Foundationsof}, welcher eng mit Feichtinger zusammenarbeitet.
\ \\
\\
In dieser Arbeit beschränken wir uns auf $ G = \R^d $.
Die Fouriertransformation ist ein Automorphismus auf dem Schwartzraum $ \S(\R^d) $.
Das Fehlen der Banachraumeigenschaft macht Beweise mit diesem Raum umständlich.
Dieses Problem lösen wir mit der Feichtingeralgebra $ S_0 $.
Im Gegensatz zu  $ \S(\R^d) $ ist $ S_0(\R^d) $ eine Banachalgebra, womit wir eine handhabbarere mathematische Struktur erhalten.
Wir werden sehen, dass der Schwartzraum $ \S(\R^d) $ dicht in $ S_0(\R^d) $ liegt.
Darüber hinaus übertragen sich einige nützliche Eigenschaften des Schwartzraums auf $ S_0 $.
Auf $ S_0 $ ist die Fouriertransformation ein Automorphismus.
Des weiteren ist die Possionformel gültig, die Translations-und Modulationsinvarianz, die Invarianz unter Automorphismen und die Invarianz unter dem Tensorprodukt sind erfüllt. Wir werden nicht alle diese Eigenschaften nachweisen.
Dies unterstreicht jedoch die Relevanz der Feichtingeralgebra $ S_0 $, da wir das Tripel $ (\S(\R^d), \L^2 , \S^\prime(\R^d) )$
durch $ (S_0, \L^2 , S_0^\prime) $ ersetzen können.
 
%Darunter sind die Invarianz unter der Fouriertransformation, Invarianz unter Automorphismen
%und 
Unsere größte Aufmerksamkeit werden wir jedoch der Minimalität von $ S_0 $ widmen.
Minimal bedeutet:
$ S_0 $ ist der kleinste nicht-triviale Banachraum, welcher unter Zeit-Frequenzverschiebungen invariant ist.
\ \\
\\
Zu Beginn dieser Arbeit führen wir die Kurzzeit-Fouriertransformation (STFT) ein.
Zusammen mit den gemischt-gewichteten $ \L^p $-Räumen liefert die STFT die Grundlage zur Definition der Modulationsräume, wozu auch $ S_0 $ gehört.
Der Weg über die Modulationsräume ist allgemeiner als nur die Feichtingeralgebra zu betrachten und die Beweise sind ein wenig aufwendiger.
Jedoch werden alle Eigenschaften dieser Räume direkt oder indirekt in den Beweis der Minimalität von $ S_0 $ einfließen
und erste Resultate zu $ S_0 $ liefern.
Nach diesem Beweis widmen wir uns einigen Eigenschaften, welche aus der Minimalität folgen.
\\


%Über diese Räume werden wir erste Eigenschaften von $ S_0 $ entdecken.
%Die Feichtingeralgebra $ S_0 $ ist einer dieser Modulationsräume. 


%Dies führt auf das Gelfandtripel $ (S_0, \L^2 , S_0^\prime) $
\blankpage
\section{Grundlagen}
\vspace{-0.5cm}
In diesem Abschnitt werden wir grundlegende Bezeichnungen, Definitionen und Sätze einführen.
Unter 
\begin{align}
\int \limits_{\R^d} f(x) \td{x}
=
\int \limits_{-\infty}^\infty \cdots \int \limits_{-\infty}^\infty f(x_1,...,x_d) \td{(x_1,...,x_d)} 
\end{align}
mit $ \td{(x_1,...,x_d)} = \td{x_1}...\td{x_d} $ verstehen wir das Lebesgueintegral auf $ \R^d $.
Sei $ f  $ eine messbare Funktion auf $ \R^d $.
Dann gilt $ f \in \L^p(\R^d) $, falls die $ \L^p $-Norm
\begin{align}
\| f \|_p = \left(  \ \int \limits_{\R^d} | f(x) |^p \td{x} \right)^{\frac{1}{p}}
\end{align}
für $ 1 \leq p < \infty $ endlich ist.
$ \L^\infty(\R^d) $ besteht aus den essentiell beschränkten messbaren Funktionen, d.h. es gilt
\begin{align*}
\|f \|_\infty 
= 
\esssupp \limits_{x \in \R^d} | f(x) | < \infty.
\end{align*}
Wir werden $ \sup $ auch für das essentielle Supremum verwenden, falls dies sinnvoll ist.
Die $ \L^p $-Räume sind für $ 1 \leq p \leq \infty $ allesamt Banachräume. 
$ \L^2(\R^d) $ ist mit dem Skalarprodukt
\begin{align}
\langle f, g \rangle
=
\int \limits_{\R^d} f(x) \overline{g(x)} \td{x}
\end{align}
ein Hilbertraum. Wenn der Kontext klar ist, werden wir $ \L^p $ statt $ \L^p(\R^d) $ schreiben.
In wenigen Fällen ist $ \| \cdot \|_2 $ die euklidische Norm auf $ \R^d $.
Mit $ x \cdot \omega = \sum_{k=1}^d x_i \omega_i$ beschreiben wir das Skalarprodukt auf $ \R^d $.
\begin{df}
	Sei $ f \in \L^1(\R^d) $. Dann definieren wir durch
	\begin{align}\label{eq:fourier_transformation}
	\hat{f}(\omega) =
	\int \limits_{\R^d} f(x) e^{-2 \pi \i x \cdot \omega} \td{x}
	\end{align}
	die \textit{Fouriertransformation} von $ f $.	
\end{df}
Für \eqref{eq:fourier_transformation} erhalten wir:
\begin{align}\label{eq:estimate_fouriertransformation}
\| \hat{f} \|_{\infty}
\leq
\sup \limits_{\omega \in \R^d} \int \limits_{\R^d} |f(x) e^{-2 \pi \i x \cdot \omega}| \td{x} = \| f \|_1
\end{align}
Falls wir die Fouriertransformation als linearen Operator auf einem Funktionenraum betrachten, schreiben wir statt $ \hat{f} $ auch $ \F f $.
\begin{genericthm}{Lemma von Riemann-Lebesgue}
	Sei $ f \in \L^1(\R^d) $.
	Dann ist $ \hat{f} $ gleichmäßig stetig und es gilt
	$ \lim_{|\omega| \to \infty} | \hat{f}(\omega) | = 0 $.
\end{genericthm}
Mit diesem Lemma können wir die Fouriertransformation durch
\begin{align*}
\F : \L^1(\R^d) \to C_0(\R^d)
\end{align*}
beschreiben. Hierbei ist $ C_0(\R^d) :=  \{ f \in C(\R^d) |   \lim_{|\omega| \to \infty} | \hat{f}(\omega) | = 0\} $. 

\begin{genericthm}{Dichtheitsprinzip}\label{th:densitiy_priniciple}
	Gegeben seien die Banachräume  $ B_1 $ und $ B_2 $, ein dichter Unterraum $ X $ von $ B_1 $ und ein linearer Operator $ A : X \to B_2 $ mit
	\begin{align}\label{eq:density_principle}
	\| A f \|_{B_2} \leq C \| f \|_{B_1}
	\end{align}
	für  alle $  f \in X $.
	Dann gilt \eqref{eq:density_principle} für alle $ f \in B_1 $ und $ A $ wird zu einem beschränkten linearen Operator von $ B_1  $ nach $ B_2 $ fortgesetzt.
\end{genericthm}
\begin{sz}
	Sei $ f \in \L^1(\R^d) \cap \L^2(\R^d) $. Dann gilt die \textit{Plancherelgleichung}
	\begin{align}\label{eq:plancherel}
	\|f \|_2 = \| \hat{f} \|_2.
	\end{align}
	.
\end{sz}
Damit lässt sich $ \F $ zu einem unitären Operator auf $ \L^2(\R^d) $ erweitern, womit auch die \textit{Parsevalgleichung}
\begin{align}\label{eq:parseval}
\langle f, g \rangle = \langle \hat{f} , \hat{g} \rangle
\end{align}
für alle $ f,g \in \L^2(\R^d) $ erfüllt ist.
Für $ f \in \L^2(\R^d)  $ können wir $ \hat{f} $ jedoch nicht mehr wie in \eqref{eq:fourier_transformation} punktweise definieren. 
Hier ist uns das Dichtheitsprinzip \ref{th:densitiy_priniciple} behilflich.
Sei $ X = \L^1 \cap \L^2 $ und $ f \in \L^2  $ beliebig. Dann ist $ X $ dicht in $ \L^2 $ und es existiert ein $f_n  \in X$ mit $ \| f - f_n \|_2 \to 0 $.
Wegen $ f_n \in \L^1 $ ist $ \hat{f}_n $ wohldefiniert und es gilt mit \eqref{eq:plancherel} $ \| f_n - f_m \|_2 = \| \hat{f}_n - \hat{f}_m \|_2 $.Damit ist $ \hat{f}_n $ eine Cauchyfolge in $ \L^2 $ und wegen der Vollständigkeit existiert ein eindeutiger Grenzwert in $ \L^2 $.
Dementsprechend definieren wir $ \hat{f} := \lim_{n \to \infty} \hat{f}_n $.

\newpage
\begin{df}
	Seien $ f, g \in \L^1(\R^d) $. Dann definieren wir durch
	\begin{equation}\label{eq:convolution}
	(f \ast g)(x) = \int \limits_{\mathbb{R}^d}
	f(y) g(x-y) \td{y}
	\end{equation}
	die \textit{Faltung} von $ f $ und $ g $.
\end{df}
Wegen
\begin{align}\label{eq:convolution_prop_1}
\| f \ast g \|_1 &\leq \| f\|_1 \|g \|_1
\end{align}
ist  $ \L^1 $ eine Banachalgebra bezüglich der Faltung. Außerdem ist
\begin{align}\label{eq:convolution_prop_2}
\widehat{(f \ast  g)} &= \hat{f} \cdot \hat{g}
\end{align}
erfüllt.

\begin{genericdf}{Translation und Modulation}
	Seien $ x, \omega \in \R^d $ und $ f : \R^d \to \C$ eine beliebige Funktion.
	Dann ist
	\begin{align}\label{eq:translation}
	T_x f(t)=  f(t-x) 
	\end{align}
	die \textit{Translation} um $ x $ und
	\begin{align}\label{eq:modulation}
	M_\omega f(t) = e^{2 \pi \mathrm{i} t \cdot \omega} f(t)
	\end{align}
	die \textit{Modulation} um $ \omega $. 
	Wir nennen Operatoren der Form $M_\omega T_x$ oder $T_x M_\omega$ \textit{Zeit-Frequenz-Verschiebungen}. 
 \end{genericdf}

Die elementaren Eigenschaften dieser Operatoren werden uns im Laufe dieser Arbeit häufig begegnen, weswegen wir diese direkt beweisen werden.

\begin{lem}\label{th:properties_TF}
	\begin{enumerate}[label =\textbf{(\roman*)}]
		\item Es gilt
		\begin{align}\label{eq:trans_mod_commutation_relation_1}
		T_x M_\omega = e^{-2 \pi \mathrm{i} x \cdot \omega} M_\omega T_x
		\end{align}
		Damit kommutieren $T_x$ und $M_\omega$ genau dann, wenn $x \cdot \omega \in \mathbb{Z}$.
		\item
		Zeit-Frequenz-Verschiebungen sind Isometrien auf $L^p$ für $1 \leq p \leq \infty$.
		\item
		Es gelten
		\begin{equation}\label{eq:trans_mod_fourier_1}
		\widehat{(T_x f)}= M_{-x} \hat{f} \ \text{und} \ \widehat{(M_\omega f)} = T_\omega \hat{f},
		\end{equation}
		woraus
		\begin{equation}\label{eq:trans_mod_fourier_2}
		\widehat{(T_x M_\omega f)} = M_{-x}T_\omega \hat{f} = e^{-2\pi \mathrm{i} x \cdot \omega} T_\omega M_{-x} \hat{f}
		\end{equation}
		folgt.
	\end{enumerate}
\end{lem}

\begin{proof}
	\begin{enumerate}[label =\textbf{(\roman*)}] 
		\item
		Durch
		\begin{align*}
		T_x M_\omega f(t) 
		&= M_\omega f(t-x)
		= e^{2\pi \mathrm{i} (t-x) \cdot \omega} f(t-x)\\
		&= e^{-2\pi \mathrm{i} x \cdot \omega} e^{2\pi \mathrm{i} t \cdot \omega} T_x f(t)
		= e^{-2\pi \mathrm{i} x \cdot \omega} M_\omega T_x f(t)
		\end{align*}
		erhalten wir die Aussage.
		
		\item
		Sei $1\leq p < \infty$.
		Es gilt
		\begin{align*}
		\| M_\omega T_x f\|_p^p
		=
		\int \limits_{\mathbb{R}^d} | M_\omega T_x f(t) |^p \td{t}
		=
		\int \limits_{\mathbb{R}^d} | T_x f(t) |^p \td{t}
		=
		\int \limits_{\mathbb{R}^d} | f(t) |^p \td{t}
		=
		\|  f\|_p^p
		\end{align*}
		für $1\leq p < \infty$.
		Der Beweis für $p = \infty$ funktioniert analog.
		
		\item
		Durch
		\begin{align*}
		\widehat{(T_x f)} (\omega) 
		=
		\int \limits_{\R^d} f(t-x) e^{-2 \pi \mathrm{i } t \cdot \omega} \td{t}
		=
		e^{-2 \i x\cdot \omega}
		\int \limits_{\R^d} f(u) e^{-2 \pi \mathrm{i} u \cdot \omega} \td{u}
		= M_{-x} \hat{f}(\omega)
		\end{align*}
		erhalten wir die erste Aussage.
		Die Zweite folgt analog.
		Durch Anwenden der beiden Aussagen
		erhalten wir die Folgerung.
	\end{enumerate}
\end{proof}


\begin{df}
Wir bezeichnen mit
\begin{align}\label{eq:involution}
f^\ast(x) = \overline{f(-x)}
\end{align}
die \textit{Involution}.	
\end{df}
Für die Involution ist 
\begin{align}\label{eq:involution_fouriertrans}
\widehat{f^\ast} = \hat{\overline{f}}
\end{align}
erfüllt. Die Faltung lässt sich nun auch durch
\begin{align}\label{eq:convolution_with_translation}
(f \ast g)(x) = \langle f, T_x g^\ast \rangle
\end{align}
beschreiben, falls beide Seiten definiert sind.
Der Differentiations -und Multiplikationsoperator ist durch
\begin{align*}
D^\alpha f
&=
\prod \limits_{j=1}^d \left(\frac{\partial^{\alpha_j} }{\partial x_j^{\alpha_j}}\right) f\\
(X^\beta f)(t)
&= t^\beta f(t) 
=
\left(\prod \limits_{j = 1}^d t_j^{\beta_j}\right) f(t)
\end{align*}
für $ \alpha,\beta \in \N_0^d $ definiert.
Für $ \alpha ,\beta\in \N_0^d $ verwenden wir außerdem:
\begin{align*}
|\alpha | &:= \sum \limits_{j =1}^d \alpha_j\\
x^\alpha &:= \sum \limits_{j=1}^d x_j^{\alpha_j}\\
\alpha \leq \beta &:\Leftrightarrow
\forall j \in \{1,...,d\} : \alpha_j \leq \beta_j.
\end{align*}
Für die Fouriertransformation gelten
\begin{align}\label{eq:fourier_derivate_1}
\widehat{(D^\alpha f)}(\omega)
=(2 \pi  \i \omega)^\alpha \hat{f}(\omega)
\end{align}
und 
\begin{align}\label{eq:fourier_derivate_2}
\widehat{((- 2 \pi \i x)^\alpha f)}(\omega)
=
D^\alpha \hat{f}(\omega).
\end{align}

Der \textit{Schwartzraum} $ \S(\R^d) $ ist über die Endlichkeit der Seminormen $ \| D^\alpha X^\beta f \|_\infty $ für $ f \in C^\infty(\R^d) $ definiert, d.h. es gilt
\begin{align*}
\| D^\alpha X^\beta f \|_\infty
=
\sup \limits_{x \in \R^d} | D^\alpha X^\beta f(x) |
 < \infty
\end{align*}
für alle $ \alpha,\beta \in \N_0^d $.
Eine Folge $ f_n \in \S(\R^d) $ konvergiert gegen $ f \in \S(\R^d) $, falls
\begin{align}
\| D^\alpha X^\beta (f_n - f ) \|_\infty \to 0
\end{align}
für alle $ \alpha,\beta \in \N_0^d $ gilt.

\newpage 
\begin{sz}
	Die Fouriertransformation $ \F : \S(\R^d) \to \S(\R^d)  $ ist ein Isomorphismus, d.h. stetig und bijektiv mit stetiger Inversen.
	Die inverse Abbildung ist durch
	\begin{align}\label{eq:fouriertransformation_inverse}
	(\F^{-1} \hat{f} )(x) = f(x) 
	=
	\int \limits_{\R^d} \hat{f}(\omega)e^{2\pi \i x \cdot \omega} \td{\omega}
	\end{align}
	gegeben.
\end{sz}
Der Dualraum $ \S^\prime(\R^d) $ ist der Raum der temperierten Distributionen und für $ f \in \S^\prime(\R^d) $ gilt folgende Stetigkeitsabschätzung:\\
Es existieren $ C >0  $ und $ N,M > 0  $, sodass
\begin{align}\label{eq:cont_distribution}
\langle f, \varphi \rangle 
\leq
C 
\sum \limits_{|\alpha| \leq M}
\sum \limits_{|\beta| \leq N}
\| D^\alpha X^\beta \varphi \|_\infty
\end{align}
für alle $ \varphi \in \S(\R^d) $ gilt. Sei $ f \in \S^\prime(\R^d) $.\\ Die Fouriertransformation von $ f $  ist durch
\begin{align}
\hat{f} (\varphi ) := \langle \hat{f}, \varphi \rangle := \langle f , \hat{\varphi} \rangle
\end{align}
für alle $ \varphi \in \S(\R^d) $ definiert. 

\begin{genericthm}{Open-Mapping-Theorem}\label{th:open_mapping}
	Gegeben seien die Banach -oder Frecheträume  $ B_1 $ und $ B_2 $ und der surjektive beschränkte Operator $ A : B_1  \to B_2 $.
	Dann ist $ A $ eine offene Abbildung.
	Insbesondere können wir diesen auf Operatoren von $ \S(\R^d)  $ nach $ \S(\R^d) $ anwenden.
\end{genericthm}

\begin{genericthm}{Closed-Range-Theorem}\label{th:closed_range}
	Gegeben seien die Banachräume $ B_1 $ und $ B_2 $
	und ein beschränkter linearer Operator $ A : B_1 \to B_2 $.
	Der Operator $ A $ hat ein abgeschlossenes Bild genau dann, wenn der 
	adjungierte Operator $ A^\prime : B_2^\prime \to B_1^\prime $
	ein abgeschlossenes Bild besitzt.
	Insbesondere ist $ A $ genau dann surjektiv, wenn $ A^\prime $ injektiv ist und ein abgeschlossenes Bild hat.
\end{genericthm}

%\newpage
%\input{sections/allgemeine_banachraum_theorie}
\newpage
%\section{Bedingte Konvergenz in Banachräumen}
\chapter{Bedingte Konvergenz in Banachräumen}
Dieser Abschnitt widmet sich der Untersuchung des Konvergenzbereiches von bedingt konvergente Reihen in beliebigen Banachräume.
Die  Struktur richtet sich nach den Werken \cite{Kadets1991} und \cite{Kadets1997} von Mikhail und Vladimir Kadets.
Wir beginnen mit dem Satz von Steinitz für endlichdimensionale Räume. 
Anschließend beweisen wir eine hinreichende Bedingung für die Steinizeigenschaft in beliebigen Banachräumen.
Diese verwenden wir um die von Pecherskii in \cite{Pecherski1989} bewiesene Aussage über die Steinitzeigenschaft zu zeigen. 
Der Satz von Pecherskii wurde unabhängig auch von Chobanyan\cite{Chobanyan1989} bewiesen. 
Unser Beweis wird sich nach dem von Chobanyan richten. Dies wird nebenbei ein Schema für den Nachweis der Steinitzeigenschaft liefern.
Beide Sätze werden über Rundungs-und Umordnungseigenschaften bewiesen. 
Diese Aussagen gliedern wir in dafür vorgesehene Abschnitte und sortieren sie nach endlichdimensionalen und unendlichdimensionalen Voraussetzungen.
Im Anschluss werden wir uns der Steinitzeigenschaft aus einer anderen Richtung nähern und den Satz von Chobanyan beweisen.
Dieser wird uns zeigen, dass zwischen unbedingter Summierbarkeit und der Steinitzeigenschaft nichts weiter als eine Nullmenge liegt.
Mit der von Chobanyan in \cite{Chobanyan1989} bewiesenen Aussage können wir Voraussetzungen für die Steinitzeigenschaft in $ \L^p $-Räumen herleiten.
Diese wurden bereits von M. I. Kadets in der Arbeit \cite{Kadets1954} bewiesen. 


\newpage
\section{Endlichdimensionale Umordnungs-und Rundungssätze}
%\subsection{Endlichdimensionale Umordnungs-und Rundungssätze}
In diesem Abschnitt sammeln wir Umordnungs-und Rundungsaussagen mit endlichdimensionalen Voraussetzungen.
Die Basis dieser Aussagen bildet das nachfolgende Lemma über Polyeder.

\begin{genericthm}{Eckenlemma für Polyeder}\label{thm:polyeder_lemma}
	Sei $ K $ ein Polyeder im $ \R^n $, welches durch die Funktionale
	\begin{align*}
		f_{i}(x) &= a_i, \quad i = 1,...,p\\
		g_{j}(x) &\leq b_j, \quad j = 1,..,q
	\end{align*}
	für $ x \in \R^n $ gegeben ist. 
	Sei $ x_0 $ eine Ecke des Polyeders $ K $ und $ A:= \{ j \ | \ g_j(x_0) = b_j \} $.
	Dann gilt $ | A | \geq n -p $.
\end{genericthm}

\begin{proof}
	Sei o.B.d.A. $ x_0 = 0 \in \R^n $ eine Ecke des Polyeders. 
	Angenommen es gilt $ |A| < n - p $.
	Dann besitzt das System
	\begin{align*}
		f_j(x) &= 0, \quad i = 1,...,p\\
		g_j(x) &= 0, \quad j \in A
	\end{align*}
	weniger Gleichungen als Unbekannte. Damit existiert eine Lösung $ x_1 \neq 0 $ und  ein $ \varepsilon > 0  $ mit $ \varepsilon x_1 \in K $.
	Dies ist ein Widerspruch zur Eckeneigenschaft von $ x_0 $.
	Damit gilt die gewünschte Aussage.
\end{proof}

\begin{genericthm}{Rundungslemma I}\label{thm:rounding_lemma}
	Sei $ X $ ein endlichdimensionaler normierter Raum, $ \{ x_i \}_{i=1}^n \subset X $ eine endliche Teilmenge, $ \{ \lambda_i\}_{i = 1}^n $ Koeffizienten mit $ \lambda_i \in [0,1] $
	und $ x = \sum_{i= 1}^n \lambda_i x_i $.
	Dann existieren Koeffizienten $ \{\Theta_i\}_{i=1}^n $ mit $ \Theta_i = 0 $ oder $ \Theta_i = 1 $, sodass 
	\begin{align}\label{eq:rounding_off_coefficients}
		\left\| x - \sum \limits_{i=1}^n \Theta_i x_i \right\| \leq \frac{\dim  X }{2} \max_{1 \leq i \leq n} \| x_i \|
	\end{align}
	gilt.
\end{genericthm}

\begin{proof}
	Sei $ m := \dim \ X $.
	Wir führen eine Fallunterscheidung durch.
	Der erste Fall ist $ n \leq m $. Mit 
	\begin{align*}
		\Theta_i = 
		\begin{cases}
			0 \ &, \ \text{falls } \lambda_i \leq \frac{1}{2}\\
			1 \ &, \ \text{falls } \lambda_i > \frac{1}{2}
		\end{cases}
	\end{align*}
	erhalten wir wegen
	\begin{align*}
		\left\| x - \sum \limits_{i=1}^n \Theta_i x_i \right\|
		\leq
		\sum \limits_{i=1}^n\underbrace{|\lambda_i -\Theta_i|}_{\leq \frac{1}{2}} \|x_i \| 
		\leq \frac{n}{2} \max_{1 \leq i \leq n} \| x_i \|
	\end{align*}
	die Aussage. In dem Fall $ n > m $ wenden wir das Polyederlemma an.	Hierbei konstruieren wir das Polyeder $ P $ durch das System
	\begin{align*}
		0 &\leq t_i \leq 1 , \ i = 1,...,n\\
		x &= 
		\sum_{i =1}^n t_i x_i
	\end{align*}
	in dem Koeffizientenraum $ \R^n $. 
	%wegen lambda
	Nach Voraussetzung ist $ P  $ nicht leer und beschränkt. 
	Also existiert eine Ecke $ x_0 = (\tilde{t}_i)_{i=1}^n $.
	Nach dem Polyederlemma sind (mindestens) $ n - m $ Koordinaten entweder $ 0 $ oder $ 1 $.
	Mit 
	\begin{align*}
		\Theta_i = 
		\begin{cases}
			1 \ &, \ \text{falls } \frac{1}{2} <\tilde{t}_i < 1\\
			0 \ &, \ \text{falls } 0 \leq  \tilde{t}_i \leq \frac{1}{2}
		\end{cases}
	\end{align*}
	erhalten wir analog zum ersten Fall
	\begin{align*}
		\left\| x - \sum \limits_{i=1}^n \Theta_i x_i \right\|
		\leq
		\sum \limits_{i=1}^n|\tilde{t}_i -\Theta_i| \|x_i \| 
		\leq
		\underbrace{\left(\sum \limits_{i=1}^n|\tilde{t}_i -\Theta_i| \right) }_{n-m \ \textrm{Summanden } = 0}
		\max_{1 \leq i \leq n} \| x_i \|
		\leq
		\frac{\dim  X}{2} \max_{1 \leq i \leq n} \| x_i \|.
	\end{align*}
\end{proof}

\begin{genericthm}{Umordnungslemma I}\label{thm:rearrangment_lemma}
	Sei $ X $ ein normierter Raum mit $ \dim X = m < \infty $, $ \{ x_i \}_{i = 1}^n \subset X $ eine endliche Teilmenge und $ x := \sum_{i = 1}^n x_i$.
	Dann existiert eine Permutation $ \pi $ von $ \{1,...,n\} $, sodass für alle $ k \leq n $ die Ungleichung
	\begin{align}\label{eq:rearrangment_lemma_inequality}
		\left\|
		\sum \limits_{i = 1}^k x_{\pi(i)} - \frac{k-m}{n} x 
		\right\|
		\leq 
		m \max \limits_{1 \leq i \leq n } \| x_i \|
	\end{align}
	gilt.
\end{genericthm}

\begin{proof}
	Der einfache Fall ist $ n \leq m $. Wegen $ k \leq n \leq m $ folgt die Aussage mit
	\begin{align*}
		\left\|
		\sum \limits_{i = 1}^k x_{\pi(i)} - \frac{k-m}{n}
		\sum \limits_{i= 1}^n x_i
		\right\|
		\leq
		\sum \limits_{i = 1}^k \| x_{\pi(i)} \| 
		+
		\frac{m-k}{n} \sum \limits_{i= 1}^n \| x_i \|
		\leq
		m \max_{1 \leq i \leq n} \| x_i \|.
	\end{align*}
	Wir betrachten nun den Fall $ n > m $.
	Da die Ungleichung \eqref{eq:rearrangment_lemma_inequality} homogen bezüglich den Komponenten $ x_i  $ ist, nehmen wir o.B.d.A. an, dass $ \max_{1 \leq i \leq n} \|x_i\| = 1 $ ist.
	Wir werden nun (rückwärts) induktiv eine Mengenkette $ A_m \subset ... \subset A_{n-1} \subset A_{n} = \{1,...,n\}$ und Koeffizienten $ \lambda_k^i $ mit $ k = m,m+1,...,n $ und $ i \in A_k $ konstruieren, sodass die Bedingungen
	\begin{equation}\label{eq:proof_rearrangement_lemma_condition_1}
		\begin{split}
			&|A_k| = k, \quad \forall i \in A_k: \ 0 \leq \lambda^i_k \leq 1 \\
			&\sum \limits_{i \in A_k} \lambda_k^i = k - m\\
			&\sum \limits_{i \in A_k} \lambda_k^i x_i = \frac{k-m}{n} x
		\end{split}
	\end{equation}
	erfüllt sind.
	Für $ k = n  $ genügt es $ A_n = \{1,...,n\} $ mit $ \lambda_n^i = \frac{n-m}{n} $ zu wählen.
	Angenommen $ A_{k+1} $ und $ \{\lambda_{k+1}^i\}_{i \in A_{k+1}} $ wurden bereits konstruiert.
	Wir betrachten die Menge $ K = \{\mu_i , i \in A_{k+1}\} $, deren Elemente 
	\begin{equation}\label{eq:proof_rearrangement_lemma_condition_2}
				\forall i \in A_{k+1}: \ 0 \leq \mu_i \leq 1, \ \
				\sum \limits_{i \in A_{k+1} } \mu_i = k- m, \ \
				\sum \limits_{i \in A_{k+1}} \mu_i x_i = \frac{k-m}{n} x
	\end{equation}
	erfüllen.
	Unser Ziel ist es das Polyederlemma anzuwenden.
	Mit der Wahl $ \mu_i = \frac{k-m}{k+1-m} \lambda_{k+1}^i $ sehen wir, dass $ K $ nicht leer ist und ein Polyeder im $ \R^{k+1} $ formt.
	Die beiden Summen liefern $ m+1 $ Gleichungen und aus den Schranken für $ \mu_i $ erhalten wir $ 2(k+1) $ Ungleichungen. 
	In der Darstellung des Polyederlemmas bedeutet dies $ q = m+1  $ und $ p = 2(k+1) $.
	Wegen $ \mu_i \in [0,1] $ für $ i \in A_{k+1} $ ist $  K $ beschränkt.
	Damit existieren Ecken des Polyeders.
	Sei $ \overline{\mu} = (\overline{\mu}_i)_{i \in A_{k+1}} $ eine (beliebige) Ecke von $ K $. 
	Dann liefert das Polyederlemma
	\begin{align*}
		| A | \geq (k+1) - (m+1 ) = k - m.
	\end{align*}
	für $ A = \{i \in A_{k+1} \ : \ \overline{\mu}_i = 0 \ \wedge \ \overline{\mu}_i = 1\} $.
	Wir werden nun zeigen, dass mindestens ein $ i \in A $ mit $ \overline{\mu}_i = 0 $ existiert.
	Angenommen es gilt $ \overline{\mu}_i = 1 $ für alle $ i \in A $.
	Dann liefert die erste Summe in der Bedingung \eqref{eq:proof_rearrangement_lemma_condition_2},
	dass $ |A | = k - m $ ist und $ \overline{\mu}_i = 1 $ für alle $ A_{k+1}\setminus A $ gilt. Wegen $ A_{k+1 } \setminus A \subset A $ erhalten wir einen Widerspruch zu der Annahme.
	Damit existiert ein $ j \in A $ mit $ \overline{\mu}_j = 0 $ und wir können
	\begin{align*}
		A_k := A_{k+1} \setminus \{j\}, \quad \lambda_k^i := \overline{\mu}_i
	\end{align*}
	für $ i \in A_k $ definieren. Hierfür ist sind die Bedingungen in \eqref{eq:proof_rearrangement_lemma_condition_1} erfüllt, womit die Konstruktion abgeschlossen ist.
	Aus der Konstruktion können wir eine Permutation $ \pi \in \mathrm{S}_{\{1,...,n\}} $ wie folgt definieren:
	\begin{align*}
		\pi(i)
		=
		\begin{cases}
			\ \quad j, &\ \text{falls } A_{i-1} := A_i \setminus \{j\} \  i = m+1,...,n\\
			\textrm{beliebig}, &\ \text{falls } i \leq m
		\end{cases}.
	\end{align*}
	Zum Schluss bleibt noch zu zeigen, dass diese Permutation die Ungleichung \eqref{eq:rearrangment_lemma_inequality} erfüllt.
	Für $ k \leq m  $ ist das Argument analog zum ersten Fall.
	Für $ k > m  $ verwenden wir die Bedingung \eqref{eq:proof_rearrangement_lemma_condition_1}:
	\begin{align*}
		\left\|
		\sum \limits_{i = 1}^k x_{\pi(i)} - \frac{k-m}{n} x 
		\right\|
		=
		\left\|
		\sum \limits_{i \in A_k} x_{i} - \sum \limits_{i \in A_k} \lambda_k^ix_{i} 
		\right\|
		=
		\left\|
		\sum \limits_{i \in A_k} ( 1 - \lambda_k^i )x_{i} 
		\right\|
		\leq
		\sum \limits_{i \in A_k} ( 1 - \lambda_k^i ) 
		%= k - (k-m ) 
		= m.
	\end{align*}
\end{proof}
Mit der Dreicksungleichung nach unten und geeigneten Abschätzungen ergibt sich die Formulierung des letzten Lemmas, welche wir für den Beweis des Satzes von Steinitz \ref{th:lemma_of_steiniz} verwenden. Durch die Zusatzvoraussetzung $ x = 0 $ erhalten wir ein von Steinitz gezeigtes Lemma.
Die dort auftretende (exakte) Konstante ist nach wie vor von Interesse.
\begin{kor}\label{th:rearrangement_lemma_kor}
	Unter den Voraussetzungen des Umordnungslemmas gilt
	\begin{align}
		\left\| \sum \limits_{i = 1}^k x_{\pi(i)} \right\|
		\leq m \max_{1 \leq i \leq n} \|x_i\| + (m+1) \|x\|.
	\end{align}
\end{kor}

\begin{genericthm}{Steinitz-Lemma}\label{th:steiniz_lemma}
	Unter den Voraussetzungen des Umordnungslemmas und
	$ x := \sum_{i = 1}^n x_i = 0$ existiert eine Permutation $ \pi $ von $ \{1,...,n\} $ und eine Konstante $ K > 0 $,
	sodass
	\begin{align*}
		\max \limits_{j \leq n} \left\|
		\sum \limits_{i = 1}^j
		x_{\pi(i)}
		\right\|
		\leq 
		K 
		\max \limits_{i} \| x_i\|
	\end{align*}
	gilt.
	Hierbei hängt die Konstante $ K $ nur von dem Raum $ X $ ab. 
\end{genericthm}

Wegen der Abhängigkeit von $ X $ schreiben wir $ K_X $ und wir bezeichnen diese Konstante als \textit{Steinitzkonstante}.
Die vorhergehenden Aussagen liefern die Dimension des Raumes als Schranke für $ K_X $. 
Damit gilt $ K_X \leq  \dim X $.
Eine Idee ist die Normstruktur der endlichdimensonalen Räume zu verwenden.
Für die Ebene mit der euklidischen Norm wurde von Banaszczyk\cite{Banaszczyk1987} durch $ \frac{\sqrt{5}}{2} $ der optimale Wert für $ K_X  $ gefunden. Außerdem existiert mit
\begin{align*}
	K_X \leq \dim X -1 + \frac{1}{\dim X}
\end{align*}
eine schärfere Schranke. Jedoch sind für $ \dim X > 2 $ keine expliziten Werte mehr bekannt.
Für genauere Abschätzungen können wir \cite{Banaszczyk1990} zu Rate ziehen.

\newpage
\begin{lem}\label{th:sign_inequality_finite_dim}
	Sei $ X $ ein normierter Raum mit $ \dim X = m < \infty $ und $ \{ x_i \}_{i = 1}^n \subset X $ eine endliche Teilmenge.
	Dann existieren $ \alpha_i = \pm 1 $   , sodass
	\begin{align}
		\max \limits_{j \leq n}
		\left\|
		\sum \limits_{i=1}^j \alpha_i x_i 		
		\right\|
		\leq 
		2 m \max \limits_{1 \leq i \leq n } \| x_i \|
	\end{align}
	gilt.
\end{lem}

\begin{proof}
	Für $ n \leq m $ folgt die Aussage unmittelbar aus der Dreiecksungleichung mit einer beliebigen Wahl von $ \alpha $.
	Wir widmen uns deswegen direkt dem Fall $ n > m $. 
	Unser Ziel ist es das Polyederlemma \ref{thm:polyeder_lemma} $ n-m $ mal anzuwenden damit eine geeignete Wahl von $  \alpha \in \{\pm 1\}^n  $ zu erreichen.
	Wir definieren das Polyeder $ K_1 \subset \R^{m+1} $ durch
	\begin{align*}
		\sum \limits_{i = 1}^{m+1 } t_i x_i&= 0, \quad
		-1 \leq t_i \leq 1 , \ i = 1,..., m+1,
	\end{align*}
	wobei $ (t_i)_{i=1}^{m+1} $ die zugehörigen Koordinaten bezeichnet und die Summe $ m $ Gleichungen liefert.
	Das Polyeder ist wegen $ 0 \in K_1 $  nicht leer und aufgrund der Ungleichungen beschränkt.
	Damit ist die Existenz von Ecken gesichert. 
	In der Notation des Polyederlemmas heißt dies
	\begin{align*}
		|A_1 | \geq m+1 - m = 1.
	\end{align*}
	Damit existiert eine Ecke $ (t^1_i)_{i=1}^{m+1} $
	mit einer Koordinate $ t^1_{i_1} = \pm 1$ für ein $ 1 \leq i_1 \leq m+1 $.
%	womit eine Ecke $ (t^1_i)_{i=1}^{m+1} $ mit einer Koordinate $ %t^1_{i_1} = \pm 1$, wobei $1 \leq i_1 \leq m+1 $, existiert.
	Hiermit definieren wir das nicht-leere und beschränkte Polyeder $ K_2 \subset \R^{m+2} $ durch
	\begin{align*}
		t_{i_1} = t_{i_1}^1
		\sum \limits_{i = 1}^{m+1 } t_i x_i = 0, \quad 
		-1 \leq t_i \leq 1 , \ i &\in \{1,...,m+2\} \setminus \{i_1\}.
	\end{align*}
	Mit dem Polyederlemma erhalten wir eine Ecke $ (t^2_i)_{i=1}^{m+2} $
	mit einer Koordinate $ t_{i_2}^2 = \pm 1 $ für ein $ i_2 \in \{1,...,m+2\} \setminus \{i_1\} $.
	Außerdem gilt nach Konstruktion $ t^2_{i_1} = t^1_{i_1} $.
	Wenn wir diese Konstruktion $ n-m $ mal durchführen, erhalten wir Indizes 
	$ i_1,i_2,...,i_{n-m} $ und Koordinaten $ (t_i)^{m-k}_{i=1} $ für $ k = 1,...,m-n $. Diese erfüllen die Bedingungen
	\begin{align*}
		\sum \limits_{i = 1}^{m+k} t_i^k x_i = 0, \
		| t_i^k | \leq 1, \
		|t_{i_k}| = 1 , \ \text{und} \
		t^k_{i_j} = t^j_{i_j} \ \text{für} \ k > j.
	\end{align*}
	Wir definieren $ \alpha \in \{\pm 1\}^n $ durch
	\begin{align*}
		\alpha_i = 
		\begin{cases}
			t^k_{i_k}, &\ \text{falls } i = i_k, k = 1, ..., n - m  \\
			1 \ , &\ \text{sonst}
		\end{cases}.
	\end{align*}
	Nun bleibt noch zu zeigen, dass die Konstruktion das Gewünschte erfüllt.
	Für $ j \leq m $ folgt
	\begin{align*}
		\left\|
		\sum \limits_{i = 1 }^j \alpha_i x_i 
		\right\|
		\leq
		m \cdot \max \limits_{1 \leq i \leq n } \| x_i \|
	\end{align*} 
	mit der Dreiecksungleichung. Für $ j > m $ setzen wir $ j = m + k $ mit $ k = 1,...,n-m $.
	Nach der Konstruktion enthält $ \sum_{i=1}^{m+l} (\alpha_i - t_i^l) x_i $ maximal $ m $ Summanden ungleich Null und es gilt $ \sum_{i=1}^{m+l} t_i^l x_i =0 $.
	Damit folgt mit 
	\begin{align*}
		\left\|
		\sum \limits_{i = 1 }^{m+l} \alpha_i x_i 
		\right\|
		=
		\left\|
		\sum \limits_{i = 1 }^{m+l} \underbrace{(\alpha_i - t_i^l)}_{| \cdot | \leq 2} x_i 
		\right\|
		\leq 2 m \cdot \max \limits_{1 \leq i \leq n } \| x_i \|
	\end{align*}
	die Aussage.
\end{proof}

\section{Der Satz von Steinitz}
%\subsection{Der Satz von Steinitz}
\begin{df}
	Sei $ (x_n) $ eine Folge in einem Banachraum  $ X $. 
	Dann ist die \textit{Menge der Partialsummen} durch
	\begin{align*}
		P_{(x_n)} = 
		\left\{ 
		\sum \limits_{k=1}^n x_{i_k} \  |  \ i_1 < ... < i_n, n \in \N
		\right\}
	\end{align*}
	definiert. Die \textit{Menge der $ [0,1] $-Kombinationen} ist durch
	\begin{align*}
		Q_{(x_n)}
		=
		\left\{ 
		\sum \limits_{i=1}^n \lambda_i x_i \ | \
		0 \leq \lambda_i \leq 1 , n \in \N
		\right\}
	\end{align*}
	gegeben. 
\end{df}

Falls keine Verwechslungsgefahr besteht, schreiben wir auch $ P $ oder $ Q $.
Wenn wir aktiv die zugrundeliegende Menge der Folgenglieder verändern, greifen wir auf die von Kadets\cite{Kadets1997} verwendete Notation $ P(\{x_i\}_{i = 1}^\infty) $ und $ Q(\{x_i\}_{i = 1}^\infty) $
zurück.

\begin{genericthm}{Eigenschaften}  
	\begin{enumerate}
		\item Es gilt $ P \subset Q $.
		\item $ Q $ ist konvex.
	\end{enumerate}
\end{genericthm}
%\begin{proof}
%	\begin{enumerate}
%		\item 
%		Folgt unmittelbar durch eine geeignete Wahl der Koeffizienten.
%		\item
%		Folgt aus der Konvexkombination der einzelnen Summanden.
%	\end{enumerate}
%\end{proof}
Die erste Eigenschaft folgt durch eine geeignete Wahl der Koeffizienten und die Zweite aus der Konvexkombination der einzelnen Summanden.

Das Rundungslemma \ref{thm:rounding_lemma} lässt sich auch folgendermaßen formulieren.
Sei $ X $ endlichdimensional und $ (x_n) $ eine Folge in $ X $.
Dann existiert zu jedem $ q \in Q $ ein $ p \in P $, sodass
\begin{align}
	\| q - p \| \leq \frac{\dim X}{2} \max_{1 \leq i \leq n} \|x_i \|
\end{align}
gilt.
Das $ n $ erhalten wir aus der Definition von $ q $. Diese kompaktere Darstellung werden wir in dem Beweis des Satzes von Steinitz \ref{th:lemma_of_steiniz} verwenden.
Das nachfolgende fundamentale Lemma für den Beweis des Satzes von Steinitz wurde von V. Fonf 1972 in \cite{Fonf1972} unter allgemeineren Voraussetzungen bewiesen.


\begin{lem}\label{thm:affine_space_is_subset}
	Sei $ X $ ein beliebiger Banachraum und $ (x_n) $ bedingt summierbar in $ X $ mit Grenzwert $ s $.
	Falls $ s \in \overline{Q} $ gilt, erhalten wir $ s + \Gamma_0 \subset \overline{Q} $.
	%	\begin{align*}
	%	x \in \overline{Q} \Rightarrow x + \Gamma_0 \subset \overline{Q}.
	%	\end{align*}
\end{lem}

Dieses Lemma bleibt auch gültig, wenn wir endlich viele Glieder der Folge $ (x_n) $ entfernen oder endlich viele Werte aus $ X $ hinzu addieren. Damit gilt
\begin{align}\label{eq:affine_space_is_subset_special}
	s + y \in \overline{Q}(\{x_i\}_{i = 1}^\infty \setminus A) +y
	\ \Rightarrow \
	s + \Gamma_0 + y \subseteq  \overline{Q}(\{x_i\}_{i = 1}^\infty \setminus A) +y
\end{align}
für eine endliche Teilmenge $ A \subset \{x_i\}_{i = 1}^\infty $ und $ y \in X $. Da der Umgang hiermit recht sperrig ist, werden wir diese Extras in dem Beweis weglassen.


\begin{proof}
	Wir beginnen mit einer Vorüberlegung.
	Sei $ x^\prime \in X^\prime \setminus \Gamma $.
	Dann ist $ (\langle x^\prime,x_i \rangle) $ über $ \R $ bedingt summierbar, aber es liegt keine absolute Konvergenz vor.
	Dann gilt nach der Fallunterscheidung in dem Beweis des Riemannschen Umordnungssatzes $ \sup_{y \in P} \langle x^\prime ,y \rangle = \infty $. 
	Wegen $ P \subset Q $ folgt $ \sup_{y \in Q} \langle x^\prime ,y \rangle = \infty $.\\
	Nun führen wir den eigentlichen Beweis.
	Sei $ s \in \overline{Q} $.
	Angenommen es existiert ein $ z \in \Gamma_0 $, sodass $ s + z \notin \overline{Q} $ gilt. Nach dem Satz von Hahn-Banach in der Trennungsversion \cite{Werner2011} existiert ein $ x^\prime \in X^\prime $, sodass
	\begin{align*}
		\sup \limits_{y \in \overline{Q}} \langle x^\prime ,y \rangle < \langle x^\prime , s+z \rangle
	\end{align*}
	gilt. 
	Wir unterscheiden nun die Fälle $ x^\prime \in \Gamma $ und $ x^\prime \in X^\prime \setminus \Gamma $.
	Für $ x^\prime \in \Gamma $ gilt $ \langle x^\prime, s + z \rangle = \langle x^\prime, s  \rangle $. Wegen $ s \in \overline{Q} $ ist
	$ \sup_{y \in \overline{Q}} \langle x^\prime ,y \rangle < \langle x^\prime , s \rangle $
	ein Widerspruch. Für $ x^\prime \in X^\prime \setminus \Gamma $ gilt nach der Vorüberlegung:
	\begin{align*}
		\infty =\sup \limits_{y \in \overline{Q}} \langle x^\prime ,y \rangle < \langle x^\prime , s+z \rangle < \infty.
	\end{align*}
	Dies ist auch ein Widerspruch.
	%$ f \in \Gamma $ und $ f(s + z) < \infty $ können wegen $  f(s + z )   < \infty $ nicht gelten. 
	Demnach war die Annahme falsch und die Aussage gilt.
\end{proof}

\newpage
\begin{lem}\label{th:lemma_apply_rounding_of_lemma_steinitz_finite_dimensional}
	Sei $ X $ ein $ m $-dimensionaler Raum und $ (x_n) $ summierbar mit dem Grenzwert $ s $. 
	Dann existiert für alle $ s^\prime  \in s + \Gamma_0 $ eine Permutation $ \pi_0 \in \mathrm{S}_\N $ und eine streng monotone Indexfolge $ (n_j) $, sodass
	\begin{align}
	\lim \limits_{j \to \infty}
	\left\| s^\prime - \sum \limits_{i = 1}^{n_j} x_{\pi_0(i)} 
	\right\| = 0
	\end{align}
	gilt.
	Dies bedeutet, dass eine Teilfolge der umgeordneten Partialsummenfolge gegen $ s^\prime $ konvergiert. 
\end{lem}
\begin{proof}
	Sei $ s^\prime \in s + \Gamma_0 $.
	Wegen $ s \in \overline{Q} $ gilt nach dem Lemma \ref{thm:affine_space_is_subset}
	$ s + \Gamma_0 \subset \overline{Q} $ und somit auch $ s^\prime \in \overline{Q} $.
	Sei $ (\varepsilon_n) $ ein streng monoton fallende Nullfolge.
	Dann existiert ein $ q_1 \in Q(\{x_i\}_{i =1}^\infty) $ und ein $ k_1 > 0 $, sodass
	\begin{align*}
		\| s^\prime - q_1 \| 
		=
		\left\|
		s^\prime - \sum \limits_{i = 1}^{k_1} \lambda_i^{(1)} x_i
		\right\|
		< \varepsilon_1
	\end{align*}
	gilt. Wir wenden nun das Rundungslemma \ref{thm:rounding_lemma} auf den Aufspann von $ x_1,...,x_{k_1} $ an. Nach diesem Lemma existiert ein $ p_1 \in P(\{x_i\}_{i =1}^\infty) $ mit
	\begin{align*}
		\| q_1 - p_1 \| = \left\| q_1 - \sum \limits_{i = 1}^{k_1} \Theta_i^{(1)} x_i \right\| \leq m \cdot \max \limits_{i \geq 1} \|x_i \|
	\end{align*}
	für entsprechende $ \Theta_i^{(1)} \in \{0,1\} $.
	Wir definieren $ S_1 := \{x_i \ : \ \Theta_i^{(1)} = 1\} \cup \{x_1\} $ und \\ $ s_1 := \sum_{x \in S_1} x $.
%	\begin{align*} 
%		S_1 := \{x_i \ : \ \Theta_i^{(1)} = 1\} \cup \{x_1\}, \quad 
%		s_1 := \sum \limits_{x \in S_1} x,
%	\end{align*}
	Dann gilt:
	\begin{align*}
		\|s^\prime - s_1 \| 
		\leq
		\| s^\prime  -q_1 \| + \| q_1 - s_1\|
		< 
		\varepsilon_1 + m \max_{i \geq 1} \| x_i \| + \| x_1\|.
	\end{align*}
	%gilt.
	Nun betrachten wir  $ s_1 + \overline{Q}(\{x_i\}_{i = 1}^\infty \setminus S_1) $. Mit dem Lemma \ref{thm:affine_space_is_subset} folgt dann $ s^\prime \in s_1 + \overline{Q}(\{x_i\}_{i = 1}^\infty \setminus S_1) $.
%	Wegen $ s \in s_1 + \overline{Q}(\{x_i\}_{i = 1}^\infty \setminus S_1) $ gilt mit Lemma \ref{thm:affine_space_is_subset} $ s^\prime \in s_1 + \overline{Q}(\{x_i\}_{i = 1}^\infty \setminus S_1) $.
	Damit gibt es auch hier ein $ q_2 \in Q(\{x_i\}_{i = 1}^\infty \setminus S_1) $ und ein $ k_2 > k_1 $, sodass
	\begin{align*}
	\|s^\prime - s_1 - q_2 \| 
	= 
	\left\| s^\prime - s_1 - \sum \limits_{i = 1, x_i \notin S_1 }^{k_2} \lambda_i^{(2)} x_i \right\| < \varepsilon_2
	\end{align*}
	gilt. Mit dem Rundungslemma erhalten wir ein $ p_2 \in P(\{x_i\}_{i = 1}^\infty \setminus S_1) $ mit
	\begin{align*}
	\| q_2 - p_2 \| 
	=
	\left\| q_2 - \sum \limits_{i=1,x_i \notin S_1}^{k_2} \Theta_i^{(2)} x_i \right\|
	\leq
	m \cdot \max \limits_{i\geq 2} \| x_i \| 
	\end{align*}
	für entsprechende $ \Theta_i^{(2)} \in \{0,1\} $. Wir definieren wieder
	\begin{align*}
	S_2 := S_1 \cup \{x_i \ : \ \Theta_i^{(2)} = 1\} \cup \{x_2\}, \quad 
	s_2 := \sum \limits_{x \in S_2} x,
	\end{align*}
	womit
	\begin{align*}
	\|s^\prime - s_2 \| 
	=
	\| s^\prime - s_1 -p_2 \|
	\leq
	\| s^\prime - s_1 -q_2 \| + \| q_2 - p_2\|
	< 
	\varepsilon_2 + m \max_{i \geq 2} \| x_i \| + \| x_2\|
	\end{align*}
	gilt. Diese Prinzip lässt sich analog für alle $ j \in \N $ fortführen.
	Damit gilt
	\begin{align*}
	\|s^\prime - s_j \| \leq \varepsilon_j + m \max_{i \geq j} \| x_i \| + \|x_j\| \rightarrow 0
	\end{align*}
	für $ j \to \infty $. Wir erhalten 
	\begin{align*}
	S_1 \subset S_2 \subset S_3 \subset ... \subset S_n \subset ..., \quad
	\bigcup \limits_{j \in \N} S_j = \{x_i\}_{i=1}^\infty.
	\end{align*}
	Die Teilmengenbeziehungen sind echt aufgrund der strengen Monotonie von $ (\varepsilon_n) $. 
	Mit $ n_j := | S_j | $ erhalten wir die streng monotone Indexfolge $ (n_j) $. Die Permutation $ \pi_0  $ ergibt sich aus der disjunkten Mengenfolge
	\begin{align*}
	\tilde{S}_1 := S_1, \quad 
	\tilde{S}_j := S_j \setminus S_{j-1} , \ j \geq 2.
	\end{align*}
	Insgesamt erhalten wir
	\begin{align*}
		\left\| s^\prime - \sum \limits_{i =1}^{n_j} x_{\pi_0(i)} \right\|
		= 
		\left\|
		s^\prime - \sum \limits_{i=1}^{j -1 } \sum \limits_{x \in \tilde{S}_i} x - \sum \limits_{x \in \tilde{S}_j} x
		\right\|
		= 
		\left\|
		s^\prime - s_j \right\|
		\rightarrow 0
	\end{align*}
	für $ j \to \infty $.
\end{proof}


\begin{genericthm}{Satz von Steinitz(1913)}\label{th:lemma_of_steiniz}
	Sei $ X $ ein $ m $-dimensionaler Raum und $ (x_n) $ summierbar mit dem Grenzwert $ s $.
	Dann besitzt $ (x_n) $ die Steinitzeigenschaft.
%	Dann ist $ \mathcal{K}\left(  x_k \right) $ ein affiner Unterraum, womit
%	\begin{align*}
%	\mathcal{K}_{\left(x_k \right)} =
%	s + \Gamma_0
%	\end{align*}
%	gilt.
\end{genericthm}
\begin{proof}
	Die Inklusion $ \mathcal{K}_{\left(x_n \right)} \subseteq
	s + \Gamma_0  $ wurde bereits in \ref{th:subset_conv_area} gezeigt. 
	Nun wenden wir uns der Beziehung $ s + \Gamma_0 \subset \mathcal{K}_{\left(  x_n \right)} $ zu.
	Sei $ s^\prime \in s + \Gamma_0 $. 
	Nach dem Lemma \ref{th:lemma_apply_rounding_of_lemma_steinitz_finite_dimensional} gibt es ein $ \pi_0 \in \mathrm{S}_\N $ und eine Indexfolge $ (n_j) $ mit
	\begin{align*}
		\lim \limits_{j \to \infty}
		\left\| s^\prime - \sum \limits_{k = 1}^{n_j} x_{\pi_0(k)} 
		\right\|
		=
		\left\| s^\prime - \sum \limits_{i = 1}^{n_j} x_{i} 
		\right\|
		 = 0.
	\end{align*}
	Um die Darstellung zu vereinfachen setzen wir $ \pi_0(k) = i $. 
	Wegen der Summierbarkeit von $ (x_n) $ gilt mit dem Cauchykriterium \ref{th:chauchy_crit}:
	\begin{align*}
		\lim \limits_{j \to \infty} \left\|  \sum \limits_{i = n_j +1}^{n_{j+1}} x_{i} 
		\right\| = 0.
	\end{align*}
	Wir wenden nun das Umordnungslemma in der Version \ref{th:rearrangement_lemma_kor} auf den endlichen Mengen
	$ \{ x_i \}_{i = n_j + 1}^{n_{j+1}} $ an.
	Damit gibt es ein $ \pi_j \in \mathrm{S}_{\{n_j + 1,..., n_{j+1}\}} $, sodass
	\begin{align*}
		\left\|
		\sum \limits_{k = n_j +1}^{r}
		x_{\pi_j(k)}
		\right\|
		\leq
		m \cdot \max \limits_{n_j + 1 \leq k \leq n_{j+1} } \| x_k \|
		+
		(m+1 ) 
		\left\| \sum \limits_{i = n_j +1}^r x_i \right\|
	\end{align*}
	für alle $ n_j < r \leq n_{j+1} $ gilt. 
	Wir definieren die globale Permutation $ \pi \in \mathrm{S}_\N $ durch\\ $ \pi\big|_{\{ x_i \}_{i = n_j + 1}^{n_{j+1}} } = \pi_j $ für $ j \in \N $ und $ \pi(i) = i $ für $ i \leq n_1 $.
	Dies führt insbesondere dazu, dass
	\begin{align*}
		\sum \limits_{k = 1}^{n_j} x_{\pi(k)}
		=
		\sum \limits_{i = 1}^{n_j} x_i
	\end{align*}
	gilt. Sei nun $ r > n_1  $ und $ j $ so gewählt, dass $ n_j + 1 \leq r \leq n_{j+1}  $ gilt.
	Dann erhalten wir mit
	\begin{align*}
		\left\|
		\sum \limits_{k=1}^r x_{\pi(k)} - s^\prime 
		\right\|
		&\leq
		\underbrace{\left\|
		\sum \limits_{k=1}^{n_j} x_{\pi(k)} - s^\prime 
		\right\| }_{
		= \| \sum x_i - s^\prime \|
		}
		+
		\left\|
		\sum \limits_{k = n_j +1}^{r}
		x_{\pi_j(k)}
		\right\|\\
		&\leq
		\left\|
		\sum \limits_{i=1}^{n_j} x_{i} - s^\prime 
		\right\|
		+
		m \cdot \max  \limits_{k > n_j } \| x_k \|
		+
		(m+1 ) 
		\left\| \sum \limits_{i = n_j +1}^r x_i \right\|
		\rightarrow 0
	\end{align*}
	für $ j \to \infty $, dass $ s^\prime \in \mathcal{K}_{\left(x_n \right)}  $ gilt.
	Damit erfüllt $ (x_n) $ die Steinitzeigenschaft.
\end{proof}



\section{Eine hinreichende Bedingung für die Steinitzeigenschaft}
%\subsection{Eine hinreichende Bedingung für die Steinizeigenschaft}
In diesem Abschnitt betrachten wir für beliebige Banachräume eine hinreichende Bedingung für die Steinitzeigenschaft.
Diese Bedingung setzt sich aus einer Umordnungs- und Rundungseigenschaft zusammen.
Damit besteht Ähnlichkeit zu dem Beweis des Satz von Steinitz, welcher aus solchen Eigenschaften folgt.
Für endlichdimensionale Räume folgt die hinreichende Bedingung unmittelbar mit dem Umordnungslemmas I \ref{thm:rearrangment_lemma} und dem Rundungslemma I \ref{thm:rounding_lemma}.
Also verkürzt sich der Beweis des Satzes von Steinitz auf diese zwei Lemmata.  
%In diesem Abschnitt betrachten wir in dem Satz \ref{th:prop_linear_conv_area} eine hinreichende Bedingung für die Steinizeigenschaft in einem beliebigen Banachraum $ X $.
%Diese Bedingung setzt sich aus einer Umordnungs-und Rundungseigenschaft zusammen.
%Damit besteht eine Ähnlichkeit zu dem Beweis  des Satzes von Steiniz.
%Für endlichdimensionale Räume folgt die hinreichende Bedingung unmittelbar durch das
%Umordnungslemmas I \ref{thm:rearrangment_lemma} und Rundungslemma I \ref{thm:rounding_lemma}.
%Also verkürzt sich der Beweis des Steinizsatzes auf diese zwei Lemmata.
%Wie für den Satz von Steiniz \ref{th:lemma_of_steiniz} benötigen wir hier eine 
%In der hinreichenden Bedingung setzen wir diese voraus. 
%Für endlichdimensionsionale Räume ergeben sich diese Aussagen unmittelbar aus dem Umordnungslemmas I \ref{thm:rearrangment_lemma} und dem Rundungslemma I \ref{thm:rounding_lemma}.
%Damit folgt der Satz von Steiniz aus dem nachfolgenden Satz. 
Die hinreichende Bedingung liefert uns eine Möglichkeit die Steinitzeigenschaft für Folgen in unendlichdimensionalen Räumen nachzuweisen.
Dafür fehlen uns noch konkrete Voraussetzungen.
Diese werden wir in den Abschnitten \ref{sc:hanani_pecherskii} und\ref{sc:lemma_of_chobanyan} herausarbeiten.

\begin{genericthm}{Satz}\label{th:prop_linear_conv_area}
	Sei  $ (x_n) $ eine Folge in einem Banachraum $ X $ mit den Eigenschaften:
	\begin{enumerate}
		\item\label{eq:prop_linear_conv_area_2_item}
		Für alle $ \varepsilon > 0 $ existiert ein $ N_\varepsilon \in \N $,
		sodass für $ \{y_i\}_{i =1}^n \subset \{x_i \}_{i = N_\varepsilon}^\infty$, $ \lambda \in [0,1]^n $
		ein $ \Theta \in \{0,1\}^n $ mit
		\begin{align}\label{eq:prop_linear_conv_area_2}
			\left\| \sum \limits_{i = 1}^n \lambda_i y_{i}- 
			\sum \limits_{i = 1}^n
			\Theta_i y_i \right\| \leq \varepsilon
		\end{align}
		existiert.
		
		\item\label{eq:prop_linear_conv_area_1_item} 
		Für alle $ \varepsilon > 0 $ existiert ein $ N_\varepsilon \in \N $ und ein $ \delta > 0 $, 
		sodass für $ \{y_i\}_{i =1}^n \subset \{x_i \}_{i = N_\varepsilon}^\infty$
		und $ \| \sum_{i=1}^n y_i \| \leq \delta_\varepsilon $ eine Permutation $ \pi \in \mathrm{S}_{\{1,...,n\}} $ mit
		\begin{align}\label{eq:prop_linear_conv_area_1}
			\max \limits_{j \leq n}
			\left\| \sum \limits_{i = 1}^j y_{\pi(i)} \right\| \leq \varepsilon
		\end{align}
		existiert.
		
	\end{enumerate}
	Dann besitzt $ (x_n) $ die Steinitzeigenschaft.
\end{genericthm}

Die erste Bedingung \ref{eq:prop_linear_conv_area_1_item} lässt sich auch anders formulieren.
Für alle $ \varepsilon > 0  $ exisitert ein $ N_\varepsilon \in \N $, sodass für alle $ q \in Q(\{x_i\}_{i = N_\varepsilon}^{\infty}) $ ein $ p \in P(\{x_i\}_{i = N_\varepsilon}^\infty) $ mit identischen Elementen aus $ X $ existiert, sodass 
\begin{align*}
	\| q - p \| 
	\leq \varepsilon
\end{align*}
gilt. 
Für endlichdimensionale Räume folgen die Bedingungen \ref{eq:prop_linear_conv_area_2_item} und \ref{eq:prop_linear_conv_area_1_item}
aus dem Rundungslemma I \ref{thm:rounding_lemma} und dem Umordnungslemmas I 
\ref{thm:rearrangment_lemma} .

%Die Bedingungen $ \ref{eq:prop_linear_conv_area_2_item} $ und \ref{eq:prop_linear_conv_area_1_item} entsprechen
%dem Steiniz-Lemma \ref{th:steiniz_lemma} als Folgerung des Umordnungslemmas I \ref{thm:rearrangment_lemma} und dem Rundungslemma I \ref{thm:rounding_lemma}. 
%Deren Problem ist die vorausgesetzte endliche Dimension.
%In dem nächsten Abschnitt werden wir ähnliche Aussagen unabhängig von der Dimension des Raumes beweisen und nebenbei ein weiteres hinreichendes Kriterium erhalten.
%Dieses Kriterium wird im Wesentlichen den Satz \ref{th:prop_linear_conv_area} implizieren.\\
%\\
Nun beweisen wir, dass der Satz \ref{th:prop_linear_conv_area} eine hinreichende Bedingung für die Steinitzeigenschaft ist.
Wir werden den Beweis analog zu dem Beweis des Satzes von Steinitz \ref{th:lemma_of_steiniz} strukturieren, da die Beweisideen identisch sind und dadurch die Unterschiede erkenntlich sind.
Wir beginnen mit einer allgemeineren Formulierung des Lemmas \ref{th:lemma_apply_rounding_of_lemma_steinitz_finite_dimensional}.

\begin{lem}\label{th:using_rounding_lemma_suff_cond}
	Sei $ (x_n) $ summierbar in einem Banachraum $ X $ mit dem Grenzwert $ s $ und der Eigenschaft \ref{eq:prop_linear_conv_area_2_item}.
	Dann existiert für alle $ s^\prime  \in s + \Gamma_0 $ eine Permutation $ \pi_0 \in \mathcal{S}_\N$  und eine streng monotone Indexfolge $ (n_j) $, sodass
	\begin{align*}
		\lim \limits_{j \to \infty}
		\left\| s^\prime - \sum \limits_{i = 1}^{n_j} x_{\pi_0(i)} 
		\right\| = 0
	\end{align*}
	gilt.
\end{lem}

\begin{proof}
	Sei $ (\varepsilon_n) $ ein streng monoton fallende reelle Nullfolge.
	Damit exisitert nach Eigenschaft \ref{eq:prop_linear_conv_area_2_item} zu $ \varepsilon_1 $ ein $ K_1 \in \N $, wofür die Ungleichung \eqref{eq:prop_linear_conv_area_2} erfüllt ist.
	Mit dem Lemma \ref{thm:affine_space_is_subset} erhalten wir die Folgerung:
	\begin{align*}
		s \in \overline{Q}(\{x_i\}_{i=K_1}^\infty)  + \sum_{i=1}^{K_1 - 1} x_i
		\ \Rightarrow \
		s + \Gamma_0 \subseteq \overline{Q}(\{x_i\}_{i=K_1}^\infty)  + \sum_{i=1}^{K_1 - 1} x_i
		\ \Rightarrow \
		s^\prime - \sum_{i=1}^{K_1 - 1} x_i \in \overline{Q}(\{x_i\}_{i=K_1}^\infty).
	\end{align*}
	Damit exisitiert ein $ q_1 \in \overline{Q}(\{x_i\}_{i=K_1}^\infty) $, sodass
	\begin{align*}
		\left\| s^\prime - \sum \limits_{i= 1}^{K_1 - 1} x_i - q_1
		\right\|
		=
		\left\| s^\prime - \sum \limits_{i = 1}^{K_1 - 1} x_i - \sum \limits_{i=K_1}^{K_1 + k_1} \lambda_{i} x_i
		\right\|
		\leq \frac{\varepsilon_1}{2}
	\end{align*}
	gilt. Insbesondere liefert die Eigenschaft \ref{eq:prop_linear_conv_area_2_item} die Rundungskoeffizienten $ \Theta_i $ mit $ K_1 \leq i \leq K_1 + k_1$, sodass
	\begin{align*}
		\left\| 
		\underbrace{\sum \limits_{i = K_1}^{K_1+ k_1} \lambda_i x_{i}}_{q_1 := }
		- 
		\underbrace{\sum \limits_{i = K_1}^{K_1+ k_1}\Theta_i x_i }_{p_1 :=}
		\right\| 
		\leq \frac{\varepsilon_1}{2}
	\end{align*}
	erfüllt ist.
	Wir definieren
	\begin{align*}
		S_1 := \underbrace{\{x_i\}_{i =1}^{K_1 - 1} \cup \{x_i \ : \  \Theta_i^{(1)} = 1, K_1 \leq i \leq K_1 + k_1\}}_{A_1 :=} \cup \{x_1\}, \quad
		s_1 := \sum \limits_{x \in S_1} x
	\end{align*}
	und erhalten 
	\begin{align*}
		\|
		s^\prime - s_1
		\|
		&\leq
		\left\|
		s^\prime  - \sum \limits_{i= 1}^{K_1 - 1} x_i - p_1
		\right\| + \|x_1\|
		\leq
			\left\|
		s^\prime  - \sum \limits_{i= 1}^{K_1 - 1} x_i - q_1
		\right\|
		+
		\|q_1 - p_1\| 
		+ 
		\|x_1\|\\
		&\leq 
		\varepsilon_1 + \|x_1\|
	\end{align*}
	mithilfe der Dreiecksungleichung. Durch unser Vorgehen ersparen wir uns insbesondere eine Fallunterscheidung für $ x_1 $, da die $ \varepsilon_1 $-Abschätzungen bereits für $ A_1 $ gelten.
	Wie im Beweis zu dem Lemma \ref{th:lemma_apply_rounding_of_lemma_steinitz_finite_dimensional} werden wir mit $ \varepsilon_2 $ fortfahren.
	Hierfür existiert ein $ K_2 \in \N $, wofür die Ungleichung \eqref{eq:prop_linear_conv_area_2} erfüllt ist.
	Wir setzen $ M_2 := \{x_i\}_{i=1}^{K_2 - 1} \cup S_1$ und erhalten mit Lemma \ref{thm:affine_space_is_subset}:
	\begin{align*}
		s \in \overline{Q}(\{x_i\}_{i=K_2}^\infty \setminus S_1)  + \sum \limits_{ x  \in M_2} x
		\ &\Rightarrow \
		s + \Gamma_0 \subseteq \overline{Q}(\{x_i\}_{i=K_2}^\infty \setminus S_1)  + \sum \limits_{ x \in M_2} x\\
		\ &\Rightarrow \
		s^\prime - \sum\limits_{ x \in M_2} x \in \overline{Q}(\{x_i\}_{i=K_2}^\infty \setminus S_1).
	\end{align*}
	Eventuelle Probleme durch das Herausschneiden von $ S_1 $ können wir ohne weiteres durch Umindizierung lösen.
	Damit existiert ein $ q_2 \in  \overline{Q}(\{x_i\}_{i=K_2}^\infty \setminus S_1)$, sodass
	\begin{align*}
		\left\| s^\prime - 
		\sum\limits_{ x \in M_2} x - q_2
		\right\| 
		=
		\left\| s^\prime - 
		\sum\limits_{ x \in M_2} x - \sum \limits_{i = K_2}^{K_2+ k_2} \lambda_i x_{i}
		\right\| 
		\leq \frac{\varepsilon_2}{2}
	\end{align*}
	gilt. Die Eigenschaft \ref{eq:prop_linear_conv_area_2_item} liefert wieder die Rundungskoeffizienten $ \Theta_i $ mit $ K_2 \leq i \leq  K_2 + k_2 $, sodass 
	\begin{align*}
		\left\| 
		\underbrace{\sum \limits_{i = K_2}^{K_2+ k_2} \lambda_i x_{i}}_{q_2 := }
		- 
		\underbrace{\sum \limits_{i = K_2}^{K_2+ k_2}\Theta_i x_i }_{p_2 :=}
		\right\| 
		\leq \frac{\varepsilon_2}{2}
	\end{align*}
	gilt. Wir definieren
	\begin{align*}
		S_2 := \{x_i\}_{i =1}^{K_2 - 1} \cup \{x_i \ : \  \Theta_i^{(1)} = 1, K_2 \leq i \leq K_2 + k_2\} \cup \{x_2\}, \quad
		s_2 := \sum \limits_{x \in S_2} x
	\end{align*}
	und erhalten analog
	\begin{align*}
		\| s^\prime - s_2 \| \leq \varepsilon_2 + \|x_2 \|.
	\end{align*}
	Diese Prinzip lässt sich analog für alle $ j \in \N $ fortführen.
	Damit gilt
	\begin{align*}
		\|s^\prime - s_j \| \leq \varepsilon_j +  \|x_j\| \rightarrow 0
	\end{align*}
	für $ j \to \infty $. Der Rest des Beweises ist identisch zu dem Beweis des Lemmas \ref{th:lemma_apply_rounding_of_lemma_steinitz_finite_dimensional}.
%	Es gilt $ s \in \overline{Q}(\{x_i\}_{i=K_1}^\infty)  + \sum_{i=1}^{K_1 - 1} x_i$.
%	 
%	Mit dem Lemma \ref{thm:affine_space_is_subset} folgt
%	$ s + \Gamma_0 \subset \overline{Q}(\{x_i\}_{i=K_1}^\infty)  + \sum_{i=1}^{K_1 - 1} x_i $ und damit auch $ s^\prime - \sum_{i=1}^{K_1 - 1} x_i \in \overline{Q}(\{x_i\}_{i=K_1}^\infty)  $
	
%	Zu $ \varepsilon_1 $ existiert ein $ K_1 \in \N $, sodass für alle endlichen Teilmengen $ \{y_i\}_{i = 1}^k \subset \{x_i\}_{i = K_1}^\infty $ und $ \lambda \in [0,1]^k $
%	ein $ \Theta \in \{0,1\}^k $ mit
%	\begin{align*}
%		\left\| \sum \limits_{i = 1}^k \lambda_i y_{i}- 
%		\sum \limits_{i = 1}^k
%		\Theta_i y_i \right\| \leq \frac{\varepsilon_1}{2}
%	\end{align*}
%	exisitert.
\end{proof}

\begin{proof}[Beweis von Satz \ref{th:prop_linear_conv_area}]
	Die Inklusion $ \mathcal{K}_{(x_n)} \subseteq s + \Gamma_0 $ folgt mit \ref{th:subset_conv_area}.
	Nun widmen wir uns der Beziehung $ s + \Gamma_0 \subseteq \mathcal{K}_{(x_n)}$.
	Sei $ s^\prime \in s + \Gamma_0 $ beliebig.
	Aus der Eigenschaft \ref{eq:prop_linear_conv_area_2_item} erhalten wir mit dem Lemma \ref{th:using_rounding_lemma_suff_cond} die Existenz einer Permutation $ \pi_0 $ und einer streng monotonen Indexfolge $ (n_j) $, sodass 
	\begin{align*}
		\lim \limits_{j \to \infty}
		\left\| s^\prime - \sum \limits_{k = 1}^{n_j} x_{\pi_0(k)} 
		\right\|
		=
		\left\| s^\prime - \sum \limits_{i = 1}^{n_j} x_{i} 
		\right\|
		= 0
	\end{align*}
	gilt.
	Um die Darstellung zu vereinfachen, setzen wir $ \pi_0(k) = i $. 
%	 Wir setzen hierbei $ \pi_0(k) = i $ um die Darstellung zu vereinfachen.
	Wir konstruieren eine korrigierende Permutation $ \pi \in \mathcal{S}_\N $, sodass $ (x_{\pi(n)}) $ gegen $ s^\prime $ konvergiert.
%	Unser Ziel ist nun eine korrigierende Permutation $ \pi  $ von $ \N $ zu konstruieren um die Konvergenz gegen $ s^\prime $ zu erreichen. 
\newpage
	Mit dem Cauchykriterium \ref{th:chauchy_crit} gilt:
	\begin{align*}
		\lim \limits_{j \to \infty}
		\left\|  \sum \limits_{i = n_j + 1}^{n_{j+1}} x_{i} 
		\right\|
		=
		0.
	\end{align*}
	Hierauf wenden wir die Eigenschaft \ref{eq:prop_linear_conv_area_1_item} an.
	Sei $ \varepsilon > 0 $ beliebig.
	Nach der Eigenschaft \ref{eq:prop_linear_conv_area_1_item} existiert ein $ N_\varepsilon \in \N $ und ein $ \delta_\varepsilon > 0 $, sodass für alle endlichen Teilmengen $ \{y_i\}_{i=1}^n \subset \{x_i \}_{ i = N_\varepsilon}^\infty $ mit $ \left\|
	\sum_{i=1}^n y_i\right\| \leq \delta_\varepsilon$
	eine Permutation $ \pi $ von $ \{1,...,n\} $ mit
	\begin{align*}
		\max \limits_{r \leq n} 
		\left\|
		\sum \limits_{i = 1}^r y_{\pi(i)}
		\right\|
		\leq 
		\varepsilon
	\end{align*}
	existiert. Damit erhalten wir ein $ J_\varepsilon \in \N $ mit
	\begin{align*}
		\left\|  \sum \limits_{i = n_j + 1}^{n_{j+1}} x_{i} 
		\right\| \leq \delta_\varepsilon
	\end{align*}
	für $ j > J_\varepsilon $.
	Für $ n_j > \max \{N_\varepsilon, n_{J_\varepsilon}\} $ existiert eine Permutation $ \pi_j $ von $ \{n_j+1,...,n_{j+1}\} $ mit
	\begin{align*}
		\max \limits_{n_j + 1 \leq r \leq n_{j+1}} 
		\left\|
		\sum \limits_{i = n_j + 1}^r y_{\pi_j(i)}
		\right\|
		\leq 
		\varepsilon.
	\end{align*} 
	Durch eine geeignete Wahl von $ \varepsilon $ erhalten wir die Permutationen $ \pi_j $ für alle $ j \in \N $.
	Wir definieren die globale Permutation $ \pi $ auf $ \N $ durch $ \pi\big|_{\{ x_i \}_{i = n_j + 1}^{n_{j+1}} } = \pi_j $ für $ j \in \N $ und $ \pi(i) = i $ für $ i \leq n_1 $.
	Dies führt insbesondere dazu, dass
	\begin{align*}
		\sum \limits_{i = 1}^{n_j} x_{\pi(i)}
		=
		\sum \limits_{i = 1}^{n_j} x_i
	\end{align*}
	gilt.
	Um den Beweis abzuschließen betrachten wir wieder $ \varepsilon > 0 $ mit $ r > n_j \geq \max \{N_\varepsilon, n_{J_\varepsilon}\}$. Wir wählen $ r > n_1  $ und $ j $ so, dass $ n_j + 1 \leq r \leq n_{j+1}  $ gilt.
	Dann erhalten wir
	\begin{align*}
		\left\|
		\sum \limits_{i=1}^r x_{\pi(i)} - s^\prime 
		\right\|
		&\leq
		\underbrace{\left\|
			\sum \limits_{i=1}^{n_j} x_{\pi(i)} - s^\prime 
			\right\| }_{
			= \| \sum x_i - s^\prime \|
		}
		+
		\left\|
		\sum \limits_{i = n_j +1}^{r}
		x_{\pi_j(i)}
		\right\| \leq 2 \varepsilon.
	\end{align*}
	Damit gilt $ s^\prime \in \mathcal{K}_{(x_n)} $ und damit auch $ \mathcal{K}_{(x_n)}  = s + \Gamma_0$.
\end{proof}

\section{Unendlichdimensionale Umordnungs-und Rundungssätze}
%\subsection{Unendlichdimensionale Umordnungs-und Rundungssätze}

\begin{genericthm}{Rundungslemma II}\label{th:lemma_for_pecherskii_2}
	Sei $ X $ ein normierter Raum und $ \varepsilon > 0 $ beliebig.
	Desweiteren sei $ \{x_i\}_{i =1}^N \subset X $ eine endliche Teilmenge mit der Eigenschaft:
	Für alle $ \{y_i\}_{i = 1}^m \subset \{x_i\}_{i = 1}^N $ existieren Vorzeichen, sodass 
	\begin{align*}
		\left\|
		\sum \limits_{i = 1}^m \alpha_i y_i
		\right\|
		\leq \varepsilon
	\end{align*}
	gilt.
%	, $ \varepsilon > 0 $ beliebig und $ \{x_i\}_{i =1}^N \subset X $ mit 
%	\begin{align*}
%		\left\|
%		\sum \limits_{i = 1}^m \alpha_i y_i
%		\right\|
%		\leq \varepsilon
%	\end{align*}
%	für alle $ \{y_i\}_{i = 1}^m \subset \{x_i\}_{i = 1}^N $ und $ \alpha_i = \pm 1 $.
	%	Außerdem sei folgende Eigenschaft erfüllt:
	%	Für alle $ \{y_i\}_{i = 1}^m \subset \{x_i\}_{i = 1}^N $ existieren Vorzeichen $ \alpha_i = \pm 1 $, sodass
	%	\begin{align*}
	%		\left\|
	%		\sum \limits_{i = 1}^m \alpha_i y_i
	%		\right\|
	%		\leq \varepsilon
	%	\end{align*}
	%	gilt.
	Dann existieren für alle Koeffizienten $ \{\lambda_i\}_{i = 1}^N \subset [0,1] $
	Rundungskoeffizienten $ \{\Theta_i\}_{i = 1}^N \subset \{0,1\} $, sodass 
	\begin{align}
		\left\|
		\sum \limits_{i = 1}^N
		\lambda_i x_i
		-
		\sum \limits_{i = 1}^N
		\Theta_i x_i
		\right\|
		\leq \varepsilon
	\end{align}
	gilt.
\end{genericthm}

\begin{proof}
	Wir beweisen die Aussage für dyadische Darstellungen zwischen $ 0 $ und $ 1 $.
	Da diese Darstellungen dicht in dem Intervall $ [0,1] $ liegen, folgt die Aussage dann aus einem Stetigkeitsargument.
	Da wir nur dyadische Darstellungen der Koeffizienten betrachten, lassen sich diese als endliche Summen der Form 
	\begin{align}\label{eq:dyadic_rep}
		\lambda_i =
		\sum \limits_{k = 1}^n
		\lambda_{i,k-1}
		\cdot \frac{1}{2^{k-1}}
		=
		\lambda_{i,0}
		+
		\lambda_{i,1} \cdot \frac{1}{2}
		+
		...
		+
		\lambda_{i,n-1} \cdot \frac{1}{2^{n-1}}
	\end{align}
	schreiben. Hierbei sind die $ \lambda_{i,k} $ entweder $ 0 $ oder $ 1 $ und die Anzahl der Summanden nennen wir Länge der dyadischen Darstellung.
	Wegen $ \lambda_{i} \in [0,1] $ folgt aus $ \lambda_{i,0} = 1 $ unmittelbar, dass alle weiteren $ \lambda_{i,k} $ gleich $ 0 $ sind.
	Wir nehmen zusätzlich an, dass alle $ \lambda_{i} $ dieselbe Länge besitzen. 
	Alternativ erreichen wir dies durch die Addition der Null.
	Da diese dyadischen Darstellungen dicht in dem Intervall $ [0,1] $ liegen, folgt die Aussage aus einem Stetigkeitsargument.\\
	Wir zeigen induktiv, dass für die dyadischen Darstellungen $ \lambda_{i} $ der Länge $ n $ Rundungskoeffizienten $ \Theta_i \in \{0,1\} $ existieren, sodass
	\begin{align}\label{eq:induction_inequaltiy}
		\left\|
		\sum \limits_{i = 1}^N
		\lambda_i x_i
		-
		\sum \limits_{i = 1}^N
		\Theta_i x_i
		\right\|
		\leq \varepsilon 
		\left( 1 - \frac{1}{2^{n-1}} \right)
	\end{align}
	gilt. 
	Für $ n = 1 $ erhalten wir für die dyadischen Darstellungen entweder $ \lambda_i = 0 $ oder $ \lambda_{i} = 1 $. Mit $ \Theta_i := \lambda_{i}  $ folgt dann der Induktionsanfang.
	Wir nehmen nun an, dass die Aussage für dyadische Darstellungen der Länge $ n $ gilt.
	Seien $ \mu_i = \mu_{i,0}+ ... + \mu_{i,n} 2^{-n} $, $ 1 \leq i \leq N $ dyadische Darstellungen der Länge $ n +1 $.
	Für die Indexmenge $ A := \{ i \ | \ 1 \leq  i \leq N \ \wedge \ \mu_{i,n} = 1 \} $ existieren $ \{ \alpha_i \}_{i \in A} $ mit $ \alpha_i = \pm 1 $, sodass
	\begin{align*}
		\left\| \sum \limits_{i \in A} \alpha_i x_i \right\| \leq \varepsilon \cdot \frac{1}{2^n}
	\end{align*}
	erfüllt ist. Wir definieren 
	\begin{align*}
		\lambda_{i} 
		:= 
		\begin{cases}
			\quad \mu_i, \ &\textrm{falls } i \notin A\\
			\mu_i + \alpha_i \cdot  2^{-n}, \ &\textrm{falls } i \in A
		\end{cases}
	\end{align*}
	und erhalten damit
	\begin{align*}
		\left\|
		\sum \limits_{i = 1 }^N
		\lambda_{i} x_i 
		- \sum \limits_{i =1 }^N
		\mu_i x_i
		\right\| 
		\leq 
		\frac{1}{2^n}
		\left\| \sum \limits_{i \in A} \alpha_i x_i \right\|
		\leq
		\varepsilon \cdot \frac{1}{2^n}.
	\end{align*}
	Die dyadischen Darstellungen $ \lambda_i $ besitzen alle die Länge $ n $.
	Damit existieren zu diesen $ \lambda_{i} $ nach der Induktionsvoraussetzung
	$ \Theta_i \in  \{0, 1\}  $, sodass \eqref{eq:induction_inequaltiy} gilt.
	Insgesamt erhalten wir mit
	\begin{align*}
		\left\|
		\sum \limits_{i = 1}^N
		\mu_i x_i
		-
		\sum \limits_{i = 1}^N
		\theta_i x_i
		\right\|
		&=
		\left\|
		\sum \limits_{i = 1}^N
		\lambda_i x_i
		-
		\sum \limits_{i = 1}^N
		\theta_i x_i
		+
		\sum \limits_{i = 1}^N
		\mu_i x_i
		-
		\sum \limits_{i = 1}^N
		\lambda_i x_i
		\right\|\\
		&\leq
		\left\|
		\sum \limits_{i = 1}^N
		\lambda_i x_i
		-
		\sum \limits_{i = 1}^N
		\theta_i x_i
		\right\|
		+
		\left\|
		\sum \limits_{i = 1}^N
		\mu_i x_i
		-
		\sum \limits_{i = 1}^N
		\lambda_i x_i
		\right\|\\
		&\leq 
		\varepsilon
		\left(1 - \frac{1}{2^{n-1}}\right)
		+ \varepsilon \cdot  \frac{1}{2^n }
		=
		\varepsilon \left(
		1 - \frac{1}{2^n}
		\right)
	\end{align*}
	den Induktionsschritt.
	Damit haben wir die Aussage gezeigt.
	
\end{proof}


\begin{genericthm}{Das Lemma von Chobanyan}\label{th:lemma_chobanyan}
	Sei $ \{x_i\}_{i = 1}^n \subset X $ mit $ \sum_{i  = 1 }^n  x_i = 0$.
	Dann existiert eine Permutation $ \sigma $, sodass für beliebige Vorzeichen $ \alpha_i = \pm 1 $
	\begin{align}\label{eq:lem_choban_ineq}
		\max \limits_{j \leq n}
		\left\|
		\sum \limits_{i = 1}^j \alpha_i x_{\sigma(i)}
		\right\|
		\geq 
		\max \limits_{j \leq n}
		\left\|
		\sum \limits_{i = 1}^j  x_{\sigma(i)}
		\right\|
	\end{align}
	erfüllt ist.
\end{genericthm}

%\newpage
\begin{proof}
	Sei $ \sigma $ die Permutation, für welche
	\begin{align*}
		\min \limits_{\nu }
		\left\{
		\max_{j \leq  n}
		\left\|
		\sum \limits_{i = 1}^j
		x_{\nu(i)}
		\right\|
		\right\}
	\end{align*}
	angenommen wird. Wir fixieren für $ i = 1,...,n $  eine beliebige Vorzeichenverteilung $ \alpha_i  $ und definieren die Mengen:
	\begin{align*}
		A 
		:=
		\{
		i \ | \ \ \alpha_i = 1
		\}
		, \ &A_j := A \cap \{1,...,j\} , A_j^C = \{1,...,n\} \setminus A_j\\
		B 
		:=
		\{
		i \ | \ \alpha_i = -1
		\}
		, \ &B_j := B \cap \{1,...,j\} , B_j^C = \{1,...,n\} \setminus B_j.
	\end{align*}
	Mit $ \sum_{i  = 1 }^n  x_{\sigma(i)}  = \sum_{i \in B_j} x_{\sigma(i)} +  \sum_{i \in B_j^C} x_{\sigma(i)} = 0$ und $ 2 \max\{a,b\} = a + b + | a -b| $ folgt
	\begin{equation}\label{eq:proof_lem_choban_1}
		\begin{split}
			\max \limits_{j \leq n }
			\left\|
			\sum \limits_{i = 1}^j
			\alpha_i x_{\sigma(i)}
			\right\|
			+\max \limits_{j \leq n }
			\left\|
			\sum \limits_{i = 1}^j
			x_{\sigma(i)}
			\right\|
			&\geq
			2 
			\max 
			\left\{
			\max \limits_{j \leq n }
			\left\|
			\sum \limits_{i \in A_j}
			x_{\sigma(i)}
			\right\|,
			\max \limits_{j \leq n }
			\left\|
			\sum \limits_{i \in B_j}
			x_{\sigma(i)}
			\right\|
			\right\}\\
			&=
			2 
			\max 
			\left\{
			\max \limits_{j \leq n }
			\left\|
			\sum \limits_{i \in A_j}
			x_{\sigma(i)}
			\right\|,
			\max \limits_{j \leq n }
			\left\|
			\sum \limits_{i \in B_j^C}
			x_{\sigma(i)}
			\right\|
			\right\}.
		\end{split}
	\end{equation}
	Nun betrachten wir die Teilmengenbeziehung
	\begin{align*}
		A_1 \subseteq A_2 \subseteq ... \subseteq A_n \subseteq B_n^C 
		\subseteq B_{n-1}^C \subseteq ... \subseteq B_1^C,
	\end{align*}
	welche genau $ n $ Gleichheiten besitzt. Dies liefert die Existenz einer Permutation $ \nu $ mit
	\begin{align}\label{eq:proof_lem_choban_2}
		\max_{j \leq  n}
		\left\|
		\sum \limits_{i = 1}^j
		x_{\nu(i)}
		\right\|
		=
		2 
		\max 
		\left\{
		\max \limits_{j \leq n }
		\left\|
		\sum \limits_{i \in A_j}
		x_{\sigma(i)}
		\right\|,
		\max \limits_{j \leq n }
		\left\|
		\sum \limits_{i \in B_j^C}
		x_{\sigma(i)}
		\right\|
		\right\},
	\end{align}
	für welche aufgrund der Wahl von $ \sigma $
	\begin{align}\label{eq:proof_lem_choban_3}
		\max_{j \leq  n}
		\left\|
		\sum \limits_{i = 1}^j
		x_{\nu(i)}
		\right\|
		\geq 
		\max_{j \leq  n}
		\left\|
		\sum \limits_{i = 1}^j
		x_{\sigma(i)}
		\right\|
	\end{align}
	gilt. Wenn wir die Aussagen von \eqref{eq:proof_lem_choban_1}, \eqref{eq:proof_lem_choban_2} und \eqref{eq:proof_lem_choban_3} kombinieren folgt
	\begin{align*}
		\max \limits_{j \leq n }
		\left\|
		\sum \limits_{i = 1}^j
		\alpha_i x_{\sigma(i)}
		\right\|
		+\max \limits_{j \leq n }
		\left\|
		\sum \limits_{i = 1}^j
		x_{\sigma(i)}
		\right\|
		&\geq 2 
		\max 
		\left\{
		\max \limits_{j \leq n }
		\left\|
		\sum \limits_{i \in A_j}
		x_{\sigma(i)}
		\right\|,
		\max \limits_{j \leq n }
		\left\|
		\sum \limits_{i \in B_j^C}
		x_{\sigma(i)}
		\right\|
		\right\}\\
		&=
		2 
		\max_{j \leq  n}
		\left\|
		\sum \limits_{i = 1}^j
		x_{\nu(i)}
		\right\|\\
		&\geq 
		2 
		\max_{j \leq  n}
		\left\|
		\sum \limits_{i = 1}^j
		x_{\sigma(i)}
		\right\|,
	\end{align*}
	womit die gewünschte Ungleichung \eqref{eq:lem_choban_ineq} folgt.
\end{proof}


\begin{genericthm}{Umordnungslemma II}\label{th:lemma_for_pecherskii_3}
	Sei $ \{x_i\}_{i = 1}^n \subset X $ mit $ \| \sum_{i=1}^n x_i \| \leq \varepsilon  $ und für alle Permutationen $ \nu  $ existieren Vorzeichen $ \alpha_i = \pm 1 $, wofür 
	\begin{align*}
		\max \limits_{j \leq n }
		\left\|
		\sum \limits_{i=1}^j \alpha_i x_{\nu(i)}
		\right\| 
		\leq \varepsilon
	\end{align*}
	gilt.
	Dann existiert eine Permutation $ \sigma $, sodass
	\begin{align}\label{eq:ineq_for_pecherskii_prop_A}
		\max \limits_{j \leq n}
		\left\|
		\sum \limits_{i = 1}^j x_{\sigma(i)}
		\right\|
		\leq 3 \varepsilon
	\end{align} 
	gilt.
\end{genericthm}

\begin{proof}
	Zuerst stellen wir die Voraussetzungen des Lemmas von Chobayan her.
	Wir setzen $ x_{n+1} := - \sum_{i = 1}^n x_i $, womit 
	$ \sum_{i = 1}^{n+1} x_i = 0 $ gilt.
	Damit existiert nach dem letzten Lemma eine Permutation $ \pi : \{1,...,n+1\} \to \{1,...,n+1\} $, sodass
	\begin{align*}
		\underbrace{\max 
			\limits_{j \leq n +1  }
			\left\|
			\sum \limits_{i = 1}^j x_{\pi(i)}
			\right\|}_{\textbf{\textrm{(I)}}}
		\leq 
		\underbrace{\min \limits_{\alpha_i = \pm 1}
			\max 
			\limits_{j \leq n +1  }
			\left\|
			\sum \limits_{i = 1}^j \alpha_i x_{\pi(i)}
			\right\|}_{\textbf{\textrm{(II)}}}
	\end{align*}
	gilt. Als nächstes entfernen wir $ x_{n+1} $ wieder. Aus diesem Schritt erhalten wir eine Permutation $ \sigma $ von $ \{1,...,n\} $, welche mit
	\begin{align*}
		\max \limits_{j \leq n} 
		\left\|
		\sum \limits_{i = 1}^j
		x_{\sigma(i)}
		\right\| 
		\leq 
		\min \limits_{\alpha_i = \pm 1}
		\max 
		\limits_{j \leq n   }
		\left\|
		\sum \limits_{i = 1}^j \alpha_i x_{\sigma(i)}
		\right\|
		+ 
		2 \| x_{n+1} \|
		\leq 3 \varepsilon 
	\end{align*} 
	die Ungleichung \eqref{eq:ineq_for_pecherskii_prop_A} liefert.
	Um dies zu untermauern schließen wir zunächst den Fall\\ $ \pi(n+1) = n+1 $ aus.
	Damit existiert ein $ l \in \{ 1,...,n\} $ mit $ \pi(l) = n+1 $. Die Permutation $ \sigma  $ erhalten wir durch das Überspringen von $ l $ und entsprechende Umindizierung.
	Die Abschätzung für \textbf{\textrm{(I)}} erhalten wir dann durch
	\begin{align*}
		\left\|
		\sum \limits_{i = 1}^j
		x_{\pi(i)}
		\right\| 
		\geq 
		\left\|
		\sum \limits_{i = 1, i \neq l}^j
		x_{\pi(i)}
		\right\|  -\| x_{n+1} \|
	\end{align*}
	für $ 1 \leq j \leq n+1 $. Die Abschätzung für \textbf{\textrm{(II)}} folgt analog.
	
	
	%Mit 
	%	\begin{align*}
	%		\sigma(i)
	%		:=
	%		\begin{cases}
	%			\pi(i), &\ \textrm{falls } \pi(i) \neq n+1\\
	%			\pi(n+1), &\ \textrm{falls } \pi(i) = n+1
	%		\end{cases}
	%	\end{align*}
	%	für $ 1 \leq i \leq n $ erhalten wir eine Permutation auf $ \{ 1,...,n\} $.
	%	 Nun wählen wir die Vorzeichen so, dass $ \max_{j \leq n }\left\|\sum_{i=1}^j \alpha_i x_{\sigma(i)}\right\| \leq \varepsilon $ erfüllt  ist.
\end{proof}
\newpage

\section{Die Sätze von Dvoretzky-Hanani und Pecherskii}\label{sc:hanani_pecherskii}
%\subsection{Die Sätze von Dvoretzky-Hanani und Pecherskii}
In diesem Abschnitt beschäftigen wir uns mit den Sätzen von Dvoretzky-Hanani\cite{Dvoretzky1947} und Pecherskii\cite{Pecherski1989}.
Gemeinsam führen beide Aussagen zu einer echten Erweiterung des Satzes von Steinitz auf unendlichdimensionale Räume.

%In \cite{Pecherski1989} hat Pecherskii gezeigt, dass eine bedingt summierbare Folge unter bestimmten Vorausetzungen die Steinizeigenschaft erfüllt.
%Durch die Aussage von Dvoretzky und Hanani wird sich diese zusätzliche im endlichdimensionalen auflösen, sodass wir von einer echten Erweiterung des Steiniz-Satzes sprechen können. 

\begin{df}
	Sei $ X $ ein Banachraum.
	Wir nennen eine Folge $ (x_n) $ \textit{perfekt unsummierbar}, falls
	$ (\alpha_n x_n) $ für alle $ \alpha_n = \pm 1 $ unsummierbar ist.
\end{df}

\begin{genericthm}{Satz von Dvoretzky-Hanani(1947)}\label{th:dvoretzkey_hanani}
	Sei $ X $ ein endlichdimensionaler normierter Raum und $ (x_n) $ perfekt unsummierbar in $ X $.
	Dann konvergiert $ (x_n) $ nicht gegen $ 0 $.
\end{genericthm}
\begin{proof}
	Wir beweisen diese Aussage über Kontraposition.\\
	Sei $ (x_n) $ eine Nullfolge.
	Unser Ziel ist es eine $ \pm 1 $ Folge  $ (\alpha_n) $ zu konstruieren, sodass
	$ \sum \alpha_i x_i $ konvergiert.
	Wir setzen
	\begin{align*}
		d_n := \max \limits_{i > n}  \| x_i \|,
	\end{align*}
	womit $ d_n \to 0  $ für $ n \to \infty $ folgt.
	Insbesondere existiert eine streng monoton wachsende Indexfolge $ (n_k) $, sodass
	\begin{align*}
		\sum \limits_{k = 1 }^\infty d_{n_k } < \infty
	\end{align*}
	gilt. 
	Nach dem Lemma \ref{th:sign_inequality_finite_dim} existieren für $ \{ x_i \}_{i = n_k +1 }^{n_{k+1}} $ Koeffizienten $ \alpha_i = \pm 1 $ für $ n_k +1 \leq i \leq n_{k+1} $, sodass
	\begin{align*}
		\max \limits_{n_k + 1 \leq j \leq n_{k+1} }
		\left\|
		\sum 
		\limits_{i = n_k +1 }^j
		\varepsilon_i  x_i 
		\right\|
		\leq 2m d_{n_k}
	\end{align*}
	erfüllt ist.
	Mit dem Majorantenkriterium erhalten wir dann
	\begin{align*}
		\left\|
		\sum \limits_{i = 1 }^\infty \alpha_i x_i 
		\right\| 
		=
		\left\|
		\sum \limits_{k = 1 }^\infty \sum \limits_{i = n_k +1 }^{n_k} \alpha_i x_i 
		\right\| 
		\leq
		\sum \limits_{k = 1 }^\infty \left\| \sum \limits_{i = n_k +1 }^{n_k} \alpha_i x_i \right\|
		\leq
		2m \sum \limits_{k = 1 }^\infty d_{n_k } < \infty.
	\end{align*}
	Damit ist $ (x_n) $ nicht perfekt unsummierbar.
\end{proof}

\begin{genericthm}{Satz von Pecherskii (1988)}\label{th:lemma_of_pecherskii}
	Sei $ X $ ein Banachraum, $ (x_n) $ bedingt summierbar in $ X $ mit Grenzwert $ s $ und es existiere keine Umordnung $ \pi  $, sodass $ (x_{\pi(n)}) $ perfekt unsummierbar ist.
	Dann erfüllt $ (x_n) $ die Steinitzeigenschaft.
\end{genericthm}
Die Nullfolgeneigenschaft bleibt für beliebige Permutationen erhalten.
Damit können wir nach dem Satz von Dvoretzky-Hanani einer echten Erweiterung des Satzes von Steinitz sprechen. 
%Für einen endlichdimensionalen normierten Raum, folgt aus dem Satz von Dvoretky-Hanani, dass wir von einer echten Erweiterung des Satzes von Steiniz sprechen können.
Die Voraussetzung, dass keine Permutation zu einer perfekt unsummierbaren Folge umsortiert, ist im Endlichdimensionalen immer erfüllt. 
%Falls $ X $ ein endlichdimensionaler normierter Raum ist, entspricht der Satz von Pecherskii dem Satz von Steinitz. 
%Wir können hier also von einer echten Erweiterung sprechen.
%Diese Tatsache folgt aus dem Satz von Dvoretzky-Hanani \ref{th:dvoretzkey_hanani}.
%Die bedingte Summierbarkeit von $ (x_n) $ liefert die Nullfolgeneigenschaft.
%Diese bleibt für jede Permutation erhalten.
%Damit ist $ (x_n) $ für jede Umordnung nicht perfekt unsummierbar.

Da das Rundungslemma II \ref{th:lemma_for_pecherskii_2} und das Umordnungslemma II \ref{th:lemma_for_pecherskii_3} den Satz \ref{th:prop_linear_conv_area} implizieren, genügt es deren Voraussetzungen unter den Pecherskii-Voraussetzungen zu zeigen.
Dann folgt der Satz von Pecherskii.

%Die Aussage \ref{th:prop_linear_conv_area} liefert zwei Bedingungen. Der Satz von Pecherskii ist bewiesen, wenn diese unter dessen Voraussetzungen erfüllt sind.
%Wir werden nun zeigen, dass die Voraussetzungen der Sätze \ref{th:lemma_for_pecherskii_2} und  \ref{th:lemma_for_pecherskii_3} unter den Pecherskii-Voraussetzungen gelten.
%Damit ist der Beweis des Satzes von Pecherskii abgeschlossen.

\begin{lem}\label{th:lemma_for_pecherskii_1}
	Sei $ (x_n) $ eine Folge, welche nicht zu einer perfekt unsummierbaren Folge umsortierbar ist.
	Dann existiert für alle $ \varepsilon >0 $ ein $ N_\varepsilon \in \N $,
	sodass wir für alle $ \{y_i\}_{i = 1}^n \subset \{x_i\}_{i = N_\varepsilon}^\infty $ Vorzeichen $ \alpha_i = \pm 1 $ mit
	\begin{align}\label{eq:prop_for_steiniz}
		\max \limits_{j \leq n}
		\left\|
		\sum \limits_{i = 1}^j \alpha_i y_i
		\right\|
		\leq \varepsilon
	\end{align}
	finden.
\end{lem}

\begin{proof}
	Wir führen einen Widerspruchsbeweis und nehmen an, dass die Aussage des Lemmas nicht gilt. Dann existiert ein $ \varepsilon > 0 $, sodass für alle $ N \in \N $ eine endliche Teilmenge $ \{y_i \}_{i = 1}^n \subset \{x_i\}_{i=N}^\infty $ existiert, wofür 
	\begin{align*}
		\max \limits_{j \leq n } 
		\left\|
		\sum \limits_{i = 1}^j \alpha_i y_i
		\right\|
		> \varepsilon
	\end{align*}
	für beliebige Vorzeichen $ \alpha_i = \pm 1 $ gilt.
	Damit finden wir eine streng monotone Indexfolge $ (N_k) $ mit $ N_1 = 1 $, sodass für eine Teilmenge  $ \{y_i^k\}_{i = 1}^{n_k} \subset \{x_i\}_{N_k}^{N_{k+1} - 1} $ die Abschätzung 
	\begin{align}\label{eq:pecherskii_lemma_1_proof}
		\min \limits_{\alpha_i = \pm 1}
		\max \limits_{j \leq n_k}
		\left\|
		\sum \limits_{i = 1}^j \alpha_i y_i
		\right\|
		> \varepsilon
	\end{align}
	gilt.
	Damit erhalten wir eine disjunkte vollständige Aufteilung der Folgenglieder.
	Diese erfolgt nach dem Schema:
	\begin{align*}
		\{y^1_i\}_{i =1}^{n_1} , \quad \
		&\quad 
		\{x_i\}_{i = N_1}^{N_2 - 1} \setminus \{y^1_i\}_{i =1}^{n_1}\\
		\underbrace{\{y^2_i\}_{i =1}^{n_2},}_{\textrm{Ordnung von \eqref{eq:pecherskii_lemma_1_proof}}}
		&\quad 
		\underbrace{\{x_i\}_{i = N_2}^{N_3 - 1} \setminus \{y^2_i\}_{i =1}^{n_2}}_{\textrm{beliebige Ordnung}} , \ ...
	\end{align*}
	Wir übernehmen die Ordnung von $\{y^k_i\}_{i =1}^{n_k}  $, sodass die Ungleichung \eqref{eq:pecherskii_lemma_1_proof} erhalten bleibt und lassen den Rest beliebig.
	Die hieraus entstandene Permutation $ \pi  $ liefert uns eine unsummierbare Folge
	$ (\alpha_n x_n) $ für eine beliebige Vorzeichenverteilung $ (\alpha_n) $,
	da wir mit \eqref{eq:pecherskii_lemma_1_proof} ein Widerspruch zum Cauchy-Kriterium erhalten.
	Damit haben wir einen Widerspruch zu der Voraussetzung, dass keine Umordnung zu einer perfekt unsummierbaren Folge existiert. Also war die Annahme falsch und das Gewünschte gilt.
\end{proof}
Durch unsere Vorarbeit benötigt der Beweis des Satzes von Pecherskii nur den Nachweis der Eigenschaft \eqref{eq:prop_for_steiniz} unter dessen Voraussetzungen.
Diese impliziert dann mit \ref{th:prop_linear_conv_area} die Steinitzeigenschaft.
Insbesondere können wir die Voraussetzungen von \ref{th:prop_linear_conv_area} zusammenfassen.
%Damit ergibt sich der nachfolgenden Satz.
\begin{genericthm}{Satz}\label{th:sufficent_condition_steiniz_prop}
	Sei $ (x_n) $ eine Folge in einem Banachraum $ X $ und
	für alle $ \varepsilon >0 $ existiert ein $ N_\varepsilon \in \N $,
	sodass für alle $ \{y_i\}_{i = 1}^n \subset \{x_i\}_{i = N_\varepsilon}^\infty $ Vorzeichen $ \alpha_i = \pm 1 $ mit
	\begin{align*}
		\max \limits_{j \leq n}
		\left\|
		\sum \limits_{i = 1}^j \alpha_i y_i
		\right\|
		\leq \varepsilon
	\end{align*}
	existieren.
	Dann besitzt $ (x_n) $ die Steinitzeigenschaft.
\end{genericthm}
%Damit lässt sich mit \eqref{eq:prop_for_steiniz} ein affiner Konvergenzbereich auch unter anderen Voraussetzungen beweisen.
Der Beweis des Satzes von Steinitz \ref{th:lemma_of_steiniz} reduziert sich mit dem letzten Satz auf den Beweis des Lemmas \ref{th:sign_inequality_finite_dim}. 











\section{Der Satz von Chobanyan}\label{sc:lemma_of_chobanyan}
%\subsection{Der Satz von Chobanyan}
In diesem Abschnitt verwenden wir ausgiebig den Begriff der Rademacherfolge $ (r_n) $.
Die zugehörige Rademachersumme für eine Banachraumfolge $ (x_n) $ kennzeichnen wir durch
\begin{align*}
	R_k := 
	\left\|
	\sum \limits_{i = 1}^k r_i x_i \right\|.
\end{align*}
%für die Folge $ (x_n) $ in einem Banachraum $ X $.



\begin{lem}
	Sei $ (r_n) $ eine Rademacherfolge und $ (x_n) $ eine Folge in $ X $.
	Dann gilt
	\begin{align}\label{eq:proab_boundary}
		\mathrm{P}
		\left(
		\sup \limits_{k \leq n } R_k > t
		\right)
		\leq 2 \cdot
		 \mathrm{P}(R_n > t)
	\end{align}
	für alle $ t > 0 $.
\end{lem}

\begin{proof}
	Sei $ \Omega $ der zugehörige Wahrscheinlichkeitsraum zu den Rademachervariablen $ r_i $.
	Dann definieren wir $ A_i \subset \Omega  $ wie folgt:
	\begin{align*}
		A_1 := (R_1 > t), \ 
		A_2 := (R_1 \leq t) \cap (R_2 > t), \
	\dots, \
		A_n = \bigcap \limits_{i = 1}^{n-1}(R_i \leq t) \cap (R_n > t).
	\end{align*}
	Die hieraus entstandene Mengenfolge ist disjunkt und wir erhalten:
	\begin{align}\label{eq:proab_boundary_proof}
		\mathrm{P}
		\left( 
		\sup \limits_{k \leq n} R_k > t
		\right)
		=
		\mathrm{P}
		\left( \bigcup \limits_{i = 1}^n A_i \right)
		=
		\sum \limits_{i = 1}^n P(A_i).
	\end{align}
	Hierbei erhalten wir die erste Identität durch das Betrachten der Gegenwahrscheinlichkeit und elementare Mengenumformungen.
	Nun setzen wir 
	\begin{align*}
		R_{n,k} := 
		\left\|
		\sum \limits_{i = 1}^k r_i x_i
		-
		\sum \limits_{i = k+1}^n r_i x_i
		\right\|.
	\end{align*}
	Wegen $ \sum_{i = 1}^k r_i x_i = \sum_{i = 1}^n  r_i x_i -\sum_{i = k+1}^n r_i x_i  $ 
	folgt mit der Dreiecksungleichung\\ $ R_k \leq \frac{1}{2} ( R_n + R_{n,k}) $.
	Dies führt durch geeignete Fallunterscheidung bezüglich $ R_n $ und $ R_{n,k} $ zu
	\begin{align*}
		\mathrm{P}(A_k)
		&=
		\mathrm{P}(A_k \cap ((R_n + R_{n,k}) > 2t))
		\leq
		\mathrm{P}((A_k \cap (R_n > t)) \cup (A_k \cap (R_{n,k} > t)) )\\
		&\leq 
		\mathrm{P}(A_k \cap (R_n > t)) + \mathrm{P}(A_k \cap (R_{n,k} > t ))\\
		&= 
		2 	\mathrm{P}(A_k \cap (R_n > t)).
	\end{align*}
	Die letzte Identität folgt, da die Wahrscheinlichkeiten der Ereignisse $ A_k \cap (R_n > t) $ und $ A_k \cap (R_{n,k} > t ) $ übereinstimmen.
	Da die $ A_i $ disjunkt sind, folgt die Aussage mit der Gleichung \eqref{eq:proab_boundary_proof}:
%	Mit der Tatsache,  dass die $ A_i $ disjunkt sind und der Gleichung \eqref{eq:proab_boundary_proof} folgt durch
	\begin{align*}
		\mathrm{P}
		\left( 
		\sup \limits_{k \leq n} R_k > t
		\right)
		\leq
		2
		\sum \limits_{i = 1}^n P(A_i \cap (R_n > t))
		\leq 
		2 
		P(R_n > t).
	\end{align*}


%	\begin{align*}
%		2 &R_k
%		= 
%		\left\| \sum \limits_{i = 1}^k r_i x_i + \sum \limits_{i = 1}^k r_i x_i  \right\|
%		=
%		\left\|  \sum \limits_{i = 1}^n  r_i x_i -\sum \limits_{i = k+1}^n r_i x_i + \sum \limits_{i = 1}^k r_i x_i \right\|
%		\leq 
%		R_n + R_{n,k} \\
%		 \Leftrightarrow \ &R_k \leq \frac{1}{2} ( R_n + R_{n,k})
%	\end{align*}
%	folgt.
\end{proof}

\begin{lem}\label{th:expectation_estimate}
	Sei $ (r_n) $ eine Rademacherfolge und $ (x_n) $ eine Folge in $ X $.
	Dann gilt:
	\begin{align*}
		\mathbb{E}
		\left(
		\sup \limits_{k \leq n } R_k
		\right)
		\leq 2 \cdot
		\mathbb{E}(R_n).
	\end{align*}
	
\end{lem}

\begin{proof}
	Für eine positive Zufallsvariable $ \eta $ gilt
	\begin{align*}
		\mathbb{E}(\eta)
		=
		\int \limits_{0}^\infty \mathrm{P}(\eta > t) \dxS{t}
		- 
		\underbrace{\int \limits_{-\infty}^0 \mathrm{P}(\eta \leq  t) \dxS{t}}_{= 0}
		=
		\int \limits_{0}^\infty \mathrm{P}(\eta > t) \dxS{t}.
	\end{align*}
	Damit ergibt sich die Aussage unmittelbar durch Einsetzen der Ungleichung \eqref{eq:proab_boundary}.
\end{proof}
\newpage
\begin{genericthm}{Ungleichung von Chobanyan}
	Sei $ (r_n) $ eine Rademacherfolge, $ X $ ein normierter Raum und $ \{x_i\}_{i=1}^n \subset X $ mit $ \sum_{i = 1}^n  x_i = 0$.
	Dann existiert eine Permutation $ \sigma $ von $ \{1,...,n\} $, sodass 
	\begin{align*}
		\sup 
		\limits_{k \leq n}
		\left\|
		\sum \limits_{i = 1}^k
		x_{\sigma(i)}
		\right\|
		\leq 
		2 \mathbb{E}
		\left(
		\left\|
		\sum \limits_{i=1}^n
		r_i x_i
		\right\|
		\right)
	\end{align*}
	gilt.
\end{genericthm}

\begin{proof}
	Nach dem Lemma von Chobanyan \ref{th:lemma_chobanyan} existiert eine Permutation $ \sigma $, sodass
	\begin{align*}
		\max\limits_{k \leq n}
		\left\|
		\sum \limits_{i = 1}^k x_{\sigma(i)}
		\right\|
		\leq 
		\max \limits_{k \leq n}
		\left\|
		\sum \limits_{i=1}^k \alpha_i x_i
		\right\|
	\end{align*}
	für alle $ \alpha_i =\pm 1   $ gilt. Dies führt zu:
	\begin{align*}
		\max\limits_{k \leq n}
		\left\|
		\sum \limits_{i = 1}^k x_{\sigma(i)}
		\right\|
		\leq
		\frac{1}{2^n}
		\sum \limits_{\alpha_i  =\pm  1}  
		\max \limits_{k \leq n }
		\left\|
		\sum \limits_{i=1}^k \alpha_i x_{\sigma(i)}
		\right\|
		= 
		\mathbb{E}
		\left(
		\max \limits_{k \leq n}
		\left\|
		\sum \limits_{i=1}^k r_i x_{\sigma(i)}
		\right\|
		\right).
	\end{align*}
	Da das Lemma von Chobanyan für jede Kombination der $ \alpha_i $ gilt, ergibt sich der Erwartungswert durch Mittelwertbildung.
	%Hierbei ist wichtig, dass das Lemma von Chobanyan für jede Kombination der $ \alpha_i $ gilt, womit sich dann der Erwartungswert ergibt.
	Mit dem Lemma \ref{th:expectation_estimate} erhalten wir durch
	\begin{align*}
		\mathbb{E}
		\left(
		\max \limits_{k \leq n}
		\left\|
		\sum \limits_{i=1}^k r_i x_{\sigma(i)}
		\right\|
		\right)
		\leq 
		2 
		\mathbb{E}
		\left(
		\left\|
		\sum \limits_{i=1}^n r_i x_{\sigma(i)}
		\right\|
		\right)
		=
		2 
		\mathbb{E}
		\left(
		\left\|
		\sum \limits_{i=1}^n r_i x_{i}
		\right\|
		\right)
	\end{align*}
	die Aussage.
\end{proof}


Die ursprüngliche Idee war, den nachfolgenden Satz nach Kadets\cite{Kadets1997} mit der Ungleichung von Chobanyan zu beweisen.
Mit dieser werden dort die Voraussetzungen von Satz \ref{th:prop_linear_conv_area} gezeigt. Jedoch steckt in dieser Ausführung eine falsche Ungleichung, welche nicht korrigiert werden konnte.

%Der nachfolgende Satz sollte eigentlich mit der Ungleichung von Chobanyan bewiesen werden.
%In \cite{Kadets1997} werden damit die Voraussetzungen des Satzes \ref{th:prop_linear_conv_area} gezeigt.
%Jedoch steckt in dieser Ausführung eine Ungleichung, wofür es ein Gegenbeispiel gibt.
%Dieses Problem konnte nicht behoben werden.

\begin{sz}
	Sei $ (r_n) $ eine Rademacherfolge und  $ (x_n) $ eine Folge in einem Banachraum $ X $, welche die Eigenschaft
	\begin{align*}
		\lim \limits_{m > n \to \infty} \mathbb{E}
		\left(
		\left\|
		\sum \limits_{i=n}^m
		r_i x_i
		\right\|
		\right) = 0
	\end{align*}
	erfüllt. Dann gilt für $ (x_n) $ die Steinitzeigenschaft.
\end{sz}
\begin{proof}
Wir werden die Aussage beweisen, indem wir die Voraussetzungen der hinreichenden Bedingung \ref{th:sufficent_condition_steiniz_prop} nachweisen.	
Sei $ \varepsilon > 0  $ beliebig. Nach Voraussetzung existiert ein $ N_\varepsilon \in \N$, sodass
\begin{align*}
	\mathbb{E}
	\left(
	\left\|
	\sum 
	\limits_{i=n}^m r_i x_i
	\right\|
	\right)
	\leq 
	\frac{\varepsilon}{2}
\end{align*}
für $ m > n \geq N_\varepsilon $ gilt.
Wir wählen $ \{y_i\}_{i=1}^k \subset \{x_i\}_{i=N_\varepsilon}^\infty$ beliebig und bezeichnen
mit $ \tilde{r}_1,...,\tilde{r}_k $ die zugehörigen Rademachervariablen in der Reihe.
Außerdem finden wir ein $ l > N_\varepsilon $ mit $ \{\tilde{r}_i y_i\}_{i=1}^k \subset \{r_i x_i\}_{i=N_\varepsilon}^l$.
Mit dem Lemma \ref{th:expectation_estimate} erhalten wir dann
\begin{align*}
	\mathbb{E}
	\left(
	\max \limits_{j \leq k}
	\left\|
	\sum \limits_{i = 1}^j
	\tilde{r}_i y_i
	\right\|
	\right)
	\leq 
	\mathbb{E}
	\left(
	\max \limits_{N_\varepsilon \leq j \leq  l}
	\left\|
	\sum \limits_{i = N_\varepsilon}^j
	r_i y_i
	\right\|
	\right)
	\leq 
	2 
	\mathbb{E}
	\left(
	\left\|
	\sum \limits_{i = N_\varepsilon}^l
	r_i y_i
	\right\|
	\right)
	\leq 
	\varepsilon.
\end{align*}
Sei $ \Omega $ der zu den Rademachervariablen gehörende Wahrscheinlichkeitsraum.
Da der Erwartungswert kleiner als $ \varepsilon $ ist, existiert ein $ \omega \in \Omega^k $, sodass
\begin{align*}
	\max \limits_{j \leq k}
	\left\|
	\sum \limits_{i = 1}^j
	\tilde{r}_i(\omega_i) y_i
	\right\|
	\leq \varepsilon
\end{align*}
gilt. Mit $ \alpha_i = \tilde{r}_i(\omega_i) $ für $  1 \leq i \leq k$ haben wir eine geeignete Vorzeichenanordnung gefunden.
Insgesamt haben wir die Voraussetzung des Lemmas \ref{th:sufficent_condition_steiniz_prop} nachgewiesen, wodurch $ (x_n) $ die Steinitzeigenschaft besitzt.
\end{proof}

\begin{df}
Sei $ (r_n) $ eine Rademacherfolge und $ (x_n) $ eine Folge in einem Banachraum $ X $. 
Wir nennen $ (r_n x_n) $ fast überall summierbar, falls 
\begin{align*}
	\sum 
	\limits_{i = 1}^\infty r_i(\omega_i) x_i
\end{align*}
bis auf eine Nullmenge existiert.
Das bedeutet
\begin{align*}
	\rho
	\left(
	\left\{
	(\omega_n) \in \Omega^\N \ | \ \sum \limits_{i = 1}^\infty r_i(\omega_i) x_i\ \textrm{ist divergent} \
	\right\}
	\right) = 0,
\end{align*}
wobei $ \Omega  $ der Wahrscheinlichkeitsraum der einzelnen Rademachervariablen und $ \rho $ das zu $ \Omega^\N $ gehörende Wahrscheinlichkeitsmaß ist.
\end{df}

%Der Beweis des nächsten Satzes folgt unmittelbar aus der Äquivalenz von 
%\begin{align*}
%	\lim \limits_{m > n \to \infty} \mathbb{E}
%	\left(
%	\left\|
%	\sum \limits_{i=n}^m
%	r_i x_i
%	\right\|
%	\right) = 0
%\end{align*}
%und fast überall Summierbarkeit von $ (r_n x_n) $.
%Diese Aussage wird in \cite[Chapter 5, \S 5]{Vakhania:1987} bewiesen.
\newpage
\begin{genericthm}{Satz von Chobanyan(1985)}\label{th:lemma_of_chobanyan}
	Sei $ (r_n) $ eine Rademacherfolge, $ (x_n) $ eine Folge in einem Banachraum $ X $
	und $ (r_n x_n) $ fast überall summierbar.
	Dann erfüllt $ (x_n) $ die Steinitzeigenschaft.
\end{genericthm}

Der Beweis dieser Aussage folgt unmittelbar aus der Äquivalenz von 
\begin{align*}
	\lim \limits_{m > n \to \infty} \mathbb{E}
	\left(
	\left\|
	\sum \limits_{i=n}^m
	r_i x_i
	\right\|
	\right) = 0
\end{align*}
und der fast überall Summierbarkeit von $ (r_n x_n) $.
Diese Aussage wird in \cite[Chapter 5, \S 5]{Vakhania:1987}  bewiesen. Nachfolgend werden wir noch eine weitere Variante für den Beweis des Satzes von Chobanyan skizzieren.
Hierfür nutzen wir den Satz von Pecherskii
\begin{lem}
	Sei $ (r_n) $ eine Rademacherfolge, $ (x_n) $ eine Folge in einem Banachraum $ X $ und $ (r_n x_n ) $ fast überall summierbar.
	Dann ist $ (r_{\pi(n)} x_{\pi(n) } ) $ für alle $ \pi \in \mathcal{S}_\N $ fast überall summierbar.
\end{lem}

\begin{proof}
	Wir werden wieder die in \cite{Vakhania:1987} bewiesene Äquivalenz verwenden.
%	Sei $ \varepsilon > 0 $ beliebig.
%	Nach Voraussetzung existiert ein $ N_\varepsilon \in \N $, sodass
%	\begin{align*}
%		\mathbb{E}
%		\left(
%		\left\|
%		\sum 
%		\limits_{i=n}^m r_i x_i
%		\right\|
%		\right)
%		\leq 
%		\varepsilon
%	\end{align*}
%	für $ m > n \geq N_\varepsilon $ gilt.
	Wir nehmen an es existiert eine Permutation $ \pi $, sodass $ (r_{\pi(n)} x_{\pi(n)})  $ nicht fast überall summierbar ist.
	Damit gilt:
	\begin{align*}
		\exists_{\kappa > 0}
		\forall_{K \in \N}
		\exists_{k_2 > k_1 \geq K}:
		\mathbb{E}
		\left(
		\left\|
		\sum 
		\limits_{i=k_1}^{k_2} r_{\pi(i)} x_{\pi(i)}
		\right\|
		\right)
		> 
		\kappa.
	\end{align*}
	Wir setzen $ \varepsilon := \kappa $. Dann exisitiert nach Voraussetzung ein $ N_\varepsilon $,
	sodass
	\begin{align*}
		\mathbb{E}
		\left(
		\left\|
		\sum 
		\limits_{i=n}^m r_i x_i
		\right\|
		\right)
		\leq 
		\varepsilon
	\end{align*}
	für $ m > n \geq N_\varepsilon $ gilt.
	Für eine beliebige Permutation $ \sigma $ mit  beliebigem $ N \in \N $ existiert ein $ L \in \N $ mit
	\begin{align*}
		\sigma
		\left( 
		\{
		n \geq L\}
		\right)
		\subset \{n \geq N\}.
	\end{align*}
	Das $ L $ ergibt sich aus
	\begin{align*}
		L := \max \{ l \in \N \ : \ 1 \leq \sigma(l) \leq N-1\} + 1.
	\end{align*} 
	Insgesamt haben wir einen Widerspruch erhalten.	Damit war die Annahme falsch und die Aussage gilt.
%	Damit existiert ein $  \kappa > 0$, sodass für alle $ K \in \N $  $ k_2 > k_1 \geq K $ mit
%	\begin{align*}
%		\mathbb{E}
%		\left(
%		\left\|
%		\sum 
%		\limits_{i=k_1}^{k_2} r_{\pi(i)} x_{\pi(i)}
%		\right\|
%		\right)
%		> 
%		\kappa
%	\end{align*}
%	exisiteren.
\end{proof}
Damit gibt es insbesondere keine Umordnung von $ (x_n) $ zu einer perfekt unsummierbaren Folge. 
Also ist die Voraussetzung des Satzes von Pecherskii \ref{th:lemma_of_pecherskii} erfüllt.
\newpage

\section{Die Steinitzeigenschaft für $ L^p $-Räume}
%\subsection{Die Steinizeigenschaft für $ L^p $-Räume}
In diesem Abschnitt werden wir den Konvergenzbereich von $ L^p $-Räumen untersuchen.
Die Bedingung für die Steinitzeigenschaft wurde von M. I. Kadets in \cite{Kadets1954}
bewiesen. 
Im Gegensatz zu dem Satz von Pecherskii \ref{th:lemma_of_pecherskii} werden wir eine handliche Bedingungen der Form 
\begin{align*}
	\sum \limits_{i = 1}^\infty \| x_i \|^p < \infty
\end{align*}
für $ 1 < p < \infty $ erhalten.
%Das zeigt insbesondere, dass $ p $-absolute Summierbarkeit die bedingte Summierbarkeit implizieren kann, womit keine allgemeine Äquivalenz zwischen unbedingter und $ p $- absoluter Konvergenz hergestellt werden kann.
Wir werden diese Bedingungen mit dem Satz von Chobanyan \ref{th:lemma_of_chobanyan} nachweisen, womit sich der Beweis des Satzes von Kadets \ref{th:lemma_of_kadets} auf den Nachweis einer Eigenschaft für $ \L^p $-Räume reduziert.
Zuerst benötigen wir jedoch geeignete Mittelwertungleichungen, welche uns die konkreten Werte von $ p $ liefern.


\begin{df}
	Seien $ x_1,..., x_n  \in \R$ und $ r_1,...,r_n$ Rademachervariablen.
	Dann setzen wir 
	\begin{align*}
		M_p 
		:=
		\left(
		\mathbb{E}
		\left|
		\sum \limits_{i = 1}^n r_i x_i 
		\right|^p \
		\right)^{\frac{1}{p}}
	\end{align*}
	für $ 1 \leq p < \infty $.
\end{df}

\begin{lem}
	Der Ausdruck $ M_p  $ entspricht dem $ p $-Mittelwert über alle möglichen Kombinationen
	\begin{align*}
		| \alpha_1 x_1 + ... + \alpha_n x_n |
	\end{align*}
	mit $ \alpha_i = \pm 1$ für $  i = 1, ..., n $. Insbesondere gilt
	\begin{align*}
		M_p = 
		\left(
		\frac{1}{2^n}
		\sum \limits_{\alpha_i = \pm 1}
		\left(\sum \limits_{i = 1}^n \alpha_i x_i\right)^p \
		\right)^\frac{1}{p}
	\end{align*}
	für $ 1 \leq p < \infty $. 
\end{lem}
\begin{proof}
	Der Erwartungswert der Zufallsvariablen $  \mu:= \left|\sum_{i = 1}^n r_i x_i \right|^p $ entspricht der Summe aller möglichen Werte multipliziert mit deren Wahrscheinlichkeiten.
	Die möglichen Werte sind durch
	\begin{align*}
		\left|\sum_{i = 1}^n \alpha_i x_i \right|^p
	\end{align*}
	für $ \alpha_i = \pm 1$ mit $ i = 1,...,n $ gegeben. 
	\newpage
	 $ \mu $ ist gleichverteilt und nimmt die Werte mit der Wahrscheinlichkeit $ 2^{-n} $ an.
	Es ergibt sich:
	\begin{align*}
		\mathbb{E}(\mu)
		=
		\sum \limits_{\alpha_i = \pm 1} \frac{1}{2^n}
		\left(\sum \limits_{i = 1}^n \alpha_i x_i\right)^p. 
	\end{align*}
	Also ist $ M_p $ der in der Aussage erwähnte Mittelwert.
\end{proof}

Da es sich hier um Mittelwerte handelt, gilt
\begin{align*}
	M_1 \leq M_p \leq M_q
\end{align*}
für $ 1 \leq p \leq q < \infty $.
Für den Fall $ p = 2 $ erhält man durch Induktion über $ n $, dass gilt:
\begin{align*}
	M_2 = \left( \sum \limits_{i = 1}^n | x_i |^2\right)^\frac{1}{2}.
\end{align*}
%gilt.\\
Nach dieser Vorbereitung widmen wir uns der Chintschin-Ungleichung.
Die exakten Werte, der dort auftretenden Konstanten, sind bekannt und finden sich in \cite{Szarek1976} und \cite{Young1976}.
Beispielsweise gilt $ a_1 = \nicefrac{1}{\sqrt{2}} $.
Es ist auch möglich $ M_p  $ für beliebige normierte Räume zu definieren. Dies findet hier jedoch keine Anwendungen.


\begin{genericthm}{Chintschin-Ungleichung I}\label{th:khinchin_ineq_1}
	Seien $ x_1,...,x_n \in \R $ beliebig. Dann existiert ein $ 0 < A_p < 1 $ und $ 1 \leq B_p < \infty $, sodass
	\begin{align*}
		A_p 
		\left(
		\sum \limits_{i = 1}^n | x_i|^2		
		\right)^\frac{1}{2}
		\leq 
		M_p
		\leq
		B_p 
		\left(
		\sum \limits_{i = 1}^n | x_i|^2		
		\right)^\frac{1}{2}
	\end{align*}
	für $ 1 \leq p \leq 2 $ gilt.  
\end{genericthm}
\begin{proof}
	\begin{description}
		\item[1. Fall:   ]
		Mit $ 1 \leq p \leq 2 $ erhalten wir durch $ M_p \leq M_2 $ die rechte Seite unmittelbar.
		Wegen $ M_1 \leq M_p $ genügt es zu zeigen, dass ein $ a > 0 $ mit
		\begin{align}\label{eq:khinchin_1_proof_1}
			a 
			\left(
			\sum \limits_{i = 1}^n | x_i |^2 \ 
			\right)^\frac{1}{2}
			\leq M_1 
			=
			\mathbb{E}
			\left(\left|
			\sum \limits_{i = 1}^n  r_i x_i
			\right|\right)
		\end{align}
		existiert. Hierbei ist $ a $ unabhängig von $ n $ und der Wahl der $ x_i $. 
		Aus der Ungleichung \eqref{eq:khinchin_1_proof_1} ergibt sich:
		\begin{align*}
			\mathbb{E}
			\left(\left|
			\sum \limits_{i = 1}^n  r_i x_i
			\right|\right)
			\geq 
			a 
			\left(
			\sum \limits_{i = 1}^n | x_i |^2 \ 
			\right)^\frac{1}{2} > 0
			\ \Leftrightarrow \
			\frac{1}{\left(\sum \limits_{i = 1}^n | x_i |^2 \ \right)^\frac{1}{2}}
			\mathbb{E}
			\left(\left|
			\sum \limits_{i = 1}^n  r_i x_i
			\right|\right)
			\geq a > 0.
		\end{align*}
		Durch die Linearität des Erwartungswerts und  $ t_i := \left(\sum_{i = 1}^n |x_i|^2\right)^{-\frac{1}{2}} x_i $ erhalten wir das System
		\begin{align}\label{eq:khinchin_1_proof_2}
			\mathbb{E}
			\left(\left|
			\sum \limits_{i = 1}^n  r_i t_i
			\right|\right) 
			\geq  a > 0, \quad
			\sum \limits_{i = 1}^n t_i^2 = 1.
		\end{align}
		Wir betrachten die Ungleichung $ |t | \geq 1- \cos t $ für alle $ t \in \R $ und die Identität
		\begin{align*}
			\sum \limits_{\alpha_i = \pm 1} 
			\cos \left(
			\sum \limits_{i = 1}^n \alpha_i t_i
			\right) 
			=
			2^n \prod \limits_{i = 1}^n \cos(t_i).
		\end{align*}
		Die Identität resultiert aus einer Induktion über $ n $ und der Anwendung eines geeigneten Additionstheorems.
		Hiermit ausgerüstet erhalten wir:
		\begin{align*}
			\mathbb{E}
			\left(\left|
			\sum \limits_{i = 1}^n  r_i t_i
			\right|\right) 
			=
			\frac{1}{2^n} \sum \limits_{\alpha_i = \pm 1}
			\left| \sum \limits_{i = 1}^n \alpha_i t_i \right|
			\geq 
			\frac{1}{2^n}
			\sum \limits_{\alpha_i = \pm 1}
			\left(
			1 - \cos
			\left(
			\sum \limits_{i = 1}^n \alpha_i t_i
			\right)
			\right)
			=
			1 - 
			\prod \limits_{i = 1}^n \cos(t_i).
		\end{align*}
		Das System \eqref{eq:khinchin_1_proof_2} ist erfüllt, wenn wir ein (geeignetes) Minimum von $ 1 - 
		\prod_{i = 1}^n \cos(t_i)$ unter der Nebenbedingung $ \sum_{i = 1}^n t_i^2 = 1 $ finden.
		Dieses Minimum erhalten wir an der Stelle $ t_1 = ... = t_n = n^{- \frac{1}{2}} $ mit dem Wert $ 1 -( \cos n^{-\frac{1}{2}})^n $.
		Zuletzt betrachten wir:
		\begin{align*}
			1 -( \cos n^{-\frac{1}{2}})^n \geq 1 - e^{-\frac{1}{2}}
			\ \Leftrightarrow \
			( \cos n^{-\frac{1}{2}})^n \leq e^{-\frac{1}{2}}
			\ \Leftrightarrow \
			\cos n^{-\frac{1}{2}} \leq e^{-\frac{1}{2n}}.
		\end{align*}
		Die letzte Ungleichung ist erfüllt. Dies erhalten wir aus der Reihendarstellung beider Seiten und der Ungleichung $ (2k)! \geq 2^k k! $ für alle $ k \in \N $.
		Damit haben wir für $ 1 \leq p \leq 2 $ ein geeignetes $ A_p \geq 1 - e^{-\frac{1}{2}} $ gefunden.
		\item[2. Fall:] 
		Für $ 2 < p < \infty $ ergibt sich
		die linke Seite der Ungleichung sich aus der Monotonie der $ M_p $.
		Wir interessieren uns also nur für die rechte Seite der Ungleichung. Analog zu dem letzten Beweis dividieren wir die Ungleichung durch $ t_i := \left(\sum_{i = 1}^n |x_i|^2\right)^{-\frac{1}{2}} x_i $ und erhalten das System
		\begin{align}\label{eq:khinchin_2_proof_1}
			\left(
			\mathbb{E} 
			\left(
			\left|
			\sum\limits_{i = 1}^n r_i t_i
			\right|^p \
			\right)
			\right)^\frac{1}{p}
			\leq B_p , \quad \sum \limits_{i = 1}^n t_i^2 = 1.
		\end{align}
		Unser Ziel ist nachzuweisen, dass ein solches $1 \leq A_p < \infty   $ existiert. Hierfür betrachten wir folgende Ungleichung:
		Es existiert ein $ C_p > 0 $ (abhängig von $ p $), sodass
		\begin{align*}
			|t|^p \leq C_p (\cosh t - 1)
		\end{align*} 
		für alle $ t \in \R $ gilt. Die Ungleichung selbst ergibt sich aus der Definition von $ \cosh  $ und den Wachstumseigenschaften der Exponential-und Potenzfunktionen.\newpage
		Durch Induktion über $ n $ erhalten wir mit den Additionstheoremen für $ \cosh $ die Gleichung
		\begin{align*}
			\sum \limits_{\alpha_i = \pm 1} 
			\cosh
			\left(
			\sum \limits_{i = 1}^n \alpha_it_i
			\right)
			=
			2^n 
			\prod \limits_{i = 1}^n \cosh(t_i),
		\end{align*}
		woraus dann insgesamt
		\begin{align*}
			\mathbb{E} 
			\left(
			\left|
			\sum\limits_{i = 1}^n r_i t_i
			\right|^p \
			\right)
			\leq 
			C_p
			\left( \prod \limits_{i = 1}^n \cosh(t_i) - 1\right)
		\end{align*}
		folgt.
		Das System \eqref{eq:khinchin_2_proof_1} ist sinnvoll, falls ein Maximum von $ 	C_p
		\left( \prod_{i = 1}^n \cosh(t_i) - 1\right) $ unter der Nebenbedingung $ \sum_{i = 1}^n t_i^2 = 1 $ existiert.
		Dieses Maximum befindet sich an der Stelle $ t_1 = ... = t_n = n^{- \frac{1}{2}} $ dem Wert $ (\cosh n^{- \frac{1}{2}} )^n - 1 $. 
		Analog zu dem Beweis der ersten Chintschin-Ungleichung erhalten wir
		\begin{align*}
			\left(\cosh n^{- \frac{1}{2}} \right)^n - 1 \leq e^{\frac{1}{2}} - 1
		\end{align*}
		durch geeignete Umformungen und anschließenden Vergleich der Potenzreihen.
		Damit ist unsere Schranke nach oben unabhängig von $ n $ und den $ x_i $, womit ein geeignetes $ A_p  $ existiert. 
	\end{description}
	
	
	
\end{proof}



\begin{df}
	Sei $ X $ ein normierter Raum und $ (r_n) $ eine Rademacherfolge.
	Wir nennen $ X $ von \textit{Typ} $ p $ mit der Konstanten $ C $, falls für alle endlichen Teilmengen $ \{x_i \}_{i=1}^n \subset X $
	\begin{align}
		\mathbf{E}
		\left(
		\left\|
		\sum  \limits_{i = 1}^n r_i x_i
		\right\|
		\right)
		\leq 
		C 
		\left(
		\sum  \limits_{i = 1}^n \|x_i\|^p
		\right)^{\frac{1}{p}}
	\end{align}
	erfüllt ist.
\end{df}

\begin{genericthm}{Folgerung von Chobanyan}\label{th:conclusion_chobanyan}
	Sei $ (x_n) $ eine Folge in einem Banachraum $ X $ von Typ $ p $ mit $ \sum_{i=1}^\infty \|x_i \|^p < \infty $.
	Dann besitzt $ (x_n) $ die Steinitzeigenschaft.
\end{genericthm}

\begin{proof}
	Da $ X $ von Typ $ p $ ist, erhalten wir
	\begin{align*}
		\mathbb{E}
		\left( \left\| \sum \limits_{i=1}^\infty r_i x_i  \right\|\right)
		\leq
		C 
		\left( \sum \limits_{i=1}^\infty\| x_i \|^p  \right)^\frac{1}{p} < \infty.
	\end{align*}
	Hierbei ist $ (r_n) $ eine Rademacherfolge. Damit ist $ (r_n x_n) $ fast sicher summierbar und die Voraussetzungen des Satzes von Chobanyan \ref{th:lemma_of_chobanyan} sind erfüllt.\qedhere
	
	
\end{proof}

\begin{sz}\label{th:Lp_type2}
	Die Räume $ \L^p(\Omega,\mu) $ mit $ 2 < p < \infty $ sind von Typ $ 2 $.
\end{sz}

\begin{proof}
Seien $ f_1,...,f_n \in \L^p(\Omega,\mu) $ beliebig und $ r_1,...,r_n $ Rademachervariablen.
Dann erhalten wir mit der Chintschin-Ungleichung \ref{th:khinchin_ineq_1}:
\begin{equation*}
	\begin{split}
		\mathbb{E}
		\left(
		\left\|
		\sum \limits_{i=1}^n r_i f_i
		\right\|
		\right)
		&\leq
		\left( 
		\mathbb{E}
		\left(
		\left\|
		\sum \limits_{i = 1}^n
		r_i f_i
		\right\|^p \ 
		\right)
		\right)^\frac{1}{p}
		=
		\left( 
		\mathbb{E}
		\int \limits_{\Omega} 
		\left| 
		\sum \limits_{i = 1}^n
		r_i f_i(t)
		\right|^p
		\dx{\mu}
		\right)^\frac{1}{p}\\
		&=
		\left(
		\int \limits_{\Omega}
		\mathbb{E}
		\left|
		\sum \limits_{i = 1}^n
		r_i f_i(t)
		\right|^p
		\dx{\mu}
		\right)^\frac{1}{p}
		\leq
		\left(
		\int \limits_{\Omega}
		A_p^p
		\left(
		\sum \limits_{i = 1}^n
		|f_i(t)|^2
		\right)^\frac{p}{2}
		\dx{\mu}
		\right)^\frac{1}{p}\\
		&=
		A_p
		\left(
		\int \limits_{\Omega}
		\left(
		\sum \limits_{i = 1}^n
		|f_i(t)|^2
		\right)^\frac{p}{2}
		\dx{\mu}
		\right)^\frac{1}{p}.
	\end{split}	
\end{equation*}
Im nächsten Schritt verwenden wir die Dreiecksungleichung mit $ |f_i| \in \L^{\frac{p}{2}}(\Omega,\mu) $ für $ i = 1,...,n $.
Damit erhalten wir
\begin{align*}
	\left(
	\int \limits_{\Omega}
	\left(
	\sum \limits_{i = 1}^n
	|f_i(t)|^2
	\right)^\frac{p}{2}
	\dx{\mu}
	\right)^\frac{1}{p}
	=
	\left\|
	\sum \limits_{i = 1}^n |f_i|^2
	\right\|_{\L^{\frac{p}{2}}}^\frac{1}{2}
	\leq
	\left(
	\sum \limits_{i = 1}^n 
	\| f_i^2 \|_{\L^\frac{p}{2}}
	\right)^\frac{1}{2}
	=
	\left(
	\sum \limits_{i = 1}^n 
	\| f_i \|_{\L^p}^2
	\right)^\frac{1}{2} ,
\end{align*}
woraus die gewünschte Ungleichung folgt. 
\end{proof}
\begin{sz}\label{th:Lp_typep}
	Die Räume $ \L^p(\Omega,\mu) $ mit $ 1 \leq p \leq  2 $ sind von Typ $ p $.
\end{sz}
\begin{proof}
Seien $ f_1,...,f_n \in \L^p(\Omega,\mu) $ beliebig und $ r_1,...,r_n $ Rademachervariablen.
Analog zum vorherigen Beweis wenden wir die Chintschin-Ungleichung \ref{th:khinchin_ineq_1} an und erhalten
\begin{align*}
	\mathbb{E}
	\left(
	\left\|
	\sum \limits_{i=1}^n r_i f_i
	\right\|
	\right)
	\leq
	\left(
	\int \limits_{\Omega}
	\left(
	\sum \limits_{i = 1}^n
	|f_i(t)|^2
	\right)^\frac{p}{2}
	\dx{\mu}
	\right)^\frac{1}{p}.	
\end{align*}

Da die Norm von $ \ell^p_{n} $ bezüglich $ p $ monoton fallend ist, erhalten wir
\begin{align*}
	\left( \sum \limits_{i = 1}^n |x_i|^2  \right)^\frac{1}{2}
	\leq 
	\left( \sum \limits_{i = 1}^n |x_i|^p  \right)^\frac{1}{p}
\end{align*}
für $ 1 \leq p \leq 2 $ und beliebige $ x_i $. \newpage
Hiermit folgt wegen
\begin{align*}
	\left(
	\int \limits_{\Omega}
	\left(
	\sum \limits_{i = 1}^n
	|f_i(t)|^2
	\right)^\frac{p}{2}
	\dx{\mu}
	\right)^\frac{1}{p}
	\leq 
	\left(
	\int \limits_{\Omega}
	\sum \limits_{i = 1}^n
	|f_i(t)|^p
	\dx{\mu}
	\right)^\frac{1}{p}
	=
	\left(
	\sum \limits_{i = 1}^n
	\| f_ i\|^p_{\L^p}
	\right)^\frac{1}{p}
\end{align*}
die gewünschte Abschätzung.

\end{proof}


\begin{genericthm}{Satz von M. I. Kadets (1954)}\label{th:lemma_of_kadets}
	Sei $ 1 \leq p < \infty $,  $ r := \min\{2,p\} $
	und $ (x_n) $ eine Folge in $ L^p(\Omega,\mu) $.
	Dann ist für die Steinitzeigenschaft die Bedingung
	\begin{align*}
	 \sum \limits_{i=1}^\infty \|x_i \|^r < \infty 
	\end{align*}
	hinreichend.
\end{genericthm}

\begin{proof}
Der Beweis dieser Aussage ergibt sich unmittelbar aus der Folgerung von Chobanyan \ref{th:conclusion_chobanyan}
und dem Nachweis der Typeigenschaften der $ \L^p $ -Räume \ref{th:Lp_type2} und \ref{th:Lp_typep}.
\end{proof}





\newpage

\chapter{Unbedingte Konvergenz in Banachräumen}
%\section{Unbedingte Konvergenz in Banachräumen}


\section{Charakterisierung der unbedingten Konvergenz}\label{sc:char_uncon_sum}
%\subsection{Charakterisierung der unbedingten Konvergenz}





Um den Begriff der unbedingten Konvergenz zu charakterisieren, führen wir weitere Konvergenzbegriffe ein und zeigen deren Äquivalenz zur unbedingten Konvergenz.
%Wir werden dann zeigen, dass diese äquivalent zu der unbedingten Konvergenz sind. 

\begin{df}
	Sei $ (x_n) $ eine Folge in einem Banachraum $ X $.
	\begin{enumerate}
	\item 
	Eine Folge $ (x_n) $ heißt \textit{perfekt summierbar}, falls
	$ (\alpha_n x_n ) $ für alle $ \alpha \in \{\pm 1\}^\N $ summierbar ist.
	Die Reihe heißt dann \textit{perfekt konvergent}.
	%		Eine Reihe $ \sum_{k=1}^\infty x_k $ heißt \textit{perfekt konvergent}, falls
	%		\begin{align*}
	%		\sum \limits_{k=1}^\infty \alpha_k x_k
	%		\end{align*}
	%		für alle $ \alpha \in \{\pm 1\}^\N $ konvergiert.
	
	\item
	Eine Folge $ (x_n) $ heißt \textit{ungeordnet summierbar}, falls zu jedem 
	$ \varepsilon >0  $ ein $ N_\varepsilon  \in \N$ existiert, sodass für jede endliche Menge $ M \subset \N $ mit $ \min \ M > N_\varepsilon $
	\begin{align*}
		\left\| \sum \limits_{i \in M } x_i \right\| < \varepsilon
	\end{align*}
	gilt.
	Die Reihe heißt dann \textit{ungeordnet konvergent}.
	
	\item
	Eine Folge $ (x_n) $ heißt \textit{teil-summierbar}, falls $ (x_{n_j}) $
	für jede Indexfolge $ (n_j) $ summierbar ist.
	Die Reihe heißt dann \textit{teil-konvergent}.\\
	%Unter einer \textit{Indexfolge} $ (k_n) $ verstehen wir eine streng monoton wachsende Folge mit Gliedern in $ \N $.
\end{enumerate}
\end{df}


\begin{sz}\label{th:equi_uncond_1}
	Sei $ X $ ein Banachraum und $ (x_n) $ eine Folge in $ X $.
	Dann sind äquivalent:
	\begin{enumerate}
		\item $  (x_n) $ ist unbedingt summierbar.
		\item $ (x_n) $ ist ungeordnet summierbar.
		\item $ (x_n) $ ist teil-summierbar.
		\item $ (x_n) $ ist perfekt summierbar.
	\end{enumerate}	
	
\end{sz}

\begin{proof}
	\begin{description}
		\item[\textit{ \itshape\textrm{(i)}} $ \Rightarrow $ \textbf{\textit{\textrm{(ii)}}}:]
		Angenommen $ (x_n) $ ist nicht ungeordnet summierbar, d.h.
		\begin{align*}
			\exists \delta > 0 \  
			\forall n \in \N \
			\exists \tilde{M}_n \subset \N \ \textrm{endlich mit}\min \tilde{M}_n > n:
			\left\| \sum_{m \in \tilde{M}_n} x_m \right\| \geq \delta. 
		\end{align*}
		Wir definieren uns eine Mengenfolge durch
		\begin{align*}
			M_1 := \tilde{M}_1, \quad 
			M_n := \tilde{M}_{n + \min\{ k \ | \ \tilde{M}_n \cap \tilde{M}_{n+k} = \emptyset \}}. 
		\end{align*} 
		Diese Konstruktion erfüllt $ \max M_n < \min M_{n+1} $ und $\left\| \sum_{m \in M_n} x_m \right\| \geq \delta $ für alle $ n \in \N $.
		Damit sind die $ M_n $ disjunkt und es existiert eine Permutation $ \pi $ mit
		\begin{align*}
			\pi(\{\min M_n,...,\min M_n + |M_n| \}) = M_n
		\end{align*}
		für alle $ n \in \N  $. Damit ist die Partialsummenfolge über $ (x_{\pi(n)}) $  keine Cauchyfolge.
		Also erhalten wir einen Widerspruch zur unbedingten Summierbarkeit.		
		\item[\textit{ \itshape\textrm{(ii)}} $ \Rightarrow $ \textbf{\textit{\textrm{(i)}}}:]
		Sei $ \varepsilon > 0  $ beliebig. Dann existiert ein $ N_\varepsilon  \in \N$, sodass
		\begin{align*}
			\left\| \sum \limits_{m \in M } x_m \right\| < \varepsilon
		\end{align*}
		für alle endlichen $ M \subset \N $ mit $ \min M > N_\varepsilon $ gilt.
		Wir wählen eine beliebige Permutation $ \pi \in \mathrm{S}_\N $.
		Dann existiert ein $ M_\varepsilon \in \N $, sodass $ \{1,...,N_\varepsilon\} \subseteq 
		\pi(\{1,...,M_\varepsilon\}) $ gilt. Damit erhalten wir insbesondere
		\begin{align*}
			\left\| \sum \limits_{k=n}^m x_{\pi(k)} \right\| < \varepsilon
		\end{align*}
		für $ m > n > M_\varepsilon $ und mit der Vollständigkeit von $ X $ ist $ (x_{\pi(n)}) $ summierbar. Da $ \pi $ beliebig gewählt wurde, erhalten wir die unbedingte Summierbarkeit.
		\item[\textit{ \itshape\textrm{(ii)}} $ \Rightarrow $ \textbf{\textit{\textrm{(iii)}}}:]
		Sei $ \varepsilon > 0  $ beliebig. Dann existiert ein $ N_\varepsilon  \in \N$, sodass
		\begin{align*}
			\left\| \sum \limits_{m \in M } x_m \right\| < \varepsilon
		\end{align*}
		für alle endlichen $ M \subset \N $ mit $ \min M > N_\varepsilon $ gilt.
		Sei $ (n_j) $ eine Indexfolge.
		Dann existiert ein $ p \in \N $ mit $ n_p > \N_\varepsilon $. 
		Insbesondere erhalten für $ q > p $ die Teil-Summierbarkeit durch
		\begin{align*}
			\left\| \sum \limits_{j = p}^q x_{n_j} \right\| < \varepsilon.
		\end{align*}
	
		
		
		\item[\textit{ \itshape\textrm{(iii)}} $ \Rightarrow $ \textbf{\textit{\textrm{(iv)}}}:]
		Sei $ (\alpha_n) \in \{\pm 1\}^\N$ beliebig.
		Wir trennen $ \N $ durch 
		\begin{align*}
			S_\pm := \{n \in \N  \ | \ \alpha_n = \pm 1 \}
		\end{align*} 
		in zwei Indexmengen auf. 
		Sei ohne Beschränkung der Allgemeinheit $ |S_+| = | S_- | = \infty $ und seien $ (n_j^\pm) $ die entsprechenden streng monotonen Indexfolgen.
		Die Teil-Summierbarkeit von $ (x_n) $ liefert die Summierbarkeit von $ (x_{n_j^+}) $ und $ (x_{n_j^-}) $.
		Aus den Grenzwertsätzen erhalten wir dann die Summierbarkeit von $ (\alpha_n x_n) $.
		\item[\textit{ \itshape\textrm{(iv)}} $ \Rightarrow $ \textbf{\textit{\textrm{(ii)}}}:]
		Sei $ (x_n) $ perfekt summierbar. Angenommen $ (x_n) $ ist nicht ungeordnet summierbar.
		Wie in dem ersten Teil des Beweises lässt sich eine Mengenfolge mit
		\begin{align*}
			\max  M_n < \min M_{n+1} 
			\ \textrm{und} \
			\left\| \sum \limits_{m \in M_n } x_m \right\| \geq \delta 
		\end{align*}
		für ein $ \delta > 0 $ und alle $ n \in \N  $ konstruieren.
		Wir definieren die Koeffizientenfolge $ (\alpha_n) $ durch
		\begin{align*}
			\alpha_n	
			:=
			\begin{cases}
				1 \quad &\quad  \textrm{falls } n \in \bigcup \limits_{i \in \N} M_i\\
				-1 \quad &\quad  \textrm{sonst.}
			\end{cases}.
		\end{align*}
		Die negativen Koeffizienten müssen auftreten, da wir ansonsten unmittelbar einen Widerspruch zur perfekten Summierbarkeit von $ (x_n) $ erhalten.
		Wegen 
		\begin{align*}
			\left\| \sum \limits_{i = \min M_n}^{\max M_n} x_i \right\| \geq \delta 
		\end{align*}
		für alle $ n \in \N $ ist die Partialsummenfolge zu $ ((1+ \alpha_n ) x_n) $ eine Cauchyfolge. Damit ist $ (x_n) $ oder $ (\alpha_n x_n)$ unsummierbar.
		Beide Fälle sind ein Widerspruch zu der perfekten Summierbarkeit.
	\end{description}
\end{proof}

\begin{genericthm}{Schranken-Test}\label{th:bounded_test}
	Sei $ X $ ein Banachraum.
	Die Folge $ (x_n) $ ist in $ X $ genau dann unbedingt summierbar, wenn
	$ (b_n x_n) $ für alle $ (b_n) \in \ell^\infty $  summierbar ist.
\end{genericthm}
\begin{proof}
	Sei $ (b_n x_n) $ für alle $ (b_n) \in \ell^\infty $ summierbar.
	Wegen $ \{\pm 1\}^\N \subset \ell^\infty $ ist $ (x_n) $ perfekt summierbar und mit dem Satz \ref{th:equi_uncond_1} folgt die unbedingte Summierbarkeit.\\
	Nun sei $ (x_n) $ unbedingt summierbar.
	Aufgrund der Vollständigkeit von $ X $ genügt es 
	\begin{align*}
		\lim \limits_{n,m \to \infty } \left\| \sum \limits_{i = m }^n b_i x_i \right\| = 0
	\end{align*}
	für alle $ b =  (b_n)  \in \ell^\infty$ zu zeigen. Mit dem Satz von Hahn-Banach gilt
	\begin{align*}
		 \left\| \sum \limits_{i = m }^n b_i x_i \right\|
		 =
		 \sup \limits_{x^\prime \in X^\prime} \left| \left\langle x^\prime, \sum \limits_{i = m }^n b_i x_i \right\rangle \right|
		 \leq
		 \| b \|_\infty \cdot \sup \limits_{x^\prime \in X^\prime} \sum \limits_{i = m }^n \left| \left\langle x^\prime,   x_i \right\rangle \right|
	\end{align*}
	für $ n > m  $.
	Wir verwenden nun die Äquivalenz zur ungeordneten Summierbarkeit \ref{th:equi_uncond_1}.
	Sei $ \varepsilon > 0  $ beliebig. Dann existiert ein $ m_\varepsilon  \in \N$, sodass
	für jede endliche Menge $ M \in \N $, mit $ \min M >m_\varepsilon $, die Abschätzung
	$ \left\| \sum_{i \in M} x_i \right\| < \varepsilon $ gilt.
	Wir nehmen an, dass $ X $ ein $ \R $-Banachraum ist.
	Ansonsten muss der Real-und Imaginärteil des Dualitätprodukts (analog) seperat untersucht werden.
	Wir fixieren  $ x^\prime $, wählen $ n >m > m_\varepsilon $ und setzen
	\begin{align*}
		M_+ &:= \{m \leq i \leq n \ | \ \langle x^\prime, x_i  \rangle \geq 0 \}\\
		M_- &:=  \{m \leq i \leq n \ | \ \langle x^\prime, x_i \rangle   <  0 \}.
	\end{align*}
	Dann folgt mit der ungeordneten Summierbarkeit:
	\begin{align*}
		\sum 
		\limits_{i = m }^n |\langle x^\prime ,x_i \rangle | 
		=
		\left|
		\sum 
		\limits_{i \in M_+ } \langle x^\prime , x_i \rangle 
		\right|
		+ 
			\left|
		\sum 
		\limits_{i \in M_- } \langle x^\prime , x_i \rangle 
		\right|
		\leq
		\left\| \sum	\limits_{i \in M_+ } x_i \right\|
		+
		\left\| \sum	\limits_{i \in M_- } x_i \right\| 
		< 2 \varepsilon.
	\end{align*}
	Damit ist $ (b_n x_n) $ summierbar.
\end{proof}

Es ist bekannt, dass die Normkonvergenz die schwache Konvergenz in jedem normierten Raum impliziert.
Umgekehrt ist dies im Allgemeinen nicht der Fall. 
Jedoch gibt es Räume, für welche diese Richtung gilt.
Für unsere Zwecke betrachten wir in diesem Kontext die von Schur\cite{Schur1920} gezeigte Aussage.


\begin{genericthm}{$ \ell^1 $-Satz von Schur(1920)}\label{th:schur_l1}
	Sei $ (x^{(n)}) $ eine schwach konvergente Folge in $ \ell^1 $.
	Dann ist $ (x^{(n)}) $ konvergent in $ \ell^1 $.
	%In $ \ell^1 $ sind schwache Konvergenz und Normkonvergenz für Folgen identisch.
\end{genericthm}
\begin{proof}
	Sei $ (x^{(n)}) $ schwach konvergent in $ \ell^1 $.
	Wir können ohne Beschränkung annehmen, dass $ (x^{(n)}) $ schwach gegen $ 0 $ konvergiert.
	Der Dualraum von $ \ell^1 $ ist gegeben durch $ \ell^\infty $. Wegen $ x_n \rightharpoonup 0 $ gilt
	\begin{align*}
		\langle
		y, x^{(n)} 
		\rangle
		= \sum \limits_{i = 1}^\infty y_i x_i^{(n)} \overset{n \to \infty}{\rightarrow} 0
	\end{align*}
	für alle $ y \in \ell^\infty $.
	Damit folgt $ x^{(n)}_i \rightarrow 0 $ für alle $ i \in \N $ und wir erhalten
	\begin{align*}
		\sum \limits_{i = 1 }^N | x^{(n)}_i | \rightarrow 0
	\end{align*}
	für alle (fixierten) $ N \in \N  $.
	Nach dem Satz von Banach-Alaoglu ist $ B_{\ell^\infty} $ schwach-$ \ast $ kompakt.
	Wegen der Separabilität von $ \ell^1 $ ist $ B_{\ell^\infty} $ schwach-$ \ast $ metrisierbar.
	Durch 
	\begin{align*}
		U(\hat{y},\delta,N)
		=
		\{
		y \in B_{\ell^\infty} \ : \
		|y_k - \hat{y}_k| < \delta , \ 1 \leq k \leq N
		\}
	\end{align*}
	für $ \delta > 0 $ und $ N \in \N $ erhalten wir eine Basis aus schwach-$ * $ Umgebungen von $ \hat{y} \in B_{\ell^\infty} $.
	%Durch diese zwei Eigenschaften lässt sich der Bairsche Kategoriensatz anwenden.
	Wir wählen $ \varepsilon > 0  $ fest und definieren
	\begin{align*}
		B_m 
		:= \bigcap \limits_{n \geq m}
			\left\{
			y \in B_{\ell^\infty} \ : \ |\langle y, x^{(n)} \rangle | \leq \frac{\varepsilon}{3}
			\right\}
	\end{align*}
	für $ m \in \N $.
	Da der beliebige Schnitt abgeschlossener Mengen abgeschlossen ist, folgt die schwach-$ \ast $ Abgeschlossenheit von $ B_m $.
	Desweiteren ist die Mengenfolge $ (B_m) $ monoton wachsend
	und $ x_n \rightharpoonup 0 $ impliziert
	\begin{align*}
		B_{\ell^\infty }
		=
		\bigcup \limits_{m \in \N} B_m.
	\end{align*}
	Damit können wir den Baireschen Kategoriensatz anwenden.
	Nach diesem existiert ein $ m_0 \in \N $, sodass das Innere von $ B_{m_0} $ nichtleer ist. Insbesondere gilt dies für alle $ B_m $ mit $ m \geq m_0 $.
	Außerdem finden wir ein $ \hat{y} \in B_{m_0} $, $ \delta > 0 $ und $ N \in \N $, sodass
	\begin{align*}
		U(\hat{y}, \delta , N)
		\subset B_{m_0} \subseteq B_m
	\end{align*}
	für $ m \geq m_0 $ gilt. Unter eventueller Anpassung von $ m_0 $ gilt dann auch
	\begin{align*}
		\sum \limits_{i =1}^N | x^{(m)}_i | < \frac{\varepsilon}{3}
	\end{align*}
	für $ m \geq m_0 $.
	Wir fixieren ein beliebiges $ m \geq m_0 $ und verwenden, dass in der Umgebung $ U(\hat{y},\delta, N) $ nur endlich viele Folgenglieder relevant sind. Damit definieren wir $ y \in B_{\ell^\infty} $ durch
	\begin{align*}
		y_i
		:=
		\begin{cases}
			\hat{y}_i , &\ 1 \leq i \leq N,\\
			\sign \  x^{m}_i , &\ i > N.
		\end{cases}
	\end{align*}
	Dies impliziert $ y \in U(\hat{y},\delta, N) \subset B_{m_0}  $
	und $ | \langle y ,x^{(m)} \rangle  | \leq \frac{\varepsilon}{3}$.
	Insgesamt erhalten wir mit
	\begin{align*}
		\| x^{(m)} \|
		&=
		\sum \limits_{i \geq 1} |x_i^{(m)}|
		=
		\sum \limits_{i \leq N} |x_i^{(m)}|
		+
		\sum \limits_{i > N} |x_i^{(m)}|
		=
		\sum \limits_{i \leq N} |x_i^{(m)}|
		+
		\left|
		\sum \limits_{i \geq 1} y_i x_i^{(m)}
		-
		\sum \limits_{i \leq N} y_i x_i^{(m)}
		\right|\\
		&\leq
		\sum \limits_{i \leq N} (1+ y_i)|x_i^{(m)}|
		+ | \langle y, x^{(m )} \rangle | 
		<
		\frac{2 \varepsilon}{3} + \frac{\varepsilon}{3}
		= \varepsilon
	\end{align*}
	die gewünschte Aussage.
	


	
\end{proof}

\begin{df}
	Sei $ (x_n) $ eine Folge in einem Banachraum $ X $. 
	\begin{enumerate}
		\item 
		Die Folge $ (x_n) $ heißt  \textit{schwach summierbar}, falls ein $ s  \in X$ existiert, sodass
		\begin{align*}
			\lim\limits_{n \to \infty}
			\left\langle x^\prime,
			\sum \limits_{i = 1}^n x_{\pi(i)}
			\right\rangle_{X^\prime}
			= 
			\langle x^\prime,s\rangle_{X^\prime}
		\end{align*}
		für alle $ x^\prime \in X^\prime $ gilt.
		
		\item 
		Die Folge $ (x_n) $ heißt  \textit{schwach unbedingt summierbar}, falls ein $ s  \in X$ existiert, sodass
		\begin{align*}
			\lim\limits_{n \to \infty}
			\left\langle x^\prime,
			\sum \limits_{i = 1}^n x_{\pi(i)}
			\right\rangle_{X^\prime}
			= 
			\langle x^\prime,s \rangle_{X^\prime}
		\end{align*}
		für alle $ x^\prime \in X^\prime $ und $ \pi \in \mathcal{S}_\N $ gilt.
		
		\item 
		Die Folge $ (x_n) $ heißt  \textit{schwach perfekt summierbar}, falls für alle $ \pm 1 $-Folgen $ (\alpha_n) $ ein $ s  \in X$ existiert, sodass
		\begin{align*}
			\lim\limits_{n \to \infty}
			\left\langle x^\prime,
			\sum \limits_{i = 1}^n \alpha_i x_i
			\right\rangle_{X^\prime}
			= 
			\langle x^\prime,s\rangle_{X^\prime}
		\end{align*}
		für alle $ x^\prime \in X^\prime $ gilt.
%		
		\item 
		Die Folge $ (x_n) $ heißt  \textit{schwach beschränkt summierbar}, falls für alle $ (b_n)  \in \ell^\infty $ ein $ s  \in X$ existiert, sodass
		\begin{align*}
			\lim\limits_{n \to \infty}
			\left\langle x^\prime,
			\sum \limits_{i = 1}^n b_i x_i
		\right\rangle_{X^\prime}
		= 
		\langle x^\prime,s\rangle_{X^\prime}
		\end{align*}
		für alle $ x^\prime \in X^\prime $ gilt.
%		
		\item 
		Die Folge $ (x_n) $ heißt  \textit{schwach teil-summierbar}, falls für eine beliebige Indexfolge $ (k_j) $ ein $ s  \in X$ existiert, sodass
		\begin{align*}
			\lim\limits_{n \to \infty}
			\left\langle
			x^\prime,
			\sum \limits_{j = 1}^n x_{k_j}
			\right\rangle_{X^\prime}
			= 
		\langle x^\prime,s\rangle_{X^\prime}
		\end{align*}
		für alle $ x^\prime \in X^\prime $ gilt.
	\end{enumerate}
	
	

\end{df}

%Im Endeffekt übertragen sich die starken Begriffe ohne Modifikation auf die schwachen Begriffe.
%Die schwache Teil-Summierbakeit bedeutet, dass die Partialsummenfolgen zu den Teilfolgen $ (x_{n_j}) $ schwach summierbar sind.
Mithilfe der schwachen Teil-Summierbarkeit lässt sich nun ein kompakter Operator definieren.
Dieser existiert dann auf natürliche Weise auch für unbedingt summierbare Folgen.


\begin{lem}\label{th:weak_subseries_implies_compact}
	Sei $ (x_n) $ schwach teil-summierbar in $ X $.
	Dann ist 
	\begin{align}\label{eq:compact_operator_charact_uncond_conv_1}
		\psi  :  X^\prime \to \ell^1, \
		x^\prime \mapsto (\langle x^\prime , x_n \rangle)
	\end{align}
	ein kompakter Operator.
\end{lem}

\begin{proof}
	Sei $ (x_n) $ schwach teil-summierbar.
	Wir treffen die zusätzliche Annahme, dass $ X $ separabel ist.
	%Dies ist ohne weiteres möglich. 
	Falls $ X $ nicht separabel sein sollte, betrachten wir den schwachen Abschluss von $ \spn \{x_n\}_{n \in \N} $.
	Nun beweisen wir, dass die Abbildung \eqref{eq:compact_operator_charact_uncond_conv_1} ein wohldefinierter linearer beschränkter Operator ist.
	Nach Voraussetzung existiert $ \mathrm{w}- \lim_{n \to \infty}
	\sum_{j =1}^n x_{k_j} $
	für eine beliebige Indexfolge $ (k_j) $.
%	Insbesondere existiert dann $ \sum_{j=1}^{\infty} \langle x^\prime, x_{k_j} \rangle $, 
	Damit ist die Skalarfolge $ (\langle x^\prime , x_n \rangle) $ unbedingt bzw. absolut summierbar.
	Also ist $ \psi $ wohldefiniert und die Linearität folgt durch Einsetzen.
	Der Satz vom abgeschlossenen Graphen liefert die Stetigkeit von $ \psi $.\\
	Nun bleibt zu zeigen, dass $ \psi $ kompakt ist.
	Sei $ (x^\prime_m) $ eine Folge in $ B_{X^\prime} $.
	Der Einheitsball $ B_{X^\prime} $ ist kompakt und metrisierbar in der schwach-$ \ast $ Topologie, da wir die Separabilität von $ X $ angenommen haben.
	Damit existiert eine schwach-$ \ast $ konvergente Teilfolge $ (x^\prime_{m_j}) $
	mit schwach-$ \ast $ Grenzwert $ x_0^\prime $.
	Falls $ \psi(x^\prime_{m_j}) \rightarrow \psi(x_0^\prime)$ gilt, haben wir unser Ziel erreicht. 
	Nach dem $ \ell^1 $-Satz von Schur \ref{th:schur_l1} ist dies äquivalent zu
	$ \psi(x^\prime_{m_j}) \rightharpoonup \psi(x_0^\prime)$.
	Da der Aufspann von $ Y := \{
	\chi_M \ | \ M \subset \N
	\} $ dicht in $ \ell^\infty $ liegt, reicht es wegen der Linearität des Dualitätsprodukts
	\begin{align*}
		\langle f, \psi x_0^\prime \rangle_{\ell^\infty} 
		=
		\lim \limits_{j \to  \infty} 
		\langle f, \psi x_{m_j} \rangle_{\ell^\infty} 
	\end{align*}
	für alle $ f \in Y $ zu zeigen.
	Mit der  schwachen Teil-Summierbarkeit von $ (x_n) $ gilt für $ f \in Y $:
	\begin{align*}
		\langle f , \psi x_0^\prime \rangle_{\ell^\infty} 
		&=
		\sum \limits_{n \in M} \langle x_0^\prime, x_n \rangle_{X^\prime }
		=
		\left\langle x_0^\prime,
		\sum \limits_{n \in M } x_n
		\right\rangle_{X^\prime }
		=
		\lim\limits_{j \to \infty}
		\left\langle x_{m_j}^\prime,
		\sum \limits_{n \in M } x_n
		\right\rangle_{X^\prime }\\
		&=
		\lim\limits_{j \to \infty}
		\sum \limits_{n \in M} \langle x_{m_j}^\prime, x_n \rangle_{X^\prime }
		=
		\langle f , \psi x_0^\prime \rangle_{\ell^\infty}. 
	\end{align*}
	Damit gilt $ \psi(x^\prime_{m_j}) \rightarrow \psi(x^\prime)$ und $ \psi $ ist kompakt.
\end{proof}

\begin{lem}\label{th:compact_implies_uncond}
	Sei $ (x_n) $ eine Folge in $ X $, sodass
	\begin{align*}
		\psi  :  X^\prime \to \ell^1, \
		x^\prime \mapsto (\langle x^\prime , x_n \rangle)
	\end{align*}
	ein kompakter Operator ist.
	Dann ist $ (x_n) $ unbedingt summierbar.
\end{lem}
\newpage
\begin{proof}
	Nach Voraussetzung ist $ \psi $ kompakt. Damit werden beschränkte Mengen von $ X^\prime $ auf relativ kompakte Mengen $ K $ in $ \ell^1 $ abgebildet.
	Diese lassen sich durch
	\begin{align*}
		\textcolor{black}{(a_n) \in K
			\ \Leftrightarrow \
			\forall_{\varepsilon  > 0}
			\exists_{n_\varepsilon \in \N}
			\forall_{(a_n) \in K} :	
			\sum \limits_{n > n_\varepsilon}
			| a_n |
			< \varepsilon}
	\end{align*}
	charakterisieren. Die relativ kompakten Mengen sind demnach die gleichmäßig verschwindenden Restterme.
	Insbesondere wissen wir, dass $ K :=  \psi(B_{X^\prime}) $ genau diese Struktur besitzt.
	Sei $  \varepsilon > 0 $ beliebig. Dann gilt für jede endliche Teilmenge $ M \subset \N $
	mit $ \min M > n_\varepsilon $:
	\begin{align*}
		\left\|
		\sum \limits_{n \in M } x_n 
		\right\|
		=
		\sup \limits_{x^\prime \in B_{X^\prime} }
		\left|\left\langle 
		x^\prime , \sum \limits_{n \in M } x_n 
		\right\rangle\right|
		\leq 
		\sup \limits_{x^\prime \in B_{X^\prime}}
		\sum \limits_{n \in M } | \langle x^\prime , x_n \rangle |
		=
		\sup \limits_{x^\prime \in B_{X^\prime}}
		\sum \limits_{n \in M } | \psi(x^\prime) | < \varepsilon.
	\end{align*}
Somit ist $ (x_n) $ ungeordnet summierbar und damit auch unbedingt summierbar.
\end{proof}


Die unbedingte Summierbarkeit impliziert die schwache Teilsummierbarkeit. 
Mit den Lemmas \ref{th:weak_subseries_implies_compact} und \ref{th:compact_implies_uncond} folgt aus der schwachen Teil-Summierbarkeit die unbedingte Summierbarkeit.
Insbesondere ist auch die Kompaktheit von $ \psi $
und die schwache Teil-Summierbarkeit äquivalent zur unbedingten Summierbarkeit.

\begin{genericthm}{Satz von Orlicz-Pettis}\label{th:orlicz_pettis}
	Sei $ (x_n) $ schwach teil-summierbar in $ X $.
	Dann ist $ (x_n) $ teil-summierbar.
\end{genericthm}

\begin{proof}
	Ergibt sich aus Lemma \ref{th:weak_subseries_implies_compact} und Lemma \ref{th:compact_implies_uncond}.
\end{proof}

%Effektiv haben wir mit den zwei vorangegangenen Aussagen den Satz von Orlicz-Pettis bereits bewiesen.
%Der Beweis ergibt sich durch unmittelbare Aneinanderreihung der Aussagen.

\begin{genericthm}{Charakterisierung der unbedingten Summierbarkeit}\label{th:equi_uncond_2}
	Sei $ X $ ein Banachraum und $ (x_n) $ eine Folge in $ X $.
	Dann sind äquivalent:
	\begin{enumerate}
		\item$  (x_n) $ ist unbedingt summierbar.
		\item $ (x_n) $ ist ungeordnet summierbar.
		\item $ (x_n) $ ist teil-summierbar.
		\item $ (x_n) $ ist perfekt summierbar.
		\item $ (b_nx_n) $ ist summierbar für alle $ (b_n) \in \ell^\infty $.
		\item $ (x_n) $ ist schwach teil-summierbar.
		\item $ (x_n) $ ist schwach perfekt summierbar.
		\item $ (b_nx_n) $ ist schwach summierbar für alle $ (b_n ) \in \ell^\infty $.
		\item $  \psi : X^\prime \to \ell^1, \ x^\prime \mapsto (\langle x^\prime , x_n \rangle_{X^\prime} )_n $ ist ein kompakter Operator.
		\item
		$ (b_n )  \mapsto \sum b_n x_n$ definiert einen kompakten Operator $ \ell^\infty \to X $.
		\item
		$ (b_n )  \mapsto \sum b_n x_n$ definiert einen kompakten Operator $ c_0 \to X $.
		\item
		$ (b_n )  \mapsto \sum b_n x_n$ definiert einen beschränkten Operator $ \ell^\infty \to X $.
	\end{enumerate}	
\end{genericthm}

Die schwache unbedingte Summierbarkeit fehlt in dieser Liste.
Der Grund ergibt sich aus nachfolgendem Gegenbeispiel. 
Sei $ X = c_0 $ mit $ X^\prime = \ell^1 $ und dem Dualitätsprodukt
\begin{align*}
	\langle a , x \rangle_{\ell^1}
	=
	\sum \limits_{ i = 1}^\infty a_i x_i
\end{align*}
für $ a  \in \ell^1 $ und $ x \in c_0 $.
Wir betrachten die Folge 
\begin{align*}
	x^{(n)} := 
	\begin{cases}
		e^{(1)}, &\ \textrm{falls } n = 1\\
		e^{(n)} - {e}^{(n-1)} , &\ \textrm{falls } n \geq 2
	\end{cases}.
\end{align*}
Hierbei gilt $ e^{(n)}_i = 1 $ für $ n= i $ und $ e^{(n)}_i = 0 $ sonst.
Dann ist $ x^{(n)} $ schwach unbedingt summierbar. 
%ist jedoch in $ X $ unsummierbar, 
Da $ \| x^{(n)} \|_{\ell^\infty} = 1$ für alle $ n \in \N $ gilt, ist $ x^{(n)} $ jedoch in $ X $ unsummierbar.



\begin{proof}
	Mit \textbf{\textit{\itshape\textrm{(viii)}}} $ \Rightarrow $ \textbf{\textit{\textrm{(vi)}}} $ \Rightarrow $ \textbf{\textit{\textrm{(iii)}}} folgt die Äquivalenz von \textbf{\textit{\itshape\textrm{(viii)}}} zur unbedingten Konvergenz. Hierbei haben wir den Satz \ref{th:orlicz_pettis} verwendet.
	\begin{description}
		\item[\textit{ \itshape\textrm{(i)}} $ \Leftrightarrow $ \textbf{\textit{\textrm{(ii)}}} $ \Leftrightarrow $ \textbf{\textit{\textrm{(iii)}}} $ \Leftrightarrow $ \textbf{\textit{\textrm{(iv)}}}:]
		Bewiesen in Satz \ref{th:equi_uncond_1}.
		
		\item[\textit{ \itshape\textrm{(i)}} $ \Leftrightarrow $ \textbf{\textit{\textrm{(v)}}}:] Bewiesen in Satz \ref{th:bounded_test}.
		
		\item[\textbf{\textit{\textrm{(i)}}} $ \Leftrightarrow $ \textit{ \itshape\textrm{(vi)}} $ \Leftrightarrow $ \textbf{\textit{\textrm{(ix)}}}:]
		Mithilfe von \ref{th:weak_subseries_implies_compact} und dem Satz von Orlicz-Pettis \ref{th:orlicz_pettis} bewiesen.
		
		\item[\textbf{\textit{\itshape\textrm{(v)}}} $ \Rightarrow $  \textbf{\textit{\textrm{(viii)}}}:] Folgt unmittelbar.
		
		\item[\textbf{\textit{\itshape\textrm{(viii)}}} $ \Rightarrow $ \textbf{\textit{\textrm{(vi)}}}:]
		Folgt durch entsprechende Wahl der $ (b_n) $.
		
		\item[\textit{ \itshape\textrm{(vi)}} $ \Rightarrow $ \textbf{\textit{\textrm{(vii)}}}:]
		Analog zu \textbf{\itshape\textrm{(iii)}} $ \Rightarrow $ \textbf{\textit{\textrm{(iv)}}} in dem Beweis zu Satz \ref{th:equi_uncond_1}.
		
		\item[\textcolor{black}{\textit{ \itshape\textrm{(vii)}} $ \Rightarrow $ \textbf{\textit{\textrm{(viii)}}}:}]
		Sei $ (x_n) $ schwach perfekt summierbar.
		Dann existiert der Grenzwert
		\begin{align*}
			\lim\limits_{n > m \to \infty}
			\left|\left\langle x^\prime , 
			\sum \limits_{i = m}^n  \alpha_i x_i
			\right\rangle \right|
			=
			\lim\limits_{n > m \to \infty}
			\left|
			\sum \limits_{i = m}^n
			\left\langle x^\prime , 
			 \alpha_i x_i
			\right\rangle
			\right|
			=0
		\end{align*}
		für alle $ \alpha_i = \pm 1 $ und $ x^\prime \in X^\prime $.
		Für jedes (reelle) $ (b_n) \in \ell^\infty $ existiert eine $ \pm 1 $-Folge $ (\alpha_n) $, sodass $ b_n = \alpha_n | b_n| $ gilt. Der komplexe Fall ist mit der Zerlegung in Real- und Imaginärteil analog.
		Sei $ x^\prime  \in X^\prime$ und $ (b_n) \in \ell^\infty $ beliebig.
		Dann gilt
		\begin{align*}
			\left| \left\langle x^\prime , 
			\sum \limits_{i = m+1}^n  b_i x_i
			\right\rangle \right|
			=
			\left| \left\langle x^\prime , 
			\sum \limits_{i = m}^n  |b_i| \alpha_i x_i
			\right\rangle \right|
			\leq 
			\| (b_n) \|_{\ell^\infty}
			\left|  \left\langle x^\prime , 
			\sum \limits_{i = m}^n   \alpha_i x_i
			\right\rangle \right|,
		\end{align*}
		womit das Cauchy-Kriterium \ref{th:chauchy_crit} die Summierbarkeit von $ (b_nx_n) $ liefert.
		
		
		
		\item[\textit{ \itshape\textrm{(ix)}} $ \Leftrightarrow $ \textbf{\textit{\textrm{(xi)}}}:]
		Wir betrachten den Operator
		\begin{align*}
			 \eta: c_0 \to X, \ (b_n) \mapsto \sum \limits_{n = 1}^\infty b_n x_n
		\end{align*}
		und den zugehörigen adjungierten Operator
		\begin{align*}
			\eta^\prime : X^\prime \to (c_0)^\prime, \
			x^\prime \mapsto  (b \mapsto x^\prime \circ \eta (b)).
		\end{align*}
		Es gilt $ (c_0)^\prime \cong \ell^1 $. 
		%mit der Dualität
		Für $ a=  (a_n)  \in \ell^1$ und $ b=  (b_n) \in c_0 $ ist das Dualitätsprodukt durch
		\begin{align*}
			\langle a, b\rangle_{\ell^1 }
			=
			\sum \limits_{n = 1}^\infty a_n b_n
		\end{align*}
		gegeben.
		%für $ a=  (a_n)  \in \ell^1$ und $ b=  (b_n) \in c_0 $. 
		Wegen
		\begin{align*}
			x^\prime \circ \eta (b)
			=
			x^\prime 
			\left(
			\sum \limits_{n=1}^\infty b_n x_n 
			\right)
			=
			\sum 
			\limits_{n=1}^\infty \underbrace{\langle x^\prime, x_n \rangle_{X^\prime } }_{=: a} b_n
		\end{align*}
		gilt $ \eta^\prime = \psi $.
		Nach dem Satz von Schauder ist $ \eta $ genau dann kompakt, wenn $ \psi $ kompakt ist.
		
		\item[\textit{ \itshape\textrm{(x)}} $ \Rightarrow $ \textbf{\textit{\textrm{(xi)}}}:]
		Folgt aus der Definition eines kompakten Operators.
		
		\item[\textit{ \itshape\textrm{(x)}} $ \Rightarrow $ \textbf{\textit{\textrm{(xii)}}}:]
		Alle kompakte Operatoren sind stetig.
		
		\item[\textit{ \itshape\textrm{(xii)}} $ \Rightarrow $ \textbf{\textit{\textrm{(v)}}}:]
		Die Stetigkeit impliziert für beliebige $ b = (b_n) \in \ell^\infty $ mit
		\begin{align*}
			\left\| 
			\sum\limits_{n \in \N} b_n x_n
			\right\|
			\leq C \| b\|_{\ell^\infty}
		\end{align*}
		die Summierbarkeit von $ (b_n x_n) $.
		
		\item[\textit{ \itshape\textrm{(ix)}} $ \Leftrightarrow $ \textbf{\textit{\textrm{(x)}}}:]
		Für diesen Schritt nehmen wir die Summierbarkeit von $ (b_nx_n) $ für beliebige $ (b_n) \in \ell^\infty  $ an.
		Dies ist möglich, da dies beide Seiten der Äquivalenz impliziert.
		Wir betrachten den adjungierten Operator zu $ \psi $:
		\begin{align*}
			\psi^\prime :
			\ell^\infty \to X^{\prime \prime},
			b \mapsto (x^\prime \mapsto \langle b, \psi(x^\prime) \rangle_{\ell^\infty} ).
		\end{align*}
		Es gilt
		\begin{align*}
			\langle b, \psi(x^\prime) \rangle_{\ell^\infty}
			= \sum \limits_{n \in \N} b_n \langle x^\prime , x_n \rangle_{X^\prime}
			=
			\left\langle x^\prime , \sum\limits_{n \in \N} b_n x_n \right\rangle_{X^\prime} 
			=
			\left\langle  J_X \left(\sum\limits_{n \in \N} b_n x_n \right), x^\prime \right\rangle_{X^{\prime \prime }}	
		\end{align*}
		für alle $ x^\prime \in X^\prime  $ und $ b = (b_n) \in \ell^\infty $.
		Da $ J_X$ eine Isometrie ist, liegt das Bild von $ \psi^\prime $ in $ X $.
		Damit lässt sich der adjungierte Operator auch durch
		\begin{align*}
			\psi^\prime :
			\ell^\infty \to X,
			(b_n) \mapsto \sum \limits_{n \in \N} b_n x_n
		\end{align*}
		beschreiben.
		Die Äquivalenz folgt dann mit dem Satz von Schauder.
	\end{description}
\end{proof}

\begin{lem}\label{th:unconditional_uniform}
	Sei $ X $ ein Banachraum und $ (x_n) $ unbedingt summierbar in $ X $.
	Dann gilt:
	Für alle $ \varepsilon > 0 $ existiert ein $ N_\varepsilon \in \N $, sodass
	\begin{align}
		\sup
		\left\{
		\left\|
		\sum \limits_{i=N_\varepsilon}^{N_\varepsilon+ m } \alpha_i x_i \right\| \ : \
		\alpha_i = \pm 1, \ m \in \N	
		\right\}
		< \varepsilon 
	\end{align}
gilt.
\end{lem}

\begin{proof}
	Angenommen die Aussage gilt nicht.
	%Unser Ziel ist es, einen Widerspruch zur unbedingten Summierbarkeit von $ (x_n) $ zu finden.
	Dann existiert ein $ \varepsilon > 0 $, sodass für alle $ N \in \N $
	\begin{align*}
		\sup\left\{
		\left\| \sum \limits_{i = N}^{N+m} \alpha_i x_i \right\| \ : \ m \in \N , \alpha_i = \pm 1
		\right\}
		=
		\sup \limits_{m \in \N}
		\max \limits_{\alpha_i = \pm 1}
		\left\| \sum \limits_{i = N}^{N+m} \alpha_i x_i \right\|
		\geq \varepsilon
	\end{align*}
	gilt.
	Insbesondere existieren streng monotone Folgen
	$ (m_j) $ und $ (r_j) $ mit \\
	$ m_1 < r_1 < m_2 < r_2 <... $, sodass
	\begin{align*}
		\max \limits_{\alpha_i = \pm 1}
		\left\| \sum \limits_{i = m_j}^{r_j} \alpha_i x_i \right\|
		\geq \varepsilon
	\end{align*}
	für alle $ j \in \N $ gilt. 
	Sei $ (\alpha^{(j)}_n) $ eine $ \pm 1 $-Folge, sodass das Maximum von $ \left\| \sum_{i = m_j}^{r_j} \alpha_i x_i \right\| $ angenommen wird.
	Da $ (\alpha_n^{(j)} x_n) $ nach Voraussetzung für alle $ j \in \N $ summierbar sein muss, erhalten wir durch hinreichend große Wahl von $ j $
	einen Widerspruch zum Cauchy-Kriterium \ref{th:chauchy_crit}.
	Demzufolge war die Annahme falsch.
%	Damit können wir die $ \pm 1 $-Folgen 
%	\begin{align*}
%		b^{(j)}_i
%		:=
%		\begin{cases}
%			\alpha_i^{(j)}
%		\end{cases}
%	\end{align*}
%	
%	Wir können nun $ \pm 1 $ Folgen definieren, sodass das Maximum auf
%	
%	
%	Wir wählen $ j \in \N $ hinreichend groß.
%	Es ist bekannt, dass das Maximum von $  $
%	
%	 definieren $ \beta_i = \pm 1 $ durch
%	\begin{align*}
%		\left\| \sum \limits_{i = m_j}^{r_j} \beta_i x_i \right\|
%		=
%		\max \limits_{\alpha_i = \pm 1}
%		\left\| \sum \limits_{i = m_j}^{r_j} \alpha_i x_i \right\|
%	\end{align*}
\end{proof}



\begin{genericthm}{Satz von Gelfand}\label{th:gelfand}
	Sei $ X $ ein Banachraum, $ (x_n) $ unbedingt summierbar in $ X $
	und $ K = \{ (\alpha_i)  :  \alpha_i = \pm 1 \} $.
	Dann liegt das Bild der Abbildung
	\begin{align*}
		s : K \to X, \ 
		\alpha \mapsto s(\alpha) =
		\sum \limits_{i = 1}^\infty \alpha_i x_i 
	\end{align*}
	kompakt in $ X $.
\end{genericthm}

\begin{proof}
	Wir versehen $ K $ mit der Topologie der koordinatenweisen Konvergenz.
	Die koordinatenweise Konvergenz $ \alpha^{(m)} \to  \alpha$ in $ K $ bedeutet, dass $ \alpha_i^{(m)} \to \alpha_i $
	für alle $ i \in \N $ und $ m \to \infty $ gilt.
	Der daraus entstandene Raum ist homöomorph zu dem kompakten Produktraum $  \{-1,1\}^\N $.
	Nun zeigen wir, dass $ s $ stetig ist .
	Sei $ \alpha^{(m)} \to \alpha $ für $ m \to \infty $ in $ K $ und $ \varepsilon > 0 $.
	Das vorangegangene Lemma \ref{th:unconditional_uniform} liefert die Existenz eines $ N_\varepsilon $, sodass
	\begin{align*}
		\left\| \sum \limits_{i = N_\varepsilon}^{\infty} \beta_i x_i \right\|
		\leq \varepsilon
	\end{align*}
	für beliebige $ (\beta_i) $ mit $ \beta_i = \pm 1 $ gilt. 
	Wegen der koordinatenweisen Konvergenz existiert ein $ m_0 \in \N $, sodass
	$ \alpha_i^{(m)} = \alpha_i $ für $ m\geq m_0 $ und $ 1 \leq i < N_\varepsilon $
	gilt. Damit erhalten wir
	\begin{align*}
		\| s(\alpha^{(m)}) - s(\alpha) \|
		=
		\left\|
		\sum \limits_{i = N_\varepsilon}^{\infty} \alpha^{(m)}_i x_i
		-
		\sum \limits_{i = N_\varepsilon}^{\infty} \alpha_i x_i
		\right\|
		\leq
		\left\|
		\sum \limits_{i = N_\varepsilon}^{\infty} \alpha^{(m)}_i x_i
		\right\|
		+\left\|
		\sum \limits_{i = N_\varepsilon}^{\infty} \alpha_i x_i
		\right\|
		< 2 \varepsilon
	\end{align*}
	für $ m \geq m_0 $.
	Damit gilt $s(\alpha^{(m)}) \to s(\alpha)  $ für $ m \to \infty $ und $ s $ ist stetig.
	Da das Bild eines kompakten Raumes unter einer stetigen Abbildung wieder kompakt ist, folgt die Aussage.
	
\end{proof}


\section{Der Satz von Dvoretzky-Rogers}\label{sc:dv_rg}
%\subsection{Der Satz von Dvoretzky-Rogers}

In diesem Abschnitt werden wir zeigen, dass im Allgemeinen aus unbedingter Konvergenz in einem unendlichdimensionalen Raum keine absolute Konvergenz folgt.
Wir richten uns nach dem von Diestel in \cite{Diestel1995} geführten Beweis.
Der Kern des Beweises ist das nachfolgende geometrische Lemma.




\begin{lem}\label{th:estimate_2n_dim_subspace}
	Sei $ E $ ein $ 2n $-dimensionaler normierter Raum.
	Dann existieren $ \{x_i\}_{i=1}^n \subset B_E $ mit $ \| x_i \| \geq\frac{1}{2} $, sodass
	\begin{align}
		\left\|
		\sum \limits_{i = 1}^n \lambda_i x_i 
		\right\|
		\leq
		\sqrt{
			\sum \limits_{i = 1}^n \lambda_i^2
		}
	\end{align}
	für alle $ \lambda = (\lambda_1,...,\lambda_n) \in \R^n $ gilt.
\end{lem}
M.I. und V.M. Kadets\cite{Kadets1997} verwenden das geometrische Lemma von Dvoretzky-Rogers\cite{DvoretzkyRogers1950}.
%Wie der Name schon verrät, wurde dies ursprünglich von Dvoretzky und Rogers 
%bewiesen. 
In deren Formulierung existieren zu einem $ n $-dimensionalen Raum $ E $ die $ n $ Vektoren $ \{x_i\}_{i=1}^n  \subset S_E $, sodass
\begin{align*}
	\left\|
	\sum \limits_{i = 1}^m \lambda_i x_i 
	\right\|
	\leq
	\left(
	1 + \sqrt{\frac{m(m-1)}{n}}
	\right)
	\sqrt{
		\sum \limits_{i = 1}^m \lambda_i^2
	}
\end{align*}
für alle $ 1 \leq m \leq  n $ und $ \lambda = (\lambda_1,...,\lambda_m) \in \R^m $ gilt.
Der Beweis hiervon ist jedoch technisch deutlich anspruchsvoller, weswegen wir den Beweis von Diestel verwenden.

\begin{proof}
	Sei $ w : V \to W $ eine lineare Abbildung mit $ \dim V = \dim W < \infty $.
	Dann ist die Spur $ \tr (w) $ invariant unter Basistransformationen.
	Für die Determinante gilt dies mit einer von der Transformation abhängigen Konstante. Dadurch können wir auf explizite Basen zunächst verzichten.\\
	Das erste Ziel ist es, einen Isomorphismus $ u : \ell^2_{2n}  \to E$ mit $ \| u \| = 1 $ und 
	\begin{align}\label{eq:proof_dvor_rog_trace_prop}
		| \tr(u^{-1} v) | \leq 2n \| v \|
	\end{align}
	für alle $ v \in \mathcal{L}(\ell^2_{2n}, E ) $ zu finden.
	Wir wählen ein beliebiges $ u $, welches 
	\begin{align*}
		\det(u)
		=
		\max \limits_{v \in \mathcal{L}(\ell^2_{2n}, E ) , \| v \| = 1  } | \det(v) |
	\end{align*}
	erfüllt. Dieses existiert, da die Einheitssphäre von $ \mathcal{L}(\ell^2_{2n} , E) $ kompakt und die Determinante stetig ist. Nun werden wir für dieses $ u $
	die Eigenschaft \eqref{eq:proof_dvor_rog_trace_prop} nachweisen.
	Sei $ \varepsilon \neq 0  $ und \\ $ v \in \mathcal{L}(\ell^2_{2n} , E) $ beliebig.
	Dann gilt
	\begin{align*}
		\left|\det \left(
		\frac{1}{\| u + \varepsilon v \|} ( u + \varepsilon v)
		\right)\right|
		=
		\frac{| \det(u + \varepsilon v ) |}{\| u + \varepsilon v \|^{2n}}
		\leq \det(u),
	\end{align*}
	womit 
	\begin{align*}
		|\det ( u + \varepsilon  v) |
		\leq \det(u) \| u + \varepsilon v \|^{2n}
		\leq  \det(u) (1 + \varepsilon \|v \|)^{2n}
	\end{align*}
	folgt.
	Die Invertierbarkeit von $ u $ und die Approximation von $ \det $ liefern
	\begin{align*}
		| \det (u + \varepsilon v ) |
		=
		\det(u) |\det(\id + \varepsilon u^{-1} v ) |
		=
		\det(u) | 1 + \varepsilon\cdot \tr(u^{-1} v) + \mathcal{O}(|\varepsilon|^2) |
	\end{align*}
	für $ \varepsilon \to 0 $.
	Die Kombination dieser Aussagen führt zu 
	\begin{align*}
		\left| 1 + \varepsilon \cdot\tr(u^{-1} v) + \mathcal{O}(|\varepsilon|^2) \right|
		\leq (1+ |\varepsilon| \| v \| )^{2n} = 1 + 2n |\varepsilon | \cdot \|v\| + \mathcal{O}(|\varepsilon|^2)
	\end{align*}
	für hinreichend kleine $ \varepsilon $. Wir verfeinern die Wahl von $ \varepsilon $, sodass $ \varepsilon \cdot\tr(u^{-1} v) = | \varepsilon \cdot\tr(u^{-1} v)| $ gilt.
	Damit folgt insbesondere
	\begin{align*}
		1 + | \varepsilon | | \tr ( u^{-1} v) | 
		\leq 
		1 + 2n | \varepsilon |  \| v \| + \mathcal{O}(|\varepsilon|^2)
		\ \Rightarrow \
		| \tr ( u^{-1} v) | 
		\leq 2n \| v \| + \mathcal{O}(|\varepsilon|^2)
	\end{align*}
	und mit $ \varepsilon \to 0 $ erhalten wir $ | \tr ( u^{-1} v) | 
	\leq 2n \| v \| $.\newpage
	Sei $ P $ eine beliebige orthogonale Projektion von $ \ell^2_{2n} $
	auf einen $ m $-dimensionaler Unterraum.
	Dann gilt mit der Abschätzung \eqref{eq:proof_dvor_rog_trace_prop}:
	\begin{align*}
		m = \tr(P)
		=\tr(u^{-1} u P)
		\leq 2n \|u P \|
		\ \Leftrightarrow \
		\| u P \| \geq \frac{m}{2n}.
	\end{align*}
	Nun haben wir die Werkzeuge um geeignete $ x_1,...,x_n $ in $ E $ zu konstruieren.
	Wegen \\ $ \| u \| = 1 $ existiert ein $ y_1 \in \ell^2_{2n} $ und $ \| u y_1 \| = 1 $.
	Sei $ P_1 $ die orthogonale Projektion von $ \ell^2_{2n} $ auf das orthogonale Komplement $ \spn\{y_1\}^\perp $.
	Damit gilt $ \|u P_1 \| \geq \frac{2n-1}{2n} $ und es existiert ein $ y_2 \in  \spn\{y_1\}^\perp $ mit $ \| y_1 \| = 1 $ und $ \| u y_2 \| = \| u P_1 y_2 \| \geq \frac{2n -1}{2n} $.
	Nun sei $ P_2 $ die orthogonale Projektion von $ \ell^2_{2n} $ auf $ \spn \{y_1,y_2\}^\perp $, womit $ \| u P_2 \| \geq \frac{2n-2}{2n} $ folgt.
	Somit existiert ein \\ 
	$ y_3 \in \spn \{y_1,y_2\}^\perp $ mit $ \| y_3 \| = 1 $ 
	und $ \| u y_3 \| =  \| u P_2 y_3 \| \geq \frac{2n-2}{2n} $.
	Dieses Argument führen wir analog  fort und erhalten die orthonormalen Vektoren $ y_1,...,y_n$ in $  \ell^2_{2n} $.
	Wir setzen $ x_i := u y_i $ für $ 1 \leq i \leq n $.
	Nach den Vorüberlegungen gilt:
	\begin{align*}
		\| x_i \| = \| u y_i \| \geq  \frac{2n -i +1}{2n} \geq \frac{1}{2}, \ 1 \leq i \leq n.
	\end{align*}
	Insgesamt liefert unsere Konstruktion
	\begin{align*}
		\left\|
		\sum \limits_{i=1}^n\lambda_i x_i
		\right\|
		= 
		\left\|
		u 
		\left(
		\sum \limits_{i=1}^n \lambda_i y_i
		\right)
		\right\| 
		\leq 
		\| u \| 
		\left\|
		\sum \limits_{i=1}^n
		\lambda_i y_i
		\right\|
		= 
		\left(\sum \limits_{i=1}^n |\lambda_i |^2 \right)^\frac{1}{2}
	\end{align*}
	für ein beliebiges $ \lambda \in \R^n $.
\end{proof}



\begin{genericthm}{Satz von Dvoretzky-Rogers(1950)}\label{th:dvoretzky_rogers}
	Sei $ X $ ein unendlichdimensionaler Banachraum und $ (\lambda_n) \in \ell^2 $.
	Dann existiert eine unbedingt summierbare Folge $ (x_n)  $ in $ X $, welche $  \| x_n \| = | \lambda_n | $ erfüllt.
\end{genericthm}
Die Wahl $ \lambda_n = \frac{1}{n} $ liefert unmittelbar, dass im Allgemeinen aus unbedingter Konvergenz keine absolute Konvergenz folgt.

\begin{proof}
	Sei $ (\lambda_n) \in \ell^2 $ beliebig, aber fest. Dann existiert eine Indexfolge $ (n_j) $, sodass
	\begin{align*}
		\sum 
		\limits_{i = n_j}^\infty | \lambda_i |^2 \leq \frac{1}{2^{2j}}
	\end{align*}
	gilt.
	Für $ 1 \leq n < n_1 $ wählen wir beliebige $ x_n \in X$ mit $ \|x_n\| = |\lambda_{n} |$.
	Wir setzen $ k_j := n_{j+1} - n_j $ für $ j \in \N $ und mit $ \dim X = \infty $ lässt sich ein beliebiger $ 2 k_j $-dimensionaler Unterraum $ E_j $ konstruieren.
	Mit dem Lemma \ref{th:estimate_2n_dim_subspace} existieren $ \{y_i \}_{i = n_j}^{n_{j+1} - 1} \subset B_{E_j} \subset B_X $ mit $ \| y_i \|_X \geq \frac{1}{2} $, sodass
	\begin{align*}
		\left\|
		\sum \limits_{i = n_j}^N \mu_i y_i
		\right\|
		\leq 
		\left(
		\sum \limits_{i = n_j}^N
		|\mu_i|^2
		\right)^\frac{1}{2}
	\end{align*}
	für $ n_j \leq N < n_{j+1} $ und $ \mu_{i} \in \R $ mit $ n_j \leq i  <n_{j+1} $ gilt.
	Die Abschätzung für $ N < n_{j+1} - 1 $ folgt durch Nullsetzen der nachfolgenden Koeffizienten.
	Insgesamt erhalten wir eine Folge $ (y_n) $ in $ B_X $ mit $ \| y_n \| \geq \frac{1}{2} $, sodass
	\begin{align*}
		\left\|
		\sum \limits_{i = n_j}^N \mu_i y_i
		\right\|
		\leq 
		\left(
		\sum \limits_{i = n_j}^N
		|\mu_i|^2
		\right)^\frac{1}{2}
	\end{align*}
	für eine beliebige Zahlenfolge $ (\mu_n) $ und $ n_j \leq N < n_{j+1} $ gilt.
	Sei $ (\alpha_n) $ mit $ \alpha_n = \pm 1 $ beliebig und 
	\begin{align*}
		x_n := \frac{\lambda_n y_n}{\|y_n\|}
	\end{align*}
	für $ n \geq n_1 $. Dann gilt $ \| x_n \| = | \lambda_n | $ für alle $ n \in \N $ und es folgt 
	\begin{align*}
		\left\|
		\sum \limits_{i = n_j}^N \alpha_i x_i
		\right\|
		=
		\left\|
		\sum \limits_{i = n_j}^N  \frac{\alpha_i \lambda_i y_i}{\|y_i\|}
		\right\|
		\leq
		\left(
		\sum \limits_{i = n_j}^N
		\left|\frac{\lambda_i }{\|y_i \|}\right|^2
		\right)^\frac{1}{2}
		\leq  
		2\left(
		\sum \limits_{i = n_j}^N
		\left|\lambda_i \right|^2
		\right)^\frac{1}{2}
		\leq \frac{1}{2^{j-1}}
	\end{align*}
	für $ n_j \leq N < n_{j+1} $.
	Damit ist $ (\alpha_n x_n) $ ein Cauchyfolge und $ (x_n) $ ist perfekt summierbar.
\end{proof}

\section{Der Satz von Orlicz}\label{sc:lemma_of_orlicz}
%\subsection{Der Satz von Orlicz}
Nach dem Satz von Dvoretzky-Rogers \ref{th:dvoretzky_rogers} ist uns bekannt, dass in unendlichdimensionalen Räumen immer eine unbedingt summierbare Folge existiert, welche nicht absolut summierbar ist.
%Eine Idee um die Äquivalenz von unbedingter und absoluter Konvergenz zu retten, ist die Norm der Folgenglieder geeignet zu potenzieren.
%Dies führt zu dem Begriff der $ p $-absoluten Konvergenz. 
In diesem Abschnitt erarbeiten wir Bedingungen, um aus unbedingter Konvergenz die $ p $-absolute Konvergenz zu schließen.
Diese Bedingungen verwenden wir, um zu zeigen, welche Form der $ p $-absoluten Konvergenz aus der unbedingten Konvergenz in den $ \L^p $-Räumen folgt.

\begin{df}
	Ein normierter Raum $ X $ heißt \textit{Orliczraum} mit Exponent $ p $,
	falls aus der unbedingten Summierbarkeit von $ (x_n) $ die 
	$ p $-absolute Summierbarkeit folgt.
	Wir sagen auch $ X $ ist ein Orliczraum der Ordnung $ p $.
\end{df}


\begin{df}
	Sei $ X $ ein normierter Raum. $ X $ hat den $ M $-\textit{Kotyp} $ p $ mit der Konstante $ \gamma > 0 $, falls
	\begin{align}\label{eq:m_kotyp_ineq_1}
		\max \limits_{\alpha_i = \pm 1}
		\left\|
		\sum \alpha_i x_i
		\right\|
		\geq
		\gamma
		\left(\sum \| x_i\|^p\right)^\frac{1}{p}
	\end{align}
	für alle $ \{x_i\}_{i=1}^n \subset X $ gilt.
	Wir sagen auch, dass $ X $ vom \textit{$ M $-Kotyp $ p $} ist.
\end{df}

\begin{sz}\label{th:orlicz_equi_kotyp}
	Die folgenden Aussagen sind äquivalent:
	\begin{enumerate}
		\item 
		$ X $ ist ein Orliczraum mit Exponent $ p $.
		\item 
		Der Raum $ X $ hat den $ M $-Kotyp $ p $.
	\end{enumerate}
\end{sz}

\begin{proof}
	\begin{description}
		\item[\textit{ \itshape\textrm{(i)}} $ \Rightarrow $ \textbf{\textit{\textrm{(ii)}}}:]
		Wir nehmen an, dass $ X $ nicht vom $ M $-Kotyp $ p $ ist.
		Für alle $ \gamma > 0 $ existiert dann eine endliche Menge $ \{x_i\} \subset X $, sodass
		\begin{align*}
			\max \limits_{\alpha_i = \pm 1}
			\left\|
			\sum \alpha_i x_i
			\right\|
			<
			\gamma
			\left(\sum \| x_i\|^p\right)^\frac{1}{p}
		\end{align*}
		gilt. Diese Tatsache verwenden wir, um eine Folge endlicher Teilmengen\\
		$ \{ x_i\}_{i = n_k + 1}^{n_{k+1}} \subset X $ mit
		\begin{align*}
			\max \limits_{\alpha_i = \pm 1}
			\left\|
			\sum \limits_{i = n_k + 1}^{n_{k+1}} \alpha_i x_i
			\right\| 
			\leq \frac{1}{2^k} \
			\textrm{und} \
			\sum \limits_{i = n_k +1}^{n_{k+1}} \| x_i \|^p \geq 1 
		\end{align*}
		zu konstruieren. 
		Dies liefert nach dem Satz \ref{th:equi_uncond_2} eine unbedingt summierbare Folge. 
		Mit der zweiten Eigenschaft erhalten wir einen Widerspruch dazu, dass $ X $ ein Orliczraum ist.
		Damit war unsere Annahme falsch.
		Um den Beweis vollständig abzuschließen geben wir die Konstruktion der endlichen Teilmengen an.
		Sei $ k \in \N $ beliebig.
		Für $ \gamma > 2^{-k} $ existiert eine endliche Teilmenge $ \{x_i\}_{i=1}^n $, sodass
		\begin{align*}
			\max \limits_{\alpha_i = \pm 1}
			\left\|
			\sum \limits_{i = 1}^n \alpha_i x_i
			\right\|
			<
			\frac{1}{2^k}
			\underbrace{\left(\sum \limits_{i = 1}^n \| x_i\|^p\right)^\frac{1}{p}}_{=: C}
			\ \Leftrightarrow \
			\max \limits_{\alpha_i = \pm 1}
			\left\|
			\sum \limits_{i = 1}^n \alpha_i \underbrace{\frac{x_i}{C}}_{=: \tilde{x}_i}
			\right\|
			<
			\frac{1}{2^k}
		\end{align*}
		gilt. 
		Desweiteren erhalten wir mit
		\begin{align*}
		\left(\sum \limits_{i = 1}^n \| \tilde{x}_i \|^p\right)^\frac{1}{p} = 1
		\end{align*}
		die zweite gewünschte Eigenschaft.
		
		\item[\textit{ \itshape\textrm{(ii)}} $ \Rightarrow $ \textbf{\textit{\textrm{(i)}}}:]
		Für die Rückrichtung nehmen wir an, dass $ X $ kein Orliczraum der Ordnung $ p $ ist.
		Damit existiert eine unbedingt summierbare Folge $ (x_n) $, für welche $ (\| x_n \|^p) $ nicht summierbar ist. 
		Nach dem Satz von Gelfand \ref{th:gelfand} ist die Menge 
		\begin{align*}
			\left\{\sum \limits_{i = 1}^\infty \alpha_i x_i\ | \ \alpha_i = \pm 1\right\}
		\end{align*}
		kompakt. Somit auch beschränkt und abgeschlossen.
		Also sind die Summen $ \sum_{i = m+1}^n \alpha_i x_i $ gleichmäßig beschränkt, womit
		\begin{align*}
			\max 
			\left\{
			\left\|
			\sum \limits_{i = m +1 }^n \alpha_i x_i
			\right\| 
			\ : \
			\alpha_i = \pm 1 , m,n\in  \N
			\right\} 
			\leq C < \infty
		\end{align*}
		gilt. Da $ (\| x_n \|^p) $ nicht summierbar ist, erhalten wir die Indexfolgen $ (n_k) $ und $ (m_k) $ mit
		\begin{align*}
			\sum \limits_{i = m_k + 1}^{n_k} \| x_i \|^p \to \infty
		\end{align*}
		und $ 1 < m_1 < n_1 < m_2< n_2<...$.
		Damit ist die Ungleichung \eqref{eq:m_kotyp_ineq_1} nicht erfüllt und $ X $ ist nicht vom $ M $-Kotyp $ p $.
		%Dies ist ein Widerspruch und wir sind fertig.
		
	\end{description}
\end{proof}

%Damit sind die Begriffe des Orliczraums und des $ M $-Kotyps identisch. 
Diese Äquivalenz verwenden wir in dem Beweis des nächsten Satzes.


\begin{genericthm}{Satz von Orlicz(1930)}
	Sei $ (x_n) $ unbedingt summierbar in $ \L^p(\Omega,\mu) $.
	Dann gilt:
	\begin{enumerate}
		\item Für $ 1 \leq p \leq 2 $ ist $  (\| x_n \|^2) $ summierbar.
		\item Für $ 2 \leq p < \infty $ ist $  (\| x_n \|^p) $ summierbar.
	\end{enumerate}
\end{genericthm}

\begin{proof}

	\begin{enumerate}
		\item Seien $ 1 \leq p \leq 2 $ und  $ \{f_i\}_{i = 1}^n \subset \L^p(\Omega,\mu)$ eine endliche Teilmenge.
		Wir werden nachweisen, dass $ \L^p $ vom $ M $-Kotyp $ 2 $ ist.
		Mithilfe der Chintschin-Ungleichung \ref{th:khinchin_ineq_1} erhalten wir:
		\begin{align*}
			\max \limits_{\alpha_i = \pm 1}
			\left\|
			\sum 
			\limits_{i = 1}^n \alpha_i f_i
			\right\|^p
			&=
			\max \limits_{\alpha_i = \pm 1}
			\int \limits_\Omega
			\left|
			\sum 
			\limits_{i = 1}^n \alpha_i f_i(t)
			\right|^p
			\dx{\mu}
			\geq 
			\frac{1}{2^n}
			\sum \limits_{\alpha_i = \pm 1 }
			\int \limits_\Omega
			\left|
			\sum 
			\limits_{i = 1}^n \alpha_i f_i(t)
			\right|^p
			\dx{\mu}\\
			&=
			\mathbb{E}
			\left(
			\int \limits_\Omega
			\left|
			\sum 
			\limits_{i = 1}^n r_i f_i(t)
			\right|^p
			\dx{\mu}
			\right)
			\geq 
			(a_p)^p
			\int \limits_\Omega
			\left(
			\sum 
			\limits_{i = 1}^n |f_i(t)|^2
			\right)^\frac{p}{2}
			\dx{\mu}.
		\end{align*}
		Hierbei sind die $ r_i $ die bekannten Rademachervariablen.
		Der Summand in dem letzten Integral kann als Norm einer vektorwertigen Funktion aufgefasst werden. Wir setzen
		\begin{align*}
			g(t) := \begin{pmatrix}
				|f_1(t)|^p ,& ...,& |f_n(t)|^p
			\end{pmatrix}
		\end{align*}
		und erhalten in dem endlichdimensionalen Raum $ \ell^{\nicefrac{2}{p}}_{(n)} $:
		\begin{align*}
			\left(
			\sum 
			\limits_{i = 1}^n |f_i(t)|^2
			\right)^\frac{p}{2}
			=
			\left(
			\sum 
			\limits_{i = 1}^n \left(|f_i(t)|^p\right)^\frac{2}{p}
			\right)^\frac{p}{2}
			=
			\| g(t) \|_{\ell^{\nicefrac{2}{p}}_{(n)}}.
		\end{align*}
		 Mithilfe der Dreiecksungleichung folgt dann:
		\begin{align*}
			\int \limits_\Omega
			\left(
			\sum 
			\limits_{i = 1}^n |f_i(t)|^2
			\right)^\frac{p}{2}
			\dx{\mu}
			&=
			\int \limits_\Omega
			\| g(t) \|_{\ell^{\nicefrac{2}{p}}_{(n)}}
			\dx{\mu}
			\geq 
			\left\|
			\int \limits_\Omega
			g(t) 
			\dx{\mu}
			\right\|_{\ell^{\nicefrac{2}{p}}_{(n)}}
			=
			\left(
			\sum \limits_{i = 1}^n
			\left(
			\int \limits_\Omega
			|f_i(t)|^p 
			\dx{\mu}
			\right)^\frac{2}{p}
			\right)^\frac{p}{2}\\
			&=
			\left(
			\sum_{i  = 1 }^n
			\| f_i \|_{\L^p}^2
			\right)^\frac{p}{2}.
		\end{align*}
		Durch unsere Überlegungen erhalten wir
		\begin{align*}
			\max \limits_{\alpha_i = \pm 1}
			\left\|
			\sum 
			\limits_{i = 1}^n \alpha_i f_i
			\right\|
			\geq 
			a_p 
			\left(
			\sum_{i  = 1 }^n
			\| f_i \|_{\L^p}^2
			\right)^\frac{1}{2}.
		\end{align*}
		Damit ist $ \L^p $ für $ 1 \leq p \leq 2 $ ein Orliczraum mit Exponent $ 2 $.
		
		\item 
		Seien $ 2 \leq p < \infty $ und $ \{f_i\}_{i = 1}^n \subset \L^p(\Omega,\mu)$ eine endliche Teilmenge.
		Analog zu dem ersten Teil des Beweises erhalten wir mit der Chintschin-Ungleichung \ref{th:khinchin_ineq_1}:
		\begin{align*}
			\max \limits_{\alpha_i = \pm 1 }
			\left\|
			\sum \limits_{i = 1}^n \alpha_i f_i
			\right\|^p
			\geq 
			\int \limits_\Omega
			\left(
			\sum \limits_{i = 1}^n
			|f_i(t)|^2
			\right)^\frac{p}{2}
			\dx{\mu}.
		\end{align*}
		Da die $ \ell^{p}_{(n)} $-Norm bezüglich $ p $ monoton fallend ist, folgt:	
		%Mit der fallenden Monotonie der $ \ell^{p}_{(n)} $-Norm bezüglich $ p $ folgt
		\begin{align*}	
			\int \limits_\Omega
			\left(
			\sum \limits_{i = 1}^n
			|f_i(t)|^2
			\right)^\frac{p}{2}
			\dx{\mu}
			\geq 
			\sum \limits_{i = 1}^n
			\int \limits_\Omega
			|f_i(t)|^p
			\dx{\mu}
			= 
			\sum \limits_{i = 1}^n \|f_i \|_{\L^p}^p.
		\end{align*}
		Zusammengesetzt ergibt sich mit
		\begin{align*}
			\max \limits_{\alpha_i = \pm 1 }
			\left\|
			\sum \limits_{i = 1}^n \alpha_i f_i
			\right\|
			\geq 
			\left(
			\sum \limits_{i = 1}^n \|f_i \|_{\L^p}^p
			\right)^\frac{1}{p}
		\end{align*}
		die gewünschte Aussage.
	\end{enumerate}
\end{proof}

\section{Absolut summierbare Operatoren}\label{sc:abs_summing}
%\subsection{Absolut summierbare Operatoren}
In diesem Abschnitt beschäftigen wir uns mit Operatoren, bei welchen aus unbedingter Summierbarkeit im Definitionsraum eine Form von absoluter Summierbarkeit im Bildraum folgt.
Wir beginnen mit der in \cite{Kadets1997} gegebenen Definition von absoluter Summierbarkeit eines Operators.
\begin{df}
	Seien $ X $ und $ Y $ Banachräume.
	Ein Operator $ T : X \to Y $ heißt \textit{absolut summierbar}, wenn eine Konstante $ K > 0$ existiert, sodass
	\begin{align}\label{eq:abs_sum_cond}
		\sum \| T x_i \|_Y \leq K \max \limits_{\alpha_i = \pm 1} \left\| \sum \alpha_i x_i \right\|_X
	\end{align}
	für jede endliche Teilmenge $ \{ x_i\} \subset X $ gilt.
\end{df}

Der Beweis des nachfolgenden Lemmas liefert uns eine andere Formulierung für die Ungleichung \eqref{eq:abs_sum_cond}.
Dies ist die übliche Definition(\cite{Albaic2006}, \cite{Diestel1995}) eines absolut summierbaren Operators.
Die Abschätzung mit dem Maximum über die Vorzeichen hat den Vorteil, dass wir unmittelbar die Äquivalenz von unbedingter und perfekter Summierbarkeit anwenden können. 

\begin{lem}\label{th:1_sum_equality}
	Sei $ \{x_i\}_{i=1}^n  $ eine endliche Teilmenge eines normierten Raumes $ X $.
	Dann gilt:
	\begin{align*}
		\sup \limits_{x^\prime \in B_{X^\prime}} 
		\sum \limits_{i = 1}^n
		|x^\prime(x_i)|
		=
		\max \limits_{\alpha_i = \pm 1}
		\left\|
		\sum \limits_{i=1}^n \alpha_i x_i
		\right\|_X.
	\end{align*}
\end{lem}

\begin{proof}
	Sei $  \{x_i\}_{i =1}^n \subset X$ beliebig.
	Dann gilt
	\begin{align*}
		\sup \limits_{x^\prime \in B_{X^\prime}} 
		\sum \limits_{i = 1}^n
		|x^\prime(x_i)|
		=
		\sup \limits_{x^\prime \in B_{X^\prime}} 
		\sum \limits_{i = 1}^n
		| \alpha_i x^\prime(x_i)|
		\geq 
		\sup \limits_{x^\prime \in B_{X^\prime}} 
		\left|
		x^\prime \left(
		\sum \limits_{i = 1}^n
		\alpha_i x_i
		\right)
		\right|
		=
		\left\| 
		\sum \limits_{i = 1}^n
		\alpha_i x_i
		\right\|_X
	\end{align*}
	für alle $ \alpha_i = \pm 1 $. Insbesondere gilt dann
	\begin{align*}
		\max \limits_{\alpha_i = \pm 1}
		\left\|
		\sum \limits_{i=1}^n \alpha_i x_i
		\right\|_X
		\leq 
		\sup \limits_{x^\prime \in B_{X^\prime}} 
		\sum \limits_{i = 1}^n
		|x^\prime(x_i)|.
	\end{align*} 
	Wenn nun eine Kombination der $ \tilde{\alpha}_i $ existiert, für welche die Gleichheit gilt, sind wir fertig.
	Hierfür verwenden wir, dass zu jedem $ x^\prime \in  B_{X^\prime} $ eine Kombination $ \alpha = \pm 1 $ existiert, sodass 
	\begin{align*}
		\sum \limits_{ i = 1}^n | x^\prime(x_i) |
		=
		\sum \limits_{ i = 1}^n \alpha _i x^\prime(x_i). 
	\end{align*}
	gilt. Damit existieren $ \tilde{\alpha}_i = \pm 1 $, wofür
	\begin{align*}
		\sup \limits_{x^\prime \in B_{X^\prime}} 
		\sum \limits_{i = 1}^n
		|x^\prime(x_i)|
		=
		\left\|
		\sum \limits_{i=1}^n \tilde{\alpha}_i x_i
		\right\|_X
	\end{align*}
	folgt.
\end{proof}
\newpage
\begin{genericdf}{Bemerkung}
	Wir nennen den kleinsten möglichen Wert der Konstanten $ K $  die \textit{absolut summierende Norm des Operators $ T $} und bezeichnen diese mit $ \pi_1(T) $.
	Falls $ T $ nicht absolut summierbar ist, setzen wir $ \pi_1(T) = \infty $.
	Die absolut summierende Norm von $ T $ lässt sich durch
	\begin{align}\label{eq:abs_sum_norm}
		\pi_1(T)
		=
		\sup \limits_{\{x_i\}_{i=1}^n \subset X, n \in \N}
		\sum \limits_{i = 1}^n \| T x_i \|_Y \cdot \frac{1}{\max \limits_{\alpha_i = \pm 1} \left\| \sum \limits_{i=1}^n \alpha_i x_i \right\|_X}
	\end{align}
	beschreiben.
\end{genericdf}

Mit dem nächsten Satz erhalten wir die gewünschte Eigenschaft, dass unbedingte Summierbarkeit auf absolute Summierbarkeit abgebildet wird.


\begin{sz}\label{th:abs_sum_equiv}
	Ein linearer Operator $ T : X \to Y $ ist genau dann \textit{absolut summierbar}, falls dieser eine unbedingt summierbare Folge $ (x_n) $ in $ X $ auf eine absolut summierbare Folge $ (Tx_n) $ in $ Y $ abbildet.
\end{sz}

\begin{proof}
	\begin{description}
		\item[\glqq$ \Rightarrow $\grqq:]
		Sei $ \varepsilon > 0  $ beliebig.
		Dann gilt 
		\begin{align*}
			\sum \limits_{i = m}^n \|Tx_i\|_Y
			\leq K \max \limits_{\alpha_i = \pm 1} 
			\left\|
			\sum \limits_{i =  m}^n \alpha_i x_i
			\right\|_X \leq \varepsilon
		\end{align*} 
		für geeignete $ n >m \geq N_\varepsilon $.
		
		\item[\glqq$ \Leftarrow $\grqq:]
		Sei $ T : X \to Y $ ein linearer Operator mit
		\begin{align*}
			\textrm{$ (x_n) $ unbedingt summierbar} 
			\ \Rightarrow \
			\textrm{$ (Tx_n) $ absolut summierbar}. 
		\end{align*}
		Damit ist $ T(X) $ ein Orliczraum mit Exponent $ 1 $. Also exisitiert ein $ \gamma > 0 $, sodass für jede endliche Teilmenge $ \{x_i\} \subset X $
		\begin{align*}
			\sum \|T x_i \|_Y
			\leq 
			\frac{1}{\gamma} 
			\max \limits_{\alpha_i = \pm 1}
			\left\|
			\sum \alpha_i T x_i
			\right\|_Y
			= 
			\frac{1}{\gamma} 
			\max \limits_{\alpha_i = \pm 1}
			\left\|
			T
			\left(
			\sum \alpha_i  x_i
			\right)
			\right\|_Y
			\leq 
			\frac{C}{\gamma}
			\max \limits_{\alpha_i = \pm 1}
			\left\|
			\sum \alpha_i  x_i
			\right\|_X
		\end{align*}
		gilt.
		Damit ist $ T $ absolut summierbar.
	\end{description}
\end{proof}

Wir können das Konzept des absolut summierbaren Operators erweitern.
Jedoch werden wir diese Erweiterung nicht nach \cite{Kadets1997} definieren, sondern die üblichere Definition aus \cite{Diestel1995} übernehmen.
\newpage
\begin{df}
	Ein Operator $ T : X \to Y $ heißt \textit{$ p $-absolut summierbar}, wenn eine Konstante $ K_p > 0 $ existiert, sodass
	\begin{align}\label{eq:abs_sum_cond_p}
		\left(\sum \| T x_i \|_Y^p\right)^\frac{1}{p} 
		\leq K_p 
		\sup \limits_{ x^\prime \in B_{X^\prime}} 
		\left(
		\sum 
		|x^\prime(x_i)|^p
		\right)^\frac{1}{p}
	\end{align}
	für jede endliche Teilmenge $ \{ x_i\} \subset X $ gilt.
\end{df}

Aufgrund der Monotonie der $ \ell_{(n)}^p $-Norm bezüglich $ p $ und der Normäquivalenz in endlichdimensionalen Räumen folgt mit dem Lemma \ref{th:1_sum_equality} für endliche Teilmengen $ \{x_i\} \subset X $:
\begin{align*}
	\left(\sum \| T x_i \|_Y^p \right)^\frac{1}{p}
	&\leq
	K_p 
	\sup \limits_{ x^\prime \in B_{X^\prime}} 
	\left(
	\sum 
	|x^\prime(x_i)|^p
	\right)^\frac{1}{p}
	\leq
	K_p 
	\sup \limits_{ x^\prime \in B_{X^\prime}} 
	\sum 
	|x^\prime(x_i)|\\
	&=
	K_p \max \limits_{\alpha_i = \pm 1} \left\| \sum \alpha_i x_i \right\|_X
	\leq
	K_p C
		\sup \limits_{ x^\prime \in B_{X^\prime}} 
	\left(
	\sum 
	|x^\prime(x_i)|^p
	\right)^\frac{1}{p}.
\end{align*}

Wie für $ p = 1 $ erhalten wir auch bei $ p $-absolut summierbaren Operatoren die gewünschte Eigenschaft.
Mit der obigen Ungleichung verläuft der Beweis des nachfolgenden Satzes analog zu dem von Satz \ref{th:abs_sum_equiv}.
Insbesondere können wir die $ p $-absolut summierbaren Operatoren auch durch
\begin{align*}
	\left(\sum \| T x_i \|_{Y}^p\right)^\frac{1}{p} 
	\leq K_p 
	\max \limits_{\alpha_i = \pm 1}
	\left\|
	\sum \alpha_i  x_i
	\right\|_{Y}
\end{align*}
für endliche Teilmengen $ \{x_i\} \subset X $ definieren.
Der nachfolgende Satz folgt dann analog zu $ p=1 $.
\begin{sz}
	Ein linearer Operator $ T : X \to Y $ ist genau dann $ p $-absolut summierbar, falls dieser eine unbedingt summierbare Folge $ (x_n) $ in $ X $ auf eine $ p $-absolut summierbare Folge $ (Tx_n) $ in $ Y $ abbildet.
\end{sz}
Ebenso definieren wir die \textit{$ p $-absolut summierende} Norm durch
\begin{align*}
\pi_p(T)
=
\sup \limits_{\{x_i\}_{i=1}^n \subset X, n \in \N}
\left(\sum \| T x_i \|_{Y}^p\right)^\frac{1}{p}  \cdot \frac{1}{\sup \limits_{ x^\prime \in B_{X^\prime}} 
	\left(
	\sum 
	|x^\prime(x_i)|^p
	\right)^\frac{1}{p}}.
\end{align*}
Mit $ \Pi_p(X,Y) $ bezeichnen wir die Menge der $ p $-absolut summierenden Operatoren.
$ \Pi_p(X,Y) $ ist ein Unterraum von $ \mathcal{L}(X,Y) $ mit der Norm $ \pi_p $ und
\begin{align*}
	\| T \| \leq \pi_p(T)
\end{align*}
für alle $ T \in \Pi_p(X,Y) $. Genauere Informationen hierzu finden sich in \cite[Chapter 2 -4]{Diestel1995}.



\begin{sz}
	Die kanonische Einbettung von $ \L^\infty([0,1]) $ nach $ \L^p([0,1]) $ mit $ 1 \leq p < \infty $ ist ein $ p $-absolut summierbarer Operator.
\end{sz}

\begin{proof}
%	Es ist 
%	\begin{align*}
%		\left(
%		\sum \| f_i \|_{\L^p}^p
%		\right)^\frac{1}{p}
%		\leq 
%		K_p
%		\max\limits_{\alpha_i = \pm 1}
%		\left\|
%		\sum \alpha_i f_i
%		\right\|_{\L^\infty}
%	\end{align*}
%	für eine endliche Teilmenge $ \{f_i\} \subset \L^\infty([0,1]) $ zu zeigen.
	Sei $ \{f_i\} \subset \L^\infty([0,1]) $ eine beliebige endliche Teilmenge.
	Dann gilt:
	\begin{align*}
		\max\limits_{\alpha_i = \pm 1}
		\left\|
		\sum \alpha_i f_i
		\right\|_{\L^\infty}
		&=
		\max\limits_{\alpha_i = \pm 1}
		\sup \limits_{t \in [0,1]}
		\left|
		\sum \alpha_i f_i(t)
		\right|
		=
		\sup \limits_{t \in [0,1]}
		\max\limits_{\alpha_i = \pm 1}
		\left|
		\sum \alpha_i f_i(t)
		\right|
		=
		\sup \limits_{t \in [0,1]}
		\sum | f_i(t) |\\
		&\geq 
		\sup \limits_{t \in [0,1]}
		\left(
		\sum |f_i(t)|^p 
		\right)^\frac{1}{p}
		\geq 
		\left(
		\int \limits_0^1
		\sum |f_i(t)|^p \dx{t}
		\right)^\frac{1}{p}\\
		&=
		\left(
		\sum \|f_i\|^p_{\L^p} 
		\right)^\frac{1}{p}.
	\end{align*}
\end{proof}

Unser nächstes Ziel ist es, den Satz von Grothendieck zu beweisen.
Dieser besagt, dass jeder beschränkte Operator von $ \ell^1 $ nach $ \ell^2 $ absolut summierbar ist.
Die absolut summierende Norm dieser Operatoren ist überraschenderweise identisch. Wir beginnen mit der Defintion einer Orthonormalfolge in $ \L^2([0,1]) $.

\begin{df}
	Die Folge der \textit{Rademacherfunktionen} ist durch
	\begin{align*}
		\rho_n : [0,1] \to \R, \  t \mapsto \sign \left(\sin (2^n \pi t)\right)
	\end{align*}
	gegeben. 
\end{df}

Wir können $ \rho_1 = \chi_{[0,\nicefrac{1}{2}]} - \chi_{[\nicefrac{1}{2},1]}  $ ohne Probleme periodisch auf die reelle Achse fortsetzen. Dies eröffnet uns die Möglichkeit induktiv
\begin{align*}
	\rho_{n+1}(t) = \rho_1(2^n t)
\end{align*}
für alle $ n \in \N $ zu zeigen. 
Außerdem ist es sinnvoll $ \rho_0 = 1 $ zu setzen. 

\begin{figure} [H]
	\centering
	\subfigure{
		\begin{tikzpicture}[scale =3.5]
			\draw[->] (-0.1,0) -- (1.1,0);
			\draw[->] (0,-0.6) -- (0,0.6) ;
			\node at (1,0.3) {$ \rho_1 $};
			%\draw (1,-.05) -- (1,.05) node[below=4pt] {$\scriptstyle 1$};
%			\draw (-.05,0.5) -- (.05,0.5) node[left=3pt] {$\scriptstyle 1$};
%			\draw (-.05,-0.5) -- (.05,-0.5) node[left=3pt] {$\scriptstyle -1$};
			\draw(0,0.5) -- (0.5,0.5);
			\draw(0.5,-0.5) -- (1,-0.5);
		\end{tikzpicture} 
		
	} 
	\subfigure{
		\begin{tikzpicture}[scale =3.5]
			\draw[->] (-0.1,0) -- (1.1,0);
			\draw[->] (0,-0.6) -- (0,0.6) ;
			\node at (1,0.3) {$ \rho_2 $};
			%\draw (1,-.05) -- (1,.05) node[below=4pt] {$\scriptstyle 1$};
%			\draw (-.05,0.5) -- (.05,0.5) node[left=3pt] {$\scriptstyle 1$};
%			\draw (-.05,-0.5) -- (.05,-0.5) node[left=3pt] {$\scriptstyle -1$};
			\draw(0,0.5) -- (0.25,0.5);
			\draw(0.25,-0.5) -- (0.5,-0.5);
			\draw(0.5,0.5) -- (0.75,0.5);
			\draw(0.75,-0.5) -- (1,-0.5);
		\end{tikzpicture}
    

}
	\subfigure{
	\begin{tikzpicture}[scale =3.5]
		\draw[->] (-0.1,0) -- (1.1,0);
		\draw[->] (0,-0.6) -- (0,0.6) ;
		%\draw (1,-.05) -- (1,.05) node[below=4pt] {$\scriptstyle 1$};
%		\draw (-.05,0.5) -- (.05,0.5) node[left=4pt] {$\scriptstyle 1$};
%		\draw (-.05,-0.5) -- (.05,-0.5) node[left=4pt] {$\scriptstyle -1$};
		
		\node at (1,0.3) {$ \rho_3 $};
		\draw(0,0.5) -- (0.125,0.5);
		\draw(0.125,-0.5) -- (0.25,-0.5);
		\draw(0.25,0.5) -- (0.375,0.5);
		\draw(0.375,-0.5) -- (0.5,-0.5);
		\draw(0.5,0.5) -- (0.625,0.5);
		\draw(0.625,-0.5) -- (0.75,-0.5);
		\draw(0.75,0.5) -- (0.875,0.5);
		\draw(0.875,-0.5) -- (1,-0.5);
	\end{tikzpicture}
	
	
}  
\end{figure} 
Die Abbildung zeigt die ersten drei Rademacherfunktionen.
Deren Struktur ergibt die Orthogonalitätseigenschaft
\begin{align*}
	\int \limits_0^1 \rho_{n_1}^{p_1}(t) \cdots \rho_{n_k}^{p_k}(t)
	\dx{t}
	=
	\begin{cases}
		1, \ \text{falls alle } p_j \text{ für } 1 \leq j \leq k \text{ gerade}\\
		0, \ \text{sonst}
	\end{cases}
\end{align*}
für $ n_1,...,n_k \in \N$ paarweise verschieden und $ p_1,...,p_k \in \N_0  $.
 Hieraus folgt insbesondere die Orthonormalfolgeneigenschaft in  $ \L^2([0,1])$ und es giltt
\begin{align*}
	\int \limits_0^1 \left|
	\sum_{ i= 1}^\infty a_i \rho_i(t)
	\right|^2
	\dx{t}
	=
	\sum_{ i = 1}^\infty
	|a_i |^2
\end{align*}
für beliebige $ (a_n) \in \ell^2 $. Für die Folge der Radermacherfunktionen erhalten wir eine weitere Chintschin-Ungleichung.

\begin{genericthm}{Chintschin-Ungleichung II}\label{th:khinchin_inequality_3}
	Für alle $ 0 \leq p \leq \infty $ existieren Konstanten $ A_p, B_p > 0 $, sodass für alle $ (a_n) \in \ell^2 $
	\begin{align*}
		A_p \|(a_n) \|_{\ell^2} \leq 
		\left\|
		\sum_{i=1}^\infty a_i \rho_i(t)
		\right\|_{\L^p([0,1])}
		\leq
		B_p \|(a_n) \|_{\ell^2}
	\end{align*}
	gilt.
\end{genericthm}

\begin{proof}
	Wir zeigen die Aussage zunächst für natürliche $ p \in \N $ und
	eine beliebige endliche Folge $ a_1,...,a_m \in \R $.
	Die Aussage selbst ergibt sich dann durch das Sandwichkriterium und die übrigen Fälle führen wir auf den natürlichen Fall zurück.
	Mit der Reihenentwicklung der Exponentialfunktion folgt die Ungleichung
	\begin{align*}
		|x|^p 
		< 
		p!\left(1 + \frac{|x|^p}{p!}\right)
		<
		p! e^{|x|}
	\end{align*}
	für $ p \in \N $ und $ x \in\R $.
	Nun definieren wir $ f(t) := \sum_{i=1}^m a_i \rho_i(t) $ und erhalten
	\begin{align*}
		\int \limits_0^1
		|f(t) |^p
		\dx{t}
		\leq 
		p!
		\int \limits_0^1
		e^{|f(t)|}
		\dx{t}
		\leq 
		p!
		\int \limits_0^1
		e^{f(t)} + e^{-f(t)}
		\dx{t}.
	\end{align*}
	Mit der Orthonormalität der Rademacherfunktionen folgt
	\begin{align*}
		\| f \|_{\L^2} = \left(\sum \limits_{i=1}^m a_i^2\right)^\frac{1}{2}.
	\end{align*}
	\newpage
	Wir nehmen wegen der Orthonormalität o.B.d.A. $ \|f \|_{\L^2} = 1$ an.
	Dann gilt:
	\begin{align*}
		\int \limits_{0}^1 e^{f(t)} \dx{t} 
		=
		\int \limits_{0}^1 
		\prod \limits_{i=1}^m e^{a_i \rho_i(t)} \dx{t} 
		=
		\prod \limits_{i=1}^m
		\int \limits_{0}^1 
		 e^{a_i \rho_i(t)} \dx{t}
		=
		\prod \limits_{i=1}^m
		\cosh(a_i)
		\leq 
		\prod \limits_{i=1}^m
		e^{\frac{a_i^2}{2}}
		=
		e^{\sum \frac{a_i^2}{2}}
		=
		e^\frac{1}{2}.
	\end{align*}
	Dass Vertauschen des Produkts und des Integrals erhalten wir mithilfe der Potenzreihe der Exponentialfunktion.
	Die Abschätzungen $ \cosh(a) \leq e^{\nicefrac{a^2}{2}} $ folgt aus dem Vergleich der Potenzreihen.
	Aufgrund der Symmetrie gilt
	\begin{align*}
		\int \limits_0^1
		|f(t) |^p
		\dx{t}
		\leq 
		p!
		\int \limits_0^1
		e^{f(t)} + e^{-f(t)}
		\dx{t}
		\leq 
		2 p! e^\frac{1}{2}
	\end{align*}
	für $ p \in \N $.
	Damit haben wir die Ungleichung für natürliche $ p $ gezeigt.
	
	Sei $ 2 \leq p < \infty $ mit $ k := \lceil p \rceil $ und sei $ x_1,...,x_m \in \R $ eine beliebige endliche Folge .
	Dann erhalten wir aufgrund der Homogenität und der Monotonie der $ \L^p $-Norm:
	\begin{align*}
		\left(\sum \limits_{ i = 1}^m a_i^2\right)^\frac{1}{2}
		=
		\|f\|_{\L^2} \leq \|f\|_{\L^p} \leq 
		\left\|
		\sum \limits_{ i = 1}^m
		a \rho_i
		\right\|_{\L^k}
		\leq
		\left(2\cdot k! \cdot e^{\frac{1}{2}}\right)^\frac{1}{k}
		\left(\sum \limits_{ i = 1}^m a_i^2\right)^\frac{1}{2}.
	\end{align*}
	Also gilt diese Form der Chintschin-Ungleichung auch in diesem Fall.
	
	Nun wenden wir uns mit $ 0 < p < 2 $ dem Abschluss des Beweises zu.
	Wir definieren 
	\begin{align*}
		\theta := \frac{1}{2- \frac{p}{2}} = \frac{2}{4- p}.
	\end{align*}
	Dann gilt $ 0 < \theta <1 $ und $ p \theta + 4(1- \theta) = 2 $. Damit liefert die Hölderungleichung:
	\begin{align*}
		\|f\|_{\L^2}^2
		&=
		\int
		\limits_0^1
		|f(t)|^2
		\dx{t}
		=
		\int
		\limits_0^1
		|f(t)|^{p\theta} 
		|f(t)|^{4(1-\theta)}
		\dx{t}
		\leq
		\left(
		\int
		\limits_0^1
		|f(t)|^p
		\dx{t}
		\right)^\theta
			\left(
		\int
		\limits_0^1
		|f(t)|^4
		\dx{t}
		\right)^{(1-\theta)}\\
		&=
		\|f\|_{\L^p}^{p\theta}
		\|f\|_{\L^4}^{4(1-\theta)}
		\leq
		\|f\|_{\L^p}^{p\theta}	B_4^{4(1-\theta)} \|f\|_{\L^2}^{4(1-\theta)}.
	\end{align*}
	Dies ist äquivalent zu
	\begin{align*}
	B_4^{\frac{-4(1-\theta)}{p\theta}}	\|f\|_{\L^2}^{\frac{2 - 4(1- \theta) }{p\theta }}
	=
	B_4^{\frac{2p - 4}{p}} \|f\|_{\L^2}
	\leq \| f \|_{\L^p}
	\end{align*}
	und mit der Monotonie der $ \L^p $-Norm folgt $ \|f\|_{\L^p} \leq \|f \|_{\L^2} $.
\end{proof}
Alternativ lässt sich die Chintschin-Ungleichung induktiv beweisen.
Man führt die Induktion über $ p = 2^n $ durch und verwendet zwischen $ 2^n $ und $ 2^{n+1} $ die Monotonie der $ \L^p([0,1])$-Norm. 
In \cite{Diestel1995} ist der Induktionsanfang für $ p = 4 $ geführt. Für $ p = 2  $ wäre dieser nach Konstruktion trival.
Die Wahl von $ p = 4 $ hat den Seiteneffekt, dass $ A_4  = 1$ und  $B_4 \leq \sqrt[4]{3} $ bekannt sind. Dies kommt uns in dem Beweis der nachfolgenden Ungleichung zugute.



\begin{genericthm}{Die Grothendieck-Ungleichung}\label{th:grothendieck_inequality}
	Es existiert eine Konstante $ \kappa_G  >0$, sodass für alle Hilberträume $ H $, $ n \in \N $, alle $ n \times n  $-Matrizen $ A =  (a_{i j}) $ und alle Vektoren $ x_1,...,x_n,y_1,...,y_n \in B_H $ die Ungleichung
	\begin{align}\label{eq:grothendieck_ineq_general}
		\left|
			\sum \limits_{i,j} a_{ij}  \langle x_i,y_j \rangle 
		\right|
		\leq
		\kappa_G
		\max 
		\limits_{\|s\|_{\infty} \leq 1 , \|t\|_{\infty} \leq 1 }
		\left|
		\langle 
		s, A t
		\rangle
		\right|
		=
		\kappa_G
		\max 
		\limits_{|s_i| \leq 1 , |t_j| \leq  1}
		\left|
		\sum \limits_{i,j} 
		a_{ij} s_i t_j
		\right|
	\end{align}
	gilt.
	
\end{genericthm}
Die Konstante $ \kappa_G $ nennen wir auch \textit{Grothendieck-Konstante}.
Das Maximum der rechten Seite können wir als Operatornorm der dualen Abbildung von $ A =(a_{ij}) : \ell_{(n)}^\infty \to \ell_{(n)}^1 $ sehen.
Wenn wir uns auf reelle Vektorräume beschränken, erhalten wir mit dem Maxiumusprinzip von Bauer \cite{Bauer1960}
\begin{align*}
	\max 
	\limits_{\|s\|_{\infty} \leq 1 , \|t\|_{\infty} \leq 1 }
	\left|
	\langle 
	s, A t
	\rangle
	\right|
	=
\max 
\limits_{\|s\|_{\infty} = 1 , \|t\|_{\infty}=  1 }
	\left|
	\langle 
	s, A t
	\rangle
	\right|
	=
	\max 
	\limits_{\alpha_i, \alpha^\prime_j = \pm 1 }
	\left|
	\sum \limits_{i,j} 
	a_{i,j} \alpha_i \alpha^\prime_j
	\right|.
\end{align*}
Dieses Maximumsprinzip besagt, dass das Maximum eines stetigen Funktionals bezüglich der Maximumsnorm an den Eckpunkten des Einheitskreises angenommen wird.
Für den reellen Fall folgt für die Grothendieckungleichung:
\begin{align}\label{eq:grothendieck_ineq_real}
	\left|
	\sum \limits_{i,j} a_{ij}  \langle x_i,y_j \rangle 
	\right|
	\leq
	\kappa_G
	\max 
	\limits_{\alpha_i, \alpha^\prime_j = \pm 1 }
	\left|
	\sum \limits_{i,j} 
	a_{i,j} \alpha_i \alpha^\prime_j
	\right|.
\end{align} 

\begin{proof}
Es genügt die Aussage für reelle Matrizen und Hilberträume zu zeigen.
Im komplexen Fall folgt die Ungleichung durch die Zerlegung in Real-und Imaginärteil. Zu Beginn setzen wir
\begin{align*}
	\| A \|_\infty
	&=
	\sup
	\limits_{|s_i| \leq 1 , |t_j| \leq  1}
	\left|
	\sum \limits_{i,j = 1}^n a_{ij} s_i t_j
	\right|
	=
	\sup
	\limits_{|s_i| \leq 1 , |t_j| \leq  1}
	\left|
	\sum \limits_{ij} a_{ij} s_i t_j
	\right|
\end{align*}
und
\begin{align*}
	 \interleave A\interleave   
	&=
	\sup \limits_{ H}
	\sup \limits_{x_i,y_j \in  B_H}
	\left|
	\sum \limits_{i,j = 1}^n a_{ij} 
	\langle x_i, y_j \rangle
	\right|
	=
	\sup \limits_{ H}
	\sup \limits_{x_i,y_j \in  B_H}
	\left|
	\sum \limits_{ij} a_{ij} 
	\langle x_i, y_j \rangle
	\right|.
\end{align*}
Hierbei kennzeichnet $ \sup_H $ das Supremum über alle Hilberträume.
Insbesondere wird das nachgeschaltete Supremum für separable Hilberträume angenommen, da dieses über endliche viele Elemente gebildet wird. Sei $ H $ ein beliebiger seperabler Hilbertraum und $ (e_n) $ eine Orthonormalbasis.
Dann besitzt jedes $ x  \in H$ die Darstellung
\begin{align*}
	x = \sum \limits_{i = 1}^\infty
	\langle x, e_i \rangle
	e_i
\end{align*}
und wir setzen $ \xi_n  = \langle x, e_n \rangle $. Mit den Rademacherfunktionen definieren wir
\begin{align*}
	X : [0,1 ] \to \R, \ t \mapsto 
	\sum \limits_{ i = 1}^\infty \xi_i \rho_i(t).
\end{align*}
Dann gilt $ X \in \L^2([0,1]) $, $ \| x \|_H = \|X\|_{\L^2} $ und
\begin{align*}
	\langle x ,y \rangle_H
	=
	\int \limits_0^1 X(t) Y(t) \dx{t}
\end{align*}
für $ x,y \in H $. Dies folgt unmittelbar aus der Orthonormalität der Rademacherfunktionen.
Wir nehmen übergangsweise an, dass die aus $ x \in B_H $ entstehenden $ X  $ durch $ M $ beschränkt sind.
Das heißt $ |X(t)| \leq M $ für alle $ t \in [0,1] $. Dann würde die Ungleichung unmittelbar durch 
\begin{align*}
	\left|
	\sum \limits_{ij} 
	a_{ij} \langle x_i,y_j \rangle
	\right|
	=
	\left|
	\int \limits_{0}^{1}
	\sum \limits_{ij} 
	a_{ij} X_i(t) Y_i(t)
	\dx{t}
	\right|
	\leq 
	M^2
	\cdot
	\int \limits_{0}^1
	\left|
	\sum \limits_{ i j } a_{ij}
	\frac{X_i(t)}{M} \frac{Y_j(t)}{M}
	\right|
	\dx{t}
	\leq 
	M^2 \cdot \| A \|_\infty
\end{align*}
folgen. Insbesondere gilt dann $ \|A\|_H \leq M^2 \|A\|_\infty $.
Leider ist dies nicht im Allgemeinen erfüllt, jedoch werden wir $ X $ geschickt aufteilen.
Zu $ x \in H $ und $ M > 0  $ definieren wir mit 
\begin{align*}
	X^B(t)
	:=
	\begin{cases}
		X(t), &\ \textrm{falls } |X(t)| \leq M\\
		M \cdot \sign X(t), &\ \textrm{sonst}
	\end{cases}
\end{align*}
den beschränkten Teil und durch
\begin{align*}
	X^U(t):= X(t) - X^B(t)
\end{align*}
den unbeschränkten Teil. Damit gilt $ X = X^B + X^U $ und wir benötigen noch eine Abschätzung für den unbeschränkten Teil. 
Für ein $ t \in [0,1] $ mit $ X^U(t) \neq 0 $ gilt $ |X(t)| > M$ und $ |X^U(t)| = |X(t)| - M  $. Hierauf wenden wir die elementare Ungleichung
\begin{align*}
	s \leq m + \frac{s^2}{4m}, \ m,s > 0
\end{align*}
an und erhalten mit der Chintschin-Ungleichung \ref{th:khinchin_inequality_3} die Abschätzung:
\begin{align*}
	\| X_U \|_{\L^2}^2
	&=
	\int \limits_0^1
	| X_U(t)|^2
	\dx{t}
	\leq
	\frac{1}{16M^2} 
	\int \limits_0^1
	| X(t)|^4
	\dx{t}
	= 
	\frac{1}{16M^2} 
	\| X \|_{\L^4}^4
	\leq 
	\frac{1}{16M^2} B_4^4 \| x \|_H^2
	\leq 
	\frac{1}{16M^2} B_4^4\\
	&\leq 
	\frac{3}{16M^2}.
\end{align*}
Hierbei geht ein, dass $ B_4 \leq \sqrt[4]{3} $ gilt.
Nun liegen alle Werkzeuge vor um den Beweis abzuschließen.
Sei $ x_1,...x_n,y_1...,y_n \in B_H $ beliebig.
Dann gilt:
\begin{align*}
	&\left|
	\sum \limits_{ i j } a_{ij} \langle x_i,y_j \rangle
	\right|
	=
	\left|
	\int \limits_0^1
	\sum \limits_{ i j } a_{ij} X_i(t) Y_j(t)
	\dx{t}
	\right|\\
	=
	&\left|
	\int \limits_0^1
	\sum \limits_{ i j } a_{ij} (X^B_i(t) + X^U_i(t) ) \cdot Y^B_i(t) + X_i(t) Y^U_i(t) )
	\dx{t}
	\right|\\
	\leq
	&\left|
	\int \limits_0^1
	\sum \limits_{ i j } a_{ij} X^B_i(t)  \cdot Y^B_i(t) 
	\dx{t}
	\right|
	+
	\left|
	\int \limits_0^1
	\sum \limits_{ i j } a_{ij} X^U_i(t)  \cdot Y^B_i(t) 
	\dx{t}
	\right|
	+\left|
	\int \limits_0^1
	\sum \limits_{ i j } a_{ij} X_i(t)  \cdot Y^U_i(t) 
	\dx{t}
	\right|\\
	\leq
	&2 M^2 \|A \|_\infty+ 
	\left|
	\int \limits_0^1
	\sum \limits_{ i j }  a_{ij}  \|X_i^U \|_{\L^2}\frac{X^U_i(t)}{\|X_i^U \|_{\L^2}}  \cdot Y^B_i(t) 
	\dx{t}
	\right|
	+
	\left|
	\int \limits_0^1
	\sum \limits_{ i j } a_{ij}
	\|Y_i^U \|_{\L^2}
	 X_i(t)  \cdot \frac{Y^U_i(t) }{\|Y_i^U\|_{\L^2}}
	\dx{t}
	\right|\\
	&M^2 \| A \|_\infty+ 
	\frac{\sqrt{3}}{4M}
	\left|
	\sum\limits_{ i j }
	a_{ij} \left\langle\frac{X^U_i(t)}{\|X_i^U \|_{\L^2}}, Y^B_i(t)\right\rangle_{\L^2}
	\right|+
	\frac{\sqrt{3}}{4M}
	\left|
	\sum\limits_{ i j }
	a_{ij} \left\langle X_i(t) \frac{Y^U_i(t)}{\|Y_i^U \|_{\L^2}}, \right\rangle_{\L^2}
	\right|\\
	\leq 
	&M^2 \| A \|_\infty+ 
	\frac{\sqrt{3}}{4M} 2 \interleave A\interleave   
	=
	M^2 \| A \|_\infty +  \frac{\sqrt{3}}{2M}  \interleave A\interleave   .
\end{align*} 
Hierbei ist zu beachten, dass nach Konstruktion $ \| X_i \|_{\L^2},\|Y_i\|_{\L^2} \leq 1 $ gilt.
Damit folgt
\begin{align*}
 \interleave A\interleave    \leq	M^2 \| A \|_\infty +  \frac{\sqrt{3}}{2M}   \interleave A\interleave   .
\ \Leftrightarrow \  \interleave A\interleave    \leq \frac{3M^3}{2M - \sqrt{3}} \| A \|_\infty
\end{align*}
für alle $ M > \nicefrac{\sqrt{3}}{2} $. Das optimale $ M $ aus diesem Vorgehen lässt sich durch Kurvendiskussion bestimmen.
\end{proof}

Der exakte Wert der Grothendieck-Konstanten $ \kappa_G $ ist unbekannt.
Jedoch kann diese durch
\begin{align*}
	\frac{\pi}{2} \leq \kappa_G \leq \frac{\pi}{2 \ln (\sqrt{2} + 1)}
\end{align*}
abgeschätzt werden. Genauere Informationen hierzu finden sich in \cite{Pietsch1980}.
Der nächste Satz zeigt, dass jede unbedingt summierbare Folge in $ \ell^1 $ absolut summierbar in $ \ell^2 $ ist.


\newpage
\begin{genericthm}{Der Satz von Grothendieck}
	Jeder beschränkte lineare Operator $ T : \ell^1 \to \ell^2 $ ist absolut summierbar mit
	\begin{align*}
		\pi_1(T) \leq \kappa_G \| T \|.
	\end{align*}
	%wobei $ K  > 0$ nicht von $ T $ abhängt.
\end{genericthm}
\begin{proof}
	Wir nehmen o.B.d.A. an, dass $ \|T\| \leq 1 $ gilt und 
	reellwertige Folgen vorliegen.
	Sei $ (x_n) $ unbedingt summierbar in $ X = \ell^1 $.
	Damit ist der Operator
	\begin{align*}
		 \psi : X^\prime = \ell^\infty \to \ell^1, \ x^\prime \mapsto (\langle x^\prime , x^{(n)} \rangle_{X^\prime} )_n 
	\end{align*}
	nach der Charakterisierung der unbedingten Konvergenz \ref{th:equi_uncond_2} kompakt. Wegen der 
	Äquivalenz zu der perfekten Summierbarkeit folgt
	für beliebige Vorzeichen $ \alpha_i = \pm 1 $
	\begin{align*}
		\left\|
		\sum 
		\limits_{ i= 1}^\infty
		\alpha_i x_i
		\right\|
		&=
		\sup \limits_{ x^\prime  \in B_{\ell^\infty } }
		\left|
		\left\langle
		x^\prime, 
		\sum 
		\limits_{ i= 1}^\infty
		\alpha_i x_i
		\right\rangle 
		\right|
		=
		\sup \limits_{ x^\prime  \in B_{\ell^\infty } }
		\left|
		\sum 
		\limits_{ i= 1}^\infty
		\alpha_i
		\left\langle 
		x^\prime, 
		 x_i
		\right\rangle 
		\right|
		\leq
		\sup \limits_{ x^\prime  \in B_{\ell^\infty } }
		\sum 
		\limits_{ i= 1}^\infty
		\left|
		\left\langle 
		x^\prime, 
		x_i
		\right\rangle 
		\right|\\
		&= 
		\| \psi \|_{\ell^\infty \to \ell^1}
	\end{align*}
	für alle $ \alpha_i = \pm 1 $.
	Unser Ziel ist nun, $ \sum \|T x_i \|_{\ell^2} < \infty $ mithilfe der Grothendieck-Ungleichung \ref{th:grothendieck_inequality} zu zeigen.
	Seien hierfür $ m \in \N $ und $ \delta > 0 $ beliebig.
	Wir wählen $ n \geq m $ und $ y_1,...,y_m \in \ell_{(n)}^1 \subset \ell^1$ mit
	\begin{align*}
		\| x_i- y_i \| \leq \frac{\delta}{2^i}
	\end{align*}
	für $ 1 \leq i \leq m $. Sollte $ n  $ echt größer als $ m $ sein, setzen wir $ y_{m+1} = ... = y_n = 0 $.
	Die $ n \times n  $-Matrix $ A = (a_{ij}) $ ergibt sich aus
	den Linearkombinationen
	\begin{align*}
		y_i 
		=
		\sum \limits_{ j= 1}^n a_{ij} e_j
	\end{align*} 
	der Standartbasis des $ \ell_{(n)}^1 $.
	Bevor wir den Beweis abschließen können, benötigen wir noch zwei Identitäten.
	Es gilt
	\begin{align*}
		\sum \limits_{ i = 1}^n \| T y_i \|_{\ell^2}
		=
		\sum \limits_{ i = 1}^n 
		\left\| 
		\sum \limits_{ j= 1}^n
		a_{ij} T e_j
		 \right\|_{\ell^2}
		=
		\sum \limits_{ i = 1}^n 
		\sup\limits_{\tilde{z}_i \in B_{\ell_{(n)}^2}}
		\left|
		a_{ij}
		\sum \limits_{ i = 1}^n 
		\langle \tilde{z}_i , T e_j  \rangle
		\right|
		=
		\sum \limits_{ i, j = 1 }
		a_{ij}
		\langle z_i , T e_j  \rangle
	\end{align*}
	für geeignet gewählte $ z_1,...,z_n  $ in dem Einheitsball von $ \ell_{(n)}^2 $. 
	Dies können wir optimal für die linke Seite der Grothendieck-Ungleichung verwenden.\newpage
	Für die rechte Seite werden wir
	\begin{align*}
		\left\|
		\sum \limits_{i = 1}^n \alpha_i y_i
		\right\|
		&=
		\left\|
		\sum \limits_{i=1}^n
		\alpha_i
		\sum \limits_{j=1}^n
		a_{ij} e_j
		\right\|_{\ell^1}
		=
		\left\|
		\sum \limits_{j=1}^n
		\left(\sum \limits_{i=1}^n
		\alpha_i
		a_{ij}\right) 
		e_j
		\right\|_{\ell^1}
		=
		\sum \limits_{j=1}^n
		\left|
		\sum \limits_{i=1}^n 
		a_{ij} \alpha_i
		\right|\\
		&=
		\max 
		\limits_{ \alpha^\prime_j = \pm 1 }
		\left|
		\sum \limits_{ j= 1}^n \alpha^\prime_j
		\sum \limits_{ i = 1}^n
		a_{ij} \alpha_i
		\right|
		=
		\max 
		\limits_{ \alpha^\prime_j = \pm 1 }
		\left|
		\sum \limits_{i, j} a_{ij} \alpha_i \alpha^\prime_j
		\right|
	\end{align*}
	für beliebige $ \alpha_1,...,\alpha_n = \pm 1 $ verwenden.
	Mit den Vorüberlegungen können wir die Grotendieck-Ungleichung anwenden:
	\begin{align*}
		\sum \limits_{i=1}^m
		\|T x_i \|_{\ell^2}
		&\leq
		\sum \limits_{i=1}^m
		\|T x_i - T y_i \|_{\ell^2}
		+
		\sum \limits_{i=1}^m
		\|T y_i \|_{\ell^2}
		\leq 
		\delta \left(1-\frac{1}{2^m} \right)
		+
		\kappa_G
		\sum \limits_{i=1}^n
		\|T y_i \|_{\ell^2}\\
		&\leq 
		\kappa_G
		\sum \limits_{i=1}^n
		\|T y_i \|_{\ell^2}
		+
		\delta 
		=
		\kappa_G
		\sum \limits_{ i, j }
		a_{ij}
		\langle z_i , T e_j  \rangle
		+ 
		\delta\\	
		&\leq 
		\kappa_G
		\max 
		\limits_{\alpha_i= \pm 1, \alpha^\prime_j = \pm 1 }
		\kappa_G
		\left|
		\sum \limits_{i, j} a_{ij} \alpha_i \alpha^\prime_j
		\right|
		+ \delta 
		=
		\kappa_G
		\max 
		\limits_{\alpha_i= \pm 1 }
		\left\|
		\sum \limits_{i = 1}^m \alpha_i y_i
		\right\|
		+ \delta\\
		&\leq \kappa_G
		\max 
		\limits_{\alpha_i= \pm 1 }
		\left\|
		\sum \limits_{i = 1}^m \alpha_i x_i
		\right\|
		+(1+ \kappa_G) \delta
		\leq
		\kappa_G 
		\| \psi \|_{\ell^\infty \to \ell^1}
		+(1+ \kappa_G) \delta
	\end{align*}
	Wenn wir die Einschränkung $ \| T \| \leq 1 $ entfernen, erhalten wir
	mit $ m \to \infty $ und $ \delta \to 0 $ 
	\begin{align*}
		\sum \| T x_i \|_{\ell^2}
		\leq 
		\kappa_G
		\|T\| 
		\sup \limits_{ \alpha_i= \pm 1}
		\left\|
		\sum \limits \alpha_i x_i
		\right\|
	\end{align*}
	für beliebige Folgen $ (x_n)  $ in $ \ell^1 $. Insbesondere gilt diese Aussage für beliebige endliche Teilmengen von $ \ell^1 $ und es folgt $ \pi_1(T) \leq \kappa_G \|T\| $.
\end{proof}







%\subsection{Arbeitstitel : Endliche Repräsentierbarkeit}
%\textcolor{red}{\textbf{TODO:
%Mit dem Begriff der endlichen Repräsentierbarkeit lässt sich die Orliczeigenschaft verallgemeinern.
%}	} 
%
%\subsection{Arbeitstitel : Folgerungen aus der Banachraumtheorie}\label{comment:bessaga}
%\textcolor{red}{\textbf{TODO: Mithilfe von Theorie über Basisfolgen kann man den Satz von Bessaga-Pelczynski beweisen. \\
%Grob besagt dieser, dass die folgenden Aussagen äquivalent sind.
%\begin{itemize}
%	\item $ X $ besitzt keinen zu $ c_0 $ isomorphen Unterraum.
%	\item Schwach absolut konvergent  $\Rightarrow$  schwach konvergent
%	\item Schwach absolut konvergent  $\Rightarrow$  unbedingt konvergent
%	\item Schwach absolut konvergent  $\Rightarrow$  Normkonvergent
%\end{itemize}
%Interessant ist auch die Tatsache:
%Falls $ X $ einen zu $ c_0 $ isomorphen Unterraum besitzt, existiert eine schwach konvergente Reihe, welche Norm-divergiert.
%}	} 
\newpage
\chapter{Ausblick}


Die Untersuchung bedingt konvergenter Reihen lässt sich weiter fortsetzen. 
Die Folgerung von Chobanyan \ref{th:conclusion_chobanyan} setzt einen Banachraum vom Typ $ p $ voraus. 
Diese Voraussetzung lässt sich auf einen Banachraum vom \textit{Infratyp} $ p $\cite[Ch. 5]{Kadets1997} abschwächen.
Außerdem gibt es für derartige Räume eine offene Vermutung.
\begin{genericthm_no_num}{Satz\cite[Ch. 7]{Kadets1997}}
	Sei $ X $ ein Banachraum vom Infratyp-$ p $ und $ (x_n) $ eine Folge in $ X $
	mit $ \sum \| x_i \|^p < \infty $.
	Dann besitzt $ (x_n) $ die Steinitzeigenschaft.
\end{genericthm_no_num}
%Zu den Räumen mit der Infratyp $ p $-Eigenschaft gibt es noch eine offene Vermutung.
\begin{generic_no_num}{Vermutung\cite[Ch. 7]{Kadets1997}}
	Sei $ X $ ein Banachraum vom Infratyp $ p $.
	Dann ist $ (\|x_n\|^p) $ für jede perfekt unsummierbare Folge $ (x_n) $ unsummierbar.
\end{generic_no_num}
Die Umkehrung dieser Vermutung gilt: 
Sei $ (\|x_n\|^p) $ für jede perfekt unsummierbare Folge $ (x_n) $ unsummierbar. Dann ist $ X $ ein Banachraum vom Infratyp $ p $.


In dieser Arbeit wurde sich auf das Konvergenzverhalten bei Umordnung bezüglich der Norm beschränkt.
Dies lässt sich auf Topologien erweitern, welche nicht durch eine Norm beschreibar sind.
\begin{genericthm_no_num}{Satz von Banaszczyk\cite{Banaszczyk1990}}
	Sei $ (x_n) $ bedingt summierbar in einem metrisierbaren nuklearen Raum.
	Dann erfüllt $ (x_n) $ die Steinitzeigenschaft.
\end{genericthm_no_num}
Wie im ursprünglichen Satz von Steinitz \ref{th:lemma_of_steiniz} genügt hier die bedingte Summierbarkeit als Voraussetzung.

Den Satz von Orlicz wurde für $ \L^p $-Räume gezeigt. 
Für ein geeignetes $ p $ sind $ C $-konvexe Räume\cite[Ch. 5]{Kadets1997} vom $ M $-Kotyp $ p$.
Nach der Äquivalenz \ref{th:orlicz_equi_kotyp} folgt die Aussage von Orlicz.
Allgemeinere Versionen dieses Satzes finden sich in \cite[Ch. 3,Ch. 10, Ch.11]{Diestel1995}.
Genaue Zusammenhänge zwischen den Begriffen Typ $ p $, $ M $-Kotyp und Infratyp $ p $ sind in \cite[Ch. 5]{Kadets1997} und 
\cite[Ch. 11, Ch. 14., Ch. 16]{Diestel1995} dargestellt.
Nachdem in dieser Arbeit absolut summierbare Operatoren skizziert wurden, kann das Studium dieser in \cite[Ch. 8]{Albaic2006} und \cite{Diestel1995} fortgesetzt werden.
Weiterführend von der unbedingten Konvergenz sind Banachräume mit der UMD-Eigenschaft\cite[Ch. 4]{Hytoenen2016}.

%Absolut summierbare Operatoren sind  umfangreich dargestellt.




%In \cite[Ch. 5]{Kadets1997} 
%Dieser Satz gilt auch für $ C $-konvexe Räume\cite[Ch. 5]{Kadets1997}, denn diese Räume sind vom $ M $-Kotyp $ p $(siehe \ref{th:orlicz_equi_kotyp}).




%Die Typ $ p $ Eigenschaft folgt dann daraus.



\newpage

\blankpage


\newpage
\nocite{*}
\bibliographystyle{abbrv}
\bibliography{bib_file}
%\allowdisplaybreaks
\newpage \blankpage
\thispagestyle{empty}
\section*{Eigenständigkeitserklärung}

Hiermit bestätige ich, Philipp Beck, dass ich die vorliegende Arbeit eigenständig und ausschließlich unter Verwendung der angegebenen Quellen erstellt habe. 
Ich versichere, dass ich alle wörtlich oder sinngemäß aus anderen Werken übernommenen Aussagen als solche gekennzeichnet habe, und dass die eingereichte Arbeit weder vollständig noch in wesentlichen Teilen Gegenstand eines anderen Prüfungsverfahren gewesen ist.\\
\newline\\
\newline
\begin{tabular}{lp{2em}l}
	\hspace{3cm}   && \hspace{3cm} \\\cline{1-1}\cline{2-3}
	Datum     && Unterschrift
\end{tabular}
\vspace{1cm}
\section*{Danksagung}
Ein großer Dank geht an meinen Betreuer PD Dr. Peter H. Lesky und dessen Kollegen Apl. Prof. Dr. Jens Wirth.
Beide hatten stets ein offenes Ohr für auftretende Problem und gute Ratschläge, um diese zu behandeln.
%Er hatte die ganze Zeit ein offenes Ohr und gute Ratschläge für meine Probleme.
Ein herzliches Dankeschön geht an Alexander Marroquin, ohne den mir manche Unstimmigkeiten nicht aufgefallen wären.
Meinen Kommilitionen Arthur Günthner, Daniel Winkle und Volkan Bodur möchte ich ebenso für das Korrekturlesen und anregenden Diskussionen danken.
Ein weiteres Dankeschön geht an Moritz Gösling für die Korrektur der Rechtschreibung.
Außerdem ist noch die moralische Unterstützung von meiner Familie und Alisa Baransegeta zu erwähnen.
%Natürlich sollte ich meine Familie nicht vergessen, die mir in der ganzen Zeit den Rücken freigehalten hat.%Ein großer Dank geht auch an meine Familie für den Halt den sie mir gibt. 

\end{document}
