\section{Zusammenfassung}
In diesem Abschnitt wollen wir noch einmal die Resultate für $ S_0(\R^d) $ zusammenfassen.

Nach \ref{th:density_schwartzspace} liegt $ \S(\R^d) $ dicht in $ S_0(\R^d) $ und \ref{th:mod_space_banach} liefert uns die Banachraumeigenschaft.
Die Lemmas \ref{th:mod_space_tf_invariance} und \ref{th:mod_space_FT_invariance} liefern die Zeit-Frequenz-Invarianz und die Invarianz unter der Fouriertransformation.
Mit den Sätzen \ref{th:independence_of_window_norm} und \ref{th:mod_space_window_extention} kann ein beliebiges Fenster aus $ S_0 \setminus \{ 0\} $ für die Norm von $ S_0 $ gewählt werden.
Mit \ref{th:algebraic_properties} und \ref{th:am_space} wird $ S_0 $ zu einer Banachalgebra.
Die Minimaltät von $ S_0 $ erhalten wir aus \ref{th:minimality_feichtinger}.
Insbesondere folgen hieraus die restlichen Eigenschaften von $ S_0 $.
 




