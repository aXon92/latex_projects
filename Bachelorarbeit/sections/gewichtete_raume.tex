\section{$ L^{p,q}_m $-Räume}

In diesem Abschnitt werden wir gewichtete $ \L^p $-Räume mit gemischten Normen betrachten und  übertragen die bekannten Eigenschaften der $ \L^p $-Räume auf diese Räume. 
%Die Definition der Modulationsräume $ \M_m^{p,q} $ hat die Form
Die Modulationsräume werden über die Norm 
\begin{align*}
\| f \|_{\M_m^{p,q}} := \| V_g f \|_{\L_m^{p,q}}
\end{align*}
definiert. 
Also ist es naheliegend, dass wir uns zunächst mit den Eigenschaften von $ \L_m^{p,q} $ beschäftigen.



\begin{df}\label{weight_function}
	Eine Funktion $ v $ auf $ \R^{2d} $ heißt \textit{Gewichtsfunktion}, falls $ v \in \L^1_\mathrm{loc}(\R^{2d}) $ und $ v \geq 0 $ gilt.
	\begin{enumerate}[label =\textbf{(\roman*)}]
		\item 
		Eine Gewichtsfunktion $ v $ heißt \textit{submultiplikativ}, falls
		\begin{equation}\label{submultiplicative_weight}
		v(z_1 + z_2) \leq v(z_1) \cdot v(z_1)
		\end{equation}
		für $ z_1, z_2 \in \R^{2d} $ gilt.
	
		\item
		Eine Gewichtsfunktion $ m $ heißt $ v $-\textit{moderiert}, falls ein $ C > 0 $ existiert, sodass
		\begin{equation}\label{eq:v_moderate}
		m(z_1 + z_2)\leq C v(z_1) m(z_2)
		\end{equation}
		für $ z_1, z_2 \in \R^{2d} $ gilt.
		
		\item
		Die Gewichtsfunktionen $ m_1, m_2 $ heißen \textit{äquivalent}, falls
		ein $ C > 0 $ existiert, sodass
		\begin{equation}\label{equivalence_weights}
		C^{-1} m_1(z) \leq m_2(z) \leq C m_1(z)
		\end{equation}
		für $ z \in \R^{2d} $ gilt. 
		Wir schreiben hierfür $ m_1 \asymp m_2 $.
	\end{enumerate}
\end{df}


Angenommen, es existiert ein $ z_1  \in \R^{2d}$ mit $ m(z_1) = 0 $.
Dann gilt
\begin{align*}
m(z_1 + z_2) \leq C v(z_2) m(z_1) = 0
\end{align*}
für ein beliebiges $ z_2 \in \R^{2d} $ und damit auch $ m= 0 $.
Da eine submultiplikative Gewichtsfunktion $ v $ sich selbst moderiert, werden wir $ v,m > 0 $ voraussetzen.
Wegen $ v(0) \leq (v(0))^2 $ gilt $ v(0) \geq 1 $. 
Insbesondere folgt damit wegen $ v(z - z) \leq (v(z))^2 $
auch $ v(z) \geq 1 $.  

\newpage
\begin{lem}\label{th:submult_cont_symm}
	Sei $ v $ eine submultiplikative Gewichtsfunktion und $ \psi \geq 0  $ eine beliebige kompakt getragene stetige Funktion ungleich der Nullfunktion.
	Dann gelten:
	\vspace{-0.1cm}
	\begin{enumerate}[label =\textbf{(\roman*)}]
		\item $ v \ast \psi $ ist stetig und es gilt $ v \ast \psi \asymp v $.
		\item $ \tilde{v}(x,\omega) := \max\lbrace v(x,\omega),v(-x,\omega), v(x,-\omega), v(-x,-\omega) \rbrace$ ist symmetrisch. Das heißt
		\begin{align}
		\tilde{v}(x, \omega)
		= 
		\tilde{v}(-x,\omega)
		=
		\tilde{v}(x,-\omega)
		=
		\tilde{v}(-x,-\omega)
		\end{align} 
		ist für alle $ x,\omega \in \R^d $ erfüllt.
	\end{enumerate} 
\end{lem}

\begin{proof}
	\begin{enumerate}[label =\textbf{(\roman*)}]
		\item 
		Die Stetigkeit folgt direkt aus
		\begin{align*}
		(v \ast \psi)(z_n) 
		= \int \limits_{\R^{2d}} v(y) \psi(z_n-y) \td{y}
		\ \rightarrow \
		(v \ast \psi)(z_0)
		\end{align*}
		für $ z_n \to z_0 $.
		Wir müssen nun zeigen, dass
		\begin{align*}
		\frac{1}{C} v  \leq  \psi \ast v \leq C v  
		\end{align*}
		gilt.
		Der rechte Teil der Ungleichung folgt durch
		\begin{align*}
		(\psi \ast v)(z) = 
		\int \limits_{\R^{2d}} \psi(y) v(z - y) \td{y}
		\leq
		\underbrace{\max \limits_{y \in \supp  \ \psi}  \psi(y)
			\int \limits_{ \supp  \ \psi} v(-y) \td{y}
			}_{=: C_2} \cdot v(z)
		\end{align*}
		unter Verwendung der Submultiplikativität und der Stetigkeit über dem kompakten Träger.
		Wegen 
		\begin{align*}
		v( 0 ) = v(z- z) \leq v(z) v(-z)
		\ \Leftrightarrow \
		\frac{1}{v(z)} \leq \frac{v(-z)}{v(0)}
		\end{align*}
		erhalten wir $ \nicefrac{1}{v} \in \L^1_\mathrm{loc}(\R^{2d}) $.
		Des Weiteren benötigen wir 
		\begin{align*}
			v(z-y) v(y) \geq v(z-y+y) = v(z)
		\end{align*}
		für die linke Abschätzung. 
		Wir wählen ein $ y_0  $ mit $ \psi(y_0 ) > 0 $.
		Dann existiert ein $ \delta > 0  $ mit $ \psi(y) \geq \frac{1}{2} \psi(y_0) $
 		für alle $ y \in B_\delta(y_0) $.
		Nun folgt:
		\begin{align*}
		(\psi \ast v)(z) 
		\geq 
		\int \limits_{\overline{B_\delta(y_0)}} \psi(y) \frac{v(z)}{v(y)} \td{y}
		\geq 
		\underbrace{\min \limits_{y \in B_\delta(y_0)} \psi(y) 
		\int \limits_{\overline{B_\delta(y_0)}} \frac{1}{v(y)} \td{y}}_{=: C_1}
		 \cdot v(z).
		\end{align*}
		Mit der Wahl $ C := \max \lbrace C_1^{-1}, C_2 \rbrace $ folgt dann die gewünschte Aussage.
		\item Diese Aussage ist klar.
		
	\end{enumerate}
\end{proof}
Nun können wir submultiplikative Gewichtsfunktionen allgemein als stetig und symmetrisch voraussetzen. Wir werden von nun an mit $ v $ eine submultiplikative und mit $ m $ eine $ v $-moderate Gewichtsfunktion bezeichnen.
In dieser Arbeit betrachten wir polynomielle Gewichte, d.h.
\begin{align}\label{eq:poly_weight}
v_s(z) = (1+|z|)^s
=
\left( 1 + \left( \sum \limits_{j=1}^{2d} z_j^2 \right)^{\frac{1}{2}} \right)^s
= (1 +  (|x|^2 + |\omega|^2)^\frac{1}{2})^s
\end{align}
für $ z= (x,w) \in \R^{2d} $ und $ s > 0 $.

\begin{lem}\label{th:equivalence_poly_weight}
	Die Gewichte $ (1 + |x| + |\omega|)^s $ und $ (1 + |z|^2)^{\frac{s}{2}} $
	sind äquivalent zu $ v_s $.
\end{lem}

\begin{proof}
	Wir erhalten die Äquivalenz von $ (1 + |z|^2)^{\frac{s}{2}}  $ und $ (1+|x|+|\omega|)^s $ durch das asymptotische Verhalten von
	\begin{align*}
	\frac{1+ |z|^2}{(1 + |x| + |\omega|)^2}
	=
	\frac{1+ |x|^2 + |\omega|^2}{(1 + |x| + |\omega|)^2}.
	\end{align*}
	Die zweite Äquivalenz folgt analog aus dem Verhalten von 
	\begin{align*}
	\frac{1+ |z|^2}{(1+|z|)^2}. 
	\end{align*}
	Damit sind alle drei Gewichte zueinander äquivalent.
\end{proof}




\begin{lem}\label{korrolary_1_weightfunctions}
	
	\begin{enumerate}[label =\textbf{(\roman*)}]
		\item
		Sei $ m $ eine $ v-$ moderierte Gewichtsfunktion. Dann ist
		\begin{align}\label{eq:v_mod_prop_1}
		\frac{1}{C} \frac{m(z)}{v(t)} \leq m(z- t) \leq Cv(t) m(z)
		\end{align}
		erfüllt. Damit folgt
		\begin{align}\label{eq:v_mod_prop_3}
		\frac{1}{\tilde{C} v(z) } \leq m(z) \leq \tilde{C}   v(z)
		\end{align}
		für $ z \in \R^{2d} $ und es gilt
		\begin{align}\label{eq:v_mod_prop_2}
		\frac{1}{C^\prime} m(l)\leq m(z) \leq C^\prime m(l)
		\end{align}
		für alle $ z \in l + [0,1]^{2d} $ mit $ l \in \Z^{2d} $.
		
		\item 
		Die polynomiellen Gewichte $ v_s $ sind submultiplikativ.
		Desweiteren sind $ v_t $ und $ (v_t)^{-1} $ für $ 0 \leq t \leq s $ von $ v_s $ moderiert.
		
%		\item
%		Sei $ s > 2d $. Dann gilt
%		\begin{equation}\label{eq:poly_weight_1}
%		\left( \frac{1}{v_s} \ast \frac{1}{v_s}\right)(z) \leq  C_s \frac{1}{v_s}(z).
%		\end{equation}
%		 
	\end{enumerate}
	
\end{lem}

\begin{proof}
	\begin{enumerate}[label =\textbf{(\roman*)}]
		\item
		Seien $ z,t \in \R^{2d} $ beliebig.
		Dann gelten nach Definition
		$ m(z-t) \leq C m(z) v(t) $ und
		\begin{align*}
		m(z -t + t) \leq C m(z-t) v(t) 
		\ \Leftrightarrow \
		\frac{1}{C v(t)} m(z) \leq m(z-t). 
		\end{align*}
		Wir erhalten \eqref{eq:v_mod_prop_3} durch $ z = 0 $ und $ t = - w $.
		Sei nun $ z = w + l  $, wobei $ w \in [0,1]^{2d} $ und $ l \in \Z^{2d} $.
		Dann gilt $ m(z) \leq C m(l) v(w) $ und
		\begin{align*}
		m(l - w + w) \leq C m(z) v(w)
		\ \Leftrightarrow \
		\frac{1}{C v(w)} m(l) \leq m(z). 
		\end{align*}
		Mit $ C^\prime := C \max_{w \in [0,1]^{2d}} v(w) $ erhalten wir \eqref{eq:v_mod_prop_2}.
		
		\item
		Sei $ 0 \leq t \leq s $. Dann gilt wegen $ 1 + |z_1 + z_2|\leq (1+|z_1|)(1+|z_2|) $ auch
		\begin{align*}
		v_t(z_1 + z_2)
		\leq
		v_t(z_1) v_t(z_2)
		\leq 
		v_t(z_1) v_s(z_2). 
		\end{align*}
		Damit ist $ v_s $ submultiplikativ und $ v_t $ wird von $ v_s $ moderiert.
		Durch eine Äquivalenzumformung erhalten wir
		\begin{align*}
		(v_t(z_1))^{-1} \leq (v_t(z_1 + z_2))^{-1} v_s(z_2)
		\end{align*}
		mit $ z_1 = w_1 + w_2 $ und $ z_2 = -w_2 $, dass $ (v_t)^{-1} $ von $ v_s $ moderiert wird.
		
%		\item 
%		\textcolor{red}{\textbf{noch ausführen falls nötig.}}
	\end{enumerate}
\end{proof}

\begin{df}\label{df:mixed_norm}
	Sei $ m $ eine Gewichtsfunktion auf $ \R^{2d} $. 
	\begin{enumerate}[label =\textbf{(\roman*)}]
		\item 
		Wir sagen eine messbare Funktion $ F : \R^{2d} \to \C $ liegt
		in dem gewichteten Mischnormraum $ \L^{p,q}_m(\R^{2d}) $, falls
		\begin{align*}
		\| F \|_{\L^{p,q}_m} :=
		\left(		
		\ \int \limits_{\R^{2d}} 
		\left(
		\ \int \limits_{\R^{2d}}
		|F(x,\omega)|^p m(x,\omega)^p \td{x} 
		\right)^{\frac{q}{p}}
		\td{\omega}
		\right)^{\frac{1}{q}} < \infty
		\end{align*}
		erfüllt ist.
		Wenn $ p = q $ gilt, erhalten wir den gewichteten $ \L^p $-Raum auf $ \R^{2d} $ und wir schreiben
		$ \L^p_m(\R^{2d}) := \L^{p,p}_m(\R^{2d}) $.
	\end{enumerate}
	
	\begin{enumerate}
		\item[\textbf{(ii)}]
		Falls $ p = \infty $ oder $ q = \infty $ ersetzen wir die entsprechende Norm durch die $ \infty $-Norm, d.h.
		\begin{align*}
		\| F \|_{\L^{ \infty,q}_m} &:=
		\left(  \
		\int 
		\limits_{\R^{d}}
		\left(
		\esssupp \limits_{x \in \R^d} |F(x,\omega)| m(x,\omega) 
		\right)^{q}
		\td{\omega}
		\right)^{\frac{1}{q}}\\
		\| F \|_{\L^{ p, \infty}_m} &:=
		\esssupp \limits_{\omega \in \R^d}
		\left( \ 
		\int \limits_{\R^{d}}
		|F(x,\omega)|^p m(x,\omega)^p \td{x} 
		\right)^{\frac{1}{p}}.
		\end{align*}
		Für $ p  = q = \infty  $ definieren wir $ L^\infty_m(\R^{2d})  $ über die Norm
		\begin{align*}
		\| F \|_{\L^{ \infty}_m} &:=
		\esssupp \limits_{z \in \R^{2d}}
		|F(z)| m(z).
		\end{align*}
		Auch hier liegt $ F $ in den entsprechenden Räumen, falls die Norm endlich ist.
	\end{enumerate}
\end{df}

Wenn der Zusammenhang klar ist, werden wir im weiteren Verlauf $ \L^{p,q}_m $ anstatt $ \L^{p,q}_m(\R^{2d}) $ schreiben. Analog werden wir mit den anderen Räumen vorgehen.

\begin{sz}\label{hoelder_mixed_norm}
	Seien $ 1 \leq p,q \leq \infty  $, $ F \in \L^{p,q}_m $ und $ H  \in \L^{p^\prime,q^\prime}_{\nicefrac{1}{m}}$
	mit $ \nicefrac{1}{p} + \nicefrac{1}{p^\prime}=1=
	\nicefrac{1}{q} + \nicefrac{1}{q^\prime} $.
	Dann gilt die \textit{Hölder-Ungleichung}
	\begin{align}\label{eq:hoelder_mixed_norm}
	\left|
	\
	\int \limits_{\R^{2d}} 
	F(z) \overline{H(z)} 
	\td{z}
	\right|
	\leq 
	\| F \cdot H \|_{\L^1}
	\leq 
	\| F \|_{\L^{p,q}_m} 
	\| H \|_{\L^{p^\prime, q^\prime}_{\nicefrac{1}{m}}}
	\end{align}
	und $ F \cdot H \in \L^1 $.
\end{sz}

\begin{proof}
	Unser Ziel ist es, die Hölder-Ungleichung direkt zu übertragen. Es gelten\\
	$ F(\cdot, \omega) m(\cdot \omega) \in \L^p(\R^d) $ und $ H(\cdot , \omega) \nicefrac{1}{m(\cdot, \omega)} \in \L^{p^\prime}(\R^d) $ für fast alle $ \omega \in \R^d $.
	Damit erhalten wir durch zweimaliges Anwenden der üblichen Hölder-Ungleichung mit
	\begin{align*}
	\| F \cdot H \|_{\L^1}
	&=
	\int 
	\limits_{\R^d} 
	\int 
	\limits_{\R^d} |F(\cdot, \omega)| m(\cdot,\omega) |H(\cdot,\omega)| \frac{1}{m(\cdot, \omega)} \td{(x,\omega)} \\
	&\leq 
	\int \limits_{\R^d}
	\| F(\cdot , \omega) m(\cdot, \omega) \|_{p}
	\left\| H(\cdot,\omega) \frac{1}{m(\cdot , \omega)} \right\|_{p^\prime} \td{\omega}
	\leq 
%	\| \| F(\cdot , \omega) m(\cdot,\omega) \|_p \|_q
%	\left\| \left\| H(\cdot,\omega) \frac{1}{m(\cdot , \omega)}\right\|_{p^\prime} \right\|_{q^\prime}
	 \| F \|_{\L^{p,q}_m} \| H \|_{\L^{p^\prime,q^\prime}_{\nicefrac{1}{m}}}
	< \infty
	\end{align*}
	die gewünschte Aussage.
\end{proof}

\newpage

\begin{sz}\label{minkowski_mixed_norm}
	Sei $ 1 \leq p,q \leq \infty  $ und $ F,G \in \L^{p,q}_m $.
	Dann gilt mit
	\begin{align*}
	\| F + G \|_{\L^{p,q}_m} \leq \| F \|_{\L^{p,q}_m} + \| G \|_{\L^{p,q}_m} 
	\end{align*}
	die \textit{Minkowski-Ungleichung}.
\end{sz}

\begin{proof}
	Diese Ungleichung folgt wieder durch Anwenden der üblichen Minkowski-Ungleichung. Wir erhalten mit
	\begin{align*}
	\| F + G \|_{\L^{p,q}_m}^q
	&=
	\int 
	\limits_{\R^d}
	\| (F+G)(\cdot, \omega) m(\cdot, \omega) \|_p^q \td{\omega} 
	\leq 
	\int 
	\limits_{\R^d}
	(\| F(\cdot, \omega) m(\cdot, \omega) \|_{p} + \| G(\cdot, \omega) m(\cdot, \omega) \|_{p})^q \td{\omega}\\
	&\leq
	(\| F \|_{\L^{p,q}_m} + \| G\|_{\L^{p,q}_m})^q
	\end{align*}
	die gewünschte Aussage.
\end{proof}


\begin{sz}
	Sei $ 1 \leq p,q \leq \infty  $. Dann ist $ \L^{p,q}_m $ ein Banachraum.
\end{sz}

\begin{proof}
	Sei $ (F_k)_{k \in \N} $ ein Folge in $ \L_m^{p,q} $ und $ \sum_{k=1}^\infty F_k $ absolut konvergent. Wir setzen $ \| \cdot \| := \| \cdot \|_{\L^{p,q}_m} $. 
	Wir betrachten nun 
	\begin{align*}
	(x,\omega) \mapsto \sum \limits_{k=1}^\infty | F_k(x,\omega) | = G(x,\omega)
	\end{align*}
	und setzen $ G_n(x, \omega) = \sum_{k=1}^n | F_k(x,\omega) | $.
	Dann ist $ G_n \in \L^{p,q}_m $ und die Minkowski-Ungleichung liefert
	\begin{align*}
	\| G_n \| \leq \sum \limits_{k=1}^n \| F_k \| 
	\leq \sum \limits_{k=1}^\infty \| F_k \| = a 
	\end{align*}
	mithilfe der absoluten Konvergenz. Unser nächstes Ziel, ist es den Satz der monotonen Konvergenz anzuwenden. Durch zweimaliges Anwenden erhalten wir 
	\begin{align*}
	\int \limits_{\R^d} \left( \ \int \limits_{\R^d}
	|G(x,\omega)|^p m(x,\omega)^p \td{x}
	\right)^{\frac{q}{p}} \td{\omega}
	&=
	\int \limits_{\R^d} 
	\lim \limits_{n \to \infty}
	\left( \ 
	\int \limits_{\R^d}
	|G_n(x,\omega)|^p m(x,\omega)^p \td{x} \right)^{\frac{q}{p}} \td{\omega}\\
	&= \lim \limits_{n \to \infty} \| G_n \|^q \leq a^q,
	\end{align*}
	womit $ G $ bis auf eine Nullmenge $ N $ endlich sein muss. Damit ist auch
	\begin{align*}
	F(x,\omega) :=
	\begin{cases}
	\sum_{k=1}^\infty F_k(x,\omega), &\quad (x,\omega) \in \R^{2d} \setminus N\\
	\quad \quad  0, \quad &\quad (x, \omega) \in N
	\end{cases}
	\end{align*}
	aufgrund der Vollständigkeit von $ \R $ bzw. $ \C $ eine messbare Funktion.
	Wegen 
	\begin{align*}
	\| F \| \leq \| G  \| < \infty
	\end{align*}
	folgt $ F \in \L^{p,q}_m $. Gilt nun noch $ \sum_{k=1}^\infty F_k = F $ bezüglich $ \| \cdot \| $, sind wir fertig.
	Für
	\begin{align*}
	H_n(\cdot, \omega) 
	:= 
	\left| \sum \limits_{k=n}^\infty F_k(\cdot, \omega) \right|^p
	\end{align*}
	gilt $ | H_n(\cdot,\omega) | \leq G(\cdot, \omega)  $, $ G(\cdot, \omega) \in \L^p $ und $ H_n(x, \omega) \to 0 $ für alle $x, \omega \in \R^d $.
	Mit majorisierter Konvergenz erhalten wir dann
	\begin{align*}
	\| H_n(\cdot, \omega) m(\cdot, \omega) \|_p \to 0
	\end{align*}
	für alle $ \omega \in \R^d $. 
	Außerdem gilt $ \| H_n(\cdot, \omega) m(\cdot, \omega)\|_p 
	\leq \| G(\cdot, \omega) \|_p$ und $ \| G(\cdot , \omega ) \|_p\in \L^q $.
	Mit majorisierter Konvergenz folgt dann
	\begin{align*}
	\lim \limits_{n \to \infty}
	\| H_n \|
	=
	\lim \limits_{n \to \infty} 
	\| \| H_n(\cdot , \omega) m(\cdot, \omega) \|_p \|_q
	=0
	\end{align*}
	und somit die gewünschte Aussage.
\end{proof}

\begin{sz}\label{th:dualtiy_mixed_norm}
	Seien $1 \leq  p,q < \infty $ und $ \nicefrac{1}{p} + \nicefrac{1}{p^\prime}=1=
	\nicefrac{1}{q} + \nicefrac{1}{q^\prime} $.
	Dann gilt $ (\L^{p,q}_m)^\prime = \L^{p^\prime,q^\prime}_m $ und die Dualität ist durch
	\begin{align}\label{eq:dualtiy_mixed_norm}
	\langle H,F \rangle
	=
	\int 
	\limits_{\R^{2d}} F(z) \overline{H(z)} \td{z} 
	\end{align} 
	für $ F \in \L^{p,q}_m $ und $ H \in \L^{p^\prime,q^\prime}_{\nicefrac{1}{m}}  $ gegeben.
	
\end{sz}

\begin{proof}
	Wird bewiesen in \cite{benedek1961}.
\end{proof}

\newpage

\begin{sz}
	Seien $ 1 \leq p,q \leq \infty  $ und $ m $ von $ v $ moderiert. Dann ist $  \L^{p,q}_m  $ invariant unter den Translationen $ T_z $ für $ z \in \R^{2d} $ und es gilt
	\begin{align}\label{eq:invariance_translation_mixed_norm}
	\| T_z F \|_{ \L^{p,q}_m }
	\leq C v(z)
	\| F \|_{\L^{p,q}_m}.
	\end{align}
\end{sz}

\begin{proof}
	Sei $ F \in \L^{p,q}_m $.
	Mit $ z = (u, \eta) $ erhalten wir direkt
	\begin{align*}
	\| T_z F \|_{\L^{p,q}_m}
	&= 
	\left(		
	\ \int \limits_{\R^{2d}} 
	\left(
	\ \int \limits_{\R^{2d}}
	|F(x - u ,\omega - \eta )|^p m(x,\omega)^p \td{x} 
	\right)^{\frac{q}{p}}
	\td{\omega}
	\right)^{\frac{1}{q}}\\
	&=
	\left(		
	\ \int \limits_{\R^{2d}} 
	\left(
	\ \int \limits_{\R^{2d}}
	|F(x,\omega)|^p m(x + u,\omega + \eta )^p \td{x} 
	\right)^{\frac{q}{p}}
	\td{\omega}
	\right)^{\frac{1}{q}}\\
	&\leq
	C \left(		
	\ \int \limits_{\R^{2d}} 
	\left(
	\ \int \limits_{\R^{2d}}
	|F(x,\omega)|^p m(x,\omega)^p v(u, \eta)^p \td{x} 
	\right)^{\frac{q}{p}}
	\td{\omega}
	\right)^{\frac{1}{q}}\\
	&=
	C v(z) \|  F \|_{\L^{p,q}_m} < \infty
	\end{align*}
	unter Ausnutzung, dass $ m $ von $ v $ moderiert wird.
\end{proof}

%\begin{sz}\label{attributes_weighted_spaces}
%	Sei $ m $ eine $ v $-moderierte Gewichtsfunktion und $ 1 \leq p,q \leq \infty  $.
%	Dann gelten:
%	\begin{enumerate}
%		\item $ \L^{p,q}_m $ ist ein Banachraum.
%		\item $  \L^{p,q}_m  $ ist invariant unter den Translationen $ T_z $ für $ z \in \R^{2d} $ und es gilt
%		\begin{align*}
%		\| T_z F \|_{ \L^{p,q}_m }
%		\leq C v(z)
%		\| F \|_{\L^{p,q}_m}.
%		\end{align*}
%		\item
%		Sei $ F \in \L^{p,q}_m $ und $ H  \in \L^{p^\prime,q^\prime}_{\nicefrac{1}{m}}$
%		mit $ \nicefrac{1}{p} + \nicefrac{1}{p^\prime}=1=
%		\nicefrac{1}{q} + \nicefrac{1}{q^\prime} $.
%		Dann gilt die Hölderungleichung
%		\begin{align*}
%		\left|
%		\
%		\int \limits_{\R^{2d}} 
%		F(z) \overline{H(z)} 
%		\td{z}
%		\right|
%		\leq 
%		\| F \|_{\L^{p,q}_m} 
%		\| H \|_{\L^{p^\prime, q^\prime}_{\nicefrac{1}{m}}}
%		\end{align*}
%		und $ F \cdot H \in \L^1 $.
%		
%		\item
%		Sei $ p,q < \infty $.
%		Dann gilt $ (\L^{p,q}_m)^\star = \L^{p^\prime,q^\prime}_m $ und die Dualität ist durch
%		\begin{align*}
%		H(F) 
%		= 
%		\langle F,H \rangle
%		=
%		\int 
%		\limits_{\R^{2d}} F(z) \overline{H(z)} \td{z} 
%		\end{align*} 
%		für $ F \in \L^{p,q}_m $ und $ H \in \L^{p^\prime,q^\prime}_{\nicefrac{1}{m}}  $ gegeben.
%	\end{enumerate}
%\end{sz}
%
%\begin{proof}
%	\begin{enumerate} 
%	\item 
%	\textcolor{red}{\textbf{TODO.}}
%	\item
%	Sei $ F \in \L^{p,q}_m $.
%	Mit $ z = (u, \eta) $ erhalten wir direkt
%	\begin{align*}
%	\| T_z F \|_{\L^{p,q}_m}
%	&= 
%	\left(		
%	\ \int \limits_{\R^{2d}} 
%	\left(
%	\ \int \limits_{\R^{2d}}
%	|F(x - u ,\omega - \eta )|^p m(x,\omega)^p \td{x} 
%	\right)^{\frac{q}{p}}
%	\td{\omega}
%	\right)^{\frac{1}{q}}\\
%	&=
%	\left(		
%	\ \int \limits_{\R^{2d}} 
%	\left(
%	\ \int \limits_{\R^{2d}}
%	|F(x,\omega)|^p m(x + u,\omega + \eta )^p \td{x} 
%	\right)^{\frac{q}{p}}
%	\td{\omega}
%	\right)^{\frac{1}{q}}\\
%	&<
%	C \left(		
%	\ \int \limits_{\R^{2d}} 
%	\left(
%	\ \int \limits_{\R^{2d}}
%	|F(x,\omega)|^p m(x,\omega)^p v(u, \eta) \td{x} 
%	\right)^{\frac{q}{p}}
%	\td{\omega}
%	\right)^{\frac{1}{q}}\\
%	&=
%	C v(z) \| T_z F \|_{\L^{p,q}_m} < \infty
%	\end{align*}
%	unter Ausnutzung, dass $ m $ von $ v $ moderiert wird.
%	\item 
%	\textcolor{red}{\textbf{TODO.}}
%	\item
%	\textcolor{red}{\textbf{TODO.}}
%\end{enumerate}
%\end{proof}

\begin{sz}\label{th:convolution_mixed_norm}
	Sei $ m $ von $ v $ moderiert, $ F \in \L^1_v $ und $ G \in \L^{p,q}_m $.
	Dann gilt
	\begin{align*}
	\| F \ast G \|_{\L^{p,q}_m}
	\leq 
	\| F \|_{\L^1_v} \|G\|_{\L^{p,q}_m},
	\end{align*}
	wodurch wir insbesondere
	$ \L^1_v \ast \L^{p,q}_m \subseteq \L^{p,q}_m $ erhalten.
\end{sz}

\begin{proof}
	Sei $ H \in \L^{p^\prime,q^\prime}_{\nicefrac{1}{m}} $. Dann gilt mit Fubini und der Hölder-Ungleichung
	\begin{align*}
	| \langle F \ast G , H \rangle |
	&\leq 
	\int \limits_{\R^{2d}}
	\int \limits_{\R^{2d}}
	| F(w)| |G(z-w)| | H(z)| \td{w} \td{z}
	=
	\int \limits_{\R^{2d}}
	|F(w)|
	\int \limits_{\R^{2d}}
	|T_w G(z)| |H(z) |\td{z} \td{w}\\
	&\leq
	\int \limits_{\R^{2d}}
	|F(w)|  
	\|T_\omega G \|_{\L^{p,q}_m} \| H \|_{\L^{p^\prime,q^\prime}_{\nicefrac{1}{m}}}
	\td{w}
	\leq 
	C \int \limits_{\R^{2d}}
	|F(w)| v(w)\td{w} \ 
	\| G \|_{\L^{p,q}_m} \| H \|_{\L^{p^\prime,q^\prime}_{\nicefrac{1}{m}}}\\
	&=
	C \| F \|_{\L^1_v}\| G \|_{\L^{p,q}_m} \| H \|_{\L^{p^\prime,q^\prime}_{\nicefrac{1}{m}}}.
	\end{align*}
	Wir beachten hier, dass die Einsfunktion von jedem submultiplikativem Gewicht moderiert wird. Insgesamt erhalten wir die Aussage mit 
	\begin{align*}
	\qquad \qquad \qquad \qquad \qquad 
	\| F \ast G \|_{\L^{p,q}_m}
	= \sup \limits_{\| H \| = 1}
	|	\langle F\ast G , H \rangle|
	\leq 
	C \| F \|_{\L^1_v} \|G\|_{\L^{p,q}_m}.
	 \qquad
	\qquad \qquad \qquad \ \ \ \qedhere
	\end{align*}
\end{proof}



