\section{Kurzzeit-Fouriertransformation(STFT)}

%Wir werden unsere Theorie vollständig auf der STFT aufbauen.
In diesem Abschnitt definieren wir die Kurzzeit-Fouriertransformation und beweisen 
ihre elementaren Eigenschaften.
Zunächst werden wir einige fundamentale Identitäten zeigen, welche uns helfen, die STFT so allgemein wie möglich zu definieren. 
Im Anschluss beweisen wir die Parseval-und Plancherelgleichung für die STFT, welche wir für die Inversionsformel verwenden werden.
Wir benötigen die STFT, um später die Modulationsräume zu definieren.


\begin{df}
	Sei $g \neq 0$ eine glatte Abschneidefunktion.
	Diese bezeichnen wir als \textit{Fensterfunktion}.\index{Fensterfunktion}
	Dann ist die \textit{Kurzzeit-Fouriertransformation} einer Funktion $f$ abhängig von $g$ durch
	\begin{align*}
	V_g f(x, \omega) 
	:=
	\int \limits_{\mathbb{R}^d}
	f(t) \overline{g (t-x) }  e^{-2\pi \mathrm{i} t\cdot \omega }\td{t}
	\end{align*}
	für $x, \omega \in \mathbb{R}^d$ definiert.
\end{df}\noindent
Die Kurzzeit-Fouriertransformation ist mit 
\begin{align*}
(f,g) \mapsto V_g f
\end{align*}
eine Sesquilinearform. Meist lassen wir die Fensterfunktion $g$ jedoch fixiert.
Wir betrachten nun die asymmetrische Koordinatentransformation
\begin{align*}
\mathcal{T}_a F(x,t) = F(t,t-x)
\end{align*}
und die partielle Fouriertransformation
\begin{align*}
\mathcal{F}_2 F(x, \omega)
= 
\int \limits_{\R^d} F(x,t) e^{- 2 \pi \i t \cdot \omega} \td{t}
\end{align*}
von $F$ auf $\R^{2d}$.
Mit dem Tensorprodukt $(f \otimes g) (x,t) = f(x) g(t)$ erhalten wir somit
\begin{align*}
V_g f = \mathcal{F}_2 \mathcal{T}_a(f \otimes \overline{g})
\end{align*}
für $f, g \in \mathrm{L}^2(\R^d)$.
\newpage
\begin{sz}\label{fundamental_identities_STFT}
	Seien $f,g \in \mathrm{L}^2(\mathbb{R}^{d})$.
	Dann gelten die Identitäten
	\begin{align}
	V_g f(x,\omega)
	&=
	\widehat{(f \cdot T_x \overline{g})} (\omega)\label{eq:identity_STFT_1}\\
	&=
	\langle f, M_\omega T_x g \rangle\label{eq:identity_STFT_2}\\
	&=
	\langle \hat{f} , T_w M_{-x} \hat g \rangle\label{eq:identity_STFT_3}\\
	&=
	e^{-2 \pi \mathrm{i} x \cdot \omega} \widehat{(\hat{f} \cdot T_\omega \overline{ \hat{g} })} (-x)\label{eq:identity_STFT_4}\\
	&=
	e^{-2\pi \mathrm{i} x \cdot \omega} V_{\hat{g}} \hat{f}(\omega,-x)\label{eq:identity_STFT_5}\\
	&=
	e^{-2\pi \mathrm{i} x \cdot \omega}  (f \ast M_\omega g^\ast)(x) \label{eq:identity_STFT_7}\\
	&=
	(\hat{f} \ast M_{-x} \hat{g}^\ast)(\omega)\label{eq:identity_STFT_8}\\
	&=
	e^{-\pi \mathrm{i} x \cdot w } \int \limits_{\mathbb{R}^d} f\left( t + \frac{x}{2} \right) \overline{g\left( t - \frac{x}{2} \right)} e^{-2 \pi \mathrm{i} t \cdot \omega} \td{t}\label{eq:identity_STFT_9}
	\end{align}
	und $V_g f$ ist gleichmäßig stetig auf $\mathbb{R}^{2d}$.
\end{sz}

\begin{proof}
	Zuerst werden wir die Identitäten zeigen.
	Wegen
	\begin{align*}
	V_g f(x,\omega)
	=
	\int \limits_{\R^d} f(t) \overline{g(t-x)} 
	e^{-2\pi \i t  \cdot \omega} \td{t}
	=
	\widehat{(f \cdot T_x \overline{g})} (\omega)
	=
	\langle f, M_\omega T_x g \rangle
	\end{align*}
	sind \eqref{eq:identity_STFT_1} und \eqref{eq:identity_STFT_2} klar.
	Mit der Parsevalgleichung \eqref{eq:parseval} und \eqref{eq:trans_mod_fourier_2} folgen durch
	\begin{align*}
	\langle f, M_\omega T_x g \rangle
	&=
	\left\langle \hat{f}, \widehat{(M_\omega T_x g)} \right\rangle
	=
	\underbrace{ \langle \hat{f},e^{2\pi \i x \cdot \omega} M_{-x} T_\omega \hat{g} \rangle}_{
		= e^{-2\pi \i x \cdot \omega} V_{\hat{g}} \hat{f}(\omega,-x)
		}
	= 
	\langle \hat{f}, T_\omega M_{-x} \hat{g} \rangle
	\end{align*}
	die Identitäten \eqref{eq:identity_STFT_3} und \eqref{eq:identity_STFT_5}.
	Durch Anwenden von \eqref{eq:identity_STFT_1} erhalten wir \eqref{eq:identity_STFT_4}.
	Die Identität \eqref{eq:identity_STFT_8} folgt durch
	\begin{align*}
	\widehat{(f \cdot T_x \overline{g})} (\omega)
	=
	(\hat{f} \ast M_{-x}  \hat{\overline{g}} )(\omega)
	=
	(\hat{f} \ast M_{-x}  \hat{g^\ast} )(\omega)
	\end{align*}
	unter Verwendung von \eqref{eq:convolution_prop_2} und \eqref{eq:involution_fouriertrans}.
	Analog liefert dies mit
	\begin{align*}
	e^{-2 \pi \mathrm{i} x \cdot \omega} \widehat{(\hat{f} \cdot T_\omega \overline{ \hat{g} })} (-x)
	=
	e^{-2\pi \mathrm{i} x \cdot \omega}  (f \ast M_\omega g^\ast)(x)
	\end{align*}
	die Identität \eqref{eq:identity_STFT_7}.
	Die letzte Identität 
	\begin{align*} 
	V_g f(x,\omega)
	=
	e^{-\pi \mathrm{i} x \cdot w } \int \limits_{\mathbb{R}^d} f\left( t + \frac{x}{2} \right) \overline{g\left( t - \frac{x}{2} \right)} e^{-2 \pi \mathrm{i} t \cdot \omega} \td{t}
	\end{align*}
	folgt durch $u = t  - \frac{x}{2}$.\\
	Im zweiten Teil des Beweises werden wir die gleichmäßige Stetigkeit auf $\R^{2d}$ von $V_g f$ zeigen.
	Zunächst gilt 
	\begin{align*}
	\lim \limits_{x \to 0 } \| T_x f - f \|_2 = 0, 
	\end{align*}
	womit durch $(M_\omega f)^{\hat{ }} = T_\omega \hat{f}$
	auch die Aussage
	\begin{align*}
	\lim \limits_{\omega \to 0} \| M_\omega f - f \|_2
	= 
	\lim \limits_{\omega \to 0} \| T_\omega \hat{f} - \hat{f } \|_2
	= 0
	\end{align*}
	folgt. 
	Unser Ziel ist nun 
	\begin{align*}
	\forall_{\varepsilon > 0 } \exists_{\delta > 0}
	\forall_{(x,\omega) , (x^\prime,\omega^\prime)\in \R^{2d}}: \ | (x,\omega) - (x^\prime,\omega^\prime) | < \delta \ \Rightarrow \ | V_g f(x,\omega) -V_g f(x^\prime,\omega^\prime) | < \varepsilon
	\end{align*}
	zu zeigen.
	Zunächst gilt
	\begin{align*}
	|V_g f(x,\omega) - V_g f(x^\prime, \omega^\prime) |
	&= | \langle f, M_\omega T_x g \rangle - \langle f, M_{\omega^\prime} T_{x^\prime} g \rangle |\\
	&\leq
	\int \limits_{\R^d} |f(t)| \ |M_\omega T_x g(t) - M_{\omega^\prime} T_{x^\prime} g(t) | \td{t}\\
	&\leq 
	\| f \|_2 \cdot \| M_\omega T_x g - M_{\omega^\prime} T_{x^\prime} g \|_2
	\end{align*}
	mit Hilfe der Hölder-Ungleichung.
	Es folgt
	\begin{align*}
	\| M_\omega T_x g - M_{\omega^\prime} T_{x^\prime} g \|_2
	&= 
	\| M_\omega T_x g - M_{\omega^\prime} T_x g + M_{\omega^\prime} T_x g - M_{\omega^\prime} T_{x^\prime} g \|_2\\
	&\leq
	\| M_\omega T_x g - M_{\omega^\prime} T_x g \|_2 
	+ \| M_{\omega^\prime} T_x g - M_{\omega^\prime} T_{x^\prime} g\|_2\\
	&=
	\| (M_\omega - M_{\omega^\prime})T_x g\|_2
	+ \|M_{\omega^\prime} ( T_x - T_{x^\prime})g \|_2
	\end{align*}
	Wir werden die zwei Summanden seperat untersuchen.
	Für den Zweiten erhalten wir
	\begin{align*}
	\| M_{\omega^\prime}(T_x - T_{x^\prime}) g \|_2
	= \|(T_x - T_{x^\prime}) g \|_2
	= \|(T_{x^\prime}( T_{x-x^\prime} - \id)g \|_2
	= \| ( T_{x-x^\prime} - \id)g \|_2 < \frac{\varepsilon}{2 \| f \|_2}
	\end{align*}
	für $| x - x^\prime| <\delta_1$.
	Das Argument für den ersten Summanden ist ähnlich.
	Mit \eqref{eq:trans_mod_fourier_1} gilt
	\begin{align*}
	\F((M_\omega - M_{\omega^\prime}) T_x g)
	= 
	(T_\omega - T_{\omega}) M_{-x} g,
	\end{align*}
	wodurch mit der Parsevalgleichung 
	\begin{align*}
	\| (M_\omega - M_{\omega^\prime}) T_x g \|_2
	=
	\| (T_\omega - T_{\omega^\prime}) M_{-x} \hat{g} \|_2
	=
	\| (T_{\omega - \omega^\prime} - \id) g \|_2.
	\end{align*}
    folgt.Sei $ \tilde{g}  \in C_0^\infty(\R^d)$.
	Dann gilt
	\begin{align*}
	\| (T_{\omega - \omega^\prime} - \id) g \|_2 \leq 
	\| T_{\omega - \omega^\prime} g \|_2 + \| g \|_2 
	= 2 \| g \|_2,
	\end{align*}
	womit wir den Satz der majorisierten Konvergenz anwenden können.
	Da $ C_0^\infty $ dicht in $ \L^2 $ ist, finden wir ein $ \tilde{g}  \in C_0^\infty(\R^d)$ mit
	\begin{align*}
	\| \hat{g} - \tilde{g} \|_2 \leq \frac{\varepsilon}{6 \| f\|}.
	\end{align*}
	Damit existiert ein $ \delta_2 > 0 $ , sodass
	\begin{align*}
	\| T_\omega M_{-x} \tilde{g} - T_{\omega^\prime} M_{-x} \tilde{g} \|_2
	< 
	\frac{\varepsilon}{6 \| f \|_2}
	\end{align*} 
	für $ | \omega - \omega^\prime| < \delta_2 $ erfüllt ist.
	Insbesondere erhalten wir
	\begin{align*}
	\| (T_\omega - T_{\omega^\prime}) M_{-x} \hat{g} \|_2
	&\leq 
	\| T_\omega M_{-x} \hat{g} - T_\omega M_{-x} \tilde{g} \|_2
	+
	\| T_\omega M_{-x} \tilde{g} - T_{\omega^\prime} M_{-x} \tilde{g} \|_2
	+
	\| T_{\omega^\prime} M_{-x} \tilde{g} - T_{\omega^\prime} M_{-x} \hat{g} \|_2\\
	&= 2 \| \hat{g} - \tilde{g} \|_2
	+ 
	\| T_\omega M_{-x} \tilde{g} - T_{\omega^\prime} M_{-x} \tilde{g} \|_2
	< 
	\frac{\varepsilon}{2 \| f\|_2}
 	\end{align*}
	für $ | \omega - \omega^\prime| < \delta_2 $.
	Insgesamt folgt durch
	\begin{align*}
	| (x,\omega) - (x^\prime, \omega^\prime) | \leq
	| (x,0) -(x^\prime,0) | + | (0,\omega) - (0, \omega^\prime) | < \delta :=\min\lbrace \delta_1 , \delta_2 \rbrace
	\end{align*}
	die gewünschte Aussage.
\end{proof}

Der letzte Satz liefert uns nun die Möglichkeit, den Definitionsbereich der STFT zu diskutieren.
Sei $ f \in \L^p(\R^d) $ und $ g \in \L^{p^\prime}(\R^d) $ mit $ \nicefrac{1}{p} + \nicefrac{1}{p^\prime} = 1 $.
Dann liefert uns die Hölderungleichung $ f \cdot T_x \overline{g} \in \L^1(\R^d) $ und die STFT ist punktweise über
\begin{align*}
V_g f(x,\omega ) = \widehat{(f \cdot T_x \overline{g})} (\omega)
\end{align*}
definiert. Die Identität
\begin{align*}
V_g f(x,\omega) = \langle f, M_\omega T_x g \rangle
\end{align*}
hilft uns, falls das Integral nicht definiert ist.
Damit ist die STFT auch für $ f \in \mathcal{S}^\prime(\R^d)  $ und $ g \in \mathcal{S}(\R^d) $ im Dualitätssinne definiert.\\
In folgendem Lemma verstehen wir unter einem Isomorphismus einen linearen Homöomorphismus.
\newpage
\begin{lem}\label{th:iso_tensor_distri}
	\begin{enumerate}[label =\textbf{(\roman*)}]
		\item 
		Die Operatoren $ \mathcal{T}_a $ und $ \mathcal{F}_2 $ sind Isomorphismen auf $ \S^\prime(\R^{2d}) $.
		\item
		Seien $ f,g \in \S^\prime(\R^d) $.
		Dann gilt $ f \otimes g \in \S^\prime(\R^{2d}) $.
	\end{enumerate}
\end{lem}

\begin{proof}
	\begin{enumerate}[label =\textbf{(\roman*)}]
		\item 
		Wir betrachten zunächst $ (\tilde{\mathcal{T}}_a) \varphi(x,t)
		:= \varphi(x-t,x)  $.
		Dann gilt
		\begin{align*}
		(\tilde{\mathcal{T}}_a \circ \mathcal{T}_a) \varphi(x,t) = 
		\varphi (x,t)
		=
		(\mathcal{T}_a \circ \tilde{\mathcal{T}}_a) \varphi(x,t)
		\end{align*}
		und somit $ \mathcal{T}_a^{-1}  = \tilde{\mathcal{T}}_a$. Insbesondere ist 
		\begin{align*}
		\mathcal{T}_a : \S(\R^{2d}) \to \S(\R^{2d}), \
		\varphi \mapsto \mathcal{T}_a \varphi
		\end{align*}
		ein Homöomorphismus. 
		Sei $ f \in \S(\R^{2d}) $ und 
		\begin{align*}
		u_f(\varphi) 
		= \int \limits_{\R^{2d}} f(x,t) \overline{\varphi(x,t)} \td{(x,t)}
		\end{align*}
		die durch $ f  $ induzierte temperierte Distribution.
		Hierfür gilt
		\begin{align*}
		u_{\mathcal{T}_a f}(\varphi) 
		&=
		\int \limits_{\R^{2d}}
		f(t,t-x) \overline{\varphi(x,t)} \td{(x,t)}
		=
		\int \limits_{\R^{2d}}
		f(x,t) \overline{\varphi(x-t,x)} \td{(x,t)}\\
		&=
		\int \limits_{\R^{2d}}
		f(x,t) \overline{(\mathcal{T}_a^{-1})\varphi(x,t)} \td{(x,t)}.
		\end{align*}
		Damit können wir $ \mathcal{T}_a $ sinnvoll durch
		\begin{align*}
		\mathcal{T}_a F( \varphi)
		&:=
		F(\mathcal{T}_a^{-1} \varphi)\\
		\mathcal{T}_a^{-1} F( \varphi)
		&:=
		F(\mathcal{T}_a \varphi)
		\end{align*}
		auf $ \S^\prime(\R^{2d}) $ definieren und $ \mathcal{T}_a $ ist auf $ \S^\prime(\R^{2d}) $ bijektiv. 
		Sei $ F_n \to F $ in $ \S^\prime(\R^{2d}) $, d.h. $ F_n(\varphi) \to F(\varphi) $
		für alle $ \varphi \in \S(\R^{2d}) $.
		Dann gilt
		\begin{align*}
		\mathcal{T}_a F_n(\varphi )
		=
		F_n(\mathcal{T}_a^{-1} \varphi) 
		\to 
		F(\mathcal{T}_a^{-1}\varphi) = \mathcal{T}_a F(\varphi)
		\end{align*}
		und $ \mathcal{T}_a $ ist ein Homöomorphismus auf $ \S^\prime(\R^{2d}) $.
		Die partielle Fouriertransformation $ \mathcal{F}_2 : \S^\prime(\R^{2d}) \to \S^\prime(\R^{2d}) $
		ist nach \cite[Seite~129]{mit} ein Isomorphismus.
		
		\item
		Diese Aussage wird in \cite[Theorem~4.13]{mit} bewiesen .
	\end{enumerate}
\end{proof}

Durch das letzte Lemma lässt sich die STFT auch durch
\begin{align}\label{eq:STFT_tensorprod}
V_g f = \mathcal{F}_2 \mathcal{T}_a (f \otimes \overline{g})
\end{align}
für $ f,g \in \S^\prime(\R^{d}) $ definieren.

Ähnlich zur Fouriertransformation gibt es eine Art Parseval -und Plancherelgleichung für die  STFT.


\begin{lem}\label{th:STFT_modulation_TF_argument}
	Ist $ V_g $ definiert, so gilt
	\begin{align}
	V_g  (T_u M_\eta f) ( x, \omega)
	= 
	e^{-2 \pi \i u \cdot \omega}
	V_g f (x - u , \omega - \eta)
	=
	e^{-2 \pi \i u \cdot \omega}
	T_{(u,\eta)} V_g f(x,\omega)
	\end{align}
	für alle $ x, u, \omega , \eta \in \R^d $.
\end{lem}

\begin{proof}
	Durch Anwenden von \eqref{eq:trans_mod_commutation_relation_1} erhalten wir
	\begin{align*}
	M_{- \eta} T_{-u} M_\omega T_x
	=
	e^{2\pi \i u \cdot \omega} M_{\omega - \eta} T_{x - u }.
	\end{align*}
	Damit folgt durch \eqref{eq:identity_STFT_2}:
	\begin{align*}
	V_g (T_u M_\eta f)(x, \omega)
	= 
	\langle T_u M_\eta f , M_\omega T_x g \rangle
	=
	\langle f , M_{-\eta } T_{-u} M_\omega T_x g \rangle
	=
	e^{-2 \pi \i u \cdot \omega}
	V_g f (x - u , \omega - \eta).
	\end{align*}
\end{proof}

\begin{genericthm}{STFT-Parseval}
	Seien $ f_1,f_2, g_1,g_2 \in \L^2(\R^d) $.
	Dann gilt 
	\begin{align}\label{orthogonality_relation}
	\langle V_{g_1} f_1, V_{g_2} f_2 \rangle =
	\langle f_1,f_2 \rangle  \overline{ \langle g_1, g_2 \rangle}.
	\end{align}
	Insbesondere gilt dann auch $ V_{g_1} f_1, V_{g_2} f_2 \in \L^2(\R^{2d}) $.
\end{genericthm}

\begin{proof}
	Wir bemerken zunächst, dass die Operatoren $ \T_a  $ und $ \F_2 $ unitär auf $ \L^2(\R^{2d}) $ sind. 
	Dass $ \T_a  $ unitär ist sehen wir direkt durch Substitution. $ \F_2 $ ist unitär, da die Fouriertransformation auf $ \L^2 $ unitär ist. 
	%\textcolor{red}{\textbf{Warum? Nachweisen!}}
	Damit gilt durch 
	\begin{align*}
	\langle V_{g_1} f_1 , V_{g_2} f_2 \rangle_{\L^2(\R^{2d})}
	= 
	\langle \F_2 \T_a (f_1 \otimes \overline{g_1}),
	\F_2 \T_a (f_2 \otimes \overline{g_2})
	\rangle
	=
	\langle
	f_1 \otimes \overline{g_1}, f_2 \otimes \overline{g_2}
	\rangle
	=
	\langle f_1,f_2 \rangle \overline{ \langle g_1, g_2 \rangle}
	\end{align*}
	die gewünschte Aussage.
\end{proof}

\begin{genericthm}{STFT-Plancherel}\label{th:orthogonality_relation_kor_1}
	Für $ f,g \in \L^2(\R^d) $ gilt
	\begin{align}
	\| V_g f \|_2 = \| f \|_2 \| g \|_2.
	\end{align}
\end{genericthm}

\begin{proof}
	Seien $ f,g \in \L^2(\R^d) $. Dann folgt direkt
	\begin{align*}
	\| V_g f \|_2 = \langle V_g f , V_g f \rangle = 
	\langle f, f \rangle 
	\overline{\langle g, g \rangle}
	=
	\| f \|_2 \| g \|_2
	\end{align*}
	mithilfe des letzten Satzes.
\end{proof}

Also ist die STFT für $ \| g \|_2 = 1 $  wegen 
\begin{align*}
\| V_g f \|_2 = \| f \|_2
\end{align*}
eine Isometrie zwischen $ \L^2(\R^d) $ und $ \L^2(\R^{2d}) $.
Damit ist $ V_g $ insbesondere injektiv und wir können $ f $ eindeutig durch $ V_g f $ bestimmen. Aus der Injektivität erhalten wir außerdem, dass
\begin{align*}
V_g f (x,\omega) = \langle f, M_\omega T_x g \rangle = 0 \
\Rightarrow
\
f = 0
\end{align*}
gilt. Damit gilt für ein fixiertes $ g \in \L^2(\R^d) $:
\begin{align*}
\lbrace
M_\omega T_x g \ | \ x, \omega \in \R^d \rbrace^\perp = \lbrace 0 \rbrace
\end{align*}
Also ist die lineare Hülle von $ \lbrace M_\omega T_x g \ | \ x, \omega \in \R^d \rbrace $ ein dichter Teilraum des $ \L^2(\R^d) $.


\begin{genericthm}{Inversionsformel}\label{th:inversion_formula}
	Seien $ g,\gamma \in \L^2(\R^d) $ mit $ \langle g,\gamma \rangle \neq 0$.
	Dann gilt
	\begin{align}\label{eq:inversion_formula}
	f = 
	\frac{1}{\langle g, \gamma \rangle}
	\int \limits_{\R^{2d}} V_g f(x,\omega) M_\omega T_x \gamma \td{(\omega,x)} 
	\end{align}
	für alle $ f \in \L^2(\R^d) $.
\end{genericthm}

Bevor wir in der Lage sind, diesen Satz zu beweisen, müssen wir schwach interpretierte Integrale diskutieren. Sei $ B $ ein beliebiger Banachraum und eine Abbildung $ g \ : \ \R^d \to B $ gegeben.
Dann wird 
\begin{align*}
f  = \int \limits_{\R^d} g(x) \td{x}
\end{align*}
als
\begin{align*}
l(h) := \langle f,h \rangle 
= 
\int \limits_{\R^d} \langle g(x) , h \rangle \td{x}
\end{align*}
für  alle $ h \in B^\prime $ interpretiert. Ist nun $ l $ ein beschränktes (konjugiert-)lineares Funktional, so ist das Integral als eindeutiges Element in $ B^{\prime \prime} $ geben.
Im Folgenden betrachten wir nur reflexive Räume, d.h. es gilt $ B = B^{\prime \prime} $. Für den Satz betrachten wir Integrale der Form 
\begin{align}\label{eq:unique_inverse_function}
f = \int \limits_{\R^{2d}} F(x,\omega) M_\omega T_x g \td{(\omega,x)}.
\end{align}
Dann ist durch
\begin{align*}
l(h) := \int \limits_{\R^{2d}} F(x,\omega) \overline{\langle h , M_\omega T_x g \rangle} \td{(\omega,x)}
\end{align*}
ein beschränktes konjugiert-lineares Funktional auf $ \L^2(\R^d) $ für $ F \in \L^2(\R^{2d}) $ gegeben.
Die Linearität ist klar und die Beschränktheit folgt durch Anwendung der Cauchy-Schwarz-Ungleichung und \ref{th:orthogonality_relation_kor_1}:
\begin{align}\label{eq:unique_inverse_bounded}
| l(h) |
\leq 
\| F \|_2 \| V_g h \|_2 = \| F \|_2 \| h \|_2 \| g \|_2 < \infty. 
\end{align}
Insgesamt liefert \eqref{eq:unique_inverse_function} also eine eindeutige Funktion in $ \L^2(\R^d) $ mit der Norm $ \| f \|_2 \leq \| F\|_2 \| g \|_2 $.
Nun können wir dazu übergehen, den Satz zu beweisen.

\begin{proof}
	Wegen \ref{th:orthogonality_relation_kor_1} gilt $ V_g f \in \L^2(\R^{2d}) $.
	Mit unseren vorherigen Überlegungen ist 
	\begin{align*}
	\tilde{f} = \frac{1}{\langle \gamma , g \rangle}\int \limits_{\R^{2d}} V_g f(x,\omega) M_\omega T_x \gamma \td{(\omega,x)}
	\end{align*}
	eine eindeutige, wohldefinierte Funktion in $ \L^2(\R^d) $. Unter Verwendung von \eqref{orthogonality_relation} erhalten wir dann
	\begin{align*}
	\langle \tilde{f}, h \rangle 
	&= 
	\frac{1}{\langle \gamma, g \rangle}
	\int \limits_{\R^{2d}} V_g f(x,\omega) \overline{\langle h , M_\omega T_x \gamma \rangle} \td{(\omega,x)}
	=
	\frac{1}{\langle \gamma, g \rangle}
	\langle V_g f , V_\gamma h \rangle\\
	&=
	\frac{1}{\langle \gamma, g \rangle} 
	\langle f, h \rangle  \langle \gamma , g \rangle = \langle f, h \rangle.
	\end{align*}
	Also gilt $ f = \tilde{ f} $.
\end{proof}

Wir definieren uns nun den Operator
\begin{align*}
A_g \ : \ \L^2(\R^{2d}) \to \L^2(\R^d), \quad
F \mapsto  \int \limits_{\R^{2d}} F(x,\omega) M_\omega T_x g \td{(\omega,x)}.
\end{align*}
Wegen 
\begin{align*}
\langle A_g F , h \rangle 
=
\int \limits_{\R^{2d}} F(x,\omega) \langle M_\omega T_x  g , h \rangle \td{(\omega,x)}
= 
\langle F , V_g h \rangle = 
\langle V_g^\prime F , h \rangle 
\end{align*}
ist dieser adjungiert zu der STFT $ V_g  :  \L^2(\R^d) \to \L^2(\R^{2d}) $.
Mit \eqref{eq:unique_inverse_bounded} ist die Beschränktheit auch klar.
Dies liefert uns die Möglichkeit, die Inversionsformel \eqref{eq:inversion_formula} auch durch
\begin{align}\label{eq:inversion_formula_operator}
\frac{1}{\langle \gamma,g \rangle } V^\prime_\gamma V_g = I
\end{align}
zu beschreiben.