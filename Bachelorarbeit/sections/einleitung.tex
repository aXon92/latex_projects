\section{Einleitung}
Im Jahre 1979 führte Hans-Georg Feichtinger die Feichtingeralgebra $ S_0(G) $ auf lokalkompakten Gruppen ein\cite{Feichtinger1979}.
Der österreichische Mathematiker hat eine Professur an der mathematische Fakultät der 
Universität Wien inne. 
Der Schwerpunkt seiner Arbeit liegt auf dem Gebiet der harmonischen Analysis.
Speziell ist hier die Zeit-Frequenz-Analyse hervorzuheben.
Die Struktur und Notation dieser Ausarbeitung richtet sich nach dem Buch  von Karlheinz Gröchenig\cite{noauthor2009Foundationsof}, welcher eng mit Feichtinger zusammenarbeitet.
\ \\
\\
In dieser Arbeit beschränken wir uns auf $ G = \R^d $.
Die Fouriertransformation ist ein Automorphismus auf dem Schwartzraum $ \S(\R^d) $.
Das Fehlen der Banachraumeigenschaft macht Beweise mit diesem Raum umständlich.
Dieses Problem lösen wir mit der Feichtingeralgebra $ S_0 $.
Im Gegensatz zu  $ \S(\R^d) $ ist $ S_0(\R^d) $ eine Banachalgebra, womit wir eine handhabbarere mathematische Struktur erhalten.
Wir werden sehen, dass der Schwartzraum $ \S(\R^d) $ dicht in $ S_0(\R^d) $ liegt.
Darüber hinaus übertragen sich einige nützliche Eigenschaften des Schwartzraums auf $ S_0 $.
Auf $ S_0 $ ist die Fouriertransformation ein Automorphismus.
Des weiteren ist die Possionformel gültig, die Translations-und Modulationsinvarianz, die Invarianz unter Automorphismen und die Invarianz unter dem Tensorprodukt sind erfüllt. Wir werden nicht alle diese Eigenschaften nachweisen.
Dies unterstreicht jedoch die Relevanz der Feichtingeralgebra $ S_0 $, da wir das Tripel $ (\S(\R^d), \L^2 , \S^\prime(\R^d) )$
durch $ (S_0, \L^2 , S_0^\prime) $ ersetzen können.
 
%Darunter sind die Invarianz unter der Fouriertransformation, Invarianz unter Automorphismen
%und 
Unsere größte Aufmerksamkeit werden wir jedoch der Minimalität von $ S_0 $ widmen.
Minimal bedeutet:
$ S_0 $ ist der kleinste nicht-triviale Banachraum, welcher unter Zeit-Frequenzverschiebungen invariant ist.
\ \\
\\
Zu Beginn dieser Arbeit führen wir die Kurzzeit-Fouriertransformation (STFT) ein.
Zusammen mit den gemischt-gewichteten $ \L^p $-Räumen liefert die STFT die Grundlage zur Definition der Modulationsräume, wozu auch $ S_0 $ gehört.
Der Weg über die Modulationsräume ist allgemeiner als nur die Feichtingeralgebra zu betrachten und die Beweise sind ein wenig aufwendiger.
Jedoch werden alle Eigenschaften dieser Räume direkt oder indirekt in den Beweis der Minimalität von $ S_0 $ einfließen
und erste Resultate zu $ S_0 $ liefern.
Nach diesem Beweis widmen wir uns einigen Eigenschaften, welche aus der Minimalität folgen.
\\


%Über diese Räume werden wir erste Eigenschaften von $ S_0 $ entdecken.
%Die Feichtingeralgebra $ S_0 $ ist einer dieser Modulationsräume. 


%Dies führt auf das Gelfandtripel $ (S_0, \L^2 , S_0^\prime) $