\section{Anhang}

%\begin{lem}
%	Sei $ (X, \| \cdot \|) $ ein normierter Raum. Dann gilt sind äquivalent:
%	\begin{enumerate}
%		\item $ X $ ist vollständig.
%		\item Jede absolut konvergente Reihe konvergiert in $ (X, \| \cdot \|) $
%	\end{enumerate}
%\end{lem}

%\begin{genericthm}{Dichtheitsprinzip}\label{th:densitiy_priniciple}
%	Gegeben seien die Banachräume  $ B_1 $ und $ B_2 $, ein dichter Unterraum $ X $ von $ B_1 $ und ein linearer Operator $ A : X \to B_2 $ mit
%	\begin{align}\label{eq:density_principle}
%	\| A f \|_{B_2} \leq C \| f \|_{B_1}
%	\end{align}
%	für  alle $  f \in X $.
%	Dann gilt \eqref{eq:density_principle} für alle $ f \in B_1 $ und $ A $ wird zu einem beschränkten linearen Operator von $ B_1  $ nach $ B_2 $ fortgesetzt.
%\end{genericthm}

%\begin{genericthm}{Open-Mapping-Theorem}\label{th:open_mapping}
%	Gegeben seien die Banach -oder Frecheträume  $ B_1 $ und $ B_2 $ und der surjektive beschränkte Operator $ A : B_1  \to B_2 $.
%	Dann ist $ A $ eine offene Abbildung.
%	Insbesondere können wir diesen auf Operatoren von $ \S(\R^d)  $ nach $ \S(\R^d) $ anwenden.
%\end{genericthm}
%
%\begin{genericthm}{Closed Range Theorem}\label{th:closed_range}
%	Gegeben seien die Banachräume $ B_1 $ und $ B_2 $
%	und ein beschränkter linearer Operator $ A : B_1 \to B_2 $.
%	Der Operator $ A $ hat ein abgeschlossenes Bild genau dann, wenn der 
%	adjungierte Operator $ A^\prime : B_2^\prime \to B_1^\prime $
%	ein abgeschlossenes Bild besitzt.
%	Insbesondere ist $ A $ genau dann surjektiv, wenn $ A^\prime $ injektiv ist und ein abgeschlossenes Bild hat.
%\end{genericthm}